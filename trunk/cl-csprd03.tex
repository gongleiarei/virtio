399 & 27 Jun 2014 & Michael S. Tsirkin & { changelog: fill changelog since draft2

This will make review easier.
 } \\
\hline
398 & 27 Jun 2014 & Michael S. Tsirkin & { acknowledgements: add draft 3 reviewers, sort

Add new reviewers and sort by name.
 } \\
\hline
397 & 27 Jun 2014 & Michael S. Tsirkin & { add draft2 acknowledgements

List people that provided comments on draft01 in the
acknowledgements section. Might be a nice way to encourage
reviews.
 } \\
396 & 26 Jun 2014 & Michael S. Tsirkin & { diff: back to green for added text

using blue does not work well for html

 } \\
\hline
393 & 26 Jun 2014 & Michael S. Tsirkin & { makediff: cleanup using begingroup/endgroup

Pawel Moll found a way to work around xetex bugs
without mangling latexdiff output using perl:

- define DIFbegin/DIFFend commands in preample

- pass --config FLOATENV= to latexdiff

Use this in preference to the fixupdiff perl script.

 } \\
\hline
391 & 26 Jun 2014 & Michael S. Tsirkin & { more latexdiff hacks

- change link color from green to pinegreen. Looks better to me.

- split footnotes out from their text, so that latexdiff
  does not consider them as a unit

- mark field command as safe for latexdiff, otherwise it's not shown in red

- hack adding DIFaddtext within footnotes could not handle
  case where latexdiff inserted multiple DIFadd within the
  footnote. Instead, detect when footnote is within
  DIFaddbegin/DIFdelbegin, add an extra DIFaddbegin/DIFdelbegin
  within the footnote.

 } \\
\hline
390 & 26 Jun 2014 & Michael S. Tsirkin & { diffpreamble: fix colors for links within diff
 } \\
\hline
389 & 26 Jun 2014 & Michael S. Tsirkin & { work around xetex bug

Too many \textbackslash color directives produce corrupted output
and this warning:

WARNING ** Color stack overflow. Just ignore.

Use script to reduce \# of these directives.

 } \\
\hline
388 & 26 Jun 2014 & Michael S. Tsirkin & { diffpreamble: remove duplicate text

latexdiff adds some

 } \\
\hline
387 & 26 Jun 2014 & Michael S. Tsirkin & { makediffpdf.sh: tool to create marked-up diff

make pdf diff using latexpand and latexdiff-fast
styles are set in diffpreamble.tex
in diff, links are coloured green instead of blue

Must be run within a git-svn clone of the spec repository.

Note: latexdiff has --flatten option, this and options
to select diff style don't seem to work well.

So flatten by script myself, and add our own preamble.

 } \\
\hline
386 & 25 Jun 2014 & Michael S. Tsirkin & { pci: minor fomatting tweak

Make table look better. Drop spaces that make
latexdiff stumble.

 } \\
\hline
385 & 25 Jun 2014 & Michael S. Tsirkin & { fixup pci: switch from subsystem id to device id

Patch sent to list (and applied by Rusty in

    pci: switch from subsystem id to device id

) did not actually implement what commit log said
it implements.

The result is wrong for transitional devices:

Adding 0xfff works for for net+block only;

for transitional pci devices there is no fixed scheme:
\~{}/projects/qemu/include \# grep VIRTIO_ID hw/virtio/*.h

hw/virtio/virtio-balloon.h:\#define VIRTIO_ID_BALLOON 5

hw/virtio/virtio-blk.h:\#define VIRTIO_ID_BLOCK 2

hw/virtio/virtio-net.h:\#define VIRTIO_ID_NET   1

hw/virtio/virtio-rng.h:\#define VIRTIO_ID_RNG    4

hw/virtio/virtio-scsi.h:\#define VIRTIO_ID_SCSI  8

hw/virtio/virtio-serial.h:\#define VIRTIO_ID_CONSOLE             3

\~{}/projects/qemu/include \# grep VIRTIO hw/pci/*.h

hw/pci/pci.h:\#define PCI_DEVICE_ID_VIRTIO_NET         0x1000

hw/pci/pci.h:\#define PCI_DEVICE_ID_VIRTIO_BLOCK       0x1001

hw/pci/pci.h:\#define PCI_DEVICE_ID_VIRTIO_BALLOON     0x1002

hw/pci/pci.h:\#define PCI_DEVICE_ID_VIRTIO_CONSOLE     0x1003

hw/pci/pci.h:\#define PCI_DEVICE_ID_VIRTIO_SCSI        0x1004

hw/pci/pci.h:\#define PCI_DEVICE_ID_VIRTIO_RNG         0x1005

hw/pci/pci.h:\#define PCI_DEVICE_ID_VIRTIO_9P          0x1009

I am guessing TC went by commit log when it approved the change,
so fixing it up directly.

Cc: Andrew Thornton <andrewth@google.com>

Cc: Rusty Russell <rusty@ozlabs.org>

Cc: Gerd Hoffmann <kraxel@redhat.com>

 } \\
\hline
384 & 17 Jun 2014 &  & { content.tex: VIRTIO-106: mention possibility of failing TMFs

This completes the review of virtio-scsi based on observations
from Google.

 } \\
\hline
383 & 16 Jun 2014 &  & { fix erroneous reference to Subsystem Device ID

Subsystem device ID only exists for PCI.

 } \\
\hline
382 & 16 Jun 2014 & Rusty Russell & { small virtio-serial fix

nr_ports does not exist in the spec.

 } \\
\hline
381 & 09 Jun 2014 &  & { virtio-scsi: support well-known logical units

The REPORT LUNS well-known logical unit is useful because it lets you
retrieve information about all targets with a single command.  It
also provides an easy way to send a no-op request.

 } \\
\hline
380 & 09 Jun 2014 &  & { consistent formatting of footnotes

Put the indicator before punctuation, and terminate the footnote with
a period.

 } \\
\hline
379 & 09 Jun 2014 &  & { virtio-scsi: additional SHOULDification

 } \\
\hline
378 & 09 Jun 2014 &  & { virtio-scsi: fixes to protection information

pi_bytesin is in the device-readable section.  Document lack of residual
field.  Use le32 instead of u32.

This matches the new patch series that Nicholas sent for vhost-scsi.

Cc: <nab@daterainc.com>

 } \\
\hline
377 & 05 Jun 2014 & Rusty Russell & { PCI: remove duplicate paragraph.

I chose the one which used the full nomenclature.

 } \\
\hline
376 & 05 Jun 2014 & Rusty Russell & { pci: switch from subsystem id to device id

Switch virtio pci to use standard device id instead of using the
subsystem id.

Unfortunately, there's no system to the way KVM allocated
device IDs to virtio devices, we'll just have to
specify these using a table, and use a new range for
future devices. For existing devices this results in
two possible IDs that all drivers will need to match.
Unfortunate, but the cost is small.

As a nice side effect, this allows us to make non-transitional
devices use IDs 0x40 and up, this reduces even further the
chance that a non transitional device will match legacy drivers.

And, it's probably a good idea to allow drivers to match
specific subsystem IDs if they

want to, so relax requirement for drivers to match all
subsystem/vendor ID configurations, but allow them to do so.
To avoid confusion, say "PCI Device ID" and
"PCI Subsystem ID" everywhere, prefix "PCI"
for other standard registers, for consistency.

VIRTIO-102

Note: issue reporter suggested 0x10XX where XX is the virtio
device ID. This would conflict with legacy devices, which seem
to have used 7 IDs in the range 0x1000 to 0x103f without any
system. Let's use a new range 0x1040 to 0x107f for
non-transitional devices, and add a table documenting the
transitional IDs used by in practice.

(Approved at 2014-06-04 meeting:

  \url{https://lists.oasis-open.org/archives/virtio/201406/msg00013.html} )

Cc: Andrew Thornton <andrewth@google.com>

 } \\
\hline
375 & 05 Jun 2014 & Rusty Russell & { pci: set ISR bit on config change with MSI-X

config changes are slow path anyway, so we
can as well set ISR bit to help drivers detect changes.
This allows sharing config interrupts which is what
issue reporter seems to ask for.

VIRTIO-104

(Approved at 2014-06-04 meeting:

  \url{https://lists.oasis-open.org/archives/virtio/201406/msg00013.html} )

 } \\
\hline
374 & 01 Jun 2014 & Michael S. Tsirkin & { NEEDS_RESET: trivial clarification

If device sets NEEDS_RESET before DRIVER_OK, it
can't send notifications to driver.

Make this clear.

 } \\
\hline
373 & 22 May 2014 & Rusty Russell & { Fix build of document

Error introduced in "VIRTIO-98: Add DEVICE_NEEDS_RESET":
seems that underscores in labels are verboten:

[133] [134] (./virtio-v1.0-csprd02.aux

! Missing \textbackslash endcsname inserted.

<to be read again>

                   \textbackslash unhbox

l.45 ...ts: Device Status Field\}\}\{subsection.1\}\{\}\}

 } \\
\hline
372 & 22 May 2014 & Rusty Russell & { content.tex: virtio-scsi review (VIRTIO-106)

As prompted by Rusty, add a few more MUST/SHOULD items for both devices
and drivers.  Clarify semantics of max_channel/max_id/max_lun, task_attr
and task management functions.

(As per minutes of meeting 2014-05-20:

    \url{https://lists.oasis-open.org/archives/virtio/201405/msg00034.html} )

 } \\
\hline
371 & 22 May 2014 & Rusty Russell & { content.tex: add support for protection information (VIRTIO-108)

This is a new feature that was suggested by Nicholas Bellinger, who

also provided a prototype implementation for vhost-scsi.

(As per minutes of meeting 2014-05-20:

	\url{https://lists.oasis-open.org/archives/virtio/201405/msg00034.html} )

 } \\
\hline
370 & 12 May 2014 & Rusty Russell & { VIRTIO-96: Assign device id to virtio input

Assign device id to virtio input

As passed at meeting 2014-05-06:

	\url{https://lists.oasis-open.org/archives/virtio/201405/msg00016.html}

 } \\
\hline
369 & 12 May 2014 & Rusty Russell & { VIRTIO-52: Make mac field read only.

As passed at meeting 2014-05-06:

	\url{https://lists.oasis-open.org/archives/virtio/201405/msg00016.html}

 } \\
\hline
368 & 12 May 2014 & Rusty Russell & { VIRTIO-107: Clarify net mac commands.

As passed at meeting 2014-05-06:

    \url{https://lists.oasis-open.org/archives/virtio/201405/msg00016.html}

 } \\
\hline
367 & 12 May 2014 & Rusty Russell & { VIRTIO-98: Add DEVICE_NEEDS_RESET.

As passed at meeting 2014-05-06:

        \url{https://lists.oasis-open.org/archives/virtio/201405/msg00016.html}

 } \\
\hline
366 & 12 May 2014 & Rusty Russell & { VIRTIO-87: limit descriptor chain length even with INDIRECT.

As passed at meeting 2014-05-06:

        \url{https://lists.oasis-open.org/archives/virtio/201405/msg00016.html}

 } \\
\hline
365 & 12 May 2014 & Rusty Russell & { VIRTIO-103: PCI: Note that turning off queue_enable is not supported.

As passed at meeting 2014-05-06:

        \url{https://lists.oasis-open.org/archives/virtio/201405/msg00016.html}

 } \\
\hline
364 & 12 May 2014 & Rusty Russell & { VIRTIO-103: PCI: require read-after-write on device_status reset.

As passed at meeting 2014-05-06:

        \url{https://lists.oasis-open.org/archives/virtio/201405/msg00016.html}

 } \\
\hline
363 & 12 May 2014 & Rusty Russell & { VIRTIO-99: Typo fixes.

As passed at meeting 2014-05-06:

	\url{https://lists.oasis-open.org/archives/virtio/201405/msg00016.html}

 } \\
\hline
362 & 07 May 2014 & Cornelia Huck & { net: fix device conformance sections

For the network device, we had two device normative sections both called
"setting up receive buffers", neither of which was referenced in the
conformance section.

Let's rename the second one to "processing of packets" which seems to
better match the actual contents and reference both of them from the
conformance statement for network devices.

Resolves VIRTIO-97.

Agreed on the 2014/05/06 TC meeting.

 } \\
\hline
361 & 07 Apr 2014 & Michael S. Tsirkin & { conformance.tex: fix references to mmio

Both device and driver conformance referred to ccw twice; let's add the
correct mmio references.

 } \\
\hline
