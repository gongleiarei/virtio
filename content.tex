\chapter{Basic Facilities of a Virtio Device}\label{sec:Basic Facilities of a Virtio Device}

A virtio device is discovered and identified by a bus-specific method
(see the bus specific sections: \ref{sec:Virtio Transport Options / Virtio Over PCI Bus}~\nameref{sec:Virtio Transport Options / Virtio Over PCI Bus},
\ref{sec:Virtio Transport Options / Virtio Over MMIO}~\nameref{sec:Virtio Transport Options / Virtio Over MMIO} and \ref{sec:Virtio Transport Options / Virtio Over Channel I/O}~\nameref{sec:Virtio Transport Options / Virtio Over Channel I/O}).  Each
device consists of the following parts:

\begin{itemize}
\item Device Status field
\item Feature bits
\item Configuration space
\item One or more virtqueues
\end{itemize}

Unless explicitly specified otherwise, all multi-byte fields are little-endian.
To reinforce this the examples use typenames like "le16" instead of "uint16_t".

\section{Device Status Field}\label{sec:Basic Facilities of a Virtio Device / Device Status Field}

The driver MUST update the Device Status field in the order below to
indicate its progress. This provides a simple low-level diagnostic:
it's most useful to imagine them hooked up to traffic lights on the
console indicating the status of each device.  The driver MUST NOT
clear a device status bit.

This field is 0 upon reset, otherwise at least one bit should be set:

\begin{description}
\item[ACKNOWLEDGE (1)] Indicates that the guest OS has found the
  device and recognized it as a valid virtio device.

\item[DRIVER (2)] Indicates that the guest OS knows how to drive the
  device. Under Linux, drivers can be loadable modules so there
  may be a significant (or infinite) delay before setting this
  bit.

\item[FEATURES_OK (8)] Indicates that the driver has acknowledged all the
  features it understands, and feature negotiation is complete.

\item[DRIVER_OK (4)] Indicates that the driver is set up and ready to
  drive the device.

\item[FAILED (128)] Indicates that something went wrong in the guest,
  and it has given up on the device. This could be an internal
  error, or the driver didn't like the device for some reason, or
  even a fatal error during device operation. The driver MUST
  reset the device before attempting to re-initialize.
\end{description}

\section{Feature Bits}\label{sec:Basic Facilities of a Virtio Device / Feature Bits}

Each virtio device offers all the features it understands.  During
device initialization, the driver reads this and tells the device the
subset that it accepts.  The only way to renegotiate is to reset
the device.

This allows for forwards and backwards compatibility: if the device is
enhanced with a new feature bit, older drivers will not write that
feature bit back to the device and it SHOULD go into backwards
compatibility mode. Similarly, if a driver is enhanced with a feature
that the device doesn't support, it see the new feature is not offered
and SHOULD go into backwards compatibility mode (or, for poor
implementations it MAY set the FAILED Device Status bit).

The driver MUST NOT accept a feature which the device did not offer,
and MUST NOT accept a feature which requires another feature which was
not accepted.

The device MUST NOT offer a feature which requires another feature
which was not offered.

Feature bits are allocated as follows:

\begin{description}
\item[0 to 23] Feature bits for the specific device type

\item[24 to 32] Feature bits reserved for extensions to the queue and
  feature negotiation mechanisms

\item[33 and above] Feature bits reserved for future extensions.
\end{description}

For example, feature bit 0 for a network device (i.e. Subsystem
Device ID 1) indicates that the device supports checksumming of
packets.

In particular, new fields in the device configuration space are
indicated by offering a feature bit, so the driver MUST check that the
feature is offered before accessing that part of the configuration
space.

\subsection{Legacy Interface: A Note on transitions from earlier drafts}\label{sec:Basic Facilities of a Virtio Device / Feature Bits / Legacy Interface: A Note on transitions from earlier drafts}

Earlier drafts of this specification (up to 0.9.X) defined a similar, but
different interface between the hypervisor and the guest.
Since these are widely deployed, this specification
accommodates optional features to simplify transition
from these earlier draft interfaces. Specifically:

\begin{description}
\item[Legacy Interface]
        is an interface specified by an earlier draft of this specification
        (up to 0.9.X)
\item[Legacy Device]
        is a device implemented before this specification was released,
        and implementing a legacy interface on the host side
\item[Legacy Driver]
        is a driver implemented before this specification was released,
        and implementing a legacy interface on the guest side
\end{description}

Legacy devices and legacy drivers are not compliant with this
specification.

To simplify transition from these earlier draft interfaces,
it is possible to implement:

\begin{description}
\item[Transitional Device]
        a device supporting both drivers conforming to this
        specification, and allowing legacy drivers.

\item[Transitional Driver]
        a driver supporting both devices conforming to this
        specification, and legacy devices.
\end{description}

Transitional devices and transitional drivers can be compliant with
this specification (ie. when not operating in legacy mode).

Devices or drivers with no legacy compatibility are referred to as
non-transitional devices and drivers, respectively.

Transitional Drivers can detect Legacy Devices by detecting that
the feature bit VIRTIO_F_VERSION_1 is not offered.
Transitional devices can detect Legacy drivers by detecting that
VIRTIO_F_VERSION_1 has not been acknowledged by the driver.
In this case device is used through the legacy interface.

To make them easier to locate, specification sections documenting
these transitional features are explicitly marked with 'Legacy
Interface' in the section title.

\section{Configuration Space}\label{sec:Basic Facilities of a Virtio Device / Configuration Space}

Configuration space is generally used for rarely-changing or
initialization-time parameters.  Drivers MUST NOT assume reads from
fields greater than 32 bits wide are atomic, nor reads from
multiple fields.

Each transport provides a generation count for the configuration
space, which must change whenever there is a possibility that two
accesses to the configuration space can see different versions of that
space.

Thus drivers SHOULD read configuration space fields like so:

\begin{lstlisting}
	u32 before, after;
	do {
		before = get_config_generation(device);
		// read config entry/entries.
		after = get_config_generation(device);
	} while (after != before);
\end{lstlisting}

Note that configuration space uses the little-endian format
for multi-byte fields.

Note that future versions of this specification will likely
extend the configuration space for devices by adding extra fields
at the tail end of some structures in configuration space.

To allow forward compatibility with such extensions, drivers MUST
NOT limit structure size and configuration space size.  Instead,
drivers SHOULD only check that configuration space is *large enough* to
contain the fields required for device operation.

For example, if the specification states that configuration
space 'includes a single 8-bit field' drivers should understand this to mean that
the configuration space might also include an arbitrary amount of
tail padding, and accept any configuration space size equal to or
greater than the specified 8-bit size.

\subsection{Legacy Interface: A Note on Configuration Space endian-ness}\label{sec:Basic Facilities of a Virtio Device / Configuration Space / Legacy Interface: A Note on Configuration Space endian-ness}

Note that for legacy interfaces, configuration space is generally the
guest's native endian, rather than PCI's little-endian.

\subsection{Legacy Interface: Configuration Space}\label{sec:Basic Facilities of a Virtio Device / Configuration Space / Legacy Interface: Configuration Space}

Legacy devices did not have a configuration generation field, thus are
susceptible to race conditions if configuration is updated.  This
effects the block capacity and network mac fields; best practice is to
read these fields multiple times until two reads generate a consistent
result.

\section{Virtqueues}\label{sec:Basic Facilities of a Virtio Device / Virtqueues}

The mechanism for bulk data transport on virtio devices is
pretentiously called a virtqueue. Each device can have zero or more
virtqueues: for example, the simplest network device has one for
transmit and one for receive.  Each queue has a 16-bit queue size
parameter, which sets the number of entries and implies the total size
of the queue.

Each virtqueue consists of three parts:

\begin{itemize}
\item Descriptor Table
\item Available Ring
\item Used Ring
\end{itemize}

where each part is physically-contiguous in guest memory,
and has different alignment requirements.

The memory aligment and size requirements, in bytes, of each part of the
virtqueue are summarized in the following table:

\begin{tabular}{|l|l|l|}
\hline
Virtqueue Part    & Alignment & Size \\
\hline \hline
Descriptor Table  & 16        & $16 * $(Queue Size) \\
\hline
Available Ring    & 2         & $6 + 2 * $(Queue Size) \\
 \hline
Used Ring         & 4         & $6 + 4 * $(Queue Size) \\
 \hline
\end{tabular}

The Alignment column gives the miminum alignment: for each part
of the virtqueue, the physical address of the first byte
MUST be a multiple of the specified alignment value.

The Size column gives the total number of bytes required for each
part of the virtqueue.

Queue Size corresponds to the maximum number of buffers in the
virtqueue.  For example, if Queue Size is 4 then at most 4 buffers
can be queued at any given time.  Queue Size value is always a
power of 2.  The maximum Queue Size value is 32768.  This value
is specified in a bus-specific way.

When the driver wants to send a buffer to the device, it fills in
a slot in the descriptor table (or chains several together), and
writes the descriptor index into the available ring.  It then
notifies the device. When the device has finished a buffer, it
writes the descriptor index into the used ring, and sends an interrupt.


\subsection{Legacy Interfaces: A Note on Virtqueue Layout}\label{sec:Basic Facilities of a Virtio Device / Virtqueues / Legacy Interfaces: A Note on Virtqueue Layout}

For Legacy Interfaces, several additional
restrictions are placed on the virtqueue layout:

Each virtqueue occupies two or more physically-contiguous pages
(usually defined as 4096 bytes, but depending on the transport)
and consists of three parts:

\begin{tabular}{|l|l|l|}
\hline
Descriptor Table & Available Ring (\ldots padding\ldots) & Used Ring \\
\hline
\end{tabular}

The bus-specific Queue Size field controls the total number of bytes
required for the virtqueue according to the following formula:

\begin{lstlisting}
	#define ALIGN(x) (((x) + PAGE_SIZE) & ~PAGE_SIZE)
	static inline unsigned vring_size(unsigned int qsz)
	{
	     return ALIGN(sizeof(struct vring_desc)*qsz + sizeof(u16)*(3 + qsz))
	          + ALIGN(sizeof(u16)*3 + sizeof(struct vring_used_elem)*qsz);
	}
\end{lstlisting}

This wastes some space with padding.
The legacy virtqueue layout structure therefore looks like this:

\begin{lstlisting}
	struct vring {
		// The actual descriptors (16 bytes each)
		struct vring_desc desc[ Queue Size ];

		// A ring of available descriptor heads with free-running index.
		struct vring_avail avail;

		// Padding to the next PAGE_SIZE boundary.
		char pad[ Padding ];

		// A ring of used descriptor heads with free-running index.
		struct vring_used used;
	};
\end{lstlisting}

\subsection{Legacy Interfaces: A Note on Virtqueue Endianness}\label{sec:Basic Facilities of a Virtio Device / Virtqueues / Legacy Interfaces: A Note on Virtqueue Endianness}

Note that the endian of fields and in the virtqueue is the native
endian of the guest, not little-endian as specified by this standard.
It is assumed that the host is already aware of the guest endian.

\subsection{Message Framing}\label{sec:Basic Facilities of a Virtio Device / Virtqueues / Message Framing}
The device MUST NOT make assumptions about the particular arrangement
of descriptors: the message framing is
independent of the contents of the buffers. For example, a network
transmit buffer consists of a 12 byte header followed by the network
packet. This could be most simply placed in the descriptor table as a
12 byte output descriptor followed by a 1514 byte output descriptor,
but it could also consist of a single 1526 byte output descriptor in
the case where the header and packet are adjacent, or even three or
more descriptors (possibly with loss of efficiency in that case).

Note that, some implementations may have large-but-reasonable
restrictions on total descriptor size (such as based on IOV_MAX in the
host OS). This has not been a problem in practice: little sympathy
will be given to drivers which create unreasonably-sized descriptors
such as by dividing a network packet into 1500 single-byte
descriptors!

\subsubsection{Legacy Interface: Message Framing}\label{sec:Basic Facilities of a Virtio Device / Virtqueues / Message Framing / Legacy Interface: Message Framing}

Regrettably, initial driver implementations used simple layouts, and
devices came to rely on it, despite this specification wording.  In
addition, the specification for virtio_blk SCSI commands required
intuiting field lengths from frame boundaries (see
 \ref{sec:Device Types / Block Device / Device Operation / Legacy Interface: Device Operation}~\nameref{sec:Device Types / Block Device / Device Operation / Legacy Interface: Device Operation})

It is thus recommended that when using legacy interfaces, transitional
drivers be conservative in their assumptions, unless the
VIRTIO_F_ANY_LAYOUT feature is accepted.

\subsection{The Virtqueue Descriptor Table}\label{sec:Basic Facilities of a Virtio Device / Virtqueues / The Virtqueue Descriptor Table}

The descriptor table refers to the buffers the driver is using for
the device. The addresses are physical addresses, and the buffers
can be chained via the next field. Each descriptor describes a
buffer which is read-only or write-only, but a chain of
descriptors can contain both read-only and write-only buffers.

The actual contents of the memory offered to the device depends on the
device type.  Most common is to begin the data with a header
(containing little-endian fields) for the device to read, and postfix
it with a status tailer for the device to write.

Drivers MUST NOT add a descriptor chain over than $2^{32}$ bytes long in total;
this implies that loops in the descriptor chain are forbidden!

\begin{lstlisting}
	struct vring_desc {
		/* Address (guest-physical). */
		le64 addr;
		/* Length. */
		le32 len;

	/* This marks a buffer as continuing via the next field. */
	#define VRING_DESC_F_NEXT   1
	/* This marks a buffer as write-only (otherwise read-only). */
	#define VRING_DESC_F_WRITE     2
	/* This means the buffer contains a list of buffer descriptors. */
	#define VRING_DESC_F_INDIRECT   4
		/* The flags as indicated above. */
		le16 flags;
		/* Next field if flags & NEXT */
		le16 next;
	};
\end{lstlisting}

The number of descriptors in the table is defined by the queue size
for this virtqueue.

\subsubsection{Indirect Descriptors}\label{sec:Basic Facilities of a Virtio Device / Virtqueues / The Virtqueue Descriptor Table / Indirect Descriptors}

Some devices benefit by concurrently dispatching a large number
of large requests. The VIRTIO_RING_F_INDIRECT_DESC feature allows this (see \ref{sec:virtio-ring.h}~\nameref{sec:virtio-ring.h}). To increase
ring capacity the driver can store a table of indirect
descriptors anywhere in memory, and insert a descriptor in main
virtqueue (with flags\&VRING_DESC_F_INDIRECT on) that refers to memory buffer
containing this indirect descriptor table; fields addr and len
refer to the indirect table address and length in bytes,
respectively.

The driver MUST NOT set the VRING_DESC_F_INDIRECT flag unless the
VIRTIO_RING_F_INDIRECT_DESC feature was negotiated.

The indirect table layout structure looks like this
(len is the length of the descriptor that refers to this table,
which is a variable, so this code won't compile):

\begin{lstlisting}
	struct indirect_descriptor_table {
		/* The actual descriptors (16 bytes each) */
		struct vring_desc desc[len / 16];
	};
\end{lstlisting}

The first indirect descriptor is located at start of the indirect
descriptor table (index 0), additional indirect descriptors are
chained by next field. An indirect descriptor without next field
(with flags\&VRING_DESC_F_NEXT off) signals the end of the descriptor.
An
indirect descriptor can not refer to another indirect descriptor
table (flags\&VRING_DESC_F_INDIRECT MUST be off). A single indirect descriptor
table can include both read-only and write-only descriptors;
the device MUST ignore the write-only flag (flags\&VRING_DESC_F_WRITE) in the descriptor that refers to it.

\subsection{The Virtqueue Available Ring}\label{sec:Basic Facilities of a Virtio Device / Virtqueues / The Virtqueue Available Ring}

\begin{lstlisting}
	struct vring_avail {
	#define VRING_AVAIL_F_NO_INTERRUPT      1
		le16 flags;
		le16 idx;
		le16 ring[ /* Queue Size */ ];
		le16 used_event;	/* Only if VIRTIO_RING_F_EVENT_IDX */
	};
\end{lstlisting}

The driver uses the available ring to offer buffers to the
device: each ring entry refers to the head of a descriptor chain.  It is only
written by the driver and read by the device.

The “idx” field indicates where the driver would put the next descriptor
entry in the ring (modulo the queue size). This starts at 0, and increases.

If the VIRTIO_RING_F_INDIRECT_DESC feature bit is not negotiated, the
“flags” field offers a crude interrupt control mechanism.  The driver
MUST set this to 0 or 1: 1 indicates that the device SHOULD NOT send
an interrupt when it consumes a descriptor chain from the available
ring.  The device MUST ignore the used_event value in this case.

Otherwise, if the VIRTIO_RING_F_EVENT_IDX feature bit is negotiated,
the driver MUST set the "flags" field to 0, and use the “used_event”
field in the used ring instead.  The driver can ask the device to delay interrupts
until an entry with an index specified by the “used_event” field is
written in the used ring (equivalently, until the idx field in the
used ring will reach the value used_event + 1).

The driver MUST handle spurious interrupts: either form of interrupt
suppression is merely an optimization; it may not suppress interrupts
entirely.

\subsection{The Virtqueue Used Ring}\label{sec:Basic Facilities of a Virtio Device / Virtqueues / The Virtqueue Used Ring}

\begin{lstlisting}
	struct vring_used {
	#define VRING_USED_F_NO_NOTIFY  1
		le16 flags;
		le16 idx;
		struct vring_used_elem ring[ /* Queue Size */];
		le16 avail_event; /* Only if VIRTIO_RING_F_EVENT_IDX */
	};

	/* le32 is used here for ids for padding reasons. */
	struct vring_used_elem {
		/* Index of start of used descriptor chain. */
		le32 id;
		/* Total length of the descriptor chain which was used (written to) */
		le32 len;
	};
\end{lstlisting}

The used ring is where the device returns buffers once it is done with
them: it is only written to by the device, and read by the driver.

Each entry in the ring is a pair: the head entry of the
descriptor chain describing the buffer (this matches an entry
placed in the available ring by the guest earlier), and the total
of bytes written into the buffer. The latter is extremely useful
for drivers using untrusted buffers: if you do not know exactly
how much has been written by the device, you usually have to zero
the buffer to ensure no data leakage occurs.

If the VIRTIO_RING_F_INDIRECT_DESC feature bit is not negotiated, the
“flags” field offers a crude interrupt control mechanism.  The driver
MUST initialize this to 0, the device MUST set this to 0 or 1: 1
indicates that the driver SHOULD NOT send an notification when it adds
a descriptor chain to the available ring.  The driver MUST ignore the
used_event value in this case.

Otherwise, if the VIRTIO_RING_F_EVENT_IDX feature bit is negotiated,
the device MUST leave the "flags" field at 0, and use the
“avail_event” field in the used ring instead.  The device can ask the
driver to delay notifications until an entry with an index specified
by the “avail_event” field is written in the available ring (equivalently,
until the idx field in the used ring will reach the value avail_event +
1).

The device MUST handle spurious notification: either form of
notification suppression is merely an optimization; it may not
suppress them entirely.

\subsection{Helpers for Operating Virtqueues}\label{sec:Basic Facilities of a Virtio Device / Virtqueues / Helpers for Operating Virtqueues}

The Linux Kernel Source code contains the definitions above and
helper routines in a more usable form, in
include/linux/virtio_ring.h. This was explicitly licensed by IBM
and Red Hat under the (3-clause) BSD license so that it can be
freely used by all other projects, and is reproduced (with slight
variation to remove Linux assumptions) in \ref{sec:virtio-ring.h}~\nameref{sec:virtio-ring.h}.

\chapter{General Initialization And Device Operation}\label{sec:General Initialization And Device Operation}

We start with an overview of device initialization, then expand on the
details of the device and how each step is preformed.  This section
should be read along with the bus-specific section which describes
how to communicate with the specific device.

\section{Device Initialization}\label{sec:General Initialization And Device Operation / Device Initialization}

The driver MUST follow this sequence to initialize a device:

\begin{enumerate}
\item Reset the device.

\item Set the ACKNOWLEDGE status bit: the guest OS has notice the device.

\item Set the DRIVER status bit: the guest OS knows how to drive the device.

\item Read device feature bits, and write the subset of feature bits
   understood by the OS and driver to the device.

\item\label{itm:General Initialization And Device Operation / Device Initialization / Set FEATURES-OK} Set the FEATURES_OK status bit.  The driver MUST not accept
   new feature bits after this step.

\item\label{itm:General Initialization And Device Operation / Device Initialization / Re-read FEATURES-OK} Re-read the status byte to ensure the FEATURES_OK bit is still
   set: otherwise, the device does not support our subset of features
   and the device is unusable.

\item\label{itm:General Initialization And Device Operation / Device Initialization / Device-specific Setup} Perform device-specific setup, including discovery of virtqueues for the
   device, optional per-bus setup, reading and possibly writing the
   device's virtio configuration space, and population of virtqueues.

\item\label{itm:General Initialization And Device Operation / Device Initialization / Set DRIVER-OK} Set the DRIVER_OK status bit.  At this point the device is
   "live".
\end{enumerate}

If any of these steps go irrecoverably wrong, the driver SHOULD
set the FAILED status bit to indicate that it has given up on the
device (it can reset the device later to restart if desired).  The
driver MUST not continue initialization in that case.

The device MUST NOT consume buffers before DRIVER_OK, and the driver
MUST NOT notify the device before it sets DRIVER_OK.

Devices SHOULD support all valid combinations of features, but we know
that implementations may well make assuptions that they will only be
used by fully-optimized drivers.  The resetting of the FEATURES_OK flag
provides a semi-graceful failure mode for this case.

\subsection{Legacy Interface: Device Initialization}\label{sec:General Initialization And Device Operation / Device Initialization / Legacy Interface: Device Initialization}
Legacy devices do not support the FEATURES_OK status bit, and thus did
not have a graceful way for the device to indicate unsupported feature
combinations.  It also did not provide a clear mechanism to end
feature negotiation, which meant that devices finalized features on
first-use, and no features could be introduced which radically changed
the initial operation of the device.

Legacy device implementations often used the device before setting the
DRIVER_OK bit.

The result was the steps \ref{itm:General Initialization And Device Operation / Device Initialization / Set FEATURES-OK} and \ref{itm:General Initialization And Device Operation / Device Initialization / Re-read FEATURES-OK} were omitted, and steps \ref{itm:General Initialization And Device Operation / Device Initialization / Device-specific Setup} and \ref{itm:General Initialization And Device Operation / Device Initialization / Set DRIVER-OK}
were conflated.

\section{Device Operation}\label{sec:General Initialization And Device Operation / Device Operation}

There are two parts to device operation: supplying new buffers to
the device, and processing used buffers from the device. As an
example, the simplest virtio network device has two virtqueues: the
transmit virtqueue and the receive virtqueue. The driver adds
outgoing (read-only) packets to the transmit virtqueue, and then
frees them after they are used. Similarly, incoming (write-only)
buffers are added to the receive virtqueue, and processed after
they are used.

\subsection{Supplying Buffers to The Device}\label{sec:General Initialization And Device Operation / Device Operation / Supplying Buffers to The Device}

The driver offers buffers to one of the device's virtqueues as follows:

\begin{enumerate}
\item\label{itm:General Initialization And Device Operation / Device Operation / Supplying Buffers to The Device / Place Buffers} The driver places the buffer into free descriptor(s) in the
   descriptor table, chaining as necessary (see \ref{sec:Basic Facilities of a Virtio Device / Virtqueues / The Virtqueue Descriptor Table}~\nameref{sec:Basic Facilities of a Virtio Device / Virtqueues / The Virtqueue Descriptor Table}).

\item\label{itm:General Initialization And Device Operation / Device Operation / Supplying Buffers to The Device / Place Index} The driver places the index of the head of the descriptor chain
   into the next ring entry of the available ring.

\item Steps \ref{itm:General Initialization And Device Operation / Device Operation / Supplying Buffers to The Device / Place Buffers} and \ref{itm:General Initialization And Device Operation / Device Operation / Supplying Buffers to The Device / Place Index} may be performed repeatedly if batching
  is possible.

\item The driver MUST perform suitable a memory barrier to ensure the device sees
  the updated descriptor table and available ring before the next
  step.

\item The available “idx” field is increased by the number of
  descriptor chain heads added to the available ring.

\item The driver MUST perform a suitable memory barrier to ensure that it updates
  the "idx" field before checking for notification suppression.

\item If notifications are not suppressed, the driver MUST notify the device
    of the new available buffers.
\end{enumerate}

Note that the above code does not take precautions against the
available ring buffer wrapping around: this is not possible since
the ring buffer is the same size as the descriptor table, so step
(1) will prevent such a condition.

In addition, the maximum queue size is 32768 (it must be a power
of 2 which fits in 16 bits), so the 16-bit “idx” value can always
distinguish between a full and empty buffer.

Here is a description of each stage in more detail.

\subsubsection{Placing Buffers Into The Descriptor Table}\label{sec:General Initialization And Device Operation / Device Operation / Supplying Buffers to The Device / Placing Buffers Into The Descriptor Table}

A buffer consists of zero or more read-only physically-contiguous
elements followed by zero or more physically-contiguous
write-only elements (it must have at least one element). This
algorithm maps it into the descriptor table to form a descriptor
chain:

for each buffer element, b:

\begin{enumerate}
\item Get the next free descriptor table entry, d
\item Set d.addr to the physical address of the start of b
\item Set d.len to the length of b.
\item If b is write-only, set d.flags to VRING_DESC_F_WRITE,
    otherwise 0.
\item If there is a buffer element after this:
    \begin{enumerate}
    \item Set d.next to the index of the next free descriptor
      element.
    \item Set the VRING_DESC_F_NEXT bit in d.flags.
    \end{enumerate}
\end{enumerate}

In practice, the d.next fields are usually used to chain free
descriptors, and a separate count kept to check there are enough
free descriptors before beginning the mappings.

\subsubsection{Updating The Available Ring}\label{sec:General Initialization And Device Operation / Device Operation / Supplying Buffers to The Device / Updating The Available Ring}

The head of the buffer we mapped is the first d in the algorithm
above (the descriptor chain head).  A naive implementation would do the following (with the
appropriate conversion to-and-from little-endian assumed):

\begin{lstlisting}
	avail->ring[avail->idx % qsz] = head;
\end{lstlisting}

However, in general we can add many descriptor chains before we update
the “idx” field (at which point they become visible to the
device), so we keep a counter of how many we've added:

\begin{lstlisting}
	avail->ring[(avail->idx + added++) % qsz] = head;
\end{lstlisting}

\subsubsection{Updating The Index Field}\label{sec:General Initialization And Device Operation / Device Operation / Supplying Buffers to The Device / Updating The Index Field}

Once the index field of the virtqueue is updated, the device will
be able to access the descriptor chains we've created and the
memory they refer to. This is why a memory barrier is generally
used before the index update, to ensure it sees the most up-to-date
copy.

The index field always increments, and we let it wrap naturally at
65536:

\begin{lstlisting}
	avail->idx += added;
\end{lstlisting}

\subsubsection{Notifying The Device}\label{sec:General Initialization And Device Operation / Device Operation / Supplying Buffers to The Device / Notifying The Device}

The actual method of device notification is bus-specific, but generally
it can be expensive.  So the device can suppress such notifications if it
doesn't need them.  The driver has to be careful to expose the new index
value before checking if notifications are suppressed: it's OK to notify
gratuitously, but not to omit a required notification. So again,
we use a memory barrier here before reading the flags or the
avail_event field.

If the VIRTIO_F_RING_EVENT_IDX feature is not negotiated, and if the
VRING_USED_F_NOTIFY flag is not set, the driver SHOULD notify the
device.

If the VIRTIO_F_RING_EVENT_IDX feature is negotiated, we read the
avail_event field in the available ring structure. If the
available index crossed_the avail_event field value since the
last notification, we go ahead and write to the PCI configuration
space.  The avail_event field wraps naturally at 65536 as well,
giving the following algorithm for calculating whether a device needs
notification:

\begin{lstlisting}
	(u16)(new_idx - avail_event - 1) < (u16)(new_idx - old_idx)
\end{lstlisting}

\subsection{Receiving Used Buffers From The Device}\label{sec:General Initialization And Device Operation / Device Operation / Receiving Used Buffers From The Device}

Once the device has used a buffer (read from or written to it, or
parts of both, depending on the nature of the virtqueue and the
device), it SHOULD send an interrupt, following an algorithm very
similar to the algorithm used for the driver to send the device a
buffer:

\begin{enumerate}
\item Write the head descriptor number to the next field in the used
  ring.

\item Update the used ring index.

\item Deliver an interrupt if necessary:

  \begin{enumerate}
  \item If the VIRTIO_F_RING_EVENT_IDX feature is not negotiated:
    check if the VRING_AVAIL_F_NO_INTERRUPT flag is not set in
    avail->flags.

  \item If the VIRTIO_F_RING_EVENT_IDX feature is negotiated: check
    whether the used index crossed the used_event field value
    since the last update. The used_event field wraps naturally
    at 65536 as well:
\begin{lstlisting}
	(u16)(new_idx - used_event - 1) < (u16)(new_idx - old_idx)
\end{lstlisting}
  \end{enumerate}
\end{enumerate}

For each ring, the driver MAY then disable interrupts by writing
VRING_AVAIL_F_NO_INTERRUPT flag in avail structure, if required.
Once it has processed the ring entries, it SHOULD re-enable
interrupts by clearing the VRING_AVAIL_F_NO_INTERRUPT flag or updating the
EVENT_IDX field in the available structure.  The driver SHOULD then
execute a memory barrier, and then recheck the ring empty
condition. This is necessary to handle the case where after the
last check and before enabling interrupts, an interrupt has been
suppressed by the device:

\begin{lstlisting}
	vring_disable_interrupts(vq);

	for (;;) {
		if (vq->last_seen_used != le16_to_cpu(vring->used.idx)) {
			vring_enable_interrupts(vq);
			mb();

			if (vq->last_seen_used != le16_to_cpu(vring->used.idx))
				break;
		}

		struct vring_used_elem *e = vring.used->ring[vq->last_seen_used%vsz];
		process_buffer(e);
		vq->last_seen_used++;
	}
\end{lstlisting}

\subsection{Notification of Device Configuration Changes}\label{sec:General Initialization And Device Operation / Device Operation / Notification of Device Configuration Changes}

For devices where the configuration information can be changed, an
interrupt is delivered when a configuration change occurs.



\chapter{Virtio Transport Options}\label{sec:Virtio Transport Options}

Virtio can use various different busses, thus the standard is split
into virtio general and bus-specific sections.

\section{Virtio Over PCI Bus}\label{sec:Virtio Transport Options / Virtio Over PCI Bus}

Virtio devices are commonly implemented as PCI devices.

\subsection{PCI Device Discovery}\label{sec:Virtio Transport Options / Virtio Over PCI Bus / PCI Device Discovery}

Any PCI device with Vendor ID 0x1AF4, and Device ID 0x1000 through
0x103F inclusive is a virtio device\footnote{The actual value within this range is ignored
}.

The Subsystem Device ID indicates which virtio device is
supported by the device. The Subsystem Vendor ID SHOULD reflect
the PCI Vendor ID of the environment (it's currently only used
for informational purposes by the driver).

All drivers MUST match devices with any Revision ID, this
is to allow devices to be versioned without breaking drivers.

\subsubsection{Legacy Interfaces: A Note on PCI Device Discovery}\label{sec:Virtio Transport Options / Virtio Over PCI Bus / PCI Device Discovery / Legacy Interfaces: A Note on PCI Device Discovery}
Transitional devices must have a Revision ID of 0 to match
legacy drivers.

Non-transitional devices must have a Revision ID of 1 or higher.

Both transitional and non-transitional drivers must match
any Revision ID value.

\subsection{PCI Device Layout}\label{sec:Virtio Transport Options / Virtio Over PCI Bus / PCI Device Layout}

The device is configured via I/O and/or memory regions (though see
VIRTIO_PCI_CAP_PCI_CFG for access via the PCI configuration space).

These regions contain the virtio header registers, the notification register, the
ISR status register and device specific registers, as specified by Virtio
Structure PCI Capabilities.

There may be different widths of accesses to the I/O region; the
“natural” access method for each field must be
used (i.e. 32-bit accesses for 32-bit fields, etc).

PCI Device Configuration Layout includes the common configuration,
ISR, notification and device specific configuration
structures.

All multi-byte fields are little-endian.

\subsubsection{Common configuration structure layout}\label{sec:Virtio Transport Options / Virtio Over PCI Bus / PCI Device Layout / Common configuration structure layout}
Common configuration structure layout is documented below:

\begin{lstlisting}
	struct virtio_pci_common_cfg {
		/* About the whole device. */
		le32 device_feature_select;	/* read-write */
		le32 device_feature;		/* read-only */
		le32 driver_feature_select;	/* read-write */
		le32 driver_feature;		/* read-write */
		le16 msix_config;		/* read-write */
		le16 num_queues;		/* read-only */
		u8 device_status;		/* read-write */
		u8 config_generation;		/* read-only */

		/* About a specific virtqueue. */
		le16 queue_select;		/* read-write */
		le16 queue_size;		/* read-write, power of 2, or 0. */
		le16 queue_msix_vector;		/* read-write */
		le16 queue_enable;		/* read-write */
		le16 queue_notify_off;		/* read-only */
		le64 queue_desc;		/* read-write */
		le64 queue_avail;		/* read-write */
		le64 queue_used;		/* read-write */
	};
\end{lstlisting}

\begin{description}
\item[device_feature_select]
        The driver uses this to select which Feature Bits the device_feature field shows.
        Value 0x0 selects Feature Bits 0 to 31, 0x1 selects Feature Bits 32 to 63.
        The device MUST present 0 on device_feature for any other value.

\item[device_feature]
        The device uses this to report Feature Bits to the driver.
        Device Feature Bits selected by device_feature_select.

\item[driver_feature_select]
        The driver uses this to select which Feature Bits the driver_feature field shows.
        Value 0x0 selects Feature Bits 0 to 31, 0x1 selects Feature Bits 32 to 63.
        When set to any other value, reads from driver_feature
        return 0, writing 0 into driver_feature has no effect.  The driver
        MUST not write any other value into driver_feature (a corollary of
        the rule that the driver can only write a subset of device features).

\item[driver_feature]
        The driver writes this to accept feature bits offered by the device.
        Driver Feature Bits selected by driver_feature_select.

\item[msix_config]
        The driver sets the Configuration Vector for MSI-X.

\item[num_queues]
        The device specifies the maximum number of virtqueues supported here.

\item[device_status]
        The driver writes the Device Status here. Writing 0 into this
        field resets the device.

\item[config_generation]
        Configuration atomicity value.  The device changes this every time the
        configuration noticeably changes.  This means the device may
        only change the value after a configuration read operation,
        but MUST change it if there is any risk of a driver seeing an
        inconsistent configuration state.

\item[queue_select]
        Queue Select. The driver selects which virtqueue the following
        fields refer to.

\item[queue_size]
        Queue Size.  On reset, specifies the maximum queue size supported by
        the hypervisor. This can be modified by driver to reduce memory requirements.
        The device MUST set this to 0 if this virtqueue is unavailable.

\item[queue_msix_vector]
        The driver uses this to specify the Queue Vector for MSI-X.

\item[queue_enable]
        The driver uses this to selectively prevent the device from executing requests from this virtqueue.
        1 - enabled; 0 - disabled.

        The driver MUST configure the other virtqueue fields before enabling
        the virtqueue.

\item[queue_notify_off]
        The driver reads this to calculate the offset from start of Notification structure at
        which this virtqueue is located.
        Note: this is *not* an offset in bytes. See notify_off_multiplier below.

\item[queue_desc]
        The driver writes the physical address of Descriptor Table here.

\item[queue_avail]
        The driver writes the physical address of Available Ring here.

\item[queue_used]
        The driver writes the physical address of Used Ring here.
\end{description}

\subsubsection{ISR status structure layout}\label{sec:Virtio Transport Options / Virtio Over PCI Bus / PCI Device Layout / ISR status structure layout}
ISR status structure includes a single 8-bit ISR status field.

\subsubsection{Notification structure layout}\label{sec:Virtio Transport Options / Virtio Over PCI Bus / PCI Device Layout / Notification structure layout}
Notification structure is always a multiple of 2 bytes in size.
It includes 2-byte Queue Notify fields for each virtqueue of
the device. Note that multiple virtqueues can use the same
Queue Notify field, if necessary: see notify_off_multiplier below.

\subsubsection{Device specific structure}\label{sec:Virtio Transport Options / Virtio Over PCI Bus / PCI Device Layout / Device specific structure}

Device specific structure is optional.

\subsubsection{Legacy Interfaces: A Note on PCI Device Layout}\label{sec:Virtio Transport Options / Virtio Over PCI Bus / PCI Device Layout / Legacy Interfaces: A Note on PCI Device Layout}

Transitional devices should present part of configuration
registers in a legacy configuration structure in BAR0 in the first I/O
region of the PCI device, as documented below.

There may be different widths of accesses to the I/O region; the
“natural” access method for each field in the virtio header must be
used (i.e. 32-bit accesses for 32-bit fields, etc), but 
when accessed through the legacy interface the
device-specific region can be accessed using any width accesses, and
should obtain the same results.

Note that this is possible because while the virtio header is PCI
(i.e. little) endian, when using the legacy interface the device-specific
region is encoded in the native endian of the guest (where such distinction is
applicable).

When used through the legacy interface, the virtio header looks as follows:

\begin{tabularx}{\textwidth}{ |X||X|X|X|X|X|X|X|X| }
\hline
 Bits & 32 & 32 & 32 & 16 & 16 & 16 & 8 & 8 \\
\hline
 Read / Write & R & R+W & R+W & R & R+W & R+W & R+W & R \\
\hline
 Purpose & Device Features bits 0:31 & Driver Features bits 0:31 &
  Queue Address & Queue Size & Queue Select & Queue Notify &
  Device Status & ISR \newline Status \\
\hline
\end{tabularx}

If MSI-X is enabled for the device, two additional fields
immediately follow this header:

\begin{tabular}{ |l||l|l| }
\hline
Bits       & 16             & 16     \\
\hline
Read/Write & R+W            & R+W    \\
\hline
Purpose (MSI-X) & Configuration Vector  & Queue Vector \\
\hline
\end{tabular}

Note: When MSI-X capability is enabled, device specific configuration starts at
byte offset 24 in virtio header structure. When MSI-X capability is not
enabled, device specific configuration starts at byte offset 20 in virtio
header.  ie. once you enable MSI-X on the device, the other fields move.
If you turn it off again, they move back!

Immediately following these general headers, there may be
device-specific headers:

\begin{tabular}{|l||l|l|}
\hline
Bits & Device Specific & \multirow{3}{*}{...} \\
\cline{1-2}
Read / Write & Device Specific & \\
\cline{1-2}
Purpose & Device Specific & \\
\hline
\end{tabular}

Note that only Feature Bits 0 to 31 are accessible through the
Legacy Interface. When used through the Legacy Interface,
Transitional Devices must assume that Feature Bits 32 to 63
are not acknowledged by Driver.

As legacy devices had no configuration generation field,
see \ref{sec:Basic Facilities of a Virtio Device / Configuration Space / Legacy Interface: Configuration Space}~\nameref{sec:Basic Facilities of a Virtio Device / Configuration Space / Legacy Interface: Configuration Space} for workarounds.

\subsection{PCI-specific Initialization And Device Operation}\label{sec:Virtio Transport Options / Virtio Over PCI Bus / PCI-specific Initialization And Device Operation}

\subsubsection{Device Initialization}\label{sec:Virtio Transport Options / Virtio Over PCI Bus / PCI-specific Initialization And Device Operation / Device Initialization}

This documents PCI-specific steps executed during Device Initialization.
As the first step, driver must detect device configuration layout
to locate configuration fields in memory, I/O or configuration space of the
device.

\paragraph{Virtio Device Configuration Layout Detection}\label{sec:Virtio Transport Options / Virtio Over PCI Bus / PCI-specific Initialization And Device Operation / Device Initialization / Virtio Device Configuration Layout Detection}

As a prerequisite to device initialization, driver executes a
PCI capability list scan, detecting virtio configuration layout using Virtio
Structure PCI capabilities.

Virtio Device Configuration Layout includes virtio configuration header, Notification
and ISR Status and device configuration structures.
Each structure can be mapped by a Base Address register (BAR) belonging to
the function, located beginning at 10h in Configuration Space,
or accessed though PCI configuration space.

Actual location of each structure is specified using vendor-specific PCI capability located
on capability list in PCI configuration space of the device.
This virtio structure capability uses little-endian format; all bits are
read-only:

\begin{lstlisting}
	struct virtio_pci_cap {
		u8 cap_vndr;	/* Generic PCI field: PCI_CAP_ID_VNDR */
		u8 cap_next;	/* Generic PCI field: next ptr. */
		u8 cap_len;	/* Generic PCI field: capability length */
		u8 cfg_type;	/* Identifies the structure. */
		u8 bar;		/* Where to find it. */
		u8 padding[3];	/* Pad to full dword. */
		le32 offset;	/* Offset within bar. */
		le32 length;	/* Length of the structure, in bytes. */
	};
\end{lstlisting}

This structure can optionally be followed by extra data, depending on
other fields, as documented below.

Note that future versions of this specification will likely
extend devices by adding extra fields at the tail end of some structures.

To allow forward compatibility with such extensions, drivers must
not limit structure size.  Instead, drivers should only
check that structures are *large enough* to contain the fields
required for device operation.

For example, if the specification states 'structure includes a
single 8-bit field' drivers should understand this to mean that
the structure can also include an arbitrary amount of tail padding,
and accept any structure size equal to or greater than the
specified 8-bit size.

The fields are interpreted as follows:

\begin{description}
\item[cap_vndr]
        0x09; Identifies a vendor-specific capability.

\item[cap_next]
        Link to next capability in the capability list in the configuration space.

\item[cap_len]
        Length of the capability structure, including the whole of
        struct virtio_pci_cap, and extra data if any.
        This length might include padding, or fields unused by the driver.

\item[cfg_type]
        identifies the structure, according to the following table.

\begin{lstlisting}
	/* Common configuration */
	#define VIRTIO_PCI_CAP_COMMON_CFG	1
	/* Notifications */
	#define VIRTIO_PCI_CAP_NOTIFY_CFG	2
	/* ISR Status */
	#define VIRTIO_PCI_CAP_ISR_CFG		3
	/* Device specific configuration */
	#define VIRTIO_PCI_CAP_DEVICE_CFG	4
	/* PCI configuration access */
	#define VIRTIO_PCI_CAP_PCI_CFG		5
\end{lstlisting}

        Any other value - reserved for future use. Drivers MUST
        ignore any vendor-specific capability structure which has
        a reserved cfg_type value.

        More than one capability can identify the same structure - this makes it
        possible for the device to expose multiple interfaces to drivers.  The order of
        the capabilities in the capability list specifies the order of preference
        suggested by the device; drivers SHOULD use the first interface that they can
        support.  For example, on some hypervisors, notifications using IO accesses are
        faster than memory accesses. In this case, hypervisor can expose two
        capabilities with cfg_type set to VIRTIO_PCI_CAP_NOTIFY_CFG:
        the first one addressing an I/O BAR, the second one addressing a memory BAR.
        Driver will use the I/O BAR if I/O resources are available, and fall back on
        memory BAR when I/O resources are unavailable.

\item[bar]
        values 0x0 to 0x5 specify a Base Address register (BAR) belonging to
        the function located beginning at 10h in Configuration Space
        and used to map the structure into Memory or I/O Space.
        The BAR is permitted to be either 32-bit or 64-bit, it can map Memory Space
        or I/O Space.

        Any other value is reserved for future use. Drivers MUST
        ignore any vendor-specific capability structure which has
        a reserved bar value.

\item[offset]
        indicates where the structure begins relative to the base address associated
        with the BAR.

\item[length]
        indicates the length of the structure.
        This size might include padding, or fields unused by the driver.
        Drivers SHOULD only map part of configuration structure
        large enough for device operation.
        For example, a future device might present a large structure size of several
        MBytes.
        As current devices never utilize structures larger than 4KBytes in size,
        driver can limit the mapped structure size to e.g.
        4KBytes to allow forward compatibility with such devices without loss of
        functionality and without wasting resources.
\end{description}

If cfg_type is VIRTIO_PCI_CAP_NOTIFY_CFG this structure is immediately followed
by additional fields:

\begin{lstlisting}
	struct virtio_pci_notify_cap {
		struct virtio_pci_cap cap;
		le32 notify_off_multiplier;	/* Multiplier for queue_notify_off. */
	};
\end{lstlisting}

\begin{description}
\item[notify_off_multiplier]

        Virtqueue offset multiplier, in bytes. Must be even and either a power of two, or 0.
        Value 0x1 is reserved.
        For a given virtqueue, the address to use for notifications is calculated as follows:

        queue_notify_off * notify_off_multiplier + offset

        If notify_off_multiplier is 0, all virtqueues use the same address in
        the Notifications structure!
\end{description}

If cfg_type is VIRTIO_PCI_CAP_PCI_CFG the fields bar, offset and length are RW
and this structure is immediately followed by an additional field:

\begin{lstlisting}
	struct virtio_pci_cfg_cap {
		__u8 pci_cfg_data[4];	/* Data for BAR access. */
	};
\end{lstlisting}

\begin{description}
\item[pci_cfg_data]

        This RW field allows an indirect access to any BAR on the
        device using PCI configuration accesses.

        The BAR to access is selected using the bar field.
        The length of the access is specified by the length
        field, which can be set to 1, 2 and 4.
        The offset within the BAR is specified by the offset
        field, which must be aligned to length bytes.

        After this field is written by driver, the first length
        bytes in pci_cfg_data are written at the selected
        offset in the selected BAR.

        When this field is read by driver, length bytes at the
        selected offset in the selected BAR are read into pci_cfg_data.
\end{description}

\subparagraph{Legacy Interface: A Note on Device Layout Detection}\label{sec:Virtio Transport Options / Virtio Over PCI Bus / PCI-specific Initialization And Device Operation / Device Initialization / Virtio Device Configuration Layout Detection / Legacy Interface: A Note on Device Layout Detection}

Legacy drivers skipped  Device Layout Detection step, assuming legacy
configuration space in BAR0 in I/O space unconditionally.

Legacy devices did not have the Virtio PCI Capability in their
capability list.

Therefore:

Transitional devices should expose the Legacy Interface in I/O
space in BAR0.

Transitional drivers should look for the Virtio PCI
Capabilities on the capability list.
If these are not present, driver should assume a legacy device.

Non-transitional drivers should look for the Virtio PCI
Capabilities on the capability list.
If these are not present, driver should assume a legacy device,
and fail gracefully.

Non-transitional devices, on a platform where a legacy driver for
a legacy device with the same ID might have previously existed,
must take the following steps to fail gracefully when a legacy
driver attempts to drive them:

\begin{enumerate}
\item Present an I/O BAR in BAR0, and
\item Respond to a single-byte zero write to offset 18
   (corresponding to Device Status register in the legacy layout)
   of BAR0 by presenting zeroes on every BAR and ignoring writes.
\end{enumerate}

\paragraph{Queue Vector Configuration}\label{sec:Virtio Transport Options / Virtio Over PCI Bus / PCI-specific Initialization And Device Operation / Device Initialization / Queue Vector Configuration}

When MSI-X capability is present and enabled in the device
(through standard PCI configuration space) Configuration/Queue
MSI-X Vector registers are used to map configuration change and queue
interrupts to MSI-X vectors. In this case, the ISR Status is unused.

Writing a valid MSI-X Table entry number, 0 to 0x7FF, to one of
Configuration/Queue Vector registers, maps interrupts triggered
by the configuration change/selected queue events respectively to
the corresponding MSI-X vector. To disable interrupts for a
specific event type, unmap it by writing a special NO_VECTOR
value:

\begin{lstlisting}
	/* Vector value used to disable MSI for queue */
	#define VIRTIO_MSI_NO_VECTOR            0xffff
\end{lstlisting}

Reading these registers returns vector mapped to a given event,
or NO_VECTOR if unmapped. All queue and configuration change
events are unmapped by default.

Note that mapping an event to vector might require allocating
internal device resources, and might fail. Devices MUST report such
failures by returning the NO_VECTOR value when the relevant
Vector field is read. After mapping an event to vector, the
driver MUST verify success by reading the Vector field value: on
success, the previously written value is returned, and on
failure, NO_VECTOR is returned. If a mapping failure is detected,
the driver can retry mapping with fewer vectors, or disable MSI-X.

\paragraph{Virtqueue Configuration}\label{sec:Virtio Transport Options / Virtio Over PCI Bus / PCI-specific Initialization And Device Operation / Device Initialization / Virtqueue Configuration}

As a device can have zero or more virtqueues for bulk data
transport (for example, the simplest network device has two), the driver
needs to configure them as part of the device-specific
configuration.

The driver does this as follows, for each virtqueue a device has:

\begin{enumerate}
\item Write the virtqueue index (first queue is 0) to the Queue
  Select field.

\item Read the virtqueue size from the Queue Size field, which MUST
   be a power of 2. This controls how big the virtqueue is
  (see \ref{sec:Basic Facilities of a Virtio Device / Virtqueues}~\nameref{sec:Basic Facilities of a Virtio Device / Virtqueues}). If this field is 0, the virtqueue does not exist.

\item Optionally, select a smaller virtqueue size and write it in the Queue Size
   field.

\item Allocate and zero Descriptor Table, Available and Used rings for the
   virtqueue in contiguous physical memory.

\item Optionally, if MSI-X capability is present and enabled on the
  device, select a vector to use to request interrupts triggered
  by virtqueue events. Write the MSI-X Table entry number
  corresponding to this vector in Queue Vector field. Read the
  Queue Vector field: on success, previously written value is
  returned; on failure, NO_VECTOR value is returned.
\end{enumerate}

\subparagraph{Legacy Interface: A Note on Virtqueue Configuration}\label{sec:Virtio Transport Options / Virtio Over PCI Bus / PCI-specific Initialization And Device Operation / Device Initialization / Virtqueue Configuration / Legacy Interface: A Note on Virtqueue Configuration}
When using the legacy interface, the page size for a virtqueue on a PCI virtio
device is defined as 4096 bytes.  Driver writes the physical address, divided
by 4096 to the Queue Address field\footnote{The 4096 is based on the x86 page size, but it's also large
enough to ensure that the separate parts of the virtqueue are on
separate cache lines.
}.

\subsubsection{Notifying The Device}\label{sec:Virtio Transport Options / Virtio Over PCI Bus / PCI-specific Initialization And Device Operation / Notifying The Device}

Device notification occurs by writing the 16-bit virtqueue index
of this virtqueue to the Queue Notify field.

\subsubsection{Virtqueue Interrupts From The Device}\label{sec:Virtio Transport Options / Virtio Over PCI Bus / PCI-specific Initialization And Device Operation / Virtqueue Interrupts From The Device}

If an interrupt is necessary:

\begin{itemize}
  \item If MSI-X capability is disabled:
    \begin{enumerate}
    \item Set the lower bit of the ISR Status field for the device.

    \item Send the appropriate PCI interrupt for the device.
    \end{enumerate}

  \item If MSI-X capability is enabled:
    \begin{enumerate}
    \item Request the appropriate MSI-X interrupt message for the
      device, Queue Vector field sets the MSI-X Table entry
      number.

    \item If Queue Vector field value is NO_VECTOR, no interrupt
      message is requested for this event.
    \end{enumerate}
\end{itemize}

The driver interrupt handler should:

\begin{itemize}
  \item If MSI-X capability is disabled: read the ISR Status field,
  which will reset it to zero. If the lower bit is zero, the
  interrupt was not for this device. Otherwise, the driver
  should look through the used rings of each virtqueue for the
  device, to see if any progress has been made by the device
  which requires servicing.

  \item If MSI-X capability is enabled: look through the used rings of
  each virtqueue mapped to the specific MSI-X vector for the
  device, to see if any progress has been made by the device
  which requires servicing.
\end{itemize}

\subsubsection{Notification of Device Configuration Changes}\label{sec:Virtio Transport Options / Virtio Over PCI Bus / PCI-specific Initialization And Device Operation / Notification of Device Configuration Changes}

Some virtio PCI devices can change the device configuration
state, as reflected in the virtio header in the PCI configuration
space. In this case:

\begin{itemize}
  \item If MSI-X capability is disabled: an interrupt is delivered and
  the second lowest bit is set in the ISR Status field to
  indicate that the driver should re-examine the configuration
  space.  Note that a single interrupt can indicate both that one
  or more virtqueue has been used and that the configuration
  space has changed: even if the config bit is set, virtqueues
  must be scanned.

  \item If MSI-X capability is enabled: an interrupt message is
  requested. The Configuration Vector field sets the MSI-X Table
  entry number to use. If Configuration Vector field value is
  NO_VECTOR, no interrupt message is requested for this event.
\end{itemize}

\section{Virtio Over MMIO}\label{sec:Virtio Transport Options / Virtio Over MMIO}

Virtual environments without PCI support (a common situation in
embedded devices models) might use simple memory mapped device
("virtio-mmio") instead of the PCI device.

The memory mapped virtio device behaviour is based on the PCI
device specification. Therefore most of operations like device
initialization, queues configuration and buffer transfers are
nearly identical. Existing differences are described in the
following sections.

\subsection{MMIO Device Discovery}\label{sec:Virtio Transport Options / Virtio Over MMIO / MMIO Device Discovery}

Unlike PCI, MMIO provides no generic device discovery.  For each
device, the guest OS will need to know the location of the registers
and interrupt(s) used.  The suggested binding for systems using
flattened device trees is shown in this example:

\begin{lstlisting}
        // EXAMPLE: virtio_block device taking 256 bytes at 0x1e000, interrupt 42.
	virtio_block@1e000 {
		compatible = "virtio,mmio";
		reg = <0x1e000 0x100>;
		interrupts = <42>;
	}
\end{lstlisting}

\subsection{MMIO Device Register Layout}\label{sec:Virtio Transport Options / Virtio Over MMIO / MMIO Device Register Layout}

MMIO virtio devices provides a set of memory mapped control
registers followed by a device-specific configuration space,
described in the table~\ref{tab:Virtio Trasport Options / Virtio Over MMIO / MMIO Device Register Layout}.
Driver MUST NOT access memory locations not explicitly described in the
table (or, in case of the configuration space, described in the device specification),
MUST NOT write to the read-only registers (direction R) and
MUST NOT read from the write-only registers (direction W).

All register values are organized as Little Endian.


\newcommand{\mmioreg}[5]{% Name Function Offset Direction Description
  {\bf#1} \newline #3 \newline #4 & {\bf#2} \newline #5 \\
}

\newcommand{\mmiodreg}[7]{% NameHigh NameLow Function OffsetHigh OffsetLow Direction Description
  {\bf#1} \newline #4 \newline {\bf#2} \newline #5 \newline #6 & {\bf#3} \newline #7 \\
}

\begin{longtable}{p{0.2\textwidth}p{0.7\textwidth}}
  \caption {MMIO Device Register Layout}
  \label{tab:Virtio Trasport Options / Virtio Over MMIO / MMIO Device Register Layout} \\
  \hline
  \mmioreg{Name}{Function}{Offset from base}{Direction}{Description} 
  \hline 
  \hline 
  \endfirsthead
  \hline
  \mmioreg{Name}{Function}{Offset from the base}{Direction}{Description} 
  \hline 
  \hline 
  \endhead
  \endfoot
  \endlastfoot
  \mmioreg{MagicValue}{Magic value}{0x000}{R}{%
    Device MUST return value 0x74726976
    (a Little Endian equivalent of the "virt" string).
    Driver MUST ignore device returning other values,
    although it MAY report an error.
  } 
  \hline
  \mmioreg{Version}{Device version number}{0x004}{R}{%
    Devices compliant with this specification MUST return value 0x2.
    Legacy device (see \ref{sec:Virtio Transport Options / Virtio Over MMIO / Legacy interface}~\nameref{sec:Virtio Transport Options / Virtio Over MMIO / Legacy interface}) MAY return value 0x1.
    Driver MUST ignore device returning other values,
    although it MAY report an error.
  }
  \hline 
  \mmioreg{DeviceID}{Virtio Subsystem Device ID}{0x008}{R}{%
    See \ref{sec:Device Types}~\nameref{sec:Device Types} for possible values.
    Value zero (0x0) is invalid and driver MUST ignore such device
    but MUST NOT report any error. This behaviour can be used to
    define a system memory map with placeholder devices at static,
    well known addresses, assigning functions to them depending
    on user's needs.
  }
  \hline 
  \mmioreg{VendorID}{Virtio Subsystem Vendor ID}{0x00c}{R}{}
  \hline 
  \mmioreg{DeviceFeatures}{Flags representing features the device supports}{0x010}{R}{%
    Reading from this register returns 32 consecutive flag bits,
    first bit depending on the last value written to the
    DeviceFeaturesSel register. Access to this register returns
    bits $DeviceFeaturesSel*32$ to $(DeviceFeaturesSel*32)+31$, eg.
    feature bits 0 to 31 if DeviceFeaturesSel is set to 0 and
    features bits 32 to 63 if DeviceFeaturesSel is set to 1.
    Also see \ref{sec:Basic Facilities of a Virtio Device / Feature Bits}~\nameref{sec:Basic Facilities of a Virtio Device / Feature Bits}.
  }
  \hline 
  \mmioreg{DeviceFeaturesSel}{Device (host) features word selection.}{0x014}{W}{%
    Writing to this register selects a set of 32 device feature bits
    accessible by reading from the DeviceFeatures register. The driver
    MUST write a value to the DeviceFeaturesSel register before
    reading from the DeviceFeatures register.
  }
  \hline 
  \mmioreg{DriverFeatures}{Flags representing device features understood and activated by the driver}{0x020}{W}{%
    Writing to this register sets 32 consecutive flag bits, first
    bit depending on the last value written to the DriverFeaturesSel
    register. Access to this register sets bits $DriverFeaturesSel*32$
    to $(DriverFeaturesSel*32)+31$, eg. feature bits 0 to 31 if
    DriverFeaturesSel is set to 0 and features bits 32 to 63 if
    DriverFeaturesSel is set to 1. Also see \ref{sec:Basic Facilities of a Virtio Device / Feature Bits}~\nameref{sec:Basic Facilities of a Virtio Device / Feature Bits}.
  }
  \hline 
  \mmioreg{DriverFeaturesSel}{Activated (guest) features word selection}{0x024}{W}{%
    Writing to this register selects a set of 32 activated feature
    bits accessible by writing to the DriverFeatures register.
    The driver MUST write a value to the DriverFeaturesSel
    register before writing to the DriverFeatures register.
  }
  \hline 
  \mmioreg{QueueSel}{Virtual queue index}{0x030}{W}{%
    Writing to this register selects the virtual queue that the
    following operations on the QueueNumMax, QueueNum, QueueReady,
    QueueDescLow, QueueDescHigh, QueueAvailLow, QueueAvailHigh,
    QueueUsedLow and QueueUsedHigh registers apply to. The index
    number of the first queue is zero (0x0). 
  }
  \hline 
  \mmioreg{QueueNumMax}{Maximum virtual queue size}{0x034}{R}{%
    Reading from the register returns the maximum size (number of
    elements) of the queue the device is ready to process or
    zero (0x0) if the queue is not available. This applies to the
    queue selected by writing to QueueSel. The driver MUST NOT
    access this register when the queue is in use (so when QueueReady
    is not zero).
  }
  \hline 
  \mmioreg{QueueNum}{Virtual queue size}{0x038}{W}{%
    Queue size is the number of elements in the queue, therefore size
    of the Descriptor Table and both Available and Used rings.
    Writing to this register notifies the device what size of the
    queue the driver will use. This applies to the queue selected by
    writing to QueueSel. The driver MUST NOT access this register when
    the queue is in use (so when QueueReady is not zero).
  }
  \hline 
  \mmioreg{QueueReady}{Virtual queue ready bit}{0x044}{RW}{%
    Writing one (0x1) to this register notifies the device that the
    virtual queue is ready to be used. Reading from this register
    returns the last value written to it. Both read and write
    accesses apply to the queue selected by writing to QueueSel.
    When the driver wants to stop using the queue it MUST write
    zero (0x0) to this register and MUST read the value back to
    ensure synchronisation.
  }
  \hline 
  \mmioreg{QueueNotify}{Queue notifier}{0x050}{W}{%
    Writing a queue index to this register notifies the device that
    there are new buffers to process in the queue.
  }
  \hline 
  \mmioreg{InterruptStatus}{Interrupt status}{0x60}{R}{%
    Reading from this register returns a bit mask of events that
    caused the device interrupt to be asserted.
    From a moment when any of these events takes place, the
    device MUST be returning a value with the related
    bits set, ie. equal one (1), and all other bits cleared, 
    ie. equal zero (0), until the driver acknowledges the interrupt
    by writing a corresponding bit mask to the InterruptACK register.
    The following events are possible:
    \begin{description}
      \item[Used Ring Update] - bit 0 - the interrupt was asserted
        because the device has updated the Used
        Ring in at least one of the active virtual queues.
      \item [Configuration Change] - bit 1 - the interrupt was
        asserted because the configuration of the device has changed.
    \end{description}
    Other bits of the value are reserved for future use and the
    driver MUST ignore them.
  }
  \hline 
  \mmioreg{InterruptACK}{Interrupt acknowledge}{0x064}{W}{%
    Writing to this register notifies the device that the interrupt
    has been handled.
    When the driver finishes handling an interrupt, it MUST write
    a value to this register with bits corresponding to the handled
    events (as defined for the InterruptStatus register) set, ie.
    equal one (1), and all other bits cleared, ie. equal zero (0).
  }
  \hline 
  \mmioreg{Status}{Device status}{0x070}{RW}{%
    Reading from this register returns the current device status
    flags.
    Writing non-zero values to this register sets the status flags,
    indicating the driver progress. Writing zero (0x0) to this
    register triggers a device reset, including clearing all
    bits in the InterruptStatus register and ready bits in the
    QueueReady register for all queues in the device. 
    See also p. \ref{sec:Virtio Transport Options / Virtio Over MMIO / MMIO-specific Initialization And Device Operation / Device Initialization}~\nameref{sec:Virtio Transport Options / Virtio Over MMIO / MMIO-specific Initialization And Device Operation / Device Initialization}.
  }
  \hline 
  \mmiodreg{QueueDescLow}{QueueDescHigh}{Virtual queue's Descriptor Table 64 bit long physical address}{0x080}{0x084}{W}{%
    Writing to these two registers (lower 32 bits of the address
    to QueueDescLow, higher 32 bits to QueueDescHigh) notifies
    the device about location of the Descriptor Table of the queue
    selected by writing to the QueueSel register. The driver MUST NOT
    access this register when the queue is in use (so when QueueReady
    is not zero).
  }
  \hline 
  \mmiodreg{QueueAvailLow}{QueueAvailHigh}{Virtual queue's Available Ring 64 bit long physical address}{0x090}{0x094}{W}{%
    Writing to these two registers (lower 32 bits of the address
    to QueueAvailLow, higher 32 bits to QueueAvailHigh) notifies
    the device about location of the Available Ring of the queue
    selected by writing to the QueueSel register. The driver MUST NOT
    access this register when the queue is in use (so when QueueReady
    is not zero).
  }
  \hline 
  \mmiodreg{QueueUsedLow}{QueueUsedHigh}{Virtual queue's Used Ring 64 bit long physical address}{0x0a0}{0x0a4}{W}{%
    Writing to these two registers (lower 32 bits of the address
    to QueueUsedLow, higher 32 bits to QueueUsedHigh) notifies
    the device about location of the Used Ring of the queue
    selected by writing to the QueueSel register. The driver MUST NOT
    access this register when the queue is in use (so when QueueReady
    is not zero).
  }
  \hline 
  \mmioreg{ConfigGeneration}{Configuration atomicity value}{0x0fc}{R}{
    Changes every time the configuration noticeably changes. This
    means the device may only change the value after a configuration
    read operation, but it MUST change if there is any risk of a
    device seeing an inconsistent configuration state.
  }
  \hline 
  \mmioreg{Config}{Configuration space}{0x100+}{RW}{
    Device-specific configuration space starts at the offset 0x100
    and is accessed with byte alignment. Its meaning and size
    depend on the device and the driver.
  }
  \hline
\end{longtable}

\subsection{MMIO-specific Initialization And Device Operation}\label{sec:Virtio Transport Options / Virtio Over MMIO / MMIO-specific Initialization And Device Operation}

\subsubsection{Device Initialization}\label{sec:Virtio Transport Options / Virtio Over MMIO / MMIO-specific Initialization And Device Operation / Device Initialization}

The driver MUST start the device initialization by reading and
checking values from the MagicValue and the Version registers.
If both values are valid, it MUST read the DeviceID register
and if its value is zero (0x0) MUST abort initialization and
MUST NOT access any other register.

Further initialization MUST follow the procedure described in
\ref{sec:General Initialization And Device Operation / Device Initialization}~\nameref{sec:General Initialization And Device Operation / Device Initialization}.

\subsubsection{Virtqueue Configuration}\label{sec:Virtio Transport Options / Virtio Over MMIO / MMIO-specific Initialization And Device Operation / Virtqueue Configuration}

The driver MUST initialize the virtual queue in the following way:

\begin{enumerate}
\item Select the queue writing its index (first queue is 0) to the
   QueueSel register.

\item Check if the queue is not already in use: read the QueueReady
   register, returned value should be zero (0x0).

\item Read maximum queue size (number of elements) from the
   QueueNumMax register. If the returned value is zero (0x0) the
   queue is not available.

\item Allocate and zero the queue pages, making sure the memory
   is physically contiguous. It is recommended to align the
   Used Ring to an optimal boundary (usually the page size).
   Size of the allocated queue MUST be smaller than or equal to
   the maximum size returned by the device.

\item Notify the device about the queue size by writing the size to
   the QueueNum register.

\item Write physical addresses of the queue's Descriptor Table,
   Available Ring and Used Ring to (respectively) the QueueDescLow/
   QueueDescHigh, QueueAvailLow/QueueAvailHigh and QueueUsedLow/
   QueueUsedHigh register pairs.

\item Write 0x1 to the QueueReady register.
\end{enumerate}

\subsubsection{Notifying The Device}\label{sec:Virtio Transport Options / Virtio Over MMIO / MMIO-specific Initialization And Device Operation / Notifying The Device}

The driver MUST notify the device about new buffers being available in
a queue by writing the index of the updated queue to the QueueNotify register.

\subsubsection{Notifications From The Device}\label{sec:Virtio Transport Options / Virtio Over MMIO / MMIO-specific Initialization And Device Operation / Notifications From The Device}

The memory mapped virtio device is using a single, dedicated
interrupt signal, which is asserted when at least one of the
bits described in the InterruptStatus register
description is set. This way the device may notify the
driver about a new used buffer being available in the queue
or about a change in the device configuration.

After receiving an interrupt, the driver MUST read the
InterruptStatus register to check what caused the interrupt
(see the register description). After the interrupt is handled,
the driver MUST acknowledge it by writing a bit mask
corresponding to the handled events to the InterruptACK register.

\subsection{Legacy interface}\label{sec:Virtio Transport Options / Virtio Over MMIO / Legacy interface}

The legacy MMIO transport used page-based addressing, resulting
in a slightly different control register layout, the device
initialization and the virtual queue configuration procedure.

Table \ref{tab:Virtio Trasport Options / Virtio Over MMIO / MMIO Device Legacy Register Layout} 
presents control registers layout, omitting
descriptions of registers which did not change their function
nor behaviour:

\begin{longtable}{p{0.2\textwidth}p{0.7\textwidth}}
  \caption {MMIO Device Legacy Register Layout}
  \label{tab:Virtio Trasport Options / Virtio Over MMIO / MMIO Device Legacy Register Layout} \\
  \hline
  \mmioreg{Name}{Function}{Offset from base}{Direction}{Description} 
  \hline 
  \hline 
  \endfirsthead
  \hline
  \mmioreg{Name}{Function}{Offset from the base}{Direction}{Description} 
  \hline 
  \hline 
  \endhead
  \endfoot
  \endlastfoot
  \mmioreg{MagicValue}{Magic value}{0x000}{R}{}
  \hline
  \mmioreg{Version}{Device version number}{0x004}{R}{Legacy device MUST return value 0x1.}
  \hline
  \mmioreg{DeviceID}{Virtio Subsystem Device ID}{0x008}{R}{}
  \hline
  \mmioreg{VendorID}{Virtio Subsystem Vendor ID}{0x00c}{R}{}
  \hline
  \mmioreg{HostFeatures}{Flags representing features the device supports}{0x010}{R}{}
  \hline
  \mmioreg{HostFeaturesSel}{Device (host) features word selection.}{0x014}{W}{}
  \hline
  \mmioreg{GuestFeatures}{Flags representing device features understood and activated by the driver}{0x020}{W}{}
  \hline
  \mmioreg{GuestFeaturesSel}{Activated (guest) features word selection}{0x024}{W}{}
  \hline 
  \mmioreg{GuestPageSize}{Guest page size}{0x028}{W}{%
    The driver MUST write the guest page size in bytes to the
    register during initialization, before any queues are used.
    This value MUST be a power of 2 and is used by the device to
    calculate the Guest address of the first queue page
    (see QueuePFN).
  }
  \hline
  \mmioreg{QueueSel}{Virtual queue index}{0x030}{W}{%
    Writing to this register selects the virtual queue that the
    following operations on the QueueNumMAx, QueueNum, QueueAlign
    and QueuePFN registers apply to. The index
    number of the first queue is zero (0x0). 
.
  }
  \hline
  \mmioreg{QueueNumMax}{Maximum virtual queue size}{0x034}{R}{%
    Reading from the register returns the maximum size of the queue
    the device is ready to process or zero (0x0) if the queue is not
    available. This applies to the queue selected by writing to the
    QueueSel and is allowed only when the QueuePFN is set to zero
    (0x0), so when the queue is not actively used.
  }
  \hline
  \mmioreg{QueueNum}{Virtual queue size}{0x038}{W}{%
    Queue size is the number of elements in the queue, therefore size
    of the descriptor table and both available and used rings.
    Writing to this register notifies the device what size of the
    queue the driver will use. This applies to the queue selected by
    writing to the QueueSel register.
  }
  \hline
  \mmioreg{QueueAlign}{Used Ring alignment in the virtual queue}{0x03c}{W}{%
    Writing to this register notifies the device about alignment
    boundary of the Used Ring in bytes. This value MUST be a power
    of 2 and applies to the queue selected by writing to the QueueSel
    register.
  }
  \hline
  \mmioreg{QueuePFN}{Guest physical page number of the virtual queue}{0x040}{RW}{%
    Writing to this register notifies the device about location of the
    virtual queue in the Guest's physical address space. This value
    is the index number of a page starting with the queue
    Descriptor Table. Value zero (0x0) means physical address zero
    (0x00000000) and is illegal. When the driver stops using the
    queue it MUST write zero (0x0) to this register.
    Reading from this register returns the currently used page
    number of the queue, therefore a value other than zero (0x0)
    means that the queue is in use.
    Both read and write accesses apply to the queue selected by
    writing to the QueueSel register.
  }
  \hline
  \mmioreg{QueueNotify}{Queue notifier}{0x050}{W}{}
  \hline
  \mmioreg{InterruptStatus}{Interrupt status}{0x60}{R}{}
  \hline
  \mmioreg{InterruptACK}{Interrupt acknowledge}{0x064}{W}{}
  \hline
  \mmioreg{Status}{Device status}{0x070}{RW}{%
    Reading from this register returns the current device status
    flags.
    Writing non-zero values to this register sets the status flags,
    indicating the OS/driver progress. Writing zero (0x0) to this
    register triggers a device reset. This should include
    setting QueuePFN to zero (0x0) for all queues in the device.
    Also see \ref{sec:General Initialization And Device Operation / Device Initialization}~\nameref{sec:General Initialization And Device Operation / Device Initialization}.
  }
  \hline
  \mmioreg{Config}{Configuration space}{0x100+}{RW}{}
  \hline
\end{longtable}

The virtual queue page size is defined by writing to the GuestPageSize
register, as written by the guest. This must be done before the
virtual queues are configured.

The virtual queue layout follows
p. \ref{sec:Basic Facilities of a Virtio Device / Virtqueues / Legacy Interfaces: A Note on Virtqueue Layout}~\nameref{sec:Basic Facilities of a Virtio Device / Virtqueues / Legacy Interfaces: A Note on Virtqueue Layout},
with the alignment defined in the QueueAlign register.

The virtual queue is configured as follows:
\begin{enumerate}
\item Select the queue writing its index (first queue is 0) to the
   QueueSel register.

\item Check if the queue is not already in use: read the QueuePFN
   register, returned value should be zero (0x0).

\item Read maximum queue size (number of elements) from the
   QueueNumMax register. If the returned value is zero (0x0) the
   queue is not available.

\item Allocate and zero the queue pages in contiguous virtual
   memory, aligning the Used Ring to an optimal boundary (usually
   page size). Size of the allocated queue may be smaller than or
   equal to the maximum size returned by the device.

\item Notify the device about the queue size by writing the size to
   the QueueNum register.

\item Notify the device about the used alignment by writing its value
   in bytes to the QueueAlign register.

\item Write the physical number of the first page of the queue to
   the QueuePFN register.
\end{enumerate}

Notification mechanisms did not change.

\section{Virtio Over Channel I/O}\label{sec:Virtio Transport Options / Virtio Over Channel I/O}

S/390 based virtual machines support neither PCI nor MMIO, so a
different transport is needed there.

virtio-ccw uses the standard channel I/O based mechanism used for
the majority of devices on S/390. A virtual channel device with a
special control unit type acts as proxy to the virtio device
(similar to the way virtio-pci uses a PCI device) and
configuration and operation of the virtio device is accomplished
(mostly) via channel commands. This means virtio devices are
discoverable via standard operating system algorithms, and adding
virtio support is mainly a question of supporting a new control
unit type.

As the S/390 is a big endian machine, the data structures transmitted
via channel commands are big-endian: this is made clear by use of
the types be16, be32 and be64.

\subsection{Basic Concepts}\label{sec:Virtio Transport Options / Virtio over channel I/O / Basic Concepts}

As a proxy device, virtio-ccw uses a channel-attached I/O control
unit with a special control unit type (0x3832) and a control unit
model corresponding to the attached virtio device's subsystem
device ID, accessed via a virtual I/O subchannel and a virtual
channel path of type 0x32. This proxy device is discoverable via
normal channel subsystem device discovery (usually a STORE
SUBCHANNEL loop) and answers to the basic channel commands, most
importantly SENSE ID.

For a virtio-ccw proxy device, SENSE ID will return the following
information:

\begin{tabular}{ |l|l|l| }
\hline
Bytes & Description & Contents \\
\hline \hline
0     & reserved              & 0xff \\
\hline
1-2   & control unit type     & 0x3832 \\
\hline
3     & control unit model    & <virtio device id> \\
\hline
4-5   & device type           & zeroes (unset) \\
\hline
6     & device model          & zeroes (unset) \\
\hline
7-255 & extended SenseId data & zeroes (unset) \\
\hline
\end{tabular}

A driver for virtio-ccw devices MUST check for a control unit
type of 0x3832 and MUST ignore the device type and model.

In addition to the basic channel commands, virtio-ccw defines a
set of channel commands related to configuration and operation of
virtio:

\begin{lstlisting}
	#define CCW_CMD_SET_VQ 0x13
	#define CCW_CMD_VDEV_RESET 0x33
	#define CCW_CMD_SET_IND 0x43
	#define CCW_CMD_SET_CONF_IND 0x53
	#define CCW_CMD_SET_IND_ADAPTER 0x73
	#define CCW_CMD_READ_FEAT 0x12
	#define CCW_CMD_WRITE_FEAT 0x11
	#define CCW_CMD_READ_CONF 0x22
	#define CCW_CMD_WRITE_CONF 0x21
	#define CCW_CMD_WRITE_STATUS 0x31
	#define CCW_CMD_READ_VQ_CONF 0x32
	#define CCW_CMD_SET_VIRTIO_REV 0x83
\end{lstlisting}

The virtio-ccw device acts like a normal channel device, as specified
in \hyperref[intro:S390 PoP]{[S390 PoP]} and \hyperref[intro:S390 Common I/O]{[S390 Common I/O]}. In particular:

\begin{itemize}
\item A device MUST post a unit check with command reject for any command
  it does not support.

\item If a driver did not suppress length checks for a channel command,
  the device MUST present a subchannel status as detailed in the
  architecture when the actual length did not match the expected length.

\item If a driver did suppress length checks for a channel command, the
  device MUST present a check condition if the transmitted data does
  not contain enough data to process the command. If the driver submitted
  a buffer that was too long, the device SHOULD accept the command.
  The driver SHOULD attempt to provide the correct length even if it
  suppresses length checks.
\end{itemize}

\subsection{Device Initialization}\label{sec:Virtio Transport Options / Virtio over channel I/O / Device Initialization}

virtio-ccw uses several channel commands to set up a device.

\subsubsection{Setting the Virtio Revision}\label{sec:Virtio Transport Options / Virtio over channel I/O / Device Initialization / Setting the Virtio Revision}

CCW_CMD_SET_VIRTIO_REV is issued by the driver to set the revision of
the virtio-ccw transport it intends to drive the device with. It uses the
following communication structure:

\begin{lstlisting}
	struct virtio_rev_info {
		be16 revision;
		be16 length;
		u8 data[];
	};
\end{lstlisting}

revision contains the desired revision id, length the length of the
data portion and data revision-dependent additional desired options.

The following values are supported:

\begin{tabular}{ |l|l|l|l| }
\hline
revision & length & data      & remarks \\
\hline \hline
0        & 0      & <empty>   & legacy interface; transitional devices only \\
\hline
1        & 0      & <empty>   & Virtio 1.0 \\
\hline
2-n      &        &           & reserved for later revisions \\
\hline
\end{tabular}

Note that a change in the virtio standard does not necessarily
correspond to a change in the virtio-ccw revision.

A device MUST post a unit check with command reject for any revision
it does not support. For any invalid combination of revision, length
and data, it MUST post a unit check with command reject as well. A
non-transitional device MUST reject revision id 0.

A driver SHOULD start with trying to set the highest revision it
supports and continue with lower revisions if it gets a command reject.

A driver MUST NOT issue any other virtio-ccw specific channel commands
prior to setting the revision.

A device MUST answer with command reject to any virtio-ccw specific
channel command that is not contained in the revision selected by the
driver.

After a revision has been successfully selected by the driver, it
MUST NOT attempt to select a different revision. A device MUST answer
to any such attempt with a command reject.

A device MUST treat the revision as unset from the time the associated
subchannel has been enabled until a revision has been successfully set
by the driver. This implies that revisions are not persistent across
disabling and enabling of the associated subchannel.

\paragraph{Legacy Interfaces: A Note on Setting the Virtio Revision}\label{sec:Virtio Transport Options / Virtio over channel I/O / Device Initialization / Setting the Virtio Revision / Legacy Interfaces: A Note on Setting the Virtio Revision}

A legacy device will not support the CCW_CMD_SET_VIRTIO_REV and answer
with a command reject. A non-transitional driver MUST stop trying to
operate this device in that case. A transitional driver MUST operate
the device as if it had been able to set revision 0.

A legacy driver will not issue the CCW_CMD_SET_VIRTIO_REV prior to
issuing other virtio-ccw specific channel commands. A non-transitional
device therefore MUST answer any such attempts with a command reject.
A transitional device MUST assume in this case that the driver is a
legacy driver and continue as if the driver selected revision 0. This
implies that the device MUST reject any command not valid for revision
0, including a subsequent CCW_CMD_SET_VIRTIO_REV.

\subsubsection{Configuring a Virtqueue}\label{sec:Virtio Transport Options / Virtio over channel I/O / Device Initialization / Configuring a Virtqueue}

CCW_CMD_READ_VQ_CONF is issued by the driver to obtain information
about a queue. It uses the following structure for communicating:

\begin{lstlisting}
	struct vq_config_block {
		be16 index;
		be16 max_num;
	} __attribute__ ((packed));
\end{lstlisting}

The requested number of buffers for queue index is returned in
max_num.

Afterwards, CCW_CMD_SET_VQ is issued by the driver to inform the
device about the location used for its queue. The transmitted
structure is

\begin{lstlisting}
	struct vq_info_block {
		be64 desc;
		be32 res0;
		be16 index;
		be16 num;
		be64 avail;
		be64 used;
	} __attribute__ ((packed));
\end{lstlisting}

desc, avail and used contain the guest addresses for the descriptor table,
available ring and used ring for queue index, respectively. The actual
virtqueue size (number of allocated buffers) is transmitted in num.
res0 is reserved and MUST be ignored by the device.

\paragraph{Legacy Interface: A Note on Configuring a Virtqueue}\label{sec:Virtio Transport Options / Virtio over channel I/O / Device Initialization / Configuring a Virtqueue / Legacy Interface: A Note on Configuring a Virtqueue}

For a legacy driver or for a driver that selected revision 0,
CCW_CMD_SET_VQ uses the following communication block:

\begin{lstlisting}
	struct vq_info_block_legacy {
		be64 queue;
		be32 align;
		be16 index;
		be16 num;
	} __attribute__ ((packed));
\end{lstlisting}

queue contains the guest address for queue index, num the number of buffers
and align the alignment.

\subsubsection{Virtqueue Layout}\label{sec:Virtio Transport Options / Virtio over channel I/O / Device Initialization / Virtqueue Layout}

The virtqueue is physically contiguous, with padding added to make the
used ring meet the align value:

\begin{tabular}{|l|l|l|}
\hline
Descriptor Table & Available Ring (\ldots padding\ldots) & Used Ring \\
\hline
\end{tabular}

The calculation for total size is as follows:

\begin{lstlisting}
	#define ALIGN(x) (((x) + align) & ~align)
	static inline unsigned vring_size(unsigned int num)
	{
	     return ALIGN(sizeof(struct vring_desc)*num
			  + sizeof(u16)*(3 + num))
	          + ALIGN(sizeof(u16)*3 + sizeof(struct vring_used_elem)*num);
	}
\end{lstlisting}

\subsubsection{Communicating Status Information}\label{sec:Virtio Transport Options / Virtio over channel I/O / Device Initialization / Communicating Status Information}

The driver changes the status of a device via the
CCW_CMD_WRITE_STATUS command, which transmits an 8 bit status
value.

\subsubsection{Handling Device Features}\label{sec:Virtio Transport Options / Virtio over channel I/O / Device Initialization / Handling Device Features}

Feature bits are arranged in an array of 32 bit values, making
for a total of 8192 feature bits. Feature bits are in
little-endian byte order.

The CCW commands dealing with features use the following
communication block:

\begin{lstlisting}
	struct virtio_feature_desc {
		le32 features;
		u8 index;
	} __attribute__ ((packed));
\end{lstlisting}

features are the 32 bits of features currently accessed, while
index describes which of the feature bit values is to be
accessed.

The guest obtains the device's device feature set via the
CCW_CMD_READ_FEAT command. The device stores the features at index
to features.

For communicating its supported features to the device, the driver
uses the CCW_CMD_WRITE_FEAT command, denoting a features/index
combination.

\subsubsection{Device Configuration}\label{sec:Virtio Transport Options / Virtio over channel I/O / Device Initialization / Device Configuration}

The device's configuration space is located in host memory. It is
the same size as the standard PCI configuration space.

To obtain information from the configuration space, the driver
uses CCW_CMD_READ_CONF, specifying the guest memory for the device
to write to.

For changing configuration information, the driver uses
CCW_CMD_WRITE_CONF, specifying the guest memory for the device to
read from.

In both cases, the complete configuration space is transmitted.  This
allows the driver to compare the new configuration space with the old
version, and keep a generation count internally whenever it changes.

\subsubsection{Setting Up Indicators}\label{sec:Virtio Transport Options / Virtio over channel I/O / Device Initialization / Setting Up Indicators}

In order to set up the indicator bits for host->guest notification,
the driver uses different channel commands depending on whether it
wishes to use traditional I/O interrupts tied to a subchannel or
adapter I/O interrupts for virtqueue notifications. For any given
device, the two mechanisms are mutually exclusive.

For the configuration change indicators, only a mechanism using
traditional I/O interrupts is provided, regardless of whether
traditional or adapter I/O interrupts are used for virtqueue
notifications.

\paragraph{Setting Up Classic Queue Indicators}\label{sec:Virtio Transport Options / Virtio over channel I/O / Device Initialization / Setting Up Indicators / Setting Up Classic Queue Indicators}

Indicators for notification via classic I/O interrupts are contained
in a 64 bit value per virtio-ccw proxy device.

To communicate the location of the indicator bits for host->guest
notification, the driver uses the CCW_CMD_SET_IND command,
pointing to a location containing the guest address of the
indicators in a 64 bit value.

If the driver has already set up two-staged queue indicators via the
CCW_CMD_SET_IND_ADAPTER command, the device MUST post a unit check
with command reject to any subsequent CCW_CMD_SET_IND command.

\paragraph{Setting Up Configuration Change Indicators}\label{sec:Virtio Transport Options / Virtio over channel I/O / Device Initialization / Setting Up Indicators / Setting Up Configuration Change Indicators}

Indicators for configuration change host->guest notification are
contained in a 64 bit value per virtio-ccw proxy device.

To communicate the location of the indicator bits used in the
configuration change host->guest notification, the driver issues the
CCW_CMD_SET_CONF_IND command, pointing to a location containing the
guest address of the indicators in a 64 bit value.

\paragraph{Setting Up Two-Stage Queue Indicators}\label{sec:Virtio Transport Options / Virtio over channel I/O / Device Initialization / Setting Up Indicators / Setting Up Two-Stage Queue Indicators}

Indicators for notification via adapter I/O interrupts consist of
two stages:
\begin{itemize}
\item a summary indicator byte covering the virtqueues for one or more
  virtio-ccw proxy devices
\item a set of contigous indicator bits for the virtqueues for a
  virtio-ccw proxy device
\end{itemize}

To communicate the location of the summary and queue indicator bits,
the driver uses the CCW_CMD_SET_IND_ADAPTER command with the following
payload:

\begin{lstlisting}
	struct virtio_thinint_area {
	       be64 summary_indicator;
	       be64 indicator;
	       be64 bit_nr;
	       u8 isc;
	} __attribute__ ((packed));
\end{lstlisting}

summary_indicator contains the guest address of the 8 bit summary
indicator.
indicator contains the guest address of an area wherin the indicators
for the devices are contained, starting at bit_nr, one bit per
virtqueue of the device. Bit numbers start at the left, i.e. the most
significant bit in the first byte is assigned the bit number 0.
isc contains the I/O interruption subclass to be used for the adapter
I/O interrupt. It may be different from the isc used by the proxy
virtio-ccw device's subchannel.

If the driver has already set up classic queue indicators via the
CCW_CMD_SET_IND command, the device MUST post a unit check with
command reject to any subsequent CCW_CMD_SET_IND_ADAPTER command.

\paragraph{Legacy Interfaces: A Note on Setting Up Indicators}\label{sec:Virtio Transport Options / Virtio over channel I/O / Device Initialization / Setting Up Indicators / Legacy Interfaces: A Note on Setting Up Indicators}

Legacy devices will only support classic queue indicators; they will
reject CCW_CMD_SET_IND_ADAPTER as they don't know that command.

\subsection{Device Operation}\label{sec:Virtio Transport Options / Virtio over channel I/O / Device Operation}

\subsubsection{Host->Guest Notification}\label{sec:Virtio Transport Options / Virtio over channel I/O / Device Operation / Host->Guest Notification}

There are two modes of operation regarding host->guest notification,
classic I/O interrupts and adapter I/O interrupts. The mode to be
used is determined by the driver by using CCW_CMD_SET_IND respectively
CCW_CMD_SET_IND_ADAPTER to set up queue indicators.

For configuration changes, the driver always uses classic I/O
interrupts.

\paragraph{Notification via Classic I/O Interrupts}\label{sec:Virtio Transport Options / Virtio over channel I/O / Device Operation / Host->Guest Notification / Notification via Classic I/O Interrupts}

If the driver used the CCW_CMD_SET_IND command to set up queue
indicators, the device will use classic I/O interrupts for
host->guest notification about virtqueue activity.

For notifying the driver of virtqueue buffers, the device sets the
corresponding bit in the guest-provided indicators. If an
interrupt is not already pending for the subchannel, the device
generates an unsolicited I/O interrupt.

If the device wants to notify the driver about configuration
changes, it sets bit 0 in the configuration indicators and
generates an unsolicited I/O interrupt, if needed. This also
applies if adapter I/O interrupts are used for queue notifications.

\paragraph{Notification via Adapter I/O Interrupts}\label{sec:Virtio Transport Options / Virtio over channel I/O / Device Operation / Host->Guest Notification / Notification via Adapter I/O Interrupts}

If the driver used the CCW_CMD_SET_IND_ADAPTER command to set up
queue indicators, the device will use adapter I/O interrupts for
host->guest notification about virtqueue activity.

For notifying the driver of virtqueue buffers, the device sets the
bit in the guest-provided indicator area at the corresponding offset.
The guest-provided summary indicator is set to 0x01. An adapter I/O
interrupt for the corresponding interruption subclass is generated.
The device SHOULD only generate an adapter I/O interrupt if the
summary indicator had not been set prior to notification. The driver
MUST clear the summary indicator after receiving an adapter I/O
interrupt before it processes the queue indicators.

\paragraph{Legacy Interfaces: A Note on Host->Guest Notification}\label{sec:Virtio Transport Options / Virtio over channel I/O / Device Operation / Host->Guest Notification / Legacy Interfaces: A Note on Host->Guest Notification}

As legacy devices and drivers support only classic queue indicators,
host->guest notification will always be done via classic I/O interrupts.

\subsubsection{Guest->Host Notification}\label{sec:Virtio Transport Options / Virtio over channel I/O / Device Operation / Guest->Host Notification}

For notifying the device of virtqueue buffers, the driver
unfortunately can't use a channel command (the asynchronous
characteristics of channel I/O interact badly with the host block
I/O backend). Instead, it uses a diagnose 0x500 call with subcode
3 specifying the queue, as follows:

\begin{tabular}{ |l|l|l| }
\hline
GPR  &   Input Value     & Output Value \\
\hline \hline
  1   &       0x3         &              \\
\hline
  2   &  Subchannel ID    & Host Cookie  \\
\hline
  3   & Virtqueue number  &              \\
\hline
  4   &   Host Cookie     &              \\
\hline
\end{tabular}

The device MUST ignore bits 0-31 (counting from the left) of GPR2.
This aligns passing the subchannel ID with the way it is passed
for the existing I/O instructions.

Host cookie is an optional per-virtqueue 64 bit value that can be
used by the hypervisor to speed up the notification execution.
For each notification, the output value is returned in GPR2 and
should be passed in GPR4 for the next notification:

\begin{lstlisting}
        info->cookie = do_notify(schid,
                                 virtqueue_get_queue_index(vq),
                                 info->cookie);
\end{lstlisting}

\subsubsection{Resetting Devices}\label{sec:Virtio Transport Options / Virtio over channel I/O / Device Operation / Resetting Devices}

In order to reset a device, a driver sends the
CCW_CMD_VDEV_RESET command.


\chapter{Device Types}\label{sec:Device Types}

On top of the queues, config space and feature negotiation facilities
built into virtio, several specific devices are defined.

The following device IDs are used to identify different types of virtio
devices.  Some device IDs are reserved for devices which are not currently
defined in this standard.

Discovering what devices are available and their type is bus-dependent.

\begin{tabular} { |l|c| }
\hline
Device ID  &  Virtio Device    \\
\hline \hline
0          & reserved (invalid) \\
\hline
1          &   network card     \\
\hline
2          &   block device     \\
\hline
3          &      console       \\
\hline
4          &  entropy source    \\
\hline
5          & memory ballooning  \\
\hline
6          &     ioMemory       \\
\hline
7          &       rpmsg        \\
\hline
8          &     SCSI host      \\
\hline
9          &   9P transport     \\
\hline
10         &   mac80211 wlan    \\
\hline
11         &   rproc serial     \\
\hline
12         &   virtio CAIF      \\
\hline
\end{tabular}

\section{Network Device}\label{sec:Device Types / Network Device}

The virtio network device is a virtual ethernet card, and is the
most complex of the devices supported so far by virtio. It has
enhanced rapidly and demonstrates clearly how support for new
features should be added to an existing device. Empty buffers are
placed in one virtqueue for receiving packets, and outgoing
packets are enqueued into another for transmission in that order.
A third command queue is used to control advanced filtering
features.

\subsection{Device ID}\label{sec:Device Types / Network Device / Device ID}

 1

\subsection{Virtqueues}\label{sec:Device Types / Network Device / Virtqueues}

\begin{description}
\item[0] receiveq0
\item[1] transmitq0
\item[...]
\item[2N] receiveqN
\item[2N+1] transmitqN
\item[2N+2] controlq
\end{description}

 N=0 if VIRTIO_NET_F_MQ is not negotiated, otherwise N is derived
 from max_virtqueue_pairs control field.

 controlq only exists if VIRTIO_NET_F_CTRL_VQ set.

\subsection{Feature bits}\label{sec:Device Types / Network Device / Feature bits}

\begin{description}
\item[VIRTIO_NET_F_CSUM (0)] Device handles packets with partial checksum

\item[VIRTIO_NET_F_GUEST_CSUM (1)] Driver handles packets with partial checksum

\item[VIRTIO_NET_F_CTRL_GUEST_OFFLOADS (2)] Control channel offloads
        reconfiguration support.

\item[VIRTIO_NET_F_MAC (5)] Device has given MAC address.

\item[VIRTIO_NET_F_GUEST_TSO4 (7)] Driver can receive TSOv4.

\item[VIRTIO_NET_F_GUEST_TSO6 (8)] Driver can receive TSOv6.

\item[VIRTIO_NET_F_GUEST_ECN (9)] Driver can receive TSO with ECN.

\item[VIRTIO_NET_F_GUEST_UFO (10)] Driver can receive UFO.

\item[VIRTIO_NET_F_HOST_TSO4 (11)] Device can receive TSOv4.

\item[VIRTIO_NET_F_HOST_TSO6 (12)] Device can receive TSOv6.

\item[VIRTIO_NET_F_HOST_ECN (13)] Device can receive TSO with ECN.

\item[VIRTIO_NET_F_HOST_UFO (14)] Device can receive UFO.

\item[VIRTIO_NET_F_MRG_RXBUF (15)] Driver can merge receive buffers.

\item[VIRTIO_NET_F_STATUS (16)] Configuration status field is
    available.

\item[VIRTIO_NET_F_CTRL_VQ (17)] Control channel is available.

\item[VIRTIO_NET_F_CTRL_RX (18)] Control channel RX mode support.

\item[VIRTIO_NET_F_CTRL_VLAN (19)] Control channel VLAN filtering.

\item[VIRTIO_NET_F_GUEST_ANNOUNCE(21)] Driver can send gratuitous
    packets.

\item[VIRTIO_NET_F_MQ(22)] Device supports multiqueue with automatic
    receive steering.

\item[VIRTIO_NET_F_CTRL_MAC_ADDR(23)] Set MAC address through control
    channel.
\end{description}

\subsubsection{Legacy Interface: Feature bits}\label{sec:Device Types / Network Device / Feature bits / Legacy Interface: Feature bits}
\begin{description}
\item[VIRTIO_NET_F_GSO (6)] Device handles packets with any GSO type.
\end{description}

This was supposed to indicate segmentation offload support, but
upon further investigation it became clear that multiple bits
were required.

\subsection{Device configuration layout}\label{sec:Device Types / Network Device / Device configuration layout}

Three configuration fields are currently defined. The mac address field
always exists (though is only valid if VIRTIO_NET_F_MAC is set), and
the status field only exists if VIRTIO_NET_F_STATUS is set. Two
read-only bits are currently defined for the status field:
VIRTIO_NET_S_LINK_UP and VIRTIO_NET_S_ANNOUNCE.

\begin{lstlisting}
	#define VIRTIO_NET_S_LINK_UP	1
	#define VIRTIO_NET_S_ANNOUNCE	2
\end{lstlisting}

The following read-only field, max_virtqueue_pairs only exists if
VIRTIO_NET_F_MQ is set. This field specifies the maximum number
of each of transmit and receive virtqueues (receiveq0..receiveqN
and transmitq0..transmitqN respectively;
 N=max_virtqueue_pairs - 1) that can be configured once VIRTIO_NET_F_MQ
is negotiated.  Legal values for this field are 1 to 0x8000.

\begin{lstlisting}
	struct virtio_net_config {
		u8 mac[6];
		le16 status;
		le16 max_virtqueue_pairs;
	};
\end{lstlisting}

\subsubsection{Legacy Interface: Device configuration layout}\label{sec:Device Types / Network Device / Device configuration layout / Legacy Interface: Device configuration layout}
For legacy devices, the status and max_virtqueue_pairs fields in struct virtio_net_config are the
native endian of the guest rather than (necessarily) little-endian.


\subsection{Device Initialization}\label{sec:Device Types / Network Device / Device Initialization}

\begin{enumerate}
\item The initialization routine should identify the receive and
  transmission virtqueues, up to N+1 of each kind. If
  VIRTIO_NET_F_MQ feature bit is negotiated,
  N=max_virtqueue_pairs-1, otherwise identify N=0.

\item If the VIRTIO_NET_F_MAC feature bit is set, the configuration
  space “mac” entry indicates the “physical” address of the the
  network card, otherwise a private MAC address should be
  assigned. All drivers are expected to negotiate this feature if
  it is set.

\item If the VIRTIO_NET_F_CTRL_VQ feature bit is negotiated,
  identify the control virtqueue.

\item If the VIRTIO_NET_F_STATUS feature bit is negotiated, the link
  status can be read from the bottom bit of the “status” config
  field. Otherwise, the link should be assumed active.

\item Only receiveq0, transmitq0 and controlq are used by default.
  To use more queues driver must negotiate the VIRTIO_NET_F_MQ
  feature; initialize up to max_virtqueue_pairs of each of
  transmit and receive queues;
  execute_VIRTIO_NET_CTRL_MQ_VQ_PAIRS_SET command specifying the
  number of the transmit and receive queues that is going to be
  used and wait until the device consumes the controlq buffer and
  acks this command.
  The receive virtqueue should be filled with receive buffers
  before multiqueue is activated
  (see \ref{sec:Device Types / Network Device / Device Operation / Control Virtqueue / Automatic receive steering in multiqueue mode}~\nameref{sec:Device Types / Network Device / Device Operation / Control Virtqueue / Automatic receive steering in multiqueue mode}).
  This is described in detail below in \nameref{sec:Device Types / Network Device / Device Operation / Setting Up Receive Buffers}.

\item A driver can indicate that it will generate checksumless
  packets by negotating the VIRTIO_NET_F_CSUM feature. This 
  “checksum offload” is a common feature on modern network cards.

\item If that feature is negotiated\footnote{ie. VIRTIO_NET_F_HOST_TSO* and VIRTIO_NET_F_HOST_UFO are
dependent on VIRTIO_NET_F_CSUM; a dvice which offers the offload
features must offer the checksum feature, and a driver which
accepts the offload features must accept the checksum feature.
Similar logic applies to the VIRTIO_NET_F_GUEST_TSO4 features
depending on VIRTIO_NET_F_GUEST_CSUM.
}, a driver can use TCP or UDP
  segmentation offload by negotiating the VIRTIO_NET_F_HOST_TSO4 (IPv4
  TCP), VIRTIO_NET_F_HOST_TSO6 (IPv6 TCP) and VIRTIO_NET_F_HOST_UFO
  (UDP fragmentation) features. It should not send TCP packets
  requiring segmentation offload which have the Explicit Congestion
  Notification bit set, unless the VIRTIO_NET_F_HOST_ECN feature is
  negotiated.\footnote{This is a common restriction in real, older network cards.
}

\item The converse features are also available: a driver can save
  the virtual device some work by negotiating these features.\footnote{For example, a network packet transported between two guests on
the same system may not require checksumming at all, nor segmentation,
if both guests are amenable.
}
   The VIRTIO_NET_F_GUEST_CSUM feature indicates that partially
  checksummed packets can be received, and if it can do that then
  the VIRTIO_NET_F_GUEST_TSO4, VIRTIO_NET_F_GUEST_TSO6,
  VIRTIO_NET_F_GUEST_UFO and VIRTIO_NET_F_GUEST_ECN are the input
  equivalents of the features described above.
  See \ref{sec:Device Types / Network Device / Device Operation / Setting Up Receive Buffers}~\nameref{sec:Device Types / Network Device / Device Operation / Setting Up Receive Buffers} and \ref{sec:Device Types / Network Device / Device Operation / Setting Up Receive Buffers}~\nameref{sec:Device Types / Network Device / Device Operation / Setting Up Receive Buffers} below.
\end{enumerate}

\subsection{Device Operation}\label{sec:Device Types / Network Device / Device Operation}

Packets are transmitted by placing them in the
transmitq0..transmitqN, and buffers for incoming packets are
placed in the receiveq0..receiveqN. In each case, the packet
itself is preceeded by a header:

\begin{lstlisting}
	struct virtio_net_hdr {
	#define VIRTIO_NET_HDR_F_NEEDS_CSUM    1
		u8 flags;
	#define VIRTIO_NET_HDR_GSO_NONE        0
	#define VIRTIO_NET_HDR_GSO_TCPV4       1
	#define VIRTIO_NET_HDR_GSO_UDP		 3
	#define VIRTIO_NET_HDR_GSO_TCPV6       4
	#define VIRTIO_NET_HDR_GSO_ECN      0x80
		u8 gso_type;
		le16 hdr_len;
		le16 gso_size;
		le16 csum_start;
		le16 csum_offset;
	/* Only if VIRTIO_NET_F_MRG_RXBUF: */
		le16 num_buffers;
	};
\end{lstlisting}

The controlq is used to control device features such as
filtering.

\subsubsection{Legacy Interface: Device Operation}\label{sec:Device Types / Network Device / Device Operation / Legacy Interface: Device Operation}
For legacy devices, the fields in struct virtio_net_hdr are the
native endian of the guest rather than (necessarily) little-endian.

\subsubsection{Packet Transmission}\label{sec:Device Types / Network Device / Device Operation / Packet Transmission}

Transmitting a single packet is simple, but varies depending on
the different features the driver negotiated.

\begin{enumerate}
\item If the driver negotiated VIRTIO_NET_F_CSUM, and the packet has
  not been fully checksummed, then the virtio_net_hdr's fields
  are set as follows. Otherwise, the packet must be fully
  checksummed, and flags is zero.
  \begin{itemize}
  \item flags has the VIRTIO_NET_HDR_F_NEEDS_CSUM set,

  \item csum_start is set to the offset within the packet to begin checksumming,
    and

  \item csum_offset indicates how many bytes after the csum_start the
    new (16 bit ones' complement) checksum should be placed.
  \end{itemize}

For example, consider a partially checksummed TCP (IPv4) packet.
It will have a 14 byte ethernet header and 20 byte IP header
followed by the TCP header (with the TCP checksum field 16 bytes
into that header). csum_start will be 14+20 = 34 (the TCP
checksum includes the header), and csum_offset will be 16. The
value in the TCP checksum field should be initialized to the sum
of the TCP pseudo header, so that replacing it by the ones'
complement checksum of the TCP header and body will give the
correct result.

\item If the driver negotiated
  VIRTIO_NET_F_HOST_TSO4, TSO6 or UFO, and the packet requires
  TCP segmentation or UDP fragmentation, then the “gso_type”
  field is set to VIRTIO_NET_HDR_GSO_TCPV4, TCPV6 or UDP.
  (Otherwise, it is set to VIRTIO_NET_HDR_GSO_NONE). In this
  case, packets larger than 1514 bytes can be transmitted: the
  metadata indicates how to replicate the packet header to cut it
  into smaller packets. The other gso fields are set:

  \begin{itemize}
  \item hdr_len is a hint to the device as to how much of the header
    needs to be kept to copy into each packet, usually set to the
    length of the headers, including the transport header.\footnote{Due to various bugs in implementations, this field is not useful
as a guarantee of the transport header size.
}

  \item gso_size is the maximum size of each packet beyond that
    header (ie. MSS).

  \item If the driver negotiated the VIRTIO_NET_F_HOST_ECN feature,
    the VIRTIO_NET_HDR_GSO_ECN bit may be set in “gso_type” as
    well, indicating that the TCP packet has the ECN bit set.\footnote{This case is not handled by some older hardware, so is called out
specifically in the protocol.
}
   \end{itemize}

\item If the driver negotiated the VIRTIO_NET_F_MRG_RXBUF feature,
  the num_buffers field is set to zero.

\item The header and packet are added as one output buffer to the
  transmitq, and the device is notified of the new entry
  (see \ref{sec:Device Types / Network Device / Device Initialization}~\nameref{sec:Device Types / Network Device / Device Initialization}).\footnote{Note that the header will be two bytes longer for the
VIRTIO_NET_F_MRG_RXBUF case.
}
\end{enumerate}

\paragraph{Packet Transmission Interrupt}\label{sec:Device Types / Network Device / Device Operation / Packet Transmission / Packet Transmission Interrupt}

Often a driver will suppress transmission interrupts using the
VRING_AVAIL_F_NO_INTERRUPT flag
 (see \ref{sec:Device Types / Block Device}~\nameref{sec:Device Types / Block Device})
and check for used packets in the transmit path of following
packets.

The normal behavior in this interrupt handler is to retrieve and
new descriptors from the used ring and free the corresponding
headers and packets.

\subsubsection{Setting Up Receive Buffers}\label{sec:Device Types / Network Device / Device Operation / Setting Up Receive Buffers}

It is generally a good idea to keep the receive virtqueue as
fully populated as possible: if it runs out, network performance
will suffer.

If the VIRTIO_NET_F_GUEST_TSO4, VIRTIO_NET_F_GUEST_TSO6 or
VIRTIO_NET_F_GUEST_UFO features are used, the Driver will need to
accept packets of up to 65550 bytes long (the maximum size of a
TCP or UDP packet, plus the 14 byte ethernet header), otherwise
1514. bytes. So unless VIRTIO_NET_F_MRG_RXBUF is negotiated, every
buffer in the receive queue needs to be at least this length.\footnote{Obviously each one can be split across multiple descriptor
elements.
}

If VIRTIO_NET_F_MRG_RXBUF is negotiated, each buffer must be at
least the size of the struct virtio_net_hdr.

If VIRTIO_NET_F_MQ is negotiated, each of receiveq0...receiveqN
that will be used should be populated with receive buffers.

\paragraph{Packet Receive Interrupt}\label{sec:Device Types / Network Device / Device Operation / Setting Up Receive Buffers / Packet Receive Interrupt}

When a packet is copied into a buffer in the receiveq, the
optimal path is to disable further interrupts for the receiveq
(see \ref{sec:General Initialization And Device Operation / Device Operation / Receiving Used Buffers From The Device}~\nameref{sec:General Initialization And Device Operation / Device Operation / Receiving Used Buffers From The Device}) and process
packets until no more are found, then re-enable them.

Processing packet involves:

\begin{enumerate}
\item If the driver negotiated the VIRTIO_NET_F_MRG_RXBUF feature,
  then the “num_buffers” field indicates how many descriptors
  this packet is spread over (including this one). This allows
  receipt of large packets without having to allocate large
  buffers. In this case, there will be at least “num_buffers” in
  the used ring, and they should be chained together to form a
  single packet. The other buffers will not begin with a struct
  virtio_net_hdr.

\item If the VIRTIO_NET_F_MRG_RXBUF feature was not negotiated, or
  the “num_buffers” field is one, then the entire packet will be
  contained within this buffer, immediately following the struct
  virtio_net_hdr.

\item If the VIRTIO_NET_F_GUEST_CSUM feature was negotiated, the
  VIRTIO_NET_HDR_F_NEEDS_CSUM bit in the “flags” field may be
  set: if so, the checksum on the packet is incomplete and the “
  csum_start” and “csum_offset” fields indicate how to calculate
  it (see Packet Transmission point 1).

\item If the VIRTIO_NET_F_GUEST_TSO4, TSO6 or UFO options were
  negotiated, then the “gso_type” may be something other than
  VIRTIO_NET_HDR_GSO_NONE, and the “gso_size” field indicates the
  desired MSS (see Packet Transmission point 2).
\end{enumerate}

\subsubsection{Control Virtqueue}\label{sec:Device Types / Network Device / Device Operation / Control Virtqueue}

The driver uses the control virtqueue (if VIRTIO_NET_F_CTRL_VQ is
negotiated) to send commands to manipulate various features of
the device which would not easily map into the configuration
space.

All commands are of the following form:

\begin{lstlisting}
	struct virtio_net_ctrl {
		u8 class;
		u8 command;
		u8 command-specific-data[];
		u8 ack;
	};

	/* ack values */
	#define VIRTIO_NET_OK     0
	#define VIRTIO_NET_ERR    1
\end{lstlisting}

The class, command and command-specific-data are set by the
driver, and the device sets the ack byte. There is little it can
do except issue a diagnostic if the ack byte is not
VIRTIO_NET_OK.

\paragraph{Packet Receive Filtering}\label{sec:Device Types / Network Device / Device Operation / Control Virtqueue / Packet Receive Filtering}

If the VIRTIO_NET_F_CTRL_RX feature is negotiated, the driver can
send control commands for promiscuous mode, multicast receiving,
and filtering of MAC addresses.

Note that in general, these commands are best-effort: unwanted
packets may still arrive.

\paragraph{Setting Promiscuous Mode}\label{sec:Device Types / Network Device / Device Operation / Control Virtqueue / Setting Promiscuous Mode}

\begin{lstlisting}
	#define VIRTIO_NET_CTRL_RX    0
	 #define VIRTIO_NET_CTRL_RX_PROMISC      0
	 #define VIRTIO_NET_CTRL_RX_ALLMULTI     1
\end{lstlisting}

The class VIRTIO_NET_CTRL_RX has two commands:
VIRTIO_NET_CTRL_RX_PROMISC turns promiscuous mode on and off, and
VIRTIO_NET_CTRL_RX_ALLMULTI turns all-multicast receive on and
off. The command-specific-data is one byte containing 0 (off) or
1 (on).

\paragraph{Setting MAC Address Filtering}\label{sec:Device Types / Network Device / Device Operation / Control Virtqueue / Setting MAC Address Filtering}

\begin{lstlisting}
	struct virtio_net_ctrl_mac {
		le32 entries;
		u8 macs[entries][ETH_ALEN];
	};

	#define VIRTIO_NET_CTRL_MAC    1
	 #define VIRTIO_NET_CTRL_MAC_TABLE_SET        0
	 #define VIRTIO_NET_CTRL_MAC_ADDR_SET         1
\end{lstlisting}

The device can filter incoming packets by any number of destination
MAC addresses.\footnote{Since there are no guarentees, it can use a hash filter or
silently switch to allmulti or promiscuous mode if it is given too
many addresses.
} This table is set using the class
VIRTIO_NET_CTRL_MAC and the command VIRTIO_NET_CTRL_MAC_TABLE_SET. The
command-specific-data is two variable length tables of 6-byte MAC
addresses. The first table contains unicast addresses, and the second
contains multicast addresses.

When VIRTIO_NET_F_MAC_ADDR is not negotiated, the mac field in
config space is writeable and is used to set the default MAC
address which rx filtering accepts.
When VIRTIO_NET_F_MAC_ADDR is negotiated, the mac field in
config space becomes read-only.
The VIRTIO_NET_CTRL_MAC_ADDR_SET command is used to set the
default MAC address which rx filtering
accepts

Depending on whether VIRTIO_NET_F_MAC_ADDR has been negotiated,
the mac field in config space or the VIRTIO_NET_CTRL_MAC_ADDR_SET
is used to set the default MAC address which rx filtering
accepts.
The command-specific-data for VIRTIO_NET_CTRL_MAC_ADDR_SET is
the 6-byte MAC address.

The
VIRTIO_NET_CTRL_MAC_ADDR_SET command is atomic whereas the
mac field in config space is not, therefore drivers
MUST negotiate VIRTIO_NET_F_MAC_ADDR if they change
mac address when device is accepting incoming packets.

\subparagraph{Legacy Interface: Setting MAC Address Filtering}\label{sec:Device Types / Network Device / Device Operation / Control Virtqueue / Setting MAC Address Filtering / Legacy Interface: Setting MAC Address Filtering}
For legacy devices, the entries field in struct virtio_net_ctrl_mac is the
native endian of the guest rather than (necessarily) little-endian.

Legacy drivers that didn't negotiate VIRTIO_NET_F_MAC_ADDR
changed the mac field in config space when NIC is accepting
incoming packets. These drivers always wrote the mac value from
first to last byte, therefore after detecting such drivers,
a transitional device CAN defer MAC update, or CAN defer
processing incoming packets until driver writes the last byte
of the mac field in config space.

\paragraph{VLAN Filtering}\label{sec:Device Types / Network Device / Device Operation / Control Virtqueue / VLAN Filtering}

If the driver negotiates the VIRTION_NET_F_CTRL_VLAN feature, it
can control a VLAN filter table in the device.

\begin{lstlisting}
	#define VIRTIO_NET_CTRL_VLAN       2
	 #define VIRTIO_NET_CTRL_VLAN_ADD             0
	 #define VIRTIO_NET_CTRL_VLAN_DEL             1
\end{lstlisting}

Both the VIRTIO_NET_CTRL_VLAN_ADD and VIRTIO_NET_CTRL_VLAN_DEL
command take a little-endian 16-bit VLAN id as the command-specific-data.

\subparagraph{Legacy Interface: VLAN Filtering}\label{sec:Device Types / Network Device / Device Operation / Control Virtqueue / VLAN Filtering / Legacy Interface: VLAN Filtering}
For legacy devices, the VLAN id is in the
native endian of the guest rather than (necessarily) little-endian.

\paragraph{Gratuitous Packet Sending}\label{sec:Device Types / Network Device / Device Operation / Control Virtqueue / Gratuitous Packet Sending}

If the driver negotiates the VIRTIO_NET_F_GUEST_ANNOUNCE (depends
on VIRTIO_NET_F_CTRL_VQ), it can ask the driver to send gratuitous
packets; this is usually done after the guest has been physically
migrated, and needs to announce its presence on the new network
links. (As hypervisor does not have the knowledge of guest
network configuration (eg. tagged vlan) it is simplest to prod
the guest in this way).

\begin{lstlisting}
	#define VIRTIO_NET_CTRL_ANNOUNCE       3
	 #define VIRTIO_NET_CTRL_ANNOUNCE_ACK             0
\end{lstlisting}

The Driver needs to check VIRTIO_NET_S_ANNOUNCE bit in status
field when it notices the changes of device configuration. The
command VIRTIO_NET_CTRL_ANNOUNCE_ACK is used to indicate that
driver has recevied the notification and device would clear the
VIRTIO_NET_S_ANNOUNCE bit in the status filed after it received
this command.

Processing this notification involves:

\begin{enumerate}
\item Sending the gratuitous packets or marking there are pending
  gratuitous packets to be sent and letting deferred routine to
  send them.

\item Sending VIRTIO_NET_CTRL_ANNOUNCE_ACK command through control
  vq.
\end{enumerate}

\paragraph{Automatic receive steering in multiqueue mode}\label{sec:Device Types / Network Device / Device Operation / Control Virtqueue / Automatic receive steering in multiqueue mode}

If the driver negotiates the VIRTIO_NET_F_MQ feature bit (depends
on VIRTIO_NET_F_CTRL_VQ), it MAY transmit outgoing packets on one
of the multiple transmitq0..transmitqN and ask the device to
queue incoming packets into one of the multiple receiveq0..receiveqN
depending on the packet flow.

\begin{lstlisting}
	struct virtio_net_ctrl_mq {
		le16 virtqueue_pairs;
	};

	#define VIRTIO_NET_CTRL_MQ    4
	 #define VIRTIO_NET_CTRL_MQ_VQ_PAIRS_SET        0
	 #define VIRTIO_NET_CTRL_MQ_VQ_PAIRS_MIN        1
	 #define VIRTIO_NET_CTRL_MQ_VQ_PAIRS_MAX        0x8000
\end{lstlisting}

Multiqueue is disabled by default. The driver enables multiqueue by
executing the VIRTIO_NET_CTRL_MQ_VQ_PAIRS_SET command, specifying
the number of the transmit and receive queues to be used; subsequently,
transmitq0..transmitqn and receiveq0..receiveqn where
n=virtqueue_pairs-1 MAY be used. All these virtqueues MUST have
been pre-configured in advance. The range of legal values for the
virtqueue_pairs field is between 1 and max_virtqueue_pairs.

When multiqueue is enabled, the device MUST use automatic receive steering
based on packet flow. Programming of the receive steering
classificator is implicit. After the driver transmitted a packet of a specific
flow on transmitqX, the device MUST cause incoming packets for this flow to
be steered to receiveqX. For uni-directional protocols, or where
no packets have been transmitted yet, the device MAY steer a packet
to a random queue out of the specified receiveq0..receiveqn.

Multiqueue is disabled by setting virtqueue_pairs = 1 (this is
the default). After the command has been consumed by the device, the
device MUST NOT steer new packets to virtqueues
receveq1..receiveqN (i.e. other than receiveq0) and MUST NOT read from
transmitq1..transmitqN (i.e. other than transmitq0); accordingly,
the driver MUST NOT transmit new packets on virtqueues other than
transmitq0.

\subparagraph{Legacy Interface: Automatic receive steering in multiqueue mode}\label{sec:Device Types / Network Device / Device Operation / Control Virtqueue / Automatic receive steering in multiqueue mode / Legacy Interface: Automatic receive steering in multiqueue mode}
For legacy devices, the virtqueue_paris field is in the
native endian of the guest rather than (necessarily) little-endian.

\paragraph{Offloads State Configuration}\label{sec:Device Types / Network Device / Device Operation / Control Virtqueue / Offloads State Configuration}

If the VIRTIO_NET_F_CTRL_GUEST_OFFLOADS feature is negotiated, the driver can
send control commands for dynamic offloads state configuration.

\subparagraph{Setting Offloads State}\label{sec:Device Types / Network Device / Device Operation / Control Virtqueue / Offloads State Configuration / Setting Offloads State}

\begin{lstlisting}
	le64 offloads;

	#define VIRTIO_NET_F_GUEST_CSUM       1
	#define VIRTIO_NET_F_GUEST_TSO4       7
	#define VIRTIO_NET_F_GUEST_TSO6       8
	#define VIRTIO_NET_F_GUEST_ECN        9
	#define VIRTIO_NET_F_GUEST_UFO        10

	#define VIRTIO_NET_CTRL_GUEST_OFFLOADS       5
	 #define VIRTIO_NET_CTRL_GUEST_OFFLOADS_SET   0
\end{lstlisting}

The class VIRTIO_NET_CTRL_GUEST_OFFLOADS has one command:
VIRTIO_NET_CTRL_GUEST_OFFLOADS_SET applies the new offloads configuration.

le64 value passed as command data is a bitmask, bits set define
offloads to be enabled, bits cleared - offloads to be disabled.

There is a corresponding device feature for each offload. Upon feature
negotiation corresponding offload gets enabled to preserve backward
compartibility.

Corresponding feature must be negotiated at startup in order to allow dynamic
change of specific offload state.


\subparagraph{Legacy Interface: Setting Offloads State}\label{sec:Device Types / Network Device / Device Operation / Control Virtqueue / Offloads State Configuration / Setting Offloads State / Legacy Interface: Setting Offloads State}
For legacy devices, the offloads field is the
native endian of the guest rather than (necessarily) little-endian.


\section{Block Device}\label{sec:Device Types / Block Device}

The virtio block device is a simple virtual block device (ie.
disk). Read and write requests (and other exotic requests) are
placed in the queue, and serviced (probably out of order) by the
device except where noted.

\subsection{Device ID}\label{sec:Device Types / Block Device / Device ID}
  2

\subsection{Virtqueues}\label{sec:Device Types / Block Device / Virtqueues}
\begin{description}
\item[0] requestq
\end{description}

\subsection{Feature bits}\label{sec:Device Types / Block Device / Feature bits}

\begin{description}
\item[VIRTIO_BLK_F_SIZE_MAX (1)] Maximum size of any single segment is
    in “size_max”.

\item[VIRTIO_BLK_F_SEG_MAX (2)] Maximum number of segments in a
    request is in “seg_max”.

\item[VIRTIO_BLK_F_GEOMETRY (4)] Disk-style geometry specified in “
    geometry”.

\item[VIRTIO_BLK_F_RO (5)] Device is read-only.

\item[VIRTIO_BLK_F_BLK_SIZE (6)] Block size of disk is in “blk_size”.

\item[VIRTIO_BLK_F_TOPOLOGY (10)] Device exports information on optimal I/O
    alignment.
\end{description}

\subsubsection{Legacy Interface: Feature bits}\label{sec:Device Types / Block Device / Feature bits / Legacy Interface: Feature bits}

\begin{description}
\item[VIRTIO_BLK_F_BARRIER (0)] Device supports request barriers.

\item[VIRTIO_BLK_F_SCSI (7)] Device supports scsi packet commands.

\item[VIRTIO_BLK_F_FLUSH (9)] Cache flush command support.

\item[VIRTIO_BLK_F_CONFIG_WCE (11)] Device can toggle its cache between writeback
    and writethrough modes.
\end{description}

VIRTIO_BLK_F_FLUSH was also called VIRTIO_BLK_F_WCE: Legacy drivers
should only negotiate this feature if they are capable of sending
VIRTIO_BLK_T_FLUSH commands.

\subsubsection{Device configuration layout}\label{sec:Device Types / Block Device / Feature bits / Device configuration layout}

The capacity of the device (expressed in 512-byte sectors) is always
present. The availability of the others all depend on various feature
bits as indicated above.

\begin{lstlisting}
	struct virtio_blk_config {
		le64 capacity;
		le32 size_max;
		le32 seg_max;
		struct virtio_blk_geometry {
			le16 cylinders;
			u8 heads;
			u8 sectors;
		} geometry;
		le32 blk_size;
		struct virtio_blk_topology {
			// # of logical blocks per physical block (log2)
			u8 physical_block_exp;
			// offset of first aligned logical block
			u8 alignment_offset;
			// suggested minimum I/O size in blocks
			le16 min_io_size;
			// optimal (suggested maximum) I/O size in blocks
			le32 opt_io_size;
		} topology;
		u8 reserved;
	};
\end{lstlisting}


\paragraph{Legacy Interface: Device configuration layout}\label{sec:Device Types / Block Device / Feature bits / Device configuration layout / Legacy Interface: Device configuration layout}
For legacy devices, the fields in struct virtio_blk_config are the
native endian of the guest rather than (necessarily) little-endian.


\subsection{Device Initialization}\label{sec:Device Types / Block Device / Device Initialization}

\begin{enumerate}
\item The device size should be read from the “capacity”
  configuration field. No requests should be submitted which goes
  beyond this limit.

\item If the VIRTIO_BLK_F_BLK_SIZE feature is negotiated, the
  blk_size field can be read to determine the optimal sector size
  for the driver to use. This does not affect the units used in
  the protocol (always 512 bytes), but awareness of the correct
  value can affect performance.

\item If the VIRTIO_BLK_F_RO feature is set by the device, any write
  requests will fail.

\item If the VIRTIO_BLK_F_TOPOLOGY feature is negotiated, the fields in the
  topology struct can be read to determine the physical block size and optimal
  I/O lengths for the driver to use. This also does not affect the units
  in the protocol, only performance.
\end{enumerate}

\subsubsection{Legacy Interface: Device Initialization}\label{sec:Device Types / Block Device / Device Initialization / Legacy Interface: Device Initialization}

The reserved field used to be called writeback.  If the
VIRTIO_BLK_F_CONFIG_WCE feature is offered, the cache mode should be
read from the writeback field of the configuration if available; the
driver can also write to the field in order to toggle the cache
between writethrough (0) and writeback (1) mode.  If the feature is
not available, the driver can instead look at the result of
negotiating VIRTIO_BLK_F_FLUSH: the cache will be in writeback mode
after reset if and only if VIRTIO_BLK_F_FLUSH is negotiated.

Some older legacy devices did not operate in writethrough mode even
after a driver announced lack of support for VIRTIO_BLK_F_FLUSH.

\subsection{Device Operation}\label{sec:Device Types / Block Device / Device Operation}

The driver queues requests to the virtqueue, and they are used by
the device (not necessarily in order). Each request is of form:

\begin{lstlisting}
	struct virtio_blk_req {
		le32 type;
		le32 reserved;
		le64 sector;
		char data[][512];
		u8 status;
	};
\end{lstlisting}

The type of the request is either a read (VIRTIO_BLK_T_IN), a write
(VIRTIO_BLK_T_OUT), or a flush (VIRTIO_BLK_T_FLUSH or
VIRTIO_BLK_T_FLUSH_OUT\footnote{The FLUSH and FLUSH_OUT types are equivalent, the device does not
distinguish between them
}).

\begin{lstlisting}
	#define VIRTIO_BLK_T_IN           0
	#define VIRTIO_BLK_T_OUT          1
	#define VIRTIO_BLK_T_FLUSH        4
	#define VIRTIO_BLK_T_FLUSH_OUT    5
\end{lstlisting}

The sector number indicates the offset (multiplied by 512) where
the read or write is to occur. This field is unused and set to 0
for scsi packet commands and for flush commands.

The final status byte is written by the device: either
VIRTIO_BLK_S_OK for success, VIRTIO_BLK_S_IOERR for device or driver
error or VIRTIO_BLK_S_UNSUPP for a request unsupported by device:

\begin{lstlisting}
	#define VIRTIO_BLK_S_OK        0
	#define VIRTIO_BLK_S_IOERR     1
	#define VIRTIO_BLK_S_UNSUPP    2
\end{lstlisting}

Any writes completed before the submission of the flush command should
be committed to non-volatile storage by the device.

\subsubsection{Legacy Interface: Device Operation}\label{sec:Device Types / Block Device / Device Operation / Legacy Interface: Device Operation}
For legacy devices, the fields in struct virtio_blk_req are the
native endian of the guest rather than (necessarily) little-endian.

The 'reserved' field was previously called ioprio.  The ioprio field
is a hint about the relative priorities of requests to the device:
higher numbers indicate more important requests.

\begin{lstlisting}
	#define VIRTIO_BLK_T_BARRIER	 0x80000000
\end{lstlisting}

If the device has VIRTIO_BLK_F_BARRIER
feature the high bit (VIRTIO_BLK_T_BARRIER) indicates that this
request acts as a barrier and that all preceeding requests must be
complete before this one, and all following requests must not be
started until this is complete. Note that a barrier does not flush
caches in the underlying backend device in host, and thus does not
serve as data consistency guarantee. Driver must use FLUSH request to
flush the host cache.

If the device has VIRTIO_BLK_F_SCSI feature, it can also support
scsi packet command requests, each of these requests is of form:

\begin{lstlisting}
	/* All fields are in guest's native endian. */
	struct virtio_scsi_pc_req {
		u32 type;
		u32 ioprio;
		u64 sector;
		char cmd[];
		char data[][512];
	#define SCSI_SENSE_BUFFERSIZE   96
		u8 sense[SCSI_SENSE_BUFFERSIZE];
		u32 errors;
		u32 data_len;
		u32 sense_len;
		u32 residual;
		u8 status;
	};
\end{lstlisting}

A request type can also be a scsi packet command (VIRTIO_BLK_T_SCSI_CMD or
VIRTIO_BLK_T_SCSI_CMD_OUT).  The two types are equivalent, the device
does not distinguish between them:

\begin{lstlisting}
	#define VIRTIO_BLK_T_SCSI_CMD     2
	#define VIRTIO_BLK_T_SCSI_CMD_OUT 3
\end{lstlisting}

The cmd field is only present for scsi packet command requests,
and indicates the command to perform. This field must reside in a
single, separate read-only buffer; command length can be derived
from the length of this buffer.

Note that these first three (four for scsi packet commands)
fields are always read-only: the data field is either read-only
or write-only, depending on the request. The size of the read or
write can be derived from the total size of the request buffers.

The sense field is only present for scsi packet command requests,
and indicates the buffer for scsi sense data.

The data_len field is only present for scsi packet command
requests, this field is deprecated, and should be ignored by the
driver. Historically, devices copied data length there.

The sense_len field is only present for scsi packet command
requests and indicates the number of bytes actually written to
the sense buffer.

The residual field is only present for scsi packet command
requests and indicates the residual size, calculated as data
length - number of bytes actually transferred.

Historically, devices assumed that the fields type, ioprio and
sector reside in a single, separate read-only buffer; the fields
errors, data_len, sense_len and residual reside in a single,
separate write-only buffer; the sense field in a separate
write-only buffer of size 96 bytes, by itself; the fields errors,
data_len, sense_len and residual in a single write-only buffer;
and the status field is a separate read-only buffer of size 1
byte, by itself.


\section{Console Device}\label{sec:Device Types / Console Device}

The virtio console device is a simple device for data input and
output. A device may have one or more ports. Each port has a pair
of input and output virtqueues. Moreover, a device has a pair of
control IO virtqueues. The control virtqueues are used to
communicate information between the device and the driver about
ports being opened and closed on either side of the connection,
indication from the device about whether a particular port is a
console port, adding new ports, port hot-plug/unplug, etc., and
indication from the driver about whether a port or a device was
successfully added, port open/close, etc.. For data IO, one or
more empty buffers are placed in the receive queue for incoming
data and outgoing characters are placed in the transmit queue.

\subsection{Device ID}\label{sec:Device Types / Console Device / Device ID}

  3

\subsection{Virtqueues}\label{sec:Device Types / Console Device / Virtqueues}

\begin{description}
\item[0] receiveq(port0)
\item[1] transmitq(port0)
\item[2] control receiveq
\item[3] control transmitq
\item[4] receiveq(port1)
\item[5] transmitq(port1)
\item[\ldots]
\end{description}

  Ports 2 onwards only exist if VIRTIO_CONSOLE_F_MULTIPORT is set.

\subsection{Feature bits}\label{sec:Device Types / Console Device / Feature bits}

\begin{description}
\item[VIRTIO_CONSOLE_F_SIZE (0)] Configuration cols and rows fields
    are valid.

\item[VIRTIO_CONSOLE_F_MULTIPORT (1)] Device has support for multiple
    ports; configuration fields nr_ports and max_nr_ports are
    valid and control virtqueues will be used.

\item[VIRTIO_CONSOLE_F_EMERG_WRITE (2)] Device has support for emergency write.
    Configuration field emerg_wr is valid.
\end{description}

\subsection{Device configuration layout}\label{sec:Device Types / Console Device / Device configuration layout}

  The size of the console is supplied
  in the configuration space if the VIRTIO_CONSOLE_F_SIZE feature
  is set. Furthermore, if the VIRTIO_CONSOLE_F_MULTIPORT feature
  is set, the maximum number of ports supported by the device can
  be fetched.

  If VIRTIO_CONSOLE_F_EMERG_WRITE is set then the driver can use emergency write
  to output a single character without initializing virtio queues, or even
  acknowledging the feature.

\begin{lstlisting}
	struct virtio_console_config {
		le16 cols;
		le16 rows;
		le32 max_nr_ports;
		le32 emerg_wr;
	};
\end{lstlisting}

\subsubsection{Legacy Interface: Device configuration layout}\label{sec:Device Types / Console Device / Device configuration layout / Legacy Interface: Device configuration layout}
For legacy devices, the fields in struct virtio_console_config are the
native endian of the guest rather than (necessarily) little-endian.

\subsection{Device Initialization}\label{sec:Device Types / Console Device / Device Initialization}

\begin{enumerate}
\item If the VIRTIO_CONSOLE_F_EMERG_WRITE feature is offered, the
  emerg_wr field of the configuration can be written at any time.
  Thus it should work for very early boot debugging output as well as
  catastophic OS failures (eg. virtio ring corruption).

\item If the VIRTIO_CONSOLE_F_SIZE feature is negotiated, the driver
  can read the console dimensions from the configuration fields.

\item If the VIRTIO_CONSOLE_F_MULTIPORT feature is negotiated, the
  driver can spawn multiple ports, not all of which may be
  attached to a console. Some could be generic ports. In this
  case, the control virtqueues are enabled and according to the
  max_nr_ports configuration-space value, the appropriate number
  of virtqueues are created. A control message indicating the
  driver is ready is sent to the device. The device can then send
  control messages for adding new ports to the device. After
  creating and initializing each port, a
  VIRTIO_CONSOLE_PORT_READY control message is sent to the device
  for that port so the device can let us know of any additional
  configuration options set for that port.

\item The receiveq for each port is populated with one or more
  receive buffers.
\end{enumerate}

\subsection{Device Operation}\label{sec:Device Types / Console Device / Device Operation}

\begin{enumerate}
\item For output, a buffer containing the characters is placed in
  the port's transmitq.\footnote{Because this is high importance and low bandwidth, the current
Linux implementation polls for the buffer to be used, rather than
waiting for an interrupt, simplifying the implementation
significantly. However, for generic serial ports with the
O_NONBLOCK flag set, the polling limitation is relaxed and the
consumed buffers are freed upon the next write or poll call or
when a port is closed or hot-unplugged.
}

\item When a buffer is used in the receiveq (signalled by an
  interrupt), the contents is the input to the port associated
  with the virtqueue for which the notification was received.

\item If the driver negotiated the VIRTIO_CONSOLE_F_SIZE feature, a
  configuration change interrupt may occur. The updated size can
  be read from the configuration fields.

\item If the driver negotiated the VIRTIO_CONSOLE_F_MULTIPORT
  feature, active ports are announced by the device using the
  VIRTIO_CONSOLE_PORT_ADD control message. The same message is
  used for port hot-plug as well.

\item If the device specified a port `name', a sysfs attribute is
  created with the name filled in, so that udev rules can be
  written that can create a symlink from the port's name to the
  char device for port discovery by applications in the driver.

\item Changes to ports' state are effected by control messages.
  Appropriate action is taken on the port indicated in the
  control message. The layout of the structure of the control
  buffer and the events associated are:

\begin{lstlisting}
	struct virtio_console_control {
		le32 id;    /* Port number */
		le16 event; /* The kind of control event */
		le16 value; /* Extra information for the event */
	};

	/* Some events for the internal messages (control packets) */
	#define VIRTIO_CONSOLE_DEVICE_READY     0
	#define VIRTIO_CONSOLE_PORT_ADD         1
	#define VIRTIO_CONSOLE_PORT_REMOVE      2
	#define VIRTIO_CONSOLE_PORT_READY       3
	#define VIRTIO_CONSOLE_CONSOLE_PORT     4
	#define VIRTIO_CONSOLE_RESIZE           5
	#define VIRTIO_CONSOLE_PORT_OPEN        6
	#define VIRTIO_CONSOLE_PORT_NAME        7
\end{lstlisting}
\end{enumerate}

\subsubsection{Legacy Interface: Device Operation}\label{sec:Device Types / Console Device / Device Operation / Legacy Interface: Device Operation}
For legacy devices, the fields in struct virtio_console_control are the
native endian of the guest rather than (necessarily) little-endian.


\section{Entropy Device}\label{sec:Device Types / Entropy Device}

The virtio entropy device supplies high-quality randomness for
guest use.

\subsection{Device ID}\label{sec:Device Types / Entropy Device / Device ID}
  4

\subsection{Virtqueues}\label{sec:Device Types / Entropy Device / Virtqueues}
\begin{description}
\item[0] requestq
\end{description}

\subsection{Feature bits}\label{sec:Device Types / Entropy Device / Feature bits}
  None currently defined

\subsection{Device configuration layout}\label{sec:Device Types / Entropy Device / Device configuration layout}
  None currently defined.

\subsection{Device Initialization}\label{sec:Device Types / Entropy Device / Device Initialization}

\begin{enumerate}
\item The virtqueue is initialized
\end{enumerate}

\subsection{Device Operation}\label{sec:Device Types / Entropy Device / Device Operation}

When the driver requires random bytes, it places the descriptor
of one or more buffers in the queue. It will be completely filled
by random data by the device.

\section{Memory Balloon Device}\label{sec:Device Types / Memory Balloon Device}

The virtio memory balloon device is a primitive device for
managing guest memory: the device asks for a certain amount of
memory, and the driver supplies it (or withdraws it, if the device
has more than it asks for). This allows the guest to adapt to
changes in allowance of underlying physical memory. If the
feature is negotiated, the device can also be used to communicate
guest memory statistics to the host.

\subsection{Device ID}\label{sec:Device Types / Memory Balloon Device / Device ID}
  5

\subsection{Virtqueues}\label{sec:Device Types / Memory Balloon Device / Virtqueues}
\begin{description}
\item[0] inflateq
\item[1] deflateq
\item[2] statsq.
\end{description}

  Virtqueue 2 only exists if VIRTIO_BALLON_F_STATS_VQ set.

\subsection{Feature bits}\label{sec:Device Types / Memory Balloon Device / Feature bits}
\begin{description}
\item[VIRTIO_BALLOON_F_MUST_TELL_HOST (0)] Host must be told before
    pages from the balloon are used.

\item[VIRTIO_BALLOON_F_STATS_VQ (1)] A virtqueue for reporting guest
    memory statistics is present.
\end{description}

\subsection{Device configuration layout}\label{sec:Device Types / Memory Balloon Device / Device configuration layout}
  Both fields of this configuration
  are always available.

\begin{lstlisting}
	struct virtio_balloon_config {
		le32 num_pages;
		le32 actual;
	};
\end{lstlisting}

\subsubsection{Legacy Interface: Device configuration layout}\label{sec:Device Types / Memory Balloon Device / Device configuration layout / Legacy Interface: Device configuration layout}
Note that these fields are always little endian, despite convention
that legacy device fields are guest endian.

\subsection{Device Initialization}\label{sec:Device Types / Memory Balloon Device / Device Initialization}

\begin{enumerate}
\item The inflate and deflate virtqueues are identified.

\item If the VIRTIO_BALLOON_F_STATS_VQ feature bit is negotiated:
  \begin{enumerate}
  \item Identify the stats virtqueue.

  \item Add one empty buffer to the stats virtqueue and notify the
    device.
  \end{enumerate}
\end{enumerate}

Device operation begins immediately.

\subsection{Device Operation}\label{sec:Device Types / Memory Balloon Device / Device Operation}

The device is driven by the receipt of a
configuration change interrupt.

\begin{enumerate}
\item The “num_pages” configuration field is examined. If this is
  greater than the “actual” number of pages, memory must be given
  to the balloon. If it is less than the “actual” number of
  pages, memory may be taken back from the balloon for general
  use.

\item To supply memory to the balloon (aka. inflate):
  \begin{enumerate}
  \item The driver constructs an array of addresses of unused memory
    pages. These addresses are divided by 4096\footnote{This is historical, and independent of the guest page size
} and the descriptor
    describing the resulting 32-bit array is added to the inflateq.
  \end{enumerate}

\item To remove memory from the balloon (aka. deflate):
  \begin{enumerate}
  \item The driver constructs an array of addresses of memory pages
    it has previously given to the balloon, as described above.
    This descriptor is added to the deflateq.

  \item If the VIRTIO_BALLOON_F_MUST_TELL_HOST feature is negotiated, the
    guest may not use these requested pages until that descriptor
    in the deflateq has been used by the device.

  \item Otherwise, the guest may begin to re-use pages previously
    given to the balloon before the device has acknowledged their
    withdrawl.\footnote{In this case, deflation advice is merely a courtesy
}
  \end{enumerate}

\item In either case, once the device has completed the inflation or
  deflation, the “actual” field of the configuration should be
  updated to reflect the new number of pages in the balloon.\footnote{As updates to configuration space are not atomic, this field
isn't particularly reliable, but can be used to diagnose buggy guests.
}
\end{enumerate}

\subsubsection{Memory Statistics}\label{sec:Device Types / Memory Balloon Device / Device Operation / Memory Statistics}

The stats virtqueue is atypical because communication is driven
by the device (not the driver). The channel becomes active at
driver initialization time when the driver adds an empty buffer
and notifies the device. A request for memory statistics proceeds
as follows:

\begin{enumerate}
\item The device pushes the buffer onto the used ring and sends an
  interrupt.

\item The driver pops the used buffer and discards it.

\item The driver collects memory statistics and writes them into a
  new buffer.

\item The driver adds the buffer to the virtqueue and notifies the
  device.

\item The device pops the buffer (retaining it to initiate a
  subsequent request) and consumes the statistics.
\end{enumerate}

  Each statistic consists of a 16 bit
  tag and a 64 bit value. All statistics are optional and the
  driver may choose which ones to supply. To guarantee backwards
  compatibility, unsupported statistics should be omitted.

\begin{lstlisting}
	struct virtio_balloon_stat {
	#define VIRTIO_BALLOON_S_SWAP_IN  0
	#define VIRTIO_BALLOON_S_SWAP_OUT 1
	#define VIRTIO_BALLOON_S_MAJFLT   2
	#define VIRTIO_BALLOON_S_MINFLT   3
	#define VIRTIO_BALLOON_S_MEMFREE  4
	#define VIRTIO_BALLOON_S_MEMTOT   5
		le16 tag;
		le64 val;
	} __attribute__((packed));
\end{lstlisting}

\paragraph{Legacy Interface: Memory Statistics}\label{sec:Device Types / Memory Balloon Device / Device Operation / Memory Statistics / Legacy Interface: Memory Statistics}
For legacy devices, the fields in struct virtio_balloon_stat are the
native endian of the guest rather than (necessarily) little-endian.


\subsubsection{Memory Statistics Tags}\label{sec:Device Types / Memory Balloon Device / Device Operation / Memory Statistics Tags}

\begin{description}
\item[VIRTIO_BALLOON_S_SWAP_IN (0)] The amount of memory that has been
  swapped in (in bytes).

\item[VIRTIO_BALLOON_S_SWAP_OUT (1)] The amount of memory that has been
  swapped out to disk (in bytes).

\item[VIRTIO_BALLOON_S_MAJFLT (2)] The number of major page faults that
  have occurred.

\item[VIRTIO_BALLOON_S_MINFLT (3)] The number of minor page faults that
  have occurred.

\item[VIRTIO_BALLOON_S_MEMFREE (4)] The amount of memory not being used
  for any purpose (in bytes).

\item[VIRTIO_BALLOON_S_MEMTOT (5)] The total amount of memory available
  (in bytes).
\end{description}

\section{SCSI Host Device}\label{sec:Device Types / SCSI Host Device}

The virtio SCSI host device groups together one or more virtual
logical units (such as disks), and allows communicating to them
using the SCSI protocol. An instance of the device represents a
SCSI host to which many targets and LUNs are attached.

The virtio SCSI device services two kinds of requests:
\begin{itemize}
\item command requests for a logical unit;

\item task management functions related to a logical unit, target or
  command.
\end{itemize}

The device is also able to send out notifications about added and
removed logical units. Together, these capabilities provide a
SCSI transport protocol that uses virtqueues as the transfer
medium. In the transport protocol, the virtio driver acts as the
initiator, while the virtio SCSI host provides one or more
targets that receive and process the requests.

\subsection{Device ID}\label{sec:Device Types / SCSI Host Device / Device ID}
  8

\subsection{Virtqueues}\label{sec:Device Types / SCSI Host Device / Virtqueues}

\begin{description}
\item[0] controlq
\item[1] eventq
\item[2\ldots n] request queues
\end{description}

\subsection{Feature bits}\label{sec:Device Types / SCSI Host Device / Feature bits}

\begin{description}
\item[VIRTIO_SCSI_F_INOUT (0)] A single request can include both
    read-only and write-only data buffers.

\item[VIRTIO_SCSI_F_HOTPLUG (1)] The host should enable
    hot-plug/hot-unplug of new LUNs and targets on the SCSI bus.

\item[VIRTIO_SCSI_F_CHANGE (2)] The host will report changes to LUN
    parameters via a VIRTIO_SCSI_T_PARAM_CHANGE event.
\end{description}

\subsection{Device configuration layout}\label{sec:Device Types / SCSI Host Device / Device configuration layout}

  All fields of this configuration are always available. sense_size
  and cdb_size are writable by the driver.

\begin{lstlisting}
	struct virtio_scsi_config {
		le32 num_queues;
		le32 seg_max;
		le32 max_sectors;
		le32 cmd_per_lun;
		le32 event_info_size;
		le32 sense_size;
		le32 cdb_size;
		le16 max_channel;
		le16 max_target;
		le32 max_lun;
	};
\end{lstlisting}

\begin{description}
\item[num_queues] is the total number of request virtqueues exposed by
    the device. The driver is free to use only one request queue,
    or it can use more to achieve better performance.

\item[seg_max] is the maximum number of segments that can be in a
    command. A bidirectional command can include seg_max input
    segments and seg_max output segments.

\item[max_sectors] is a hint to the driver about the maximum transfer
    size it should use.

\item[cmd_per_lun] is a hint to the driver about the maximum number of
    linked commands it should send to one LUN. The actual value
    to be used is the minimum of cmd_per_lun and the virtqueue
    size.

\item[event_info_size] is the maximum size that the device will fill
    for buffers that the driver places in the eventq. The driver
    should always put buffers at least of this size. It is
    written by the device depending on the set of negotated
    features.

\item[sense_size] is the maximum size of the sense data that the
    device will write. The default value is written by the device
    and will always be 96, but the driver can modify it. It is
    restored to the default when the device is reset.

\item[cdb_size] is the maximum size of the CDB that the driver will
    write. The default value is written by the device and will
    always be 32, but the driver can likewise modify it. It is
    restored to the default when the device is reset.

\item[max_channel, max_target and max_lun] can be used by the driver
    as hints to constrain scanning the logical units on the
    host.h
\end{description}

\subsubsection{Legacy Interface: Device configuration layout}\label{sec:Device Types / SCSI Host Device / Device configuration layout / Legacy Interface: Device configuration layout}
For legacy devices, the fields in struct virtio_scsi_config are the
native endian of the guest rather than (necessarily) little-endian.

\subsection{Device Initialization}\label{sec:Device Types / SCSI Host Device / Device Initialization}

The initialization routine should first of all discover the
device's virtqueues.

If the driver uses the eventq, it should then place at least a
buffer in the eventq.

The driver can immediately issue requests (for example, INQUIRY
or REPORT LUNS) or task management functions (for example, I_T
RESET).

\subsection{Device Operation}\label{sec:Device Types / SCSI Host Device / Device Operation}

Device operation consists of operating request queues, the control
queue and the event queue.

\subsubsection{Device Operation: Request Queues}\label{sec:Device Types / SCSI Host Device / Device Operation / Device Operation: Request Queues}

The driver queues requests to an arbitrary request queue, and
they are used by the device on that same queue. It is the
responsibility of the driver to ensure strict request ordering
for commands placed on different queues, because they will be
consumed with no order constraints.

Requests have the following format:

\begin{lstlisting}
	struct virtio_scsi_req_cmd {
		// Read-only
		u8 lun[8];
		le64 id;
		u8 task_attr;
		u8 prio;
		u8 crn;
		char cdb[cdb_size];
		char dataout[];
		// Write-only part
		le32 sense_len;
		le32 residual;
		le16 status_qualifier;
		u8 status;
		u8 response;
		u8 sense[sense_size];
		char datain[];
	};


	/* command-specific response values */
	#define VIRTIO_SCSI_S_OK                0
	#define VIRTIO_SCSI_S_OVERRUN           1
	#define VIRTIO_SCSI_S_ABORTED           2
	#define VIRTIO_SCSI_S_BAD_TARGET        3
	#define VIRTIO_SCSI_S_RESET             4
	#define VIRTIO_SCSI_S_BUSY              5
	#define VIRTIO_SCSI_S_TRANSPORT_FAILURE 6
	#define VIRTIO_SCSI_S_TARGET_FAILURE    7
	#define VIRTIO_SCSI_S_NEXUS_FAILURE     8
	#define VIRTIO_SCSI_S_FAILURE           9

	/* task_attr */
	#define VIRTIO_SCSI_S_SIMPLE            0
	#define VIRTIO_SCSI_S_ORDERED           1
	#define VIRTIO_SCSI_S_HEAD              2
	#define VIRTIO_SCSI_S_ACA               3
\end{lstlisting}

The lun field addresses a target and logical unit in the
virtio-scsi device's SCSI domain. The only supported format for
the LUN field is: first byte set to 1, second byte set to target,
third and fourth byte representing a single level LUN structure,
followed by four zero bytes. With this representation, a
virtio-scsi device can serve up to 256 targets and 16384 LUNs per
target.

The id field is the command identifier (“tag”).

task_attr, prio and crn should be left to zero. task_attr defines
the task attribute as in the table above, but all task attributes
may be mapped to SIMPLE by the device; crn may also be provided
by clients, but is generally expected to be 0. The maximum CRN
value defined by the protocol is 255, since CRN is stored in an
8-bit integer.

All of these fields are defined in SAM. They are always
read-only, as are the cdb and dataout field. The cdb_size is
taken from the configuration space.

sense and subsequent fields are always write-only. The sense_len
field indicates the number of bytes actually written to the sense
buffer. The residual field indicates the residual size,
calculated as “data_length - number_of_transferred_bytes”, for
read or write operations. For bidirectional commands, the
number_of_transferred_bytes includes both read and written bytes.
A residual field that is less than the size of datain means that
the dataout field was processed entirely. A residual field that
exceeds the size of datain means that the dataout field was
processed partially and the datain field was not processed at
all.

The status byte is written by the device to be the status code as
defined in SAM.

The response byte is written by the device to be one of the
following:

\begin{description}

\item[VIRTIO_SCSI_S_OK] when the request was completed and the status
  byte is filled with a SCSI status code (not necessarily
  "GOOD").

\item[VIRTIO_SCSI_S_OVERRUN] if the content of the CDB requires
  transferring more data than is available in the data buffers.

\item[VIRTIO_SCSI_S_ABORTED] if the request was cancelled due to an
  ABORT TASK or ABORT TASK SET task management function.

\item[VIRTIO_SCSI_S_BAD_TARGET] if the request was never processed
  because the target indicated by the lun field does not exist.

\item[VIRTIO_SCSI_S_RESET] if the request was cancelled due to a bus
  or device reset (including a task management function).

\item[VIRTIO_SCSI_S_TRANSPORT_FAILURE] if the request failed due to a
  problem in the connection between the host and the target
  (severed link).

\item[VIRTIO_SCSI_S_TARGET_FAILURE] if the target is suffering a
  failure and the driver should not retry on other paths.

\item[VIRTIO_SCSI_S_NEXUS_FAILURE] if the nexus is suffering a failure
  but retrying on other paths might yield a different result.

\item[VIRTIO_SCSI_S_BUSY] if the request failed but retrying on the
  same path should work.

\item[VIRTIO_SCSI_S_FAILURE] for other host or driver error. In
  particular, if neither dataout nor datain is empty, and the
  VIRTIO_SCSI_F_INOUT feature has not been negotiated, the
  request will be immediately returned with a response equal to
  VIRTIO_SCSI_S_FAILURE.
\end{description}

\paragraph{Legacy Interface: Device Operation: Request Queues}\label{sec:Device Types / SCSI Host Device / Device Operation / Device Operation: Request Queues / Legacy Interface: Device Operation: Request Queues}
For legacy devices, the fields in struct virtio_scsi_req_cmd are the
native endian of the guest rather than (necessarily) little-endian.

\subsubsection{Device Operation: controlq}\label{sec:Device Types / SCSI Host Device / Device Operation / Device Operation: controlq}

The controlq is used for other SCSI transport operations.
Requests have the following format:

\begin{lstlisting}
	struct virtio_scsi_ctrl {
		le32 type;
	...
		u8 response;
	};

	/* response values valid for all commands */
	#define VIRTIO_SCSI_S_OK                       0
	#define VIRTIO_SCSI_S_BAD_TARGET               3
	#define VIRTIO_SCSI_S_BUSY                     5
	#define VIRTIO_SCSI_S_TRANSPORT_FAILURE        6
	#define VIRTIO_SCSI_S_TARGET_FAILURE           7
	#define VIRTIO_SCSI_S_NEXUS_FAILURE            8
	#define VIRTIO_SCSI_S_FAILURE                  9
	#define VIRTIO_SCSI_S_INCORRECT_LUN            12
\end{lstlisting}

The type identifies the remaining fields.

The following commands are defined:

  Task management function
\begin{lstlisting}
	#define VIRTIO_SCSI_T_TMF                      0

	#define VIRTIO_SCSI_T_TMF_ABORT_TASK           0
	#define VIRTIO_SCSI_T_TMF_ABORT_TASK_SET       1
	#define VIRTIO_SCSI_T_TMF_CLEAR_ACA            2
	#define VIRTIO_SCSI_T_TMF_CLEAR_TASK_SET       3
	#define VIRTIO_SCSI_T_TMF_I_T_NEXUS_RESET      4
	#define VIRTIO_SCSI_T_TMF_LOGICAL_UNIT_RESET   5
	#define VIRTIO_SCSI_T_TMF_QUERY_TASK           6
	#define VIRTIO_SCSI_T_TMF_QUERY_TASK_SET       7

	struct virtio_scsi_ctrl_tmf
	{
		// Read-only part
		le32 type;
		le32 subtype;
		u8   lun[8];
		le64 id;
		// Write-only part
		u8   response;
	}

	/* command-specific response values */
	#define VIRTIO_SCSI_S_FUNCTION_COMPLETE        0
	#define VIRTIO_SCSI_S_FUNCTION_SUCCEEDED       10
	#define VIRTIO_SCSI_S_FUNCTION_REJECTED        11
\end{lstlisting}

  The type is VIRTIO_SCSI_T_TMF; the subtype field defines. All
  fields except response are filled by the driver. The subtype
  field must always be specified and identifies the requested
  task management function.

  Other fields may be irrelevant for the requested TMF; if so,
  they are ignored but they should still be present. The lun
  field is in the same format specified for request queues; the
  single level LUN is ignored when the task management function
  addresses a whole I_T nexus. When relevant, the value of the id
  field is matched against the id values passed on the requestq.

  The outcome of the task management function is written by the
  device in the response field. The command-specific response
  values map 1-to-1 with those defined in SAM.

  Asynchronous notification query

\begin{lstlisting}
	#define VIRTIO_SCSI_T_AN_QUERY                    1

	struct virtio_scsi_ctrl_an {
	    // Read-only part
	    le32 type;
	    u8   lun[8];
	    le32 event_requested;
	    // Write-only part
	    le32 event_actual;
	    u8   response;
	}

	#define VIRTIO_SCSI_EVT_ASYNC_OPERATIONAL_CHANGE  2
	#define VIRTIO_SCSI_EVT_ASYNC_POWER_MGMT          4
	#define VIRTIO_SCSI_EVT_ASYNC_EXTERNAL_REQUEST    8
	#define VIRTIO_SCSI_EVT_ASYNC_MEDIA_CHANGE        16
	#define VIRTIO_SCSI_EVT_ASYNC_MULTI_HOST          32
	#define VIRTIO_SCSI_EVT_ASYNC_DEVICE_BUSY         64
\end{lstlisting}

  By sending this command, the driver asks the device which
  events the given LUN can report, as described in paragraphs 6.6
  and A.6 of the SCSI MMC specification. The driver writes the
  events it is interested in into the event_requested; the device
  responds by writing the events that it supports into
  event_actual.

  The type is VIRTIO_SCSI_T_AN_QUERY. The lun and event_requested
  fields are written by the driver. The event_actual and response
  fields are written by the device.

  No command-specific values are defined for the response byte.

  Asynchronous notification subscription
\begin{lstlisting}
	#define VIRTIO_SCSI_T_AN_SUBSCRIBE                2

	struct virtio_scsi_ctrl_an {
		// Read-only part
		le32 type;
		u8   lun[8];
		le32 event_requested;
		// Write-only part
		le32 event_actual;
		u8   response;
	}
\end{lstlisting}

  By sending this command, the driver asks the specified LUN to
  report events for its physical interface, again as described in
  the SCSI MMC specification. The driver writes the events it is
  interested in into the event_requested; the device responds by
  writing the events that it supports into event_actual.

  Event types are the same as for the asynchronous notification
  query message.

  The type is VIRTIO_SCSI_T_AN_SUBSCRIBE. The lun and
  event_requested fields are written by the driver. The
  event_actual and response fields are written by the device.

  No command-specific values are defined for the response byte.

\paragraph{Legacy Interface: Device Operation: controlq}\label{sec:Device Types / SCSI Host Device / Device Operation / Device Operation: controlq / Legacy Interface: Device Operation: controlq}

For legacy devices, the fields in struct virtio_scsi_ctrl, struct
virtio_scsi_ctrl_tmf, struct virtio_scsi_ctrl_an and struct
virtio_scsi_ctrl_an are the native endian of the guest rather than
(necessarily) little-endian.


\subsubsection{Device Operation: eventq}\label{sec:Device Types / SCSI Host Device / Device Operation / Device Operation: eventq}

The eventq is used by the device to report information on logical
units that are attached to it. The driver should always leave a
few buffers ready in the eventq. In general, the device will not
queue events to cope with an empty eventq, and will end up
dropping events if it finds no buffer ready. However, when
reporting events for many LUNs (e.g. when a whole target
disappears), the device can throttle events to avoid dropping
them. For this reason, placing 10-15 buffers on the event queue
should be enough.

Buffers are placed in the eventq and filled by the device when
interesting events occur. The buffers should be strictly
write-only (device-filled) and the size of the buffers should be
at least the value given in the device's configuration
information.

Buffers returned by the device on the eventq will be referred to
as "events" in the rest of this section. Events have the
following format:

\begin{lstlisting}
	#define VIRTIO_SCSI_T_EVENTS_MISSED   0x80000000

	struct virtio_scsi_event {
		// Write-only part
		le32 event;
		u8  lun[8];
		le32 reason;
	}
\end{lstlisting}

If bit 31 is set in the event field, the device failed to report
an event due to missing buffers. In this case, the driver should
poll the logical units for unit attention conditions, and/or do
whatever form of bus scan is appropriate for the guest operating
system.

The meaning of the reason field depends on the
contents of the event field. The following events are defined:

  No event
\begin{lstlisting}
	#define VIRTIO_SCSI_T_NO_EVENT         0
\end{lstlisting}

  This event is fired in the following cases:

\begin{itemize}
\item When the device detects in the eventq a buffer that is
    shorter than what is indicated in the configuration field, it
    might use it immediately and put this dummy value in the
    event field. A well-written driver will never observe this
    situation.

\item When events are dropped, the device may signal this event as
    soon as the drivers makes a buffer available, in order to
    request action from the driver. In this case, of course, this
    event will be reported with the VIRTIO_SCSI_T_EVENTS_MISSED
    flag.
\end{itemize}

  Transport reset
\begin{lstlisting}
	#define VIRTIO_SCSI_T_TRANSPORT_RESET  1

	#define VIRTIO_SCSI_EVT_RESET_HARD         0
	#define VIRTIO_SCSI_EVT_RESET_RESCAN       1
	#define VIRTIO_SCSI_EVT_RESET_REMOVED      2
\end{lstlisting}

  By sending this event, the device signals that a logical unit
  on a target has been reset, including the case of a new device
  appearing or disappearing on the bus.The device fills in all
  fields. The event field is set to
  VIRTIO_SCSI_T_TRANSPORT_RESET. The lun field addresses a
  logical unit in the SCSI host.

  The reason value is one of the three \#define values appearing
  above:

  \begin{itemize}
  \item VIRTIO_SCSI_EVT_RESET_REMOVED (“LUN/target removed”) is used
    if the target or logical unit is no longer able to receive
    commands.

  \item VIRTIO_SCSI_EVT_RESET_HARD (“LUN hard reset”) is used if the
    logical unit has been reset, but is still present.

  \item VIRTIO_SCSI_EVT_RESET_RESCAN (“rescan LUN/target”) is used if
    a target or logical unit has just appeared on the device.
  \end{itemize}

  The “removed” and “rescan” events, when sent for LUN 0, may
  apply to the entire target. After receiving them the driver
  should ask the initiator to rescan the target, in order to
  detect the case when an entire target has appeared or
  disappeared. These two events will never be reported unless the
  VIRTIO_SCSI_F_HOTPLUG feature was negotiated between the device
  and the driver.

  Events will also be reported via sense codes (this obviously
  does not apply to newly appeared buses or targets, since the
  application has never discovered them):

  \begin{itemize}
  \item “LUN/target removed” maps to sense key ILLEGAL REQUEST, asc
    0x25, ascq 0x00 (LOGICAL UNIT NOT SUPPORTED)

  \item “LUN hard reset” maps to sense key UNIT ATTENTION, asc 0x29
    (POWER ON, RESET OR BUS DEVICE RESET OCCURRED)

  \item “rescan LUN/target” maps to sense key UNIT ATTENTION, asc
    0x3f, ascq 0x0e (REPORTED LUNS DATA HAS CHANGED)
  \end{itemize}

  The preferred way to detect transport reset is always to use
  events, because sense codes are only seen by the driver when it
  sends a SCSI command to the logical unit or target. However, in
  case events are dropped, the initiator will still be able to
  synchronize with the actual state of the controller if the
  driver asks the initiator to rescan of the SCSI bus. During the
  rescan, the initiator will be able to observe the above sense
  codes, and it will process them as if it the driver had
  received the equivalent event.

  Asynchronous notification
\begin{lstlisting}
	#define VIRTIO_SCSI_T_ASYNC_NOTIFY     2
\end{lstlisting}

  By sending this event, the device signals that an asynchronous
  event was fired from a physical interface.

  All fields are written by the device. The event field is set to
  VIRTIO_SCSI_T_ASYNC_NOTIFY. The lun field addresses a logical
  unit in the SCSI host. The reason field is a subset of the
  events that the driver has subscribed to via the "Asynchronous
  notification subscription" command.

  When dropped events are reported, the driver should poll for
  asynchronous events manually using SCSI commands.

  LUN parameter change
\begin{lstlisting}
	#define VIRTIO_SCSI_T_PARAM_CHANGE  3
\end{lstlisting}

  By sending this event, the device signals that the configuration parameters
  (for example the capacity) of a logical unit have changed.
  The event field is set to VIRTIO_SCSI_T_PARAM_CHANGE.
  The lun field addresses a logical unit in the SCSI host.

  The same event is also reported as a unit attention condition.
  The reason field contains the additional sense code and additional sense code qualifier,
  respectively in bits 0..7 and 8..15.
  For example, a change in capacity will be reported as asc 0x2a, ascq 0x09
  (CAPACITY DATA HAS CHANGED).

  For MMC devices (inquiry type 5) there would be some overlap between this
  event and the asynchronous notification event.
  For simplicity, as of this version of the specification the host must
  never report this event for MMC devices.

\paragraph{Legacy Interface: Device Operation: eventq}\label{sec:Device Types / SCSI Host Device / Device Operation / Device Operation: eventq / Legacy Interface: Device Operation: eventq}
For legacy devices, the fields in struct virtio_scsi_event are the
native endian of the guest rather than (necessarily) little-endian.

\chapter{Reserved Feature Bits}\label{sec:Reserved Feature Bits}

Currently there are four device-independent feature bits defined:

\begin{description}
  \item[VIRTIO_F_RING_INDIRECT_DESC (28)] Negotiating this feature indicates
  that the driver can use descriptors with the VRING_DESC_F_INDIRECT
  flag set, as described in \ref{sec:Basic Facilities of a Virtio Device / Virtqueues / The Virtqueue Descriptor Table / Indirect Descriptors}~\nameref{sec:Basic Facilities of a Virtio Device / Virtqueues / The Virtqueue Descriptor Table / Indirect Descriptors}.

  \item[VIRTIO_F_RING_EVENT_IDX(29)] This feature enables the used_event
  and the avail_event fields. If set, it indicates that the
  device should ignore the flags field in the available ring
  structure. Instead, the used_event field in this structure is
  used by driver to suppress device interrupts. Further, the
  driver should ignore the flags field in the used ring
  structure. Instead, the avail_event field in this structure is
  used by the device to suppress notifications. If unset, the
  driver should ignore the used_event field; the device should
  ignore the avail_event field; the flags field is used

  \item[VIRTIO_F_VERSION_1(32)] This feature must be offered by any device
  compliant with this specification, and acknowledged by all device
  drivers.
\end{description}

In addition, bit 30 is used by qemu's implementation to check for experimental
early versions of virtio which did not perform correct feature negotiation,
and should not be used.

\section{Legacy Interface: Reserved Feature Bits}\label{sec:Reserved Feature Bits / Legacy Interface: Reserved Feature Bits}

Legacy or transitional devices may offer the following:

\begin{description}
\item[VIRTIO_F_NOTIFY_ON_EMPTY (24)] Negotiating this feature
  indicates that the driver wants an interrupt if the device runs
  out of available descriptors on a virtqueue, even though
  interrupts are suppressed using the VRING_AVAIL_F_NO_INTERRUPT
  flag or the used_event field. An example of this is the
  networking driver: it doesn't need to know every time a packet
  is transmitted, but it does need to free the transmitted
  packets a finite time after they are transmitted. It can avoid
  using a timer if the device interrupts it when all the packets
  are transmitted.

\item[VIRTIO_F_ANY_LAYOUT (27)] This feature indicates that the device
  accepts arbitrary descriptor layouts, as described in Section
  \ref{sec:Basic Facilities of a Virtio Device / Virtqueues / Message Framing / Legacy Interface: Message Framing}~\nameref{sec:Basic Facilities of a Virtio Device / Virtqueues / Message Framing / Legacy Interface: Message Framing}.
\end{description}

\chapter{virtio_ring.h}\label{sec:virtio-ring.h}

This file is also available at the link
\url{\virtiourlbase/listings/virtio_ring.h}.
All definitions in this section are for non-normative reference
only.

\lstinputlisting{virtio-ring.h}

\chapter{Creating New Device Types}\label{sec:Creating New Device Types}

Various considerations are necessary when creating a new device
type.

\section{How Many Virtqueues?}\label{sec:Creating New Device Types / How Many Virtqueues?}

It is possible that a very simple device will operate entirely
through its configuration space, but most will need at least one
virtqueue in which it will place requests. A device with both
input and output (eg. console and network devices described here)
need two queues: one which the driver fills with buffers to
receive input, and one which the driver places buffers to
transmit output.

\section{What Configuration Space Layout?}\label{sec:Creating New Device Types / What Configuration Space Layout?}

Configuration space should only be used for initialization-time
parameters.  It is a limited resource with no synchronization between
writable fields, so for most uses it is better to use a virtqueue to update
configuration information (the network device does this for filtering,
otherwise the table in the config space could potentially be very
large).

Devices must not assume that configuration fields over 32 bits wide
are atomically writable.

\section{What Device Number?}\label{sec:Creating New Device Types / What Device Number?}

Device numbers can be reserved by the OASIS committee: email
virtio-dev@lists.oasis-open.org to secure a unique one.

Meanwhile for experimental drivers, use 65535 and work backwards.

\section{How many MSI-X vectors?  (for PCI)}\label{sec:Creating New Device Types / How many MSI-X vectors?  (for PCI)}

Using the optional MSI-X capability devices can speed up
interrupt processing by removing the need to read ISR Status
register by guest driver (which might be an expensive operation),
reducing interrupt sharing between devices and queues within the
device, and handling interrupts from multiple CPUs. However, some
systems impose a limit (which might be as low as 256) on the
total number of MSI-X vectors that can be allocated to all
devices. Devices and/or drivers should take this into
account, limiting the number of vectors used unless the device is
expected to cause a high volume of interrupts. Devices can
control the number of vectors used by limiting the MSI-X Table
Size or not presenting MSI-X capability in PCI configuration
space. Drivers can control this by mapping events to as small
number of vectors as possible, or disabling MSI-X capability
altogether.

\section{Device Improvements}\label{sec:Creating New Device Types / Device Improvements}

Any change to configuration space, or new virtqueues, or
behavioural changes, should be indicated by negotiation of a new
feature bit. This establishes clarity\footnote{Even if it does mean documenting design or implementation
mistakes!
} and avoids future expansion problems.

Clusters of functionality which are always implemented together
can use a single bit, but if one feature makes sense without the
others they should not be gratuitously grouped together to
conserve feature bits.


