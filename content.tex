\chapter{Basic Facilities of a Virtio Device}\label{sec:Basic Facilities of a Virtio Device}

A virtio device is discovered and identified by a bus-specific method
(see the bus specific sections: \ref{sec:Virtio Transport Options / Virtio Over PCI Bus}~\nameref{sec:Virtio Transport Options / Virtio Over PCI Bus},
\ref{sec:Virtio Transport Options / Virtio Over MMIO}~\nameref{sec:Virtio Transport Options / Virtio Over MMIO} and \ref{sec:Virtio Transport Options / Virtio Over Channel I/O}~\nameref{sec:Virtio Transport Options / Virtio Over Channel I/O}).  Each
device consists of the following parts:

\begin{itemize}
\item Device status field
\item Feature bits
\item Device Configuration space
\item One or more virtqueues
\end{itemize}

\section{\field{Device Status} Field}\label{sec:Basic Facilities of a Virtio Device / Device Status Field}
During device initialization by a driver,
the driver follows the sequence of steps specified in
\ref{sec:General Initialization And Device Operation / Device
Initialization}.

The \field{device status} field provides a simple low-level
indication of the completed steps of this sequence.
It's most useful to imagine it hooked up to traffic
lights on the console indicating the status of each device.  The
following bits are defined:
\begin{description}
\item[ACKNOWLEDGE (1)] Indicates that the guest OS has found the
  device and recognized it as a valid virtio device.

\item[DRIVER (2)] Indicates that the guest OS knows how to drive the
  device.
  \begin{note}
    There could be a significant (or infinite) delay before setting
    this bit.  For example, under Linux, drivers can be loadable modules.
  \end{note}

\item[FEATURES_OK (8)] Indicates that the driver has acknowledged all the
  features it understands, and feature negotiation is complete.

\item[DRIVER_OK (4)] Indicates that the driver is set up and ready to
  drive the device.

\item[DEVICE_NEEDS_RESET (64)] Indicates that the device has experienced
  an error from which it can't recover.

\item[FAILED (128)] Indicates that something went wrong in the guest,
  and it has given up on the device. This could be an internal
  error, or the driver didn't like the device for some reason, or
  even a fatal error during device operation.
\end{description}

\drivernormative{\subsection}{Device Status Field}{Basic Facilities of a Virtio Device / Device Status Field}
The driver MUST update \field{device status},
setting bits to indicate the completed steps of the driver
initialization sequence specified in
\ref{sec:General Initialization And Device Operation / Device
Initialization}.
The driver MUST NOT clear a
\field{device status} bit.  If the driver sets the FAILED bit,
the driver MUST later reset the device before attempting to re-initialize.

The driver SHOULD NOT rely on completion of operations of a
device if DEVICE_NEEDS_RESET is set.
\begin{note}
For example, the driver can't assume requests in flight will be
completed if DEVICE_NEEDS_RESET is set, nor can it assume that
they have not been completed.  A good implementation will try to
recover by issuing a reset.
\end{note}

\devicenormative{\subsection}{Device Status Field}{Basic Facilities of a Virtio Device / Device Status Field}
The device MUST initialize \field{device status} to 0 upon reset.

The device MUST NOT consume buffers or notify the driver before DRIVER_OK.

\label{sec:Basic Facilities of a Virtio Device / Device Status Field / DEVICENEEDSRESET}The device SHOULD set DEVICE_NEEDS_RESET when it enters an error state
that a reset is needed.  If DRIVER_OK is set, after it sets DEVICE_NEEDS_RESET, the device
MUST send a device configuration change notification to the driver.

\section{Feature Bits}\label{sec:Basic Facilities of a Virtio Device / Feature Bits}

Each virtio device offers all the features it understands.  During
device initialization, the driver reads this and tells the device the
subset that it accepts.  The only way to renegotiate is to reset
the device.

This allows for forwards and backwards compatibility: if the device is
enhanced with a new feature bit, older drivers will not write that
feature bit back to the device.  Similarly, if a driver is enhanced with a feature
that the device doesn't support, it see the new feature is not offered.

Feature bits are allocated as follows:

\begin{description}
\item[0 to 23] Feature bits for the specific device type

\item[24 to 32] Feature bits reserved for extensions to the queue and
  feature negotiation mechanisms

\item[33 and above] Feature bits reserved for future extensions.
\end{description}

\begin{note}
For example, feature bit 0 for a network device (i.e.
Device ID 1) indicates that the device supports checksumming of
packets.
\end{note}

In particular, new fields in the device configuration space are
indicated by offering a new feature bit.

\drivernormative{\subsection}{Feature Bits}{Basic Facilities of a Virtio Device / Feature Bits}
The driver MUST NOT accept a feature which the device did not offer,
and MUST NOT accept a feature which requires another feature which was
not accepted.

The driver SHOULD go into backwards compatibility mode
if the device does not offer a feature it understands, otherwise MUST
set the FAILED \field{device status} bit and cease initialization.

\devicenormative{\subsection}{Feature Bits}{Basic Facilities of a Virtio Device / Feature Bits}
The device MUST NOT offer a feature which requires another feature
which was not offered.  The device SHOULD accept any valid subset
of features the driver accepts, otherwise it MUST fail to set the
FEATURES_OK \field{device status} bit when the driver writes it.

\subsection{Legacy Interface: A Note on Feature
Bits}\label{sec:Basic Facilities of a Virtio Device / Feature
Bits / Legacy Interface: A Note on Feature Bits}

Transitional Drivers MUST detect Legacy Devices by detecting that
the feature bit VIRTIO_F_VERSION_1 is not offered.
Transitional devices MUST detect Legacy drivers by detecting that
VIRTIO_F_VERSION_1 has not been acknowledged by the driver.

In this case device is used through the legacy interface.

Legacy interface support is OPTIONAL.
Thus, both transitional and non-transitional devices and
drivers are compliant with this specification.

Requirements pertaining to transitional devices and drivers
is contained in sections named 'Legacy Interface' like this one.

When device is used through the legacy interface, transitional
devices and transitional drivers MUST operate according to the
requirements documented within these legacy interface sections.
Specification text within these sections generally does not apply
to non-transitional devices.

\section{Device Configuration Space}\label{sec:Basic Facilities of a Virtio Device / Device Configuration Space}

Device configuration space is generally used for rarely-changing or
initialization-time parameters.  Where configuration fields are
optional, their existence is indicated by feature bits: Future
versions of this specification will likely extend the device
configuration space by adding extra fields at the tail.

\begin{note}
The device configuration space uses the little-endian format
for multi-byte fields.
\end{note}

Each transport also provides a generation count for the device configuration
space, which will change whenever there is a possibility that two
accesses to the device configuration space can see different versions of that
space.

\drivernormative{\subsection}{Device Configuration Space}{Basic Facilities of a Virtio Device / Device Configuration Space}
Drivers MUST NOT assume reads from
fields greater than 32 bits wide are atomic, nor are reads from
multiple fields: drivers SHOULD read device configuration space fields like so:

\begin{lstlisting}
u32 before, after;
do {
        before = get_config_generation(device);
        // read config entry/entries.
        after = get_config_generation(device);
} while (after != before);
\end{lstlisting}

For optional configuration space fields, the driver MUST check that the
corresponding feature is offered before accessing that part of the configuration
space.
\begin{note}
See section \ref{sec:General Initialization And Device Operation / Device Initialization} for details on feature negotiation.
\end{note}

Drivers MUST
NOT limit structure size and device configuration space size.  Instead,
drivers SHOULD only check that device configuration space is {\em large enough} to
contain the fields necessary for device operation.

\begin{note}
For example, if the specification states that device configuration
space 'includes a single 8-bit field' drivers should understand this to mean that
the device configuration space might also include an arbitrary amount of
tail padding, and accept any device configuration space size equal to or
greater than the specified 8-bit size.
\end{note}

\devicenormative{\subsection}{Device Configuration Space}{Basic Facilities of a Virtio Device / Device Configuration Space}
The device MUST allow reading of any device-specific configuration
field before FEATURES_OK is set by the driver.  This includes fields which are
conditional on feature bits, as long as those feature bits are offered
by the device.

\subsection{Legacy Interface: A Note on Device Configuration Space endian-ness}\label{sec:Basic Facilities of a Virtio Device / Device Configuration Space / Legacy Interface: A Note on Configuration Space endian-ness}

Note that for legacy interfaces, device configuration space is generally the
guest's native endian, rather than PCI's little-endian.
The correct endian-ness is documented for each device.

\subsection{Legacy Interface: Device Configuration Space}\label{sec:Basic Facilities of a Virtio Device / Device Configuration Space / Legacy Interface: Device Configuration Space}

Legacy devices did not have a configuration generation field, thus are
susceptible to race conditions if configuration is updated.  This
affects the block \field{capacity} (see \ref{sec:Device Types /
Block Device / Device configuration layout}) and
network \field{mac} (see \ref{sec:Device Types / Network Device /
Device configuration layout}) fields;
when using the legacy interface, drivers SHOULD
read these fields multiple times until two reads generate a consistent
result.

\section{Virtqueues}\label{sec:Basic Facilities of a Virtio Device / Virtqueues}

The mechanism for bulk data transport on virtio devices is
pretentiously called a virtqueue. Each device can have zero or more
virtqueues\footnote{For example, the simplest network device has one virtqueue for
transmit and one for receive.}.  Each queue has a 16-bit queue size
parameter, which sets the number of entries and implies the total size
of the queue.

Each virtqueue consists of three parts:

\begin{itemize}
\item Descriptor Table
\item Available Ring
\item Used Ring
\end{itemize}

where each part is physically-contiguous in guest memory,
and has different alignment requirements.

The memory aligment and size requirements, in bytes, of each part of the
virtqueue are summarized in the following table:

\begin{tabular}{|l|l|l|}
\hline
Virtqueue Part    & Alignment & Size \\
\hline \hline
Descriptor Table  & 16        & $16 * $(Queue Size) \\
\hline
Available Ring    & 2         & $6 + 2 * $(Queue Size) \\
 \hline
Used Ring         & 4         & $6 + 8 * $(Queue Size) \\
 \hline
\end{tabular}

The Alignment column gives the minimum alignment for each part
of the virtqueue.

The Size column gives the total number of bytes for each
part of the virtqueue.

Queue Size corresponds to the maximum number of buffers in the
virtqueue\footnote{For example, if Queue Size is 4 then at most 4 buffers
can be queued at any given time.}.  Queue Size value is always a
power of 2.  The maximum Queue Size value is 32768.  This value
is specified in a bus-specific way.

When the driver wants to send a buffer to the device, it fills in
a slot in the descriptor table (or chains several together), and
writes the descriptor index into the available ring.  It then
notifies the device. When the device has finished a buffer, it
writes the descriptor index into the used ring, and sends an interrupt.

\drivernormative{\subsection}{Virtqueues}{Basic Facilities of a Virtio Device / Virtqueues}
The driver MUST ensure that the physical address of the first byte
of each virtqueue part is a multiple of the specified alignment value
in the above table.

\subsection{Legacy Interfaces: A Note on Virtqueue Layout}\label{sec:Basic Facilities of a Virtio Device / Virtqueues / Legacy Interfaces: A Note on Virtqueue Layout}

For Legacy Interfaces, several additional
restrictions are placed on the virtqueue layout:

Each virtqueue occupies two or more physically-contiguous pages
(usually defined as 4096 bytes, but depending on the transport;
henceforth referred to as Queue Align)
and consists of three parts:

\begin{tabular}{|l|l|l|}
\hline
Descriptor Table & Available Ring (\ldots padding\ldots) & Used Ring \\
\hline
\end{tabular}

The bus-specific Queue Size field controls the total number of bytes
for the virtqueue.
When using the legacy interface, the transitional
driver MUST retrieve the Queue Size field from the device
and MUST allocate the total number of bytes for the virtqueue
according to the following formula (Queue Align given in qalign and
Queue Size given in qsz):

\begin{lstlisting}
#define ALIGN(x) (((x) + qalign) & ~qalign)
static inline unsigned virtq_size(unsigned int qsz)
{
     return ALIGN(sizeof(struct virtq_desc)*qsz + sizeof(u16)*(3 + qsz))
          + ALIGN(sizeof(u16)*3 + sizeof(struct virtq_used_elem)*qsz);
}
\end{lstlisting}

This wastes some space with padding.
When using the legacy interface, both transitional
devices and drivers MUST use the following virtqueue layout
structure to locate elements of the virtqueue:

\begin{lstlisting}
struct virtq {
        // The actual descriptors (16 bytes each)
        struct virtq_desc desc[ Queue Size ];

        // A ring of available descriptor heads with free-running index.
        struct virtq_avail avail;

        // Padding to the next Queue Align boundary.
        u8 pad[ Padding ];

        // A ring of used descriptor heads with free-running index.
        struct virtq_used used;
};
\end{lstlisting}

\subsection{Legacy Interfaces: A Note on Virtqueue Endianness}\label{sec:Basic Facilities of a Virtio Device / Virtqueues / Legacy Interfaces: A Note on Virtqueue Endianness}

Note that when using the legacy interface, transitional
devices and drivers MUST use the native
endian of the guest as the endian of fields and in the virtqueue.
This is opposed to little-endian for non-legacy interface as
specified by this standard.
It is assumed that the host is already aware of the guest endian.

\subsection{Message Framing}\label{sec:Basic Facilities of a Virtio Device / Virtqueues / Message Framing}
The framing of messages with descriptors is
independent of the contents of the buffers. For example, a network
transmit buffer consists of a 12 byte header followed by the network
packet. This could be most simply placed in the descriptor table as a
12 byte output descriptor followed by a 1514 byte output descriptor,
but it could also consist of a single 1526 byte output descriptor in
the case where the header and packet are adjacent, or even three or
more descriptors (possibly with loss of efficiency in that case).

Note that, some device implementations have large-but-reasonable
restrictions on total descriptor size (such as based on IOV_MAX in the
host OS). This has not been a problem in practice: little sympathy
will be given to drivers which create unreasonably-sized descriptors
such as by dividing a network packet into 1500 single-byte
descriptors!

\devicenormative{\subsubsection}{Message Framing}{Basic Facilities of a Virtio Device / Message Framing}
The device MUST NOT make assumptions about the particular arrangement
of descriptors.  The device MAY have a reasonable limit of descriptors
it will allow in a chain.

\drivernormative{\subsubsection}{Message Framing}{Basic Facilities of a Virtio Device / Message Framing}
The driver MUST place any device-writable descriptor elements after
any device-readable descriptor elements.

The driver SHOULD NOT use an excessive number of descriptors to
describe a buffer.

\subsubsection{Legacy Interface: Message Framing}\label{sec:Basic Facilities of a Virtio Device / Virtqueues / Message Framing / Legacy Interface: Message Framing}

Regrettably, initial driver implementations used simple layouts, and
devices came to rely on it, despite this specification wording.  In
addition, the specification for virtio_blk SCSI commands required
intuiting field lengths from frame boundaries (see
 \ref{sec:Device Types / Block Device / Device Operation / Legacy Interface: Device Operation}~\nameref{sec:Device Types / Block Device / Device Operation / Legacy Interface: Device Operation})

Thus when using the legacy interface, the VIRTIO_F_ANY_LAYOUT
feature indicates to both the device and the driver that no
assumptions were made about framing.  Requirements for
transitional drivers when this is not negotiated are included in
each device section.

\subsection{The Virtqueue Descriptor Table}\label{sec:Basic Facilities of a Virtio Device / Virtqueues / The Virtqueue Descriptor Table}

The descriptor table refers to the buffers the driver is using for
the device. \field{addr} is a physical address, and the buffers
can be chained via \field{next}. Each descriptor describes a
buffer which is read-only for the device (``device-readable'') or write-only for the device (``device-writable''), but a chain of
descriptors can contain both device-readable and device-writable buffers.

The actual contents of the memory offered to the device depends on the
device type.  Most common is to begin the data with a header
(containing little-endian fields) for the device to read, and postfix
it with a status tailer for the device to write.

\begin{lstlisting}
struct virtq_desc {
        /* Address (guest-physical). */
        le64 addr;
        /* Length. */
        le32 len;

/* This marks a buffer as continuing via the next field. */
#define VIRTQ_DESC_F_NEXT   1
/* This marks a buffer as device write-only (otherwise device read-only). */
#define VIRTQ_DESC_F_WRITE     2
/* This means the buffer contains a list of buffer descriptors. */
#define VIRTQ_DESC_F_INDIRECT   4
        /* The flags as indicated above. */
        le16 flags;
        /* Next field if flags & NEXT */
        le16 next;
};
\end{lstlisting}

The number of descriptors in the table is defined by the queue size
for this virtqueue: this is the maximum possible descriptor chain length.

\begin{note}
The legacy \hyperref[intro:Virtio PCI Draft]{[Virtio PCI Draft]}
referred to this structure as vring_desc, and the constants as
VRING_DESC_F_NEXT, etc, but the layout and values were identical.
\end{note}

\devicenormative{\subsubsection}{The Virtqueue Descriptor Table}{Basic Facilities of a Virtio Device / Virtqueues / The Virtqueue Descriptor Table}
A device MUST NOT write to a device-readable buffer, and a device SHOULD NOT
read a device-writable buffer (it MAY do so for debugging or diagnostic
purposes).

\drivernormative{\subsubsection}{The Virtqueue Descriptor Table}{Basic Facilities of a Virtio Device / Virtqueues / The Virtqueue Descriptor Table}
Drivers MUST NOT add a descriptor chain over than $2^{32}$ bytes long in total;
this implies that loops in the descriptor chain are forbidden!

\subsubsection{Indirect Descriptors}\label{sec:Basic Facilities of a Virtio Device / Virtqueues / The Virtqueue Descriptor Table / Indirect Descriptors}

Some devices benefit by concurrently dispatching a large number
of large requests. The VIRTIO_F_INDIRECT_DESC feature allows this (see \ref{sec:virtio-ring.h}~\nameref{sec:virtio-ring.h}). To increase
ring capacity the driver can store a table of indirect
descriptors anywhere in memory, and insert a descriptor in main
virtqueue (with \field{flags}\&VIRTQ_DESC_F_INDIRECT on) that refers to memory buffer
containing this indirect descriptor table; \field{addr} and \field{len}
refer to the indirect table address and length in bytes,
respectively.

The indirect table layout structure looks like this
(\field{len} is the length of the descriptor that refers to this table,
which is a variable, so this code won't compile):

\begin{lstlisting}
struct indirect_descriptor_table {
        /* The actual descriptors (16 bytes each) */
        struct virtq_desc desc[len / 16];
};
\end{lstlisting}

The first indirect descriptor is located at start of the indirect
descriptor table (index 0), additional indirect descriptors are
chained by \field{next}. An indirect descriptor without a valid \field{next}
(with \field{flags}\&VIRTQ_DESC_F_NEXT off) signals the end of the descriptor.
A single indirect descriptor
table can include both device-readable and device-writable descriptors.

\drivernormative{\paragraph}{Indirect Descriptors}{Basic Facilities of a Virtio Device / Virtqueues / The Virtqueue Descriptor Table / Indirect Descriptors}
The driver MUST NOT set the VIRTQ_DESC_F_INDIRECT flag unless the
VIRTIO_F_INDIRECT_DESC feature was negotiated.   The driver MUST NOT
set the VIRTQ_DESC_F_INDIRECT flag within an indirect descriptor (ie. only
one table per descriptor).

A driver MUST NOT create a descriptor chain longer than the Queue Size of
the device.

A driver MUST NOT set both VIRTQ_DESC_F_INDIRECT and VIRTQ_DESC_F_NEXT
in \field{flags}.

\devicenormative{\paragraph}{Indirect Descriptors}{Basic Facilities of a Virtio Device / Virtqueues / The Virtqueue Descriptor Table / Indirect Descriptors}
The device MUST ignore the write-only flag (\field{flags}\&VIRTQ_DESC_F_WRITE) in the descriptor that refers to an indirect table.

The device MUST handle the case of zero or more normal chained
descriptors followed by a single descriptor with \field{flags}\&VIRTQ_DESC_F_INDIRECT.

\begin{note}
While unusual (most implementations either create a chain solely using
non-indirect descriptors, or use a single indirect element), such a
layout is valid.
\end{note}

\subsection{The Virtqueue Available Ring}\label{sec:Basic Facilities of a Virtio Device / Virtqueues / The Virtqueue Available Ring}

\begin{lstlisting}
struct virtq_avail {
#define VIRTQ_AVAIL_F_NO_INTERRUPT      1
        le16 flags;
        le16 idx;
        le16 ring[ /* Queue Size */ ];
        le16 used_event; /* Only if VIRTIO_F_EVENT_IDX */
};
\end{lstlisting}

The driver uses the available ring to offer buffers to the
device: each ring entry refers to the head of a descriptor chain.  It is only
written by the driver and read by the device.

\field{idx} field indicates where the driver would put the next descriptor
entry in the ring (modulo the queue size). This starts at 0, and increases.

\begin{note}
The legacy \hyperref[intro:Virtio PCI Draft]{[Virtio PCI Draft]}
referred to this structure as vring_avail, and the constant as
VRING_AVAIL_F_NO_INTERRUPT, but the layout and value were identical.
\end{note}

\subsection{Virtqueue Interrupt Suppression}\label{sec:Basic Facilities of a Virtio Device / Virtqueues / Virtqueue Interrupt Suppression}

If the VIRTIO_F_EVENT_IDX feature bit is not negotiated,
the \field{flags} field in the available ring offers a crude mechanism for the driver to inform
the device that it doesn't want interrupts when buffers are used.  Otherwise
\field{used_event} is a more performant alterative where the driver
specifies how far the device can progress before interrupting.

Neither of these interrupt suppression methods are reliable, as they
are not synchronized with the device, but they serve as
useful optimizations.

\drivernormative{\subsubsection}{Virtqueue Interrupt Suppression}{Basic Facilities of a Virtio Device / Virtqueues / Virtqueue Interrupt Suppression}
If the VIRTIO_F_EVENT_IDX feature bit is not negotiated:
\begin{itemize}
\item The driver MUST set \field{flags} to 0 or 1.
\item The driver MAY set \field{flags} to 1 to advise
the device that interrupts are not needed.
\end{itemize}

Otherwise, if the VIRTIO_F_EVENT_IDX feature bit is negotiated:
\begin{itemize}
\item The driver MUST set \field{flags} to 0.
\item The driver MAY use \field{used_event} to advise the device that interrupts are unnecessary until the device writes entry with an index specified by \field{used_event} into the used ring (equivalently, until \field{idx} in the
used ring will reach the value \field{used_event} + 1).
\end{itemize}

The driver MUST handle spurious interrupts from the device.

\devicenormative{\subsubsection}{Virtqueue Interrupt Suppression}{Basic Facilities of a Virtio Device / Virtqueues / Virtqueue Interrupt Suppression}

If the VIRTIO_F_EVENT_IDX feature bit is not negotiated:
\begin{itemize}
\item The device MUST ignore the \field{used_event} value.
\item After the device writes a descriptor index into the used ring:
  \begin{itemize}
  \item If \field{flags} is 1, the device SHOULD NOT send an interrupt.
  \item If \field{flags} is 0, the device MUST send an interrupt.
  \end{itemize}
\end{itemize}

Otherwise, if the VIRTIO_F_EVENT_IDX feature bit is negotiated:
\begin{itemize}
\item The device MUST ignore the lower bit of \field{flags}.
\item After the device writes a descriptor index into the used ring:
  \begin{itemize}
  \item If the \field{idx} field in the used ring (which determined
    where that descriptor index was placed) was equal to
    \field{used_event}, the device MUST send an interrupt.
  \item Otherwise the device SHOULD NOT send an interrupt.
  \end{itemize}
\end{itemize}

\begin{note}
For example, if \field{used_event} is 0, then a device using
  VIRTIO_F_EVENT_IDX would interrupt after the first buffer is
  used (and again after the 65536th buffer, etc).
\end{note}

\subsection{The Virtqueue Used Ring}\label{sec:Basic Facilities of a Virtio Device / Virtqueues / The Virtqueue Used Ring}

\begin{lstlisting}
struct virtq_used {
#define VIRTQ_USED_F_NO_NOTIFY  1
        le16 flags;
        le16 idx;
        struct virtq_used_elem ring[ /* Queue Size */];
        le16 avail_event; /* Only if VIRTIO_F_EVENT_IDX */
};

/* le32 is used here for ids for padding reasons. */
struct virtq_used_elem {
        /* Index of start of used descriptor chain. */
        le32 id;
        /* Total length of the descriptor chain which was used (written to) */
        le32 len;
};
\end{lstlisting}

The used ring is where the device returns buffers once it is done with
them: it is only written to by the device, and read by the driver.

Each entry in the ring is a pair: \field{id} indicates the head entry of the
descriptor chain describing the buffer (this matches an entry
placed in the available ring by the guest earlier), and \field{len} the total
of bytes written into the buffer.

\begin{note}
\field{len} is particularly useful
for drivers using untrusted buffers: if a driver does not know exactly
how much has been written by the device, the driver would have to zero
the buffer in advance to ensure no data leakage occurs.

For example, a network driver may hand a received buffer directly to
an unprivileged userspace application.  If the network device has not
overwritten the bytes which were in that buffer, this could leak the
contents of freed memory from other processes to the application.
\end{note}

\field{idx} field indicates where the driver would put the next descriptor
entry in the ring (modulo the queue size). This starts at 0, and increases.

\begin{note}
The legacy \hyperref[intro:Virtio PCI Draft]{[Virtio PCI Draft]}
referred to these structures as vring_used and vring_used_elem, and
the constant as VRING_USED_F_NO_NOTIFY, but the layout and value were
identical.
\end{note}

\subsubsection{Legacy Interface: The Virtqueue Used
Ring}\label{sec:Basic Facilities of a Virtio Device / Virtqueues
/ The Virtqueue Used Ring/ Legacy Interface: The Virtqueue Used
Ring}

Historically, many drivers ignored the \field{len} value, as a
result, many devices set \field{len} incorrectly.  Thus, when
using the legacy interface, it is generally a good idea to ignore
the \field{len} value in used ring entries if possible.  Specific
known issues are listed per device type.

\devicenormative{\subsubsection}{The Virtqueue Used Ring}{Basic Facilities of a Virtio Device / Virtqueues / The Virtqueue Used Ring}

The device MUST set \field{len} prior to updating the used \field{idx}.

The device MUST write at least \field{len} bytes to descriptor,
beginning at the first device-writable buffer,
prior to updating the used \field{idx}.

The device MAY write more than \field{len} bytes to descriptor.

\begin{note}
There are potential error cases where a device might not know what
parts of the buffers have been written.  This is why \field{len} is
permitted to be an underestimate: that's preferable to the driver believing
that uninitialized memory has been overwritten when it has not.
\end{note}

\drivernormative{\subsubsection}{The Virtqueue Used Ring}{Basic Facilities of a Virtio Device / Virtqueues / The Virtqueue Used Ring}

The driver MUST NOT make assumptions about data in device-writable buffers
beyond the first \field{len} bytes, and SHOULD ignore this data.

\subsection{Virtqueue Notification Suppression}\label{sec:Basic Facilities of a Virtio Device / Virtqueues / Virtqueue Notification Suppression}

The device can suppress notifications in a manner analogous to the way
drivers can suppress interrupts as detailed in section \ref{sec:Basic Facilities of a Virtio Device / Virtqueues / Virtqueue Interrupt Suppression}.
The device manipulates \field{flags} or \field{avail_event} in the used ring the
same way the driver manipulates \field{flags} or \field{used_event} in the available ring.

\drivernormative{\subsubsection}{Virtqueue Notification Suppression}{Basic Facilities of a Virtio Device / Virtqueues / Virtqueue Notification Suppression}

The driver MUST initialize \field{flags} in the used ring to 0 when
allocating the used ring.

If the VIRTIO_F_EVENT_IDX feature bit is not negotiated:
\begin{itemize}
\item The driver MUST ignore the \field{avail_event} value.
\item After the driver writes a descriptor index into the available ring:
  \begin{itemize}
        \item If \field{flags} is 1, the driver SHOULD NOT send a notification.
        \item If \field{flags} is 0, the driver MUST send a notification.
  \end{itemize}
\end{itemize}

Otherwise, if the VIRTIO_F_EVENT_IDX feature bit is negotiated:
\begin{itemize}
\item The driver MUST ignore the lower bit of \field{flags}.
\item After the driver writes a descriptor index into the available ring:
  \begin{itemize}
        \item If the \field{idx} field in the available ring (which determined
          where that descriptor index was placed) was equal to
          \field{avail_event}, the driver MUST send a notification.
        \item Otherwise the driver SHOULD NOT send a notification.
  \end{itemize}
\end{itemize}

\devicenormative{\subsubsection}{Virtqueue Notification Suppression}{Basic Facilities of a Virtio Device / Virtqueues / Virtqueue Notification Suppression}
If the VIRTIO_F_EVENT_IDX feature bit is not negotiated:
\begin{itemize}
\item The device MUST set \field{flags} to 0 or 1.
\item The device MAY set \field{flags} to 1 to advise
the driver that notifications are not needed.
\end{itemize}

Otherwise, if the VIRTIO_F_EVENT_IDX feature bit is negotiated:
\begin{itemize}
\item The device MUST set \field{flags} to 0.
\item The device MAY use \field{avail_event} to advise the driver that notifications are unnecessary until the driver writes entry with an index specified by \field{avail_event} into the available ring (equivalently, until \field{idx} in the
available ring will reach the value \field{avail_event} + 1).
\end{itemize}

The device MUST handle spurious notifications from the driver.

\subsection{Helpers for Operating Virtqueues}\label{sec:Basic Facilities of a Virtio Device / Virtqueues / Helpers for Operating Virtqueues}

The Linux Kernel Source code contains the definitions above and
helper routines in a more usable form, in
include/uapi/linux/virtio_ring.h. This was explicitly licensed by IBM
and Red Hat under the (3-clause) BSD license so that it can be
freely used by all other projects, and is reproduced (with slight
variation to remove Linux assumptions) in \ref{sec:virtio-ring.h}~\nameref{sec:virtio-ring.h}.

\chapter{General Initialization And Device Operation}\label{sec:General Initialization And Device Operation}

We start with an overview of device initialization, then expand on the
details of the device and how each step is preformed.  This section
is best read along with the bus-specific section which describes
how to communicate with the specific device.

\section{Device Initialization}\label{sec:General Initialization And Device Operation / Device Initialization}

\drivernormative{\subsection}{Device Initialization}{General Initialization And Device Operation / Device Initialization}
The driver MUST follow this sequence to initialize a device:

\begin{enumerate}
\item Reset the device.

\item Set the ACKNOWLEDGE status bit: the guest OS has notice the device.

\item Set the DRIVER status bit: the guest OS knows how to drive the device.

\item\label{itm:General Initialization And Device Operation /
Device Initialization / Read feature bits} Read device feature bits, and write the subset of feature bits
   understood by the OS and driver to the device.  During this step the
   driver MAY read (but MUST NOT write) the device-specific configuration fields to check that it can support the device before accepting it.

\item\label{itm:General Initialization And Device Operation / Device Initialization / Set FEATURES-OK} Set the FEATURES_OK status bit.  The driver MUST NOT accept
   new feature bits after this step.

\item\label{itm:General Initialization And Device Operation / Device Initialization / Re-read FEATURES-OK} Re-read \field{device status} to ensure the FEATURES_OK bit is still
   set: otherwise, the device does not support our subset of features
   and the device is unusable.

\item\label{itm:General Initialization And Device Operation / Device Initialization / Device-specific Setup} Perform device-specific setup, including discovery of virtqueues for the
   device, optional per-bus setup, reading and possibly writing the
   device's virtio configuration space, and population of virtqueues.

\item\label{itm:General Initialization And Device Operation / Device Initialization / Set DRIVER-OK} Set the DRIVER_OK status bit.  At this point the device is
   ``live''.
\end{enumerate}

If any of these steps go irrecoverably wrong, the driver SHOULD
set the FAILED status bit to indicate that it has given up on the
device (it can reset the device later to restart if desired).  The
driver MUST NOT continue initialization in that case.

The driver MUST NOT notify the device before setting DRIVER_OK.

\subsection{Legacy Interface: Device Initialization}\label{sec:General Initialization And Device Operation / Device Initialization / Legacy Interface: Device Initialization}
Legacy devices did not support the FEATURES_OK status bit, and thus did
not have a graceful way for the device to indicate unsupported feature
combinations.  They also did not provide a clear mechanism to end
feature negotiation, which meant that devices finalized features on
first-use, and no features could be introduced which radically changed
the initial operation of the device.

Legacy driver implementations often used the device before setting the
DRIVER_OK bit, and sometimes even before writing the feature bits
to the device.

The result was the steps \ref{itm:General Initialization And
Device Operation / Device Initialization / Set FEATURES-OK} and
\ref{itm:General Initialization And Device Operation / Device
Initialization / Re-read FEATURES-OK} were omitted, and steps
\ref{itm:General Initialization And Device Operation /
Device Initialization / Read feature bits},
\ref{itm:General Initialization And Device Operation / Device Initialization / Device-specific Setup} and \ref{itm:General Initialization And Device Operation / Device Initialization / Set DRIVER-OK}
were conflated.

Therefore, when using the legacy interface:
\begin{itemize}
\item
The transitional driver MUST execute the initialization
sequence as described in \ref{sec:General Initialization And Device
Operation / Device Initialization}
but omitting the steps \ref{itm:General Initialization And Device
Operation / Device Initialization / Set FEATURES-OK} and
\ref{itm:General Initialization And Device Operation / Device
Initialization / Re-read FEATURES-OK}.

\item
The transitional device MUST support the driver
writing device configuration fields
before the step \ref{itm:General Initialization And Device Operation /
Device Initialization / Read feature bits}.
\item
The transitional device MUST support the driver
using the device before the step \ref{itm:General Initialization
And Device Operation / Device Initialization / Set DRIVER-OK}.
\end{itemize}

\section{Device Operation}\label{sec:General Initialization And Device Operation / Device Operation}

There are two parts to device operation: supplying new buffers to
the device, and processing used buffers from the device.

\begin{note} As an
example, the simplest virtio network device has two virtqueues: the
transmit virtqueue and the receive virtqueue. The driver adds
outgoing (device-readable) packets to the transmit virtqueue, and then
frees them after they are used. Similarly, incoming (device-writable)
buffers are added to the receive virtqueue, and processed after
they are used.
\end{note}

\subsection{Supplying Buffers to The Device}\label{sec:General Initialization And Device Operation / Device Operation / Supplying Buffers to The Device}

The driver offers buffers to one of the device's virtqueues as follows:

\begin{enumerate}
\item\label{itm:General Initialization And Device Operation / Device Operation / Supplying Buffers to The Device / Place Buffers} The driver places the buffer into free descriptor(s) in the
   descriptor table, chaining as necessary (see \ref{sec:Basic Facilities of a Virtio Device / Virtqueues / The Virtqueue Descriptor Table}~\nameref{sec:Basic Facilities of a Virtio Device / Virtqueues / The Virtqueue Descriptor Table}).

\item\label{itm:General Initialization And Device Operation / Device Operation / Supplying Buffers to The Device / Place Index} The driver places the index of the head of the descriptor chain
   into the next ring entry of the available ring.

\item Steps \ref{itm:General Initialization And Device Operation / Device Operation / Supplying Buffers to The Device / Place Buffers} and \ref{itm:General Initialization And Device Operation / Device Operation / Supplying Buffers to The Device / Place Index} MAY be performed repeatedly if batching
  is possible.

\item The driver performs suitable a memory barrier to ensure the device sees
  the updated descriptor table and available ring before the next
  step.

\item The available \field{idx} is increased by the number of
  descriptor chain heads added to the available ring.

\item The driver performs a suitable memory barrier to ensure that it updates
  the \field{idx} field before checking for notification suppression.

\item If notifications are not suppressed, the driver notifies the device
    of the new available buffers.
\end{enumerate}

Note that the above code does not take precautions against the
available ring buffer wrapping around: this is not possible since
the ring buffer is the same size as the descriptor table, so step
(1) will prevent such a condition.

In addition, the maximum queue size is 32768 (the highest power
of 2 which fits in 16 bits), so the 16-bit \field{idx} value can always
distinguish between a full and empty buffer.

What follows is the requirements of each stage in more detail.

\subsubsection{Placing Buffers Into The Descriptor Table}\label{sec:General Initialization And Device Operation / Device Operation / Supplying Buffers to The Device / Placing Buffers Into The Descriptor Table}

A buffer consists of zero or more device-readable physically-contiguous
elements followed by zero or more physically-contiguous
device-writable elements (each has at least one element). This
algorithm maps it into the descriptor table to form a descriptor
chain:

for each buffer element, b:

\begin{enumerate}
\item Get the next free descriptor table entry, d
\item Set \field{d.addr} to the physical address of the start of b
\item Set \field{d.len} to the length of b.
\item If b is device-writable, set \field{d.flags} to VIRTQ_DESC_F_WRITE,
    otherwise 0.
\item If there is a buffer element after this:
    \begin{enumerate}
    \item Set \field{d.next} to the index of the next free descriptor
      element.
    \item Set the VIRTQ_DESC_F_NEXT bit in \field{d.flags}.
    \end{enumerate}
\end{enumerate}

In practice, \field{d.next} is usually used to chain free
descriptors, and a separate count kept to check there are enough
free descriptors before beginning the mappings.

\subsubsection{Updating The Available Ring}\label{sec:General Initialization And Device Operation / Device Operation / Supplying Buffers to The Device / Updating The Available Ring}

The descriptor chain head is the first d in the algorithm
above, ie. the index of the descriptor table entry referring to the first
part of the buffer.  A naive driver implementation MAY do the following (with the
appropriate conversion to-and-from little-endian assumed):

\begin{lstlisting}
avail->ring[avail->idx % qsz] = head;
\end{lstlisting}

However, in general the driver MAY add many descriptor chains before it updates
\field{idx} (at which point they become visible to the
device), so it is common to keep a counter of how many the driver has added:

\begin{lstlisting}
avail->ring[(avail->idx + added++) % qsz] = head;
\end{lstlisting}

\subsubsection{Updating \field{idx}}\label{sec:General Initialization And Device Operation / Device Operation / Supplying Buffers to The Device / Updating idx}

\field{idx} always increments, and wraps naturally at
65536:

\begin{lstlisting}
avail->idx += added;
\end{lstlisting}

Once available \field{idx} is updated by the driver, this exposes the
descriptor and its contents.  The device MAY
access the descriptor chains the driver created and the
memory they refer to immediately.

\drivernormative{\paragraph}{Updating idx}{General Initialization And Device Operation / Device Operation / Supplying Buffers to The Device / Updating idx}
The driver MUST perform a suitable memory barrier before the \field{idx} update, to ensure the
device sees the most up-to-date copy.

\subsubsection{Notifying The Device}\label{sec:General Initialization And Device Operation / Device Operation / Supplying Buffers to The Device / Notifying The Device}

The actual method of device notification is bus-specific, but generally
it can be expensive.  So the device MAY suppress such notifications if it
doesn't need them, as detailed in section \ref{sec:Basic Facilities of a Virtio Device / Virtqueues / Virtqueue Notification Suppression}.

The driver has to be careful to expose the new \field{idx}
value before checking if notifications are suppressed.

\drivernormative{\paragraph}{Notifying The Device}{General Initialization And Device Operation / Device Operation / Supplying Buffers to The Device / Notifying The Device}
The driver MUST perform a suitable memory barrier before reading \field{flags} or
\field{avail_event}, to avoid missing a notification.

\subsection{Receiving Used Buffers From The Device}\label{sec:General Initialization And Device Operation / Device Operation / Receiving Used Buffers From The Device}

Once the device has used buffers referred to by a descriptor (read from or written to them, or
parts of both, depending on the nature of the virtqueue and the
device), it interrupts the driver as detailed in section \ref{sec:Basic Facilities of a Virtio Device / Virtqueues / Virtqueue Interrupt Suppression}.

\begin{note}
For optimal performance, a driver MAY disable interrupts while processing
the used ring, but beware the problem of missing interrupts between
emptying the ring and reenabling interrupts.  This is usually handled by
re-checking for more used buffers after interrups are re-enabled:

\begin{lstlisting}
virtq_disable_interrupts(vq);

for (;;) {
        if (vq->last_seen_used != le16_to_cpu(virtq->used.idx)) {
                virtq_enable_interrupts(vq);
                mb();

                if (vq->last_seen_used != le16_to_cpu(virtq->used.idx))
                        break;

                virtq_disable_interrupts(vq);
        }

        struct virtq_used_elem *e = virtq.used->ring[vq->last_seen_used%vsz];
        process_buffer(e);
        vq->last_seen_used++;
}
\end{lstlisting}
\end{note}

\subsection{Notification of Device Configuration Changes}\label{sec:General Initialization And Device Operation / Device Operation / Notification of Device Configuration Changes}

For devices where the device-specific configuration information can be changed, an
interrupt is delivered when a device-specific configuration change occurs.

In addition, this interrupt is triggered by the device setting
DEVICE_NEEDS_RESET (see \ref{sec:Basic Facilities of a Virtio Device / Device Status Field / DEVICENEEDSRESET}).

\section{Device Cleanup}\label{sec:General Initialization And Device Operation / Device Cleanup}

Once the driver has set the DRIVER_OK status bit, all the configured
virtqueue of the device are considered live.  None of the virtqueues
of a device are live once the device has been reset.

\drivernormative{\subsection}{Device Cleanup}{General Initialization And Device Operation / Device Cleanup}

A driver MUST NOT alter descriptor table entries which have been
exposed in the available ring (and not marked consumed by the device
in the used ring) of a live virtqueue.

A driver MUST NOT decrement the available \field{idx} on a live virtqueue (ie.
there is no way to ``unexpose'' buffers).

Thus a driver MUST ensure a virtqueue isn't live (by device reset) before removing exposed buffers.

\chapter{Virtio Transport Options}\label{sec:Virtio Transport Options}

Virtio can use various different buses, thus the standard is split
into virtio general and bus-specific sections.

\section{Virtio Over PCI Bus}\label{sec:Virtio Transport Options / Virtio Over PCI Bus}

Virtio devices are commonly implemented as PCI devices.

A Virtio device can be implemented as any kind of PCI device:
a Conventional PCI device or a PCI Express
device.  To assure designs meet the latest level
requirements, see 
the PCI-SIG home page at \url{http://www.pcisig.com} for any
approved changes.

\devicenormative{\subsection}{Virtio Over PCI Bus}{Virtio Transport Options / Virtio Over PCI Bus}
A Virtio device using Virtio Over PCI Bus MUST expose to
guest an interface that meets the specification requirements of
the appropriate PCI specification: \hyperref[intro:PCI]{[PCI]}
and \hyperref[intro:PCIe]{[PCIe]}
respectively. 

\subsection{PCI Device Discovery}\label{sec:Virtio Transport Options / Virtio Over PCI Bus / PCI Device Discovery}

Any PCI device with PCI Vendor ID 0x1AF4, and PCI Device ID 0x1000 through
0x107F inclusive is a virtio device. The actual value within this range
indicates which virtio device is supported by the device.
The PCI Device ID is calculated by adding 0x1040 to the Virtio Device ID,
as indicated in section \ref{sec:Device Types}.
Additionally, devices MAY utilize a Transitional PCI Device ID range,
0x1000 to 0x103F depending on the device type.

\devicenormative{\subsubsection}{PCI Device Discovery}{Virtio Transport Options / Virtio Over PCI Bus / PCI Device Discovery}

Devices MUST have the PCI Vendor ID 0x1AF4.
Devices MUST either have the PCI Device ID calculated by adding 0x1040
to the Virtio Device ID, as indicated in section \ref{sec:Device
Types} or have the Transitional PCI Device ID depending on the device type,
as follows:

\begin{tabular}{|l|c|}
\hline
Transitional PCI Device ID  &  Virtio Device    \\
\hline \hline
0x1000      &   network card     \\
\hline
0x1001     &   block device     \\
\hline
0x1002     & memory ballooning (traditional)  \\
\hline
0x1003     &      console       \\
\hline
0x1004     &     SCSI host      \\
\hline
0x1005     &  entropy source    \\
\hline
0x1009     &   9P transport     \\
\hline
\end{tabular}

For example, the network card device with the Virtio Device ID 1
has the PCI Device ID 0x1041 or the Transitional PCI Device ID 0x1000.

The PCI Subsystem Vendor ID and the PCI Subsystem Device ID MAY reflect
the PCI Vendor and Device ID of the environment (for informational purposes by the driver).

Non-transitional devices SHOULD have a PCI Device ID in the range
0x1040 to 0x107f.
Non-transitional devices SHOULD have a PCI Revision ID of 1 or higher.
Non-transitional devices SHOULD have a PCI Subsystem Device ID of 0x40 or higher.

This is to reduce the chance of a legacy driver attempting
to drive the device.

\drivernormative{\subsubsection}{PCI Device Discovery}{Virtio Transport Options / Virtio Over PCI Bus / PCI Device Discovery}
Drivers MUST match devices with the PCI Vendor ID 0x1AF4 and
the PCI Device ID in the range 0x1040 to 0x107f,
calculated by adding 0x1040 to the Virtio Device ID,
as indicated in section \ref{sec:Device Types}.
Drivers for device types listed in section \ref{sec:Virtio
Transport Options / Virtio Over PCI Bus / PCI Device Discovery}
MUST match devices with the PCI Vendor ID 0x1AF4 and
the Transitional PCI Device ID indicated in section
 \ref{sec:Virtio
Transport Options / Virtio Over PCI Bus / PCI Device Discovery}.

Drivers MUST match any PCI Revision ID value.
Drivers MAY match any PCI Subsystem Vendor ID and any
PCI Subsystem Device ID value.

\subsubsection{Legacy Interfaces: A Note on PCI Device Discovery}\label{sec:Virtio Transport Options / Virtio Over PCI Bus / PCI Device Discovery / Legacy Interfaces: A Note on PCI Device Discovery}
Transitional devices MUST have a PCI Revision ID of 0.
Transitional devices MUST have the PCI Subsystem Device ID
matching the Virtio Device ID, as indicated in section \ref{sec:Device Types}.
Transitional devices MUST have the Transitional PCI Device ID in
the range 0x1000 to 0x103f.

This is to match legacy drivers.

\subsection{PCI Device Layout}\label{sec:Virtio Transport Options / Virtio Over PCI Bus / PCI Device Layout}

The device is configured via I/O and/or memory regions (though see
\ref{sec:Virtio Transport Options / Virtio Over PCI Bus / PCI Device Layout / PCI configuration access capability}
for access via the PCI configuration space), as specified by Virtio
Structure PCI Capabilities.

Fields of different sizes are present in the device
configuration regions.
All 64-bit, 32-bit and 16-bit fields are little-endian.
64-bit fields are to be treated as two 32-bit fields,
with low 32 bit part followed by the high 32 bit part.

\drivernormative{\subsubsection}{PCI Device Layout}{Virtio Transport Options / Virtio Over PCI Bus / PCI Device Layout}

For device configuration access, the driver MUST use 8-bit wide
accesses for 8-bit wide fields, 16-bit wide and aligned accesses
for 16-bit wide fields and 32-bit wide and aligned accesses for
32-bit and 64-bit wide fields. For 64-bit fields, the driver MAY
access each of the high and low 32-bit parts of the field
independently.

\devicenormative{\subsubsection}{PCI Device Layout}{Virtio Transport Options / Virtio Over PCI Bus / PCI Device Layout}

For 64-bit device configuration fields, the device MUST allow driver
independent access to high and low 32-bit parts of the field.

\subsection{Virtio Structure PCI Capabilities}\label{sec:Virtio Transport Options / Virtio Over PCI Bus / Virtio Structure PCI Capabilities}

The virtio device configuration layout includes several structures:
\begin{itemize}
\item Common configuration
\item Notifications
\item ISR Status
\item Device-specific configuration (optional)
\end{itemize}

Each structure can be mapped by a Base Address register (BAR) belonging to
the function, or accessed via the special VIRTIO_PCI_CAP_PCI_CFG field in the PCI configuration space.

The location of each structure is specified using a vendor-specific PCI capability located
on the capability list in PCI configuration space of the device.
This virtio structure capability uses little-endian format; all fields are
read-only for the driver unless stated otherwise:

\begin{lstlisting}
struct virtio_pci_cap {
        u8 cap_vndr;    /* Generic PCI field: PCI_CAP_ID_VNDR */
        u8 cap_next;    /* Generic PCI field: next ptr. */
        u8 cap_len;     /* Generic PCI field: capability length */
        u8 cfg_type;    /* Identifies the structure. */
        u8 bar;         /* Where to find it. */
        u8 padding[3];  /* Pad to full dword. */
        le32 offset;    /* Offset within bar. */
        le32 length;    /* Length of the structure, in bytes. */
};
\end{lstlisting}

This structure can be followed by extra data, depending on
\field{cfg_type}, as documented below.

The fields are interpreted as follows:

\begin{description}
\item[\field{cap_vndr}]
        0x09; Identifies a vendor-specific capability.

\item[\field{cap_next}]
        Link to next capability in the capability list in the PCI configuration space.

\item[\field{cap_len}]
        Length of this capability structure, including the whole of
        struct virtio_pci_cap, and extra data if any.
        This length MAY include padding, or fields unused by the driver.

\item[\field{cfg_type}]
        identifies the structure, according to the following table:

\begin{lstlisting}
/* Common configuration */
#define VIRTIO_PCI_CAP_COMMON_CFG        1
/* Notifications */
#define VIRTIO_PCI_CAP_NOTIFY_CFG        2
/* ISR Status */
#define VIRTIO_PCI_CAP_ISR_CFG           3
/* Device specific configuration */
#define VIRTIO_PCI_CAP_DEVICE_CFG        4
/* PCI configuration access */
#define VIRTIO_PCI_CAP_PCI_CFG           5
\end{lstlisting}

        Any other value is reserved for future use.

        Each structure is detailed individually below.

        The device MAY offer more than one structure of any type - this makes it
        possible for the device to expose multiple interfaces to drivers.  The order of
        the capabilities in the capability list specifies the order of preference
        suggested by the device.
        \begin{note}
          For example, on some hypervisors, notifications using IO accesses are
        faster than memory accesses. In this case, the device would expose two
        capabilities with \field{cfg_type} set to VIRTIO_PCI_CAP_NOTIFY_CFG:
        the first one addressing an I/O BAR, the second one addressing a memory BAR.
        In this example, the driver would use the I/O BAR if I/O resources are available, and fall back on
        memory BAR when I/O resources are unavailable.
        \end{note}

\item[\field{bar}]
        values 0x0 to 0x5 specify a Base Address register (BAR) belonging to
        the function located beginning at 10h in PCI Configuration Space
        and used to map the structure into Memory or I/O Space.
        The BAR is permitted to be either 32-bit or 64-bit, it can map Memory Space
        or I/O Space.

        Any other value is reserved for future use.

\item[\field{offset}]
        indicates where the structure begins relative to the base address associated
        with the BAR.  The alignment requirements of \field{offset} are indicated
        in each structure-specific section below.

\item[\field{length}]
        indicates the length of the structure.

        \field{length} MAY include padding, or fields unused by the driver, or
        future extensions.

        \begin{note}
        For example, a future device might present a large structure size of several
        MBytes.
        As current devices never utilize structures larger than 4KBytes in size,
        driver MAY limit the mapped structure size to e.g.
        4KBytes (thus ignoring parts of structure after the first
        4KBytes) to allow forward compatibility with such devices without loss of
        functionality and without wasting resources.
        \end{note}
\end{description}

\drivernormative{\subsubsection}{Virtio Structure PCI Capabilities}{Virtio Transport Options / Virtio Over PCI Bus / Virtio Structure PCI Capabilities}

The driver MUST ignore any vendor-specific capability structure which has
a reserved \field{cfg_type} value.

The driver SHOULD use the first instance of each virtio structure type they can
support.

The driver MUST accept a \field{cap_len} value which is larger than specified here.

The driver MUST ignore any vendor-specific capability structure which has
a reserved \field{bar} value.

        The drivers SHOULD only map part of configuration structure
        large enough for device operation.  The drivers MUST handle
        an unexpectedly large \field{length}, but MAY check that \field{length}
        is large enough for device operation.

The driver MUST NOT write into any field of the capability structure,
with the exception of those with \field{cap_type} VIRTIO_PCI_CAP_PCI_CFG as
detailed in \ref{drivernormative:Virtio Transport Options / Virtio Over PCI Bus / PCI Device Layout / PCI configuration access capability}.

\devicenormative{\subsubsection}{Virtio Structure PCI Capabilities}{Virtio Transport Options / Virtio Over PCI Bus / Virtio Structure PCI Capabilities}

The device MUST include any extra data (from the beginning of the \field{cap_vndr} field
through end of the extra data fields if any) in \field{cap_len}.
The device MAY append extra data
or padding to any structure beyond that.

If the device presents multiple structures of the same type, it SHOULD order
them from optimal (first) to least-optimal (last).

\subsubsection{Common configuration structure layout}\label{sec:Virtio Transport Options / Virtio Over PCI Bus / PCI Device Layout / Common configuration structure layout}

The common configuration structure is found at the \field{bar} and \field{offset} within the VIRTIO_PCI_CAP_COMMON_CFG capability; its layout is below.

\begin{lstlisting}
struct virtio_pci_common_cfg {
        /* About the whole device. */
        le32 device_feature_select;     /* read-write */
        le32 device_feature;            /* read-only for driver */
        le32 driver_feature_select;     /* read-write */
        le32 driver_feature;            /* read-write */
        le16 msix_config;               /* read-write */
        le16 num_queues;                /* read-only for driver */
        u8 device_status;               /* read-write */
        u8 config_generation;           /* read-only for driver */

        /* About a specific virtqueue. */
        le16 queue_select;              /* read-write */
        le16 queue_size;                /* read-write, power of 2, or 0. */
        le16 queue_msix_vector;         /* read-write */
        le16 queue_enable;              /* read-write */
        le16 queue_notify_off;          /* read-only for driver */
        le64 queue_desc;                /* read-write */
        le64 queue_avail;               /* read-write */
        le64 queue_used;                /* read-write */
};
\end{lstlisting}

\begin{description}
\item[\field{device_feature_select}]
        The driver uses this to select which feature bits \field{device_feature} shows.
        Value 0x0 selects Feature Bits 0 to 31, 0x1 selects Feature Bits 32 to 63, etc.

\item[\field{device_feature}]
        The device uses this to report which feature bits it is
        offering to the driver: the driver writes to
        \field{device_feature_select} to select which feature bits are presented.

\item[\field{driver_feature_select}]
        The driver uses this to select which feature bits \field{driver_feature} shows.
        Value 0x0 selects Feature Bits 0 to 31, 0x1 selects Feature Bits 32 to 63, etc.

\item[\field{driver_feature}]
        The driver writes this to accept feature bits offered by the device.
        Driver Feature Bits selected by \field{driver_feature_select}.

\item[\field{config_msix_vector}]
        The driver sets the Configuration Vector for MSI-X.

\item[\field{num_queues}]
        The device specifies the maximum number of virtqueues supported here.

\item[\field{device_status}]
        The driver writes the device status here (see \ref{sec:Basic Facilities of a Virtio Device / Device Status Field}). Writing 0 into this
        field resets the device.

\item[\field{config_generation}]
        Configuration atomicity value.  The device changes this every time the
        configuration noticeably changes.

\item[\field{queue_select}]
        Queue Select. The driver selects which virtqueue the following
        fields refer to.

\item[\field{queue_size}]
        Queue Size.  On reset, specifies the maximum queue size supported by
        the hypervisor. This can be modified by driver to reduce memory requirements.
        A 0 means the queue is unavailable.

\item[\field{queue_msix_vector}]
        The driver uses this to specify the queue vector for MSI-X.

\item[\field{queue_enable}]
        The driver uses this to selectively prevent the device from executing requests from this virtqueue.
        1 - enabled; 0 - disabled.

\item[\field{queue_notify_off}]
        The driver reads this to calculate the offset from start of Notification structure at
        which this virtqueue is located.
        \begin{note} this is \em{not} an offset in bytes.
        See \ref{sec:Virtio Transport Options / Virtio Over PCI Bus / PCI Device Layout / Notification capability} below.
        \end{note}

\item[\field{queue_desc}]
        The driver writes the physical address of Descriptor Table here.  See section \ref{sec:Basic Facilities of a Virtio Device / Virtqueues}.

\item[\field{queue_avail}]
        The driver writes the physical address of Available Ring here.  See section \ref{sec:Basic Facilities of a Virtio Device / Virtqueues}.

\item[\field{queue_used}]
        The driver writes the physical address of Used Ring here.  See section \ref{sec:Basic Facilities of a Virtio Device / Virtqueues}.
\end{description}

\devicenormative{\paragraph}{Common configuration structure layout}{Virtio Transport Options / Virtio Over PCI Bus / PCI Device Layout / Common configuration structure layout}
\field{offset} MUST be 4-byte aligned.

The device MUST present at least one common configuration capability.

The device MUST present the feature bits it is offering in \field{device_feature}, starting at bit \field{device_feature_select} $*$ 32 for any \field{device_feature_select} written by the driver.
\begin{note}
  This means that it will present 0 for any \field{device_feature_select} other than 0 or 1, since no feature defined here exceeds 63.
\end{note}

The device MUST present any valid feature bits the driver has written in \field{driver_feature}, starting at bit \field{driver_feature_select} $*$ 32 for any \field{driver_feature_select} written by the driver.  Valid feature bits are those which are subset of the corresponding \field{device_feature} bits.  The device MAY present invalid bits written by the driver.

\begin{note}
  This means that a device can ignore writes for feature bits it never
  offers, and simply present 0 on reads.  Or it can just mirror what the driver wrote
  (but it will still have to check them when the driver sets FEATURES_OK).
\end{note}

\begin{note}
  A driver shouldn't write invalid bits anyway, as per \ref{drivernormative:General Initialization And Device Operation / Device Initialization}, but this attempts to handle it.
\end{note}

The device MUST present a changed \field{config_generation} after the
driver has read a device-specific configuration value which has
changed since any part of the device-specific configuration was last
read.
\begin{note}
As \field{config_generation} is an 8-bit value, simply incrementing it
on every configuration change could violate this requirement due to wrap.
Better would be to set an internal flag when it has changed,
and if that flag is set when the driver reads from the device-specific
configuration, increment \field{config_generation} and clear the flag.
\end{note}

The device MUST reset when 0 is written to \field{device_status}, and
present a 0 in \field{device_status} once that is done.

The device MUST present a 0 in \field{queue_enable} on reset.

The device MUST present a 0 in \field{queue_size} if the virtqueue
corresponding to the current \field{queue_select} is unavailable.

\drivernormative{\paragraph}{Common configuration structure layout}{Virtio Transport Options / Virtio Over PCI Bus / PCI Device Layout / Common configuration structure layout}

The driver MUST NOT write to \field{device_feature}, \field{num_queues}, \field{config_generation} or \field{queue_notify_off}.

The driver MUST NOT write a value which is not a power of 2 to \field{queue_size}.

The driver MUST configure the other virtqueue fields before enabling the virtqueue
with \field{queue_enable}.

After writing 0 to \field{device_status}, the driver MUST wait for a read of
\field{device_status} to return 0 before reinitializing the device.

The driver MUST NOT write a 0 to \field{queue_enable}.

\subsubsection{Notification structure layout}\label{sec:Virtio Transport Options / Virtio Over PCI Bus / PCI Device Layout / Notification capability}

The notification location is found using the VIRTIO_PCI_CAP_NOTIFY_CFG
capability.  This capability is immediately followed by an additional
field, like so:

\begin{lstlisting}
struct virtio_pci_notify_cap {
        struct virtio_pci_cap cap;
        le32 notify_off_multiplier; /* Multiplier for queue_notify_off. */
};
\end{lstlisting}

\field{notify_off_multiplier} is combined with the \field{queue_notify_off} to
derive the Queue Notify address within a BAR for a virtqueue:

\begin{lstlisting}
        cap.offset + queue_notify_off * notify_off_multiplier
\end{lstlisting}

The \field{cap.offset} and \field{notify_off_multiplier} are taken from the
notification capability structure above, and the \field{queue_notify_off} is
taken from the common configuration structure.

\begin{note}
For example, if \field{notifier_off_multiplier} is 0, the device uses
the same Queue Notify address for all queues.
\end{note}

\devicenormative{\paragraph}{Notification capability}{Virtio Transport Options / Virtio Over PCI Bus / PCI Device Layout / Notification capability}
The device MUST present at least one notification capability.

The \field{cap.offset} MUST be 2-byte aligned.  

The device MUST either present \field{notify_off_multiplier} as an even power of 2,
or present \field{notify_off_multiplier} as 0.

The value \field{cap.length} presented by the device MUST be at least 2
and MUST be large enough to support queue notification offsets
for all supported queues in all possible configurations.

For all queues, the value \field{cap.length} presented by the device MUST satisfy:
\begin{lstlisting}
cap.length >= queue_notify_off * notify_off_multiplier + 2
\end{lstlisting}

\subsubsection{ISR status capability}\label{sec:Virtio Transport Options / Virtio Over PCI Bus / PCI Device Layout / ISR status capability}

The VIRTIO_PCI_CAP_ISR_CFG capability
refers to at least a single byte, which contains the 8-bit ISR status field
to be used for INT\#x interrupt handling.

The \field{offset} for the \field{ISR status} has no alignment requirements.

The ISR bits allow the device to distinguish between device-specific configuration
change interrupts and normal virtqueue interrupts:

\begin{tabular}{ |l||l|l|l| }
\hline
Bits       & 0                               & 1               &  2 to 31 \\
\hline
Purpose    & Queue Interrupt  & Device Configuration Interrupt & Reserved \\
\hline
\end{tabular}

To avoid an extra access, simply reading this register resets it to 0 and
causes the device to de-assert the interrupt.

In this way, driver read of ISR status causes the device to de-assert
an interrupt.

See sections \ref{sec:Virtio Transport Options / Virtio Over PCI Bus / PCI-specific Initialization And Device Operation / Virtqueue Interrupts From The Device} and \ref{sec:Virtio Transport Options / Virtio Over PCI Bus / PCI-specific Initialization And Device Operation / Notification of Device Configuration Changes} for how this is used.

\devicenormative{\paragraph}{ISR status capability}{Virtio Transport Options / Virtio Over PCI Bus / PCI Device Layout / ISR status capability}

The device MUST present at least one VIRTIO_PCI_CAP_ISR_CFG capability.  

The device MUST set the Device Configuration Interrupt bit
in \field{ISR status} before sending a device configuration
change notification to the driver.

If MSI-X capability is disabled, the device MUST set the Queue
Interrupt bit in \field{ISR status} before sending a virtqueue
notification to the driver.

If MSI-X capability is disabled, the device MUST set the Interrupt Status
bit in the PCI Status register in the PCI Configuration Header of
the device to the logical OR of all bits in \field{ISR status} of
the device.  The device then asserts/deasserts INT\#x interrupts unless masked
according to standard PCI rules \hyperref[intro:PCI]{[PCI]}.

The device MUST reset \field{ISR status} to 0 on driver read.

\drivernormative{\paragraph}{ISR status capability}{Virtio Transport Options / Virtio Over PCI Bus / PCI Device Layout / ISR status capability}

If MSI-X capability is enabled, the driver SHOULD NOT access
\field{ISR status} upon detecting a Queue Interrupt.

\subsubsection{Device-specific configuration}\label{sec:Virtio Transport Options / Virtio Over PCI Bus / PCI Device Layout / Device-specific configuration}

The device MUST present at least one VIRTIO_PCI_CAP_DEVICE_CFG capability for
any device type which has a device-specific configuration.

\devicenormative{\paragraph}{Device-specific configuration}{Virtio Transport Options / Virtio Over PCI Bus / PCI Device Layout / Device-specific configuration}

The \field{offset} for the device-specific configuration MUST be 4-byte aligned.

\subsubsection{PCI configuration access capability}\label{sec:Virtio Transport Options / Virtio Over PCI Bus / PCI Device Layout / PCI configuration access capability}

The VIRTIO_PCI_CAP_PCI_CFG capability
creates an alternative (and likely suboptimal) access method to the
common configuration, notification, ISR and device-specific configuration regions.

The capability is immediately followed by an additional field like so:

\begin{lstlisting}
struct virtio_pci_cfg_cap {
        struct virtio_pci_cap cap;
        u8 pci_cfg_data[4]; /* Data for BAR access. */
};
\end{lstlisting}

The fields \field{cap.bar}, \field{cap.length}, \field{cap.offset} and
\field{pci_cfg_data} are read-write (RW) for the driver.

To access a device region, the driver writes into the capability
structure (ie. within the PCI configuration space) as follows:

\begin{itemize}
\item The driver sets the BAR to access by writing to \field{cap.bar}.

\item The driver sets the size of the access by writing 1, 2 or 4 to
  \field{cap.length}.

\item The driver sets the offset within the BAR by writing to
  \field{cap.offset}.
\end{itemize}

At that point, \field{pci_cfg_data} will provide a window of size
\field{cap.length} into the given \field{cap.bar} at offset \field{cap.offset}.

\devicenormative{\paragraph}{PCI configuration access capability}{Virtio Transport Options / Virtio Over PCI Bus / PCI Device Layout / PCI configuration access capability}

The device MUST present at least one VIRTIO_PCI_CAP_PCI_CFG capability.

Upon detecting driver write access
to \field{pci_cfg_data}, the device MUST execute a write access
at offset \field{cap.offset} at BAR selected by \field{cap.bar} using the first \field{cap.length}
bytes from \field{pci_cfg_data}.

Upon detecting driver read access
to \field{pci_cfg_data}, the device MUST
execute a read access of length cap.length at offset \field{cap.offset}
at BAR selected by \field{cap.bar} and store the first \field{cap.length} bytes in
\field{pci_cfg_data}.

\drivernormative{\paragraph}{PCI configuration access capability}{Virtio Transport Options / Virtio Over PCI Bus / PCI Device Layout / PCI configuration access capability}

The driver MUST NOT write a \field{cap.offset} which is not
a multiple of \field{cap.length} (ie. all accesses MUST be aligned).

\subsubsection{Legacy Interfaces: A Note on PCI Device Layout}\label{sec:Virtio Transport Options / Virtio Over PCI Bus / PCI Device Layout / Legacy Interfaces: A Note on PCI Device Layout}

Transitional devices MUST present part of configuration
registers in a legacy configuration structure in BAR0 in the first I/O
region of the PCI device, as documented below.
When using the legacy interface, transitional drivers
MUST use the legacy configuration structure in BAR0 in the first
I/O region of the PCI device, as documented below.

When using the legacy interface the driver MAY access
the device-specific configuration region using any width accesses, and
a transitional device MUST present driver with the same results as
when accessed using the ``natural'' access method (i.e.
32-bit accesses for 32-bit fields, etc).

Note that this is possible because while the virtio common configuration structure is PCI
(i.e. little) endian, when using the legacy interface the device-specific
configuration region is encoded in the native endian of the guest (where such distinction is
applicable).

When used through the legacy interface, the virtio common configuration structure looks as follows:

\begin{tabularx}{\textwidth}{ |X||X|X|X|X|X|X|X|X| }
\hline
 Bits & 32 & 32 & 32 & 16 & 16 & 16 & 8 & 8 \\
\hline
 Read / Write & R & R+W & R+W & R & R+W & R+W & R+W & R \\
\hline
 Purpose & Device Features bits 0:31 & Driver Features bits 0:31 &
  Queue Address & \field{queue_size} & \field{queue_select} & Queue Notify &
  Device Status & ISR \newline Status \\
\hline
\end{tabularx}

If MSI-X is enabled for the device, two additional fields
immediately follow this header:

\begin{tabular}{ |l||l|l| }
\hline
Bits       & 16             & 16     \\
\hline
Read/Write & R+W            & R+W    \\
\hline
Purpose (MSI-X) & \field{config_msix_vector}  & \field{queue_msix_vector} \\
\hline
\end{tabular}

Note: When MSI-X capability is enabled, device-specific configuration starts at
byte offset 24 in virtio common configuration structure structure. When MSI-X capability is not
enabled, device-specific configuration starts at byte offset 20 in virtio
header.  ie. once you enable MSI-X on the device, the other fields move.
If you turn it off again, they move back!

Any device-specific configuration space immediately follows
these general headers:

\begin{tabular}{|l||l|l|}
\hline
Bits & Device Specific & \multirow{3}{*}{\ldots} \\
\cline{1-2}
Read / Write & Device Specific & \\
\cline{1-2}
Purpose & Device Specific & \\
\hline
\end{tabular}

When accessing the device-specific configuration space
using the legacy interface, transitional
drivers MUST access the device-specific configuration space
at an offset immediately following the general headers.

When using the legacy interface, transitional
devices MUST present the device-specific configuration space
if any at an offset immediately following the general headers.

Note that only Feature Bits 0 to 31 are accessible through the
Legacy Interface. When used through the Legacy Interface,
Transitional Devices MUST assume that Feature Bits 32 to 63
are not acknowledged by Driver.

As legacy devices had no \field{config_generation} field,
see \ref{sec:Basic Facilities of a Virtio Device / Device
Configuration Space / Legacy Interface: Device Configuration
Space}~\nameref{sec:Basic Facilities of a Virtio Device / Device Configuration Space / Legacy Interface: Device Configuration Space} for workarounds.

\subsubsection{Non-transitional Device With Legacy Driver: A Note
on PCI Device Layout}\label{sec:Virtio Transport Options / Virtio
Over PCI Bus / PCI Device Layout / Non-transitional Device With
Legacy Driver: A Note on PCI Device Layout}

All known legacy drivers check either the PCI Revision or the
Device and Vendor IDs, and thus won't attempt to drive a
non-transitional device.

A buggy legacy driver might mistakenly attempt to drive a
non-transitional device. If support for such drivers is required
(as opposed to fixing the bug), the following would be the
recommended way to detect and handle them.
\begin{note}
Such buggy drivers are not currently known to be used in
production.
\end{note}

\subparagraph{
\DIFdeltextcstwo{Driver Requirements: Non-transitional Device With Legacy Driver}
\DIFaddtextcstwo{Device Requirements: Non-transitional Device With Legacy Driver}
}
\label{drivernormative:Virtio Transport Options / Virtio Over PCI
Bus / PCI-specific Initialization And Device Operation /
Device Initialization / Non-transitional Device With Legacy
Driver}
\label{devicenormative:Virtio Transport Options / Virtio Over PCI
Bus / PCI-specific Initialization And Device Operation /
Device Initialization / Non-transitional Device With Legacy
Driver}

Non-transitional devices, on a platform where a legacy driver for
a legacy device with the same ID (including PCI Revision, Device
and Vendor IDs) is known to have previously existed,
SHOULD take the following steps to cause the legacy driver to
fail gracefully when it attempts to drive them:

\begin{enumerate}
\item Present an I/O BAR in BAR0, and
\item Respond to a single-byte zero write to offset 18
   (corresponding to Device Status register in the legacy layout)
   of BAR0 by presenting zeroes on every BAR and ignoring writes.
\end{enumerate}

\subsection{PCI-specific Initialization And Device Operation}\label{sec:Virtio Transport Options / Virtio Over PCI Bus / PCI-specific Initialization And Device Operation}

\subsubsection{Device Initialization}\label{sec:Virtio Transport Options / Virtio Over PCI Bus / PCI-specific Initialization And Device Operation / Device Initialization}

This documents PCI-specific steps executed during Device Initialization.

\paragraph{Virtio Device Configuration Layout Detection}\label{sec:Virtio Transport Options / Virtio Over PCI Bus / PCI-specific Initialization And Device Operation / Device Initialization / Virtio Device Configuration Layout Detection}

As a prerequisite to device initialization, the driver scans the
PCI capability list, detecting virtio configuration layout using Virtio
Structure PCI capabilities as detailed in \ref{sec:Virtio Transport Options / Virtio Over PCI Bus / Virtio Structure PCI Capabilities}

\subparagraph{Legacy Interface: A Note on Device Layout Detection}\label{sec:Virtio Transport Options / Virtio Over PCI Bus / PCI-specific Initialization And Device Operation / Device Initialization / Virtio Device Configuration Layout Detection / Legacy Interface: A Note on Device Layout Detection}

Legacy drivers skipped the Device Layout Detection step, assuming legacy
device configuration space in BAR0 in I/O space unconditionally.

Legacy devices did not have the Virtio PCI Capability in their
capability list.

Therefore:

Transitional devices MUST expose the Legacy Interface in I/O
space in BAR0.

Transitional drivers MUST look for the Virtio PCI
Capabilities on the capability list.
If these are not present, driver MUST assume a legacy device,
and use it through the legacy interface.

Non-transitional drivers MUST look for the Virtio PCI
Capabilities on the capability list.
If these are not present, driver MUST assume a legacy device,
and fail gracefully.

\paragraph{MSI-X Vector Configuration}\label{sec:Virtio Transport Options / Virtio Over PCI Bus / PCI-specific Initialization And Device Operation / Device Initialization / MSI-X Vector Configuration}

When MSI-X capability is present and enabled in the device
(through standard PCI configuration space) \field{config_msix_vector} and \field{queue_msix_vector} are used to map configuration change and queue
interrupts to MSI-X vectors. In this case, the ISR Status is unused.

Writing a valid MSI-X Table entry number, 0 to 0x7FF, to
\field{config_msix_vector}/\field{queue_msix_vector} maps interrupts triggered
by the configuration change/selected queue events respectively to
the corresponding MSI-X vector. To disable interrupts for an
event type, the driver unmaps this event by writing a special NO_VECTOR
value:

\begin{lstlisting}
/* Vector value used to disable MSI for queue */
#define VIRTIO_MSI_NO_VECTOR            0xffff
\end{lstlisting}

Note that mapping an event to vector might require device to
allocate internal device resources, and thus could fail. 

\devicenormative{\subparagraph}{MSI-X Vector Configuration}{Virtio Transport Options / Virtio Over PCI Bus / PCI-specific Initialization And Device Operation / Device Initialization / MSI-X Vector Configuration}

A device that has an MSI-X capability SHOULD support at least 2
and at most 0x800 MSI-X vectors.
Device MUST report the number of vectors supported in
\field{Table Size} in the MSI-X Capability as specified in
\hyperref[intro:PCI]{[PCI]}.
The device SHOULD restrict the reported MSI-X Table Size field
to a value that might benefit system performance.
\begin{note}
For example, a device which does not expect to send
interrupts at a high rate might only specify 2 MSI-X vectors.
\end{note}
Device MUST support mapping any event type to any valid
vector 0 to MSI-X \field{Table Size}.
Device MUST support unmapping any event type.

The device MUST return vector mapped to a given event,
(NO_VECTOR if unmapped) on read of \field{config_msix_vector}/\field{queue_msix_vector}.
The device MUST have all queue and configuration change
events are unmapped upon reset.

Devices SHOULD NOT cause mapping an event to vector to fail
unless it is impossible for the device to satisfy the mapping
request.  Devices MUST report mapping
failures by returning the NO_VECTOR value when the relevant
\field{config_msix_vector}/\field{queue_msix_vector} field is read. 

\drivernormative{\subparagraph}{MSI-X Vector Configuration}{Virtio Transport Options / Virtio Over PCI Bus / PCI-specific Initialization And Device Operation / Device Initialization / MSI-X Vector Configuration}

Driver MUST support device with any MSI-X Table Size 0 to 0x7FF.
Driver MAY fall back on using INT\#x interrupts for a device
which only supports one MSI-X vector (MSI-X Table Size = 0).

Driver MAY intepret the Table Size as a hint from the device
for the suggested number of MSI-X vectors to use.

Driver MUST NOT attempt to map an event to a vector
outside the MSI-X Table supported by the device,
as reported by \field{Table Size} in the MSI-X Capability.

After mapping an event to vector, the
driver MUST verify success by reading the Vector field value: on
success, the previously written value is returned, and on
failure, NO_VECTOR is returned. If a mapping failure is detected,
the driver MAY retry mapping with fewer vectors, disable MSI-X
or report device failure.

\paragraph{Virtqueue Configuration}\label{sec:Virtio Transport Options / Virtio Over PCI Bus / PCI-specific Initialization And Device Operation / Device Initialization / Virtqueue Configuration}

As a device can have zero or more virtqueues for bulk data
transport\footnote{For example, the simplest network device has two virtqueues.}, the driver
needs to configure them as part of the device-specific
configuration.

The driver typically does this as follows, for each virtqueue a device has:

\begin{enumerate}
\item Write the virtqueue index (first queue is 0) to \field{queue_select}.

\item Read the virtqueue size from \field{queue_size}. This controls how big the virtqueue is
  (see \ref{sec:Basic Facilities of a Virtio Device / Virtqueues}~\nameref{sec:Basic Facilities of a Virtio Device / Virtqueues}). If this field is 0, the virtqueue does not exist.

\item Optionally, select a smaller virtqueue size and write it to \field{queue_size}.

\item Allocate and zero Descriptor Table, Available and Used rings for the
   virtqueue in contiguous physical memory.

\item Optionally, if MSI-X capability is present and enabled on the
  device, select a vector to use to request interrupts triggered
  by virtqueue events. Write the MSI-X Table entry number
  corresponding to this vector into \field{queue_msix_vector}. Read
  \field{queue_msix_vector}: on success, previously written value is
  returned; on failure, NO_VECTOR value is returned.
\end{enumerate}

\subparagraph{Legacy Interface: A Note on Virtqueue Configuration}\label{sec:Virtio Transport Options / Virtio Over PCI Bus / PCI-specific Initialization And Device Operation / Device Initialization / Virtqueue Configuration / Legacy Interface: A Note on Virtqueue Configuration}
When using the legacy interface, the queue layout follows \ref{sec:Basic Facilities of a Virtio Device / Virtqueues / Legacy Interfaces: A Note on Virtqueue Layout}~\nameref{sec:Basic Facilities of a Virtio Device / Virtqueues / Legacy Interfaces: A Note on Virtqueue Layout} with an alignment of 4096.
Driver writes the physical address, divided
by 4096 to the Queue Address field\footnote{The 4096 is based on the x86 page size, but it's also large
enough to ensure that the separate parts of the virtqueue are on
separate cache lines.
}.  There was no mechanism to negotiate the queue size.

\subsubsection{Notifying The Device}\label{sec:Virtio Transport Options / Virtio Over PCI Bus / PCI-specific Initialization And Device Operation / Notifying The Device}

The driver notifies the device by writing the 16-bit virtqueue index
of this virtqueue to the Queue Notify address.  See \ref{sec:Virtio Transport Options / Virtio Over PCI Bus / PCI Device Layout / Notification capability} for how to calculate this address.

\subsubsection{Virtqueue Interrupts From The Device}\label{sec:Virtio Transport Options / Virtio Over PCI Bus / PCI-specific Initialization And Device Operation / Virtqueue Interrupts From The Device}

If an interrupt is necessary for a virtqueue, the device would typically act as follows:

\begin{itemize}
  \item If MSI-X capability is disabled:
    \begin{enumerate}
    \item Set the lower bit of the ISR Status field for the device.

    \item Send the appropriate PCI interrupt for the device.
    \end{enumerate}

  \item If MSI-X capability is enabled:
    \begin{enumerate}
    \item If \field{queue_msix_vector} is not NO_VECTOR,
      request the appropriate MSI-X interrupt message for the
      device, \field{queue_msix_vector} sets the MSI-X Table entry
      number.
    \end{enumerate}
\end{itemize}

\devicenormative{\paragraph}{Virtqueue Interrupts From The Device}{Virtio Transport Options / Virtio Over PCI Bus / PCI-specific Initialization And Device Operation / Virtqueue Interrupts From The Device}

If MSI-X capability is enabled and \field{queue_msix_vector} is
NO_VECTOR for a virtqueue, the device MUST NOT deliver an interrupt
for that virtqueue.

\subsubsection{Notification of Device Configuration Changes}\label{sec:Virtio Transport Options / Virtio Over PCI Bus / PCI-specific Initialization And Device Operation / Notification of Device Configuration Changes}

Some virtio PCI devices can change the device configuration
state, as reflected in the device-specific configuration region of the device. In this case:

\begin{itemize}
  \item If MSI-X capability is disabled:
    \begin{enumerate}
    \item Set the second lower bit of the ISR Status field for the device.

    \item Send the appropriate PCI interrupt for the device.
    \end{enumerate}

  \item If MSI-X capability is enabled:
    \begin{enumerate}
    \item If \field{config_msix_vector} is not NO_VECTOR,
      request the appropriate MSI-X interrupt message for the
      device, \field{config_msix_vector} sets the MSI-X Table entry
      number.
    \end{enumerate}
\end{itemize}

A single interrupt MAY indicate both that one or more virtqueue has
been used and that the configuration space has changed.

\devicenormative{\paragraph}{Notification of Device Configuration Changes}{Virtio Transport Options / Virtio Over PCI Bus / PCI-specific Initialization And Device Operation / Notification of Device Configuration Changes}

If MSI-X capability is enabled and \field{config_msix_vector} is
NO_VECTOR, the device MUST NOT deliver an interrupt
for device configuration space changes.

\drivernormative{\paragraph}{Notification of Device Configuration Changes}{Virtio Transport Options / Virtio Over PCI Bus / PCI-specific Initialization And Device Operation / Notification of Device Configuration Changes}

A driver MUST handle the case where the same interrupt is used to indicate
both device configuration space change and one or more virtqueues being used.

\subsubsection{Driver Handling Interrupts}\label{sec:Virtio Transport Options / Virtio Over PCI Bus / PCI-specific Initialization And Device Operation / Driver Handling Interrupts}
The driver interrupt handler would typically:

\begin{itemize}
  \item If MSI-X capability is disabled:
    \begin{itemize}
      \item Read the ISR Status field, which will reset it to zero.
      \item If the lower bit is set:
        look through the used rings of all virtqueues for the
        device, to see if any progress has been made by the device
        which requires servicing.
      \item If the second lower bit is set:
        re-examine the configuration space to see what changed.
    \end{itemize}
  \item If MSI-X capability is enabled:
    \begin{itemize}
      \item
        Look through the used rings of
        all virtqueues mapped to that MSI-X vector for the
        device, to see if any progress has been made by the device
        which requires servicing.
      \item
        If the MSI-X vector is equal to \field{config_msix_vector},
        re-examine the configuration space to see what changed.
    \end{itemize}
\end{itemize}

\section{Virtio Over MMIO}\label{sec:Virtio Transport Options / Virtio Over MMIO}

Virtual environments without PCI support (a common situation in
embedded devices models) might use simple memory mapped device
(``virtio-mmio'') instead of the PCI device.

The memory mapped virtio device behaviour is based on the PCI
device specification. Therefore most operations including device
initialization, queues configuration and buffer transfers are
nearly identical. Existing differences are described in the
following sections.

\subsection{MMIO Device Discovery}\label{sec:Virtio Transport Options / Virtio Over MMIO / MMIO Device Discovery}

Unlike PCI, MMIO provides no generic device discovery mechanism.  For each
device, the guest OS will need to know the location of the registers
and interrupt(s) used.  The suggested binding for systems using
flattened device trees is shown in this example:

\begin{lstlisting}
// EXAMPLE: virtio_block device taking 512 bytes at 0x1e000, interrupt 42.
virtio_block@1e000 {
        compatible = "virtio,mmio";
        reg = <0x1e000 0x200>;
        interrupts = <42>;
}
\end{lstlisting}

\subsection{MMIO Device Register Layout}\label{sec:Virtio Transport Options / Virtio Over MMIO / MMIO Device Register Layout}

MMIO virtio devices provide a set of memory mapped control
registers followed by a device-specific configuration space,
described in the table~\ref{tab:Virtio Trasport Options / Virtio Over MMIO / MMIO Device Register Layout}.

All register values are organized as Little Endian.

\newcommand{\mmioreg}[5]{% Name Function Offset Direction Description
  {\field{#1}} \newline #3 \newline #4 & {\bf#2} \newline #5 \\
}

\newcommand{\mmiodreg}[7]{% NameHigh NameLow Function OffsetHigh OffsetLow Direction Description
  {\field{#1}} \newline #4 \newline {\field{#2}} \newline #5 \newline #6 & {\bf#3} \newline #7 \\
}

\begin{longtable}{p{0.2\textwidth}p{0.7\textwidth}}
  \caption {MMIO Device Register Layout}
  \label{tab:Virtio Trasport Options / Virtio Over MMIO / MMIO Device Register Layout} \\
  \hline
  \mmioreg{Name}{Function}{Offset from base}{Direction}{Description} 
  \hline 
  \hline 
  \endfirsthead
  \hline
  \mmioreg{Name}{Function}{Offset from the base}{Direction}{Description} 
  \hline 
  \hline 
  \endhead
  \endfoot
  \endlastfoot
  \mmioreg{MagicValue}{Magic value}{0x000}{R}{%
    0x74726976
    (a Little Endian equivalent of the ``virt'' string).
  } 
  \hline
  \mmioreg{Version}{Device version number}{0x004}{R}{%
    0x2.
    \begin{note}
      Legacy devices (see \ref{sec:Virtio Transport Options / Virtio Over MMIO / Legacy interface}~\nameref{sec:Virtio Transport Options / Virtio Over MMIO / Legacy interface}) used 0x1.
    \end{note}
  }
  \hline 
  \mmioreg{DeviceID}{Virtio Subsystem Device ID}{0x008}{R}{%
    See \ref{sec:Device Types}~\nameref{sec:Device Types} for possible values.
    Value zero (0x0) is used to
    define a system memory map with placeholder devices at static,
    well known addresses, assigning functions to them depending
    on user's needs.
  }
  \hline 
  \mmioreg{VendorID}{Virtio Subsystem Vendor ID}{0x00c}{R}{}
  \hline 
  \mmioreg{DeviceFeatures}{Flags representing features the device supports}{0x010}{R}{%
    Reading from this register returns 32 consecutive flag bits,
    the least significant bit depending on the last value written to
    \field{DeviceFeaturesSel}. Access to this register returns
    bits $\field{DeviceFeaturesSel}*32$ to $(\field{DeviceFeaturesSel}*32)+31$, eg.
    feature bits 0 to 31 if \field{DeviceFeaturesSel} is set to 0 and
    features bits 32 to 63 if \field{DeviceFeaturesSel} is set to 1.
    Also see \ref{sec:Basic Facilities of a Virtio Device / Feature Bits}~\nameref{sec:Basic Facilities of a Virtio Device / Feature Bits}.
  }
  \hline 
  \mmioreg{DeviceFeaturesSel}{Device (host) features word selection.}{0x014}{W}{%
    Writing to this register selects a set of 32 device feature bits
    accessible by reading from \field{DeviceFeatures}.
  }
  \hline 
  \mmioreg{DriverFeatures}{Flags representing device features understood and activated by the driver}{0x020}{W}{%
    Writing to this register sets 32 consecutive flag bits, the least significant
    bit depending on the last value written to \field{DriverFeaturesSel}.
     Access to this register sets bits $\field{DriverFeaturesSel}*32$
    to $(\field{DriverFeaturesSel}*32)+31$, eg. feature bits 0 to 31 if
    \field{DriverFeaturesSel} is set to 0 and features bits 32 to 63 if
    \field{DriverFeaturesSel} is set to 1. Also see \ref{sec:Basic Facilities of a Virtio Device / Feature Bits}~\nameref{sec:Basic Facilities of a Virtio Device / Feature Bits}.
  }
  \hline 
  \mmioreg{DriverFeaturesSel}{Activated (guest) features word selection}{0x024}{W}{%
    Writing to this register selects a set of 32 activated feature
    bits accessible by writing to \field{DriverFeatures}.
  }
  \hline 
  \mmioreg{QueueSel}{Virtual queue index}{0x030}{W}{%
    Writing to this register selects the virtual queue that the
    following operations on \field{QueueNumMax}, \field{QueueNum}, \field{QueueReady},
    \field{QueueDescLow}, \field{QueueDescHigh}, \field{QueueAvailLow}, \field{QueueAvailHigh},
    \field{QueueUsedLow} and \field{QueueUsedHigh} apply to. The index
    number of the first queue is zero (0x0). 
  }
  \hline 
  \mmioreg{QueueNumMax}{Maximum virtual queue size}{0x034}{R}{%
    Reading from the register returns the maximum size (number of
    elements) of the queue the device is ready to process or
    zero (0x0) if the queue is not available. This applies to the
    queue selected by writing to \field{QueueSel}.
  }
  \hline 
  \mmioreg{QueueNum}{Virtual queue size}{0x038}{W}{%
    Queue size is the number of elements in the queue, therefore in each
    of the Descriptor Table, the Available Ring and the Used Ring.
    Writing to this register notifies the device what size of the
    queue the driver will use. This applies to the queue selected by
    writing to \field{QueueSel}.
  }
  \hline 
  \mmioreg{QueueReady}{Virtual queue ready bit}{0x044}{RW}{%
    Writing one (0x1) to this register notifies the device that it can
    execute requests from this virtual queue. Reading from this register
    returns the last value written to it. Both read and write
    accesses apply to the queue selected by writing to \field{QueueSel}.
  }
  \hline 
  \mmioreg{QueueNotify}{Queue notifier}{0x050}{W}{%
    Writing a queue index to this register notifies the device that
    there are new buffers to process in the queue.
  }
  \hline 
  \mmioreg{InterruptStatus}{Interrupt status}{0x60}{R}{%
    Reading from this register returns a bit mask of events that
    caused the device interrupt to be asserted.
    The following events are possible:
    \begin{description}
      \item[Used Ring Update] - bit 0 - the interrupt was asserted
        because the device has updated the Used
        Ring in at least one of the active virtual queues.
      \item [Configuration Change] - bit 1 - the interrupt was
        asserted because the configuration of the device has changed.
    \end{description}
  }
  \hline 
  \mmioreg{InterruptACK}{Interrupt acknowledge}{0x064}{W}{%
    Writing a value with bits set as defined in \field{InterruptStatus}
    to this register notifies the device that events causing
    the interrupt have been handled.
  }
  \hline 
  \mmioreg{Status}{Device status}{0x070}{RW}{%
    Reading from this register returns the current device status
    flags.
    Writing non-zero values to this register sets the status flags,
    indicating the driver progress. Writing zero (0x0) to this
    register triggers a device reset. 
    See also p. \ref{sec:Virtio Transport Options / Virtio Over MMIO / MMIO-specific Initialization And Device Operation / Device Initialization}~\nameref{sec:Virtio Transport Options / Virtio Over MMIO / MMIO-specific Initialization And Device Operation / Device Initialization}.
  }
  \hline 
  \mmiodreg{QueueDescLow}{QueueDescHigh}{Virtual queue's Descriptor Table 64 bit long physical address}{0x080}{0x084}{W}{%
    Writing to these two registers (lower 32 bits of the address
    to \field{QueueDescLow}, higher 32 bits to \field{QueueDescHigh}) notifies
    the device about location of the Descriptor Table of the queue
    selected by writing to \field{QueueSel} register.
  }
  \hline 
  \mmiodreg{QueueAvailLow}{QueueAvailHigh}{Virtual queue's Available Ring 64 bit long physical address}{0x090}{0x094}{W}{%
    Writing to these two registers (lower 32 bits of the address
    to \field{QueueAvailLow}, higher 32 bits to \field{QueueAvailHigh}) notifies
    the device about location of the Available Ring of the queue
    selected by writing to \field{QueueSel}.
  }
  \hline 
  \mmiodreg{QueueUsedLow}{QueueUsedHigh}{Virtual queue's Used Ring 64 bit long physical address}{0x0a0}{0x0a4}{W}{%
    Writing to these two registers (lower 32 bits of the address
    to \field{QueueUsedLow}, higher 32 bits to \field{QueueUsedHigh}) notifies
    the device about location of the Used Ring of the queue
    selected by writing to \field{QueueSel}.
  }
  \hline 
  \mmioreg{ConfigGeneration}{Configuration atomicity value}{0x0fc}{R}{
    Reading from this register returns a value describing a version of the device-specific configuration space (see \field{Config}).
    The driver can then access the configuration space and, when finished, read \field{ConfigGeneration} again.
    If no part of the configuration space has changed between these two \field{ConfigGeneration} reads, the returned values are identical.
    If the values are different, the configuration space accesses were not atomic and the driver has to perform the operations again.
    See also \ref {sec:Basic Facilities of a Virtio Device / Device Configuration Space}.
  }
  \hline 
  \mmioreg{Config}{Configuration space}{0x100+}{RW}{
    Device-specific configuration space starts at the offset 0x100
    and is accessed with byte alignment. Its meaning and size
    depend on the device and the driver.
  }
  \hline
\end{longtable}

\devicenormative{\subsubsection}{MMIO Device Register Layout}{Virtio Transport Options / Virtio Over MMIO / MMIO Device Register Layout}

The device MUST return 0x74726976 in \field{MagicValue}.

The device MUST return value 0x2 in \field{Version}.

The device MUST present each event by setting the corresponding bit in \field{InterruptStatus} from the
moment it takes place, until the driver acknowledges the interrupt
by writing a corresponding bit mask to the \field{InterruptACK} register.  Bits which
do not represent events which took place MUST be zero.

Upon reset, the device MUST clear all bits in \field{InterruptStatus} and ready bits in the
\field{QueueReady} register for all queues in the device.

The device MUST change value returned in \field{ConfigGeneration} if there is any risk of a
driver seeing an inconsistent configuration state.

The device MUST NOT access virtual queue contents when \field{QueueReady} is zero (0x0).

\drivernormative{\subsubsection}{MMIO Device Register Layout}{Virtio Transport Options / Virtio Over MMIO / MMIO Device Register Layout}
The driver MUST NOT access memory locations not described in the
table \ref{tab:Virtio Trasport Options / Virtio Over MMIO / MMIO Device Register Layout}
(or, in case of the configuration space, described in the device specification),
MUST NOT write to the read-only registers (direction R) and
MUST NOT read from the write-only registers (direction W).

The driver MUST only use 32 bit wide and aligned reads and writes to access the control registers
described in table \ref{tab:Virtio Trasport Options / Virtio Over MMIO / MMIO Device Register Layout}.
For the device-specific configuration space, the driver MUST use 8 bit wide accesses for
8 bit wide fields, 16 bit wide and aligned accesses for 16 bit wide fields and 32 bit wide and
aligned accesses for 32 and 64 bit wide fields.

The driver MUST ignore a device with \field{MagicValue} which is not 0x74726976,
although it MAY report an error.

The driver MUST ignore a device with \field{Version} which is not 0x2,
although it MAY report an error.

The driver MUST ignore a device with \field{DeviceID} 0x0,
but MUST NOT report any error.

Before reading from \field{DeviceFeatures}, the driver MUST write a value to \field{DeviceFeaturesSel}.

Before writing to the \field{DriverFeatures} register, the driver MUST write a value to the \field{DriverFeaturesSel} register.

The driver MUST write a value to \field{QueueNum} which is less than
or equal to the value presented by the device in \field{QueueNumMax}.

When \field{QueueReady} is not zero, the driver MUST NOT access
\field{QueueNum}, \field{QueueDescLow}, \field{QueueDescHigh},
\field{QueueAvailLow}, \field{QueueAvailHigh}, \field{QueueUsedLow}, \field{QueueUsedHigh}.

To stop using the queue the driver MUST write zero (0x0) to this
\field{QueueReady} and MUST read the value back to ensure
synchronization.

The driver MUST ignore undefined bits in \field{InterruptStatus}.

The driver MUST write a value with a bit mask describing events it handled into \field{InterruptACK} when
it finishes handling an interrupt and MUST NOT set any of the undefined bits in the value.

\subsection{MMIO-specific Initialization And Device Operation}\label{sec:Virtio Transport Options / Virtio Over MMIO / MMIO-specific Initialization And Device Operation}

\subsubsection{Device Initialization}\label{sec:Virtio Transport Options / Virtio Over MMIO / MMIO-specific Initialization And Device Operation / Device Initialization}

\drivernormative{\paragraph}{Device Initialization}{Virtio Transport Options / Virtio Over MMIO / MMIO-specific Initialization And Device Operation / Device Initialization}

The driver MUST start the device initialization by reading and
checking values from \field{MagicValue} and \field{Version}.
If both values are valid, it MUST read \field{DeviceID}
and if its value is zero (0x0) MUST abort initialization and
MUST NOT access any other register.

Further initialization MUST follow the procedure described in
\ref{sec:General Initialization And Device Operation / Device Initialization}~\nameref{sec:General Initialization And Device Operation / Device Initialization}.

\subsubsection{Virtqueue Configuration}\label{sec:Virtio Transport Options / Virtio Over MMIO / MMIO-specific Initialization And Device Operation / Virtqueue Configuration}

The driver will typically initialize the virtual queue in the following way:

\begin{enumerate}
\item Select the queue writing its index (first queue is 0) to
   \field{QueueSel}.

\item Check if the queue is not already in use: read \field{QueueReady},
   and expect a returned value of zero (0x0).

\item Read maximum queue size (number of elements) from
   \field{QueueNumMax}. If the returned value is zero (0x0) the
   queue is not available.

\item Allocate and zero the queue pages, making sure the memory
   is physically contiguous. It is recommended to align the
   Used Ring to an optimal boundary (usually the page size).

\item Notify the device about the queue size by writing the size to
   \field{QueueNum}.

\item Write physical addresses of the queue's Descriptor Table,
   Available Ring and Used Ring to (respectively) the
   \field{QueueDescLow}/\field{QueueDescHigh},
   \field{QueueAvailLow}/\field{QueueAvailHigh} and
   \field{QueueUsedLow}/\field{QueueUsedHigh} register pairs.

\item Write 0x1 to \field{QueueReady}.
\end{enumerate}

\subsubsection{Notifying The Device}\label{sec:Virtio Transport Options / Virtio Over MMIO / MMIO-specific Initialization And Device Operation / Notifying The Device}

The driver notifies the device about new buffers being available in
a queue by writing the index of the updated queue to \field{QueueNotify}.

\subsubsection{Notifications From The Device}\label{sec:Virtio Transport Options / Virtio Over MMIO / MMIO-specific Initialization And Device Operation / Notifications From The Device}

The memory mapped virtio device is using a single, dedicated
interrupt signal, which is asserted when at least one of the
bits described in the description of \field{InterruptStatus}
is set. This is how the device notifies the
driver about a new used buffer being available in the queue
or about a change in the device configuration.

\drivernormative{\paragraph}{Notifications From The Device}{Virtio Transport Options / Virtio Over MMIO / MMIO-specific Initialization And Device Operation / Notifications From The Device}
After receiving an interrupt, the driver MUST read
\field{InterruptStatus} to check what caused the interrupt
(see the register description). After the interrupt is handled,
the driver MUST acknowledge it by writing a bit mask
corresponding to the handled events to the InterruptACK register.

\subsection{Legacy interface}\label{sec:Virtio Transport Options / Virtio Over MMIO / Legacy interface}

The legacy MMIO transport used page-based addressing, resulting
in a slightly different control register layout, the device
initialization and the virtual queue configuration procedure.

Table \ref{tab:Virtio Trasport Options / Virtio Over MMIO / MMIO Device Legacy Register Layout} 
presents control registers layout, omitting
descriptions of registers which did not change their function
nor behaviour:

\begin{longtable}{p{0.2\textwidth}p{0.7\textwidth}}
  \caption {MMIO Device Legacy Register Layout}
  \label{tab:Virtio Trasport Options / Virtio Over MMIO / MMIO Device Legacy Register Layout} \\
  \hline
  \mmioreg{Name}{Function}{Offset from base}{Direction}{Description} 
  \hline 
  \hline 
  \endfirsthead
  \hline
  \mmioreg{Name}{Function}{Offset from the base}{Direction}{Description} 
  \hline 
  \hline 
  \endhead
  \endfoot
  \endlastfoot
  \mmioreg{MagicValue}{Magic value}{0x000}{R}{}
  \hline
  \mmioreg{Version}{Device version number}{0x004}{R}{Legacy device returns value 0x1.}
  \hline
  \mmioreg{DeviceID}{Virtio Subsystem Device ID}{0x008}{R}{}
  \hline
  \mmioreg{VendorID}{Virtio Subsystem Vendor ID}{0x00c}{R}{}
  \hline
  \mmioreg{HostFeatures}{Flags representing features the device supports}{0x010}{R}{}
  \hline
  \mmioreg{HostFeaturesSel}{Device (host) features word selection.}{0x014}{W}{}
  \hline
  \mmioreg{GuestFeatures}{Flags representing device features understood and activated by the driver}{0x020}{W}{}
  \hline
  \mmioreg{GuestFeaturesSel}{Activated (guest) features word selection}{0x024}{W}{}
  \hline 
  \mmioreg{GuestPageSize}{Guest page size}{0x028}{W}{%
    The driver writes the guest page size in bytes to the
    register during initialization, before any queues are used.
    This value should be a power of 2 and is used by the device to
    calculate the Guest address of the first queue page
    (see QueuePFN).
  }
  \hline
  \mmioreg{QueueSel}{Virtual queue index}{0x030}{W}{%
    Writing to this register selects the virtual queue that the
    following operations on the \field{QueueNumMax}, \field{QueueNum}, \field{QueueAlign}
    and \field{QueuePFN} registers apply to. The index
    number of the first queue is zero (0x0). 
.
  }
  \hline
  \mmioreg{QueueNumMax}{Maximum virtual queue size}{0x034}{R}{%
    Reading from the register returns the maximum size of the queue
    the device is ready to process or zero (0x0) if the queue is not
    available. This applies to the queue selected by writing to
    \field{QueueSel} and is allowed only when \field{QueuePFN} is set to zero
    (0x0), so when the queue is not actively used.
  }
  \hline
  \mmioreg{QueueNum}{Virtual queue size}{0x038}{W}{%
    Queue size is the number of elements in the queue, therefore size
    of the descriptor table and both available and used rings.
    Writing to this register notifies the device what size of the
    queue the driver will use. This applies to the queue selected by
    writing to \field{QueueSel}.
  }
  \hline
  \mmioreg{QueueAlign}{Used Ring alignment in the virtual queue}{0x03c}{W}{%
    Writing to this register notifies the device about alignment
    boundary of the Used Ring in bytes. This value should be a power
    of 2 and applies to the queue selected by writing to \field{QueueSel}.
  }
  \hline
  \mmioreg{QueuePFN}{Guest physical page number of the virtual queue}{0x040}{RW}{%
    Writing to this register notifies the device about location of the
    virtual queue in the Guest's physical address space. This value
    is the index number of a page starting with the queue
    Descriptor Table. Value zero (0x0) means physical address zero
    (0x00000000) and is illegal. When the driver stops using the
    queue it writes zero (0x0) to this register.
    Reading from this register returns the currently used page
    number of the queue, therefore a value other than zero (0x0)
    means that the queue is in use.
    Both read and write accesses apply to the queue selected by
    writing to \field{QueueSel}.
  }
  \hline
  \mmioreg{QueueNotify}{Queue notifier}{0x050}{W}{}
  \hline
  \mmioreg{InterruptStatus}{Interrupt status}{0x60}{R}{}
  \hline
  \mmioreg{InterruptACK}{Interrupt acknowledge}{0x064}{W}{}
  \hline
  \mmioreg{Status}{Device status}{0x070}{RW}{%
    Reading from this register returns the current device status
    flags.
    Writing non-zero values to this register sets the status flags,
    indicating the OS/driver progress. Writing zero (0x0) to this
    register triggers a device reset. The device
    sets \field{QueuePFN} to zero (0x0) for all queues in the device.
    Also see \ref{sec:General Initialization And Device Operation / Device Initialization}~\nameref{sec:General Initialization And Device Operation / Device Initialization}.
  }
  \hline
  \mmioreg{Config}{Configuration space}{0x100+}{RW}{}
  \hline
\end{longtable}

The virtual queue page size is defined by writing to \field{GuestPageSize},
as written by the guest. The driver does this before the
virtual queues are configured.

The virtual queue layout follows
p. \ref{sec:Basic Facilities of a Virtio Device / Virtqueues / Legacy Interfaces: A Note on Virtqueue Layout}~\nameref{sec:Basic Facilities of a Virtio Device / Virtqueues / Legacy Interfaces: A Note on Virtqueue Layout},
with the alignment defined in \field{QueueAlign}.

The virtual queue is configured as follows:
\begin{enumerate}
\item Select the queue writing its index (first queue is 0) to
   \field{QueueSel}.

\item Check if the queue is not already in use: read \field{QueuePFN},
   expecting a returned value of zero (0x0).

\item Read maximum queue size (number of elements) from
   \field{QueueNumMax}. If the returned value is zero (0x0) the
   queue is not available.

\item Allocate and zero the queue pages in contiguous virtual
   memory, aligning the Used Ring to an optimal boundary (usually
   page size). The driver should choose a queue size smaller than or
   equal to \field{QueueNumMax}.

\item Notify the device about the queue size by writing the size to
   \field{QueueNum}.

\item Notify the device about the used alignment by writing its value
   in bytes to \field{QueueAlign}.

\item Write the physical number of the first page of the queue to
   the \field{QueuePFN} register.
\end{enumerate}

Notification mechanisms did not change.

\section{Virtio Over Channel I/O}\label{sec:Virtio Transport Options / Virtio Over Channel I/O}

S/390 based virtual machines support neither PCI nor MMIO, so a
different transport is needed there.

virtio-ccw uses the standard channel I/O based mechanism used for
the majority of devices on S/390. A virtual channel device with a
special control unit type acts as proxy to the virtio device
(similar to the way virtio-pci uses a PCI device) and
configuration and operation of the virtio device is accomplished
(mostly) via channel commands. This means virtio devices are
discoverable via standard operating system algorithms, and adding
virtio support is mainly a question of supporting a new control
unit type.

As the S/390 is a big endian machine, the data structures transmitted
via channel commands are big-endian: this is made clear by use of
the types be16, be32 and be64.

\subsection{Basic Concepts}\label{sec:Virtio Transport Options / Virtio over channel I/O / Basic Concepts}

As a proxy device, virtio-ccw uses a channel-attached I/O control
unit with a special control unit type (0x3832) and a control unit
model corresponding to the attached virtio device's subsystem
device ID, accessed via a virtual I/O subchannel and a virtual
channel path of type 0x32. This proxy device is discoverable via
normal channel subsystem device discovery (usually a STORE
SUBCHANNEL loop) and answers to the basic channel commands:

\begin{itemize}
\item NO-OPERATION (0x03)
\item BASIC SENSE (0x04)
\item TRANSFER IN CHANNEL (0x08)
\item SENSE ID (0xe4)
\end{itemize}

For a virtio-ccw proxy device, SENSE ID will return the following
information:

\begin{tabular}{ |l|l|l| }
\hline
Bytes & Description & Contents \\
\hline \hline
0     & reserved              & 0xff \\
\hline
1-2   & control unit type     & 0x3832 \\
\hline
3     & control unit model    & <virtio device id> \\
\hline
4-5   & device type           & zeroes (unset) \\
\hline
6     & device model          & zeroes (unset) \\
\hline
7-255 & extended SenseId data & zeroes (unset) \\
\hline
\end{tabular}

In addition to the basic channel commands, virtio-ccw defines a
set of channel commands related to configuration and operation of
virtio:

\begin{lstlisting}
#define CCW_CMD_SET_VQ 0x13
#define CCW_CMD_VDEV_RESET 0x33
#define CCW_CMD_SET_IND 0x43
#define CCW_CMD_SET_CONF_IND 0x53
#define CCW_CMD_SET_IND_ADAPTER 0x73
#define CCW_CMD_READ_FEAT 0x12
#define CCW_CMD_WRITE_FEAT 0x11
#define CCW_CMD_READ_CONF 0x22
#define CCW_CMD_WRITE_CONF 0x21
#define CCW_CMD_WRITE_STATUS 0x31
#define CCW_CMD_READ_VQ_CONF 0x32
#define CCW_CMD_SET_VIRTIO_REV 0x83
\end{lstlisting}

\devicenormative{\subsubsection}{Basic Concepts}{Virtio Transport Options / Virtio over channel I/O / Basic Concepts}

The virtio-ccw device acts like a normal channel device, as specified
in \hyperref[intro:S390 PoP]{[S390 PoP]} and \hyperref[intro:S390 Common I/O]{[S390 Common I/O]}. In particular:

\begin{itemize}
\item A device MUST post a unit check with command reject for any command
  it does not support.

\item If a driver did not suppress length checks for a channel command,
  the device MUST present a subchannel status as detailed in the
  architecture when the actual length did not match the expected length.

\item If a driver did suppress length checks for a channel command, the
  device MUST present a check condition if the transmitted data does
  not contain enough data to process the command. If the driver submitted
  a buffer that was too long, the device SHOULD accept the command.
\end{itemize}

\drivernormative{\subsubsection}{Basic Concepts}{Virtio Transport Options / Virtio over channel I/O / Basic Concepts}

A driver for virtio-ccw devices MUST check for a control unit
type of 0x3832 and MUST ignore the device type and model.

A driver SHOULD attempt to provide the correct length in a channel
command even if it suppresses length checks for that command.

\subsection{Device Initialization}\label{sec:Virtio Transport Options / Virtio over channel I/O / Device Initialization}

virtio-ccw uses several channel commands to set up a device.

\subsubsection{Setting the Virtio Revision}\label{sec:Virtio Transport Options / Virtio over channel I/O / Device Initialization / Setting the Virtio Revision}

CCW_CMD_SET_VIRTIO_REV is issued by the driver to set the revision of
the virtio-ccw transport it intends to drive the device with. It uses the
following communication structure:

\begin{lstlisting}
struct virtio_rev_info {
        be16 revision;
        be16 length;
        u8 data[];
};
\end{lstlisting}

\field{revision} contains the desired revision id, \field{length} the length of the
data portion and \field{data} revision-dependent additional desired options.

The following values are supported:

\begin{tabular}{ |l|l|l|l| }
\hline
\field{revision} & \field{length} & \field{data}      & remarks \\
\hline \hline
0        & 0      & <empty>   & legacy interface; transitional devices only \\
\hline
1        & 0      & <empty>   & Virtio 1.0 \\
\hline
2-n      &        &           & reserved for later revisions \\
\hline
\end{tabular}

Note that a change in the virtio standard does not necessarily
correspond to a change in the virtio-ccw revision.

\devicenormative{\paragraph}{Setting the Virtio Revision}{Virtio Transport Options / Virtio over channel I/O / Device Initialization / Setting the Virtio Revision}

A device MUST post a unit check with command reject for any \field{revision}
it does not support. For any invalid combination of \field{revision}, \field{length}
and \field{data}, it MUST post a unit check with command reject as well. A
non-transitional device MUST reject revision id 0.

A device MUST answer with command reject to any virtio-ccw specific
channel command that is not contained in the revision selected by the
driver.

A device MUST answer with command reject to any attempt to select a different revision
after a revision has been successfully selected by the driver.

A device MUST treat the revision as unset from the time the associated
subchannel has been enabled until a revision has been successfully set
by the driver. This implies that revisions are not persistent across
disabling and enabling of the associated subchannel.

\drivernormative{\paragraph}{Setting the Virtio Revision}{Virtio Transport Options / Virtio over channel I/O / Device Initialization / Setting the Virtio Revision}

A driver SHOULD start with trying to set the highest revision it
supports and continue with lower revisions if it gets a command reject.

A driver MUST NOT issue any other virtio-ccw specific channel commands
prior to setting the revision.

After a revision has been successfully selected by the driver, it
MUST NOT attempt to select a different revision.

\paragraph{Legacy Interfaces: A Note on Setting the Virtio Revision}\label{sec:Virtio Transport Options / Virtio over channel I/O / Device Initialization / Setting the Virtio Revision / Legacy Interfaces: A Note on Setting the Virtio Revision}

A legacy device will not support the CCW_CMD_SET_VIRTIO_REV and answer
with a command reject. A non-transitional driver MUST stop trying to
operate this device in that case. A transitional driver MUST operate
the device as if it had been able to set revision 0.

A legacy driver will not issue the CCW_CMD_SET_VIRTIO_REV prior to
issuing other virtio-ccw specific channel commands. A non-transitional
device therefore MUST answer any such attempts with a command reject.
A transitional device MUST assume in this case that the driver is a
legacy driver and continue as if the driver selected revision 0. This
implies that the device MUST reject any command not valid for revision
0, including a subsequent CCW_CMD_SET_VIRTIO_REV.

\subsubsection{Configuring a Virtqueue}\label{sec:Virtio Transport Options / Virtio over channel I/O / Device Initialization / Configuring a Virtqueue}

CCW_CMD_READ_VQ_CONF is issued by the driver to obtain information
about a queue. It uses the following structure for communicating:

\begin{lstlisting}
struct vq_config_block {
        be16 index;
        be16 max_num;
};
\end{lstlisting}

The requested number of buffers for queue \field{index} is returned in
\field{max_num}.

Afterwards, CCW_CMD_SET_VQ is issued by the driver to inform the
device about the location used for its queue. The transmitted
structure is

\begin{lstlisting}
struct vq_info_block {
        be64 desc;
        be32 res0;
        be16 index;
        be16 num;
        be64 avail;
        be64 used;
};
\end{lstlisting}

\field{desc}, \field{avail} and \field{used} contain the guest addresses for the descriptor table,
available ring and used ring for queue \field{index}, respectively. The actual
virtqueue size (number of allocated buffers) is transmitted in \field{num}.

\devicenormative{\paragraph}{Configuring a Virtqueue}{Virtio Transport Options / Virtio over channel I/O / Device Initialization / Configuring a Virtqueue}

\field{res0} is reserved and MUST be ignored by the device.

\paragraph{Legacy Interface: A Note on Configuring a Virtqueue}\label{sec:Virtio Transport Options / Virtio over channel I/O / Device Initialization / Configuring a Virtqueue / Legacy Interface: A Note on Configuring a Virtqueue}

For a legacy driver or for a driver that selected revision 0,
CCW_CMD_SET_VQ uses the following communication block:

\begin{lstlisting}
struct vq_info_block_legacy {
        be64 queue;
        be32 align;
        be16 index;
        be16 num;
};
\end{lstlisting}

\field{queue} contains the guest address for queue \field{index}, \field{num} the number of buffers
and \field{align} the alignment. The queue layout follows \ref{sec:Basic Facilities of a Virtio Device / Virtqueues / Legacy Interfaces: A Note on Virtqueue Layout}~\nameref{sec:Basic Facilities of a Virtio Device / Virtqueues / Legacy Interfaces: A Note on Virtqueue Layout}.

\subsubsection{Communicating Status Information}\label{sec:Virtio Transport Options / Virtio over channel I/O / Device Initialization / Communicating Status Information}

The driver changes the status of a device via the
CCW_CMD_WRITE_STATUS command, which transmits an 8 bit status
value.

As described in
\ref{devicenormative:Basic Facilities of a Virtio Device / Feature Bits},
a device sometimes fails to set the \field{status} field: For example, it
might fail to accept the FEATURES_OK status bit during device initialization.

\drivernormative{\paragraph}{Communicating Status Information}{Virtio Transport Options / Virtio over channel I/O / Device Initialization / Communicating Status Information}

If the device posts a unit check with command reject in response to the
CCW_CMD_WRITE_STATUS command, the driver MUST assume that the device failed
to set the status and the \field{status} field retained its previous value.

\devicenormative{\paragraph}{Communicating Status Information}{Virtio Transport Options / Virtio over channel I/O / Device Initialization / Communicating Status Information}

If the device fails to set the \field{status} field to the value written by
the driver, the device MUST assure that the \field{status} field is left
unchanged and MUST post a unit check with command reject.

\subsubsection{Handling Device Features}\label{sec:Virtio Transport Options / Virtio over channel I/O / Device Initialization / Handling Device Features}

Feature bits are arranged in an array of 32 bit values, making
for a total of 8192 feature bits. Feature bits are in
little-endian byte order.

The CCW commands dealing with features use the following
communication block:

\begin{lstlisting}
struct virtio_feature_desc {
        le32 features;
        u8 index;
};
\end{lstlisting}

\field{features} are the 32 bits of features currently accessed, while
\field{index} describes which of the feature bit values is to be
accessed. No padding is added at the end of the structure, it is
exactly 5 bytes in length.

The guest obtains the device's device feature set via the
CCW_CMD_READ_FEAT command. The device stores the features at \field{index}
to \field{features}.

For communicating its supported features to the device, the driver
uses the CCW_CMD_WRITE_FEAT command, denoting a \field{features}/\field{index}
combination.

\subsubsection{Device Configuration}\label{sec:Virtio Transport Options / Virtio over channel I/O / Device Initialization / Device Configuration}

The device's configuration space is located in host memory.

To obtain information from the configuration space, the driver
uses CCW_CMD_READ_CONF, specifying the guest memory for the device
to write to.

For changing configuration information, the driver uses
CCW_CMD_WRITE_CONF, specifying the guest memory for the device to
read from.

In both cases, the complete configuration space is transmitted.  This
allows the driver to compare the new configuration space with the old
version, and keep a generation count internally whenever it changes.

\subsubsection{Setting Up Indicators}\label{sec:Virtio Transport Options / Virtio over channel I/O / Device Initialization / Setting Up Indicators}

In order to set up the indicator bits for host->guest notification,
the driver uses different channel commands depending on whether it
wishes to use traditional I/O interrupts tied to a subchannel or
adapter I/O interrupts for virtqueue notifications. For any given
device, the two mechanisms are mutually exclusive.

For the configuration change indicators, only a mechanism using
traditional I/O interrupts is provided, regardless of whether
traditional or adapter I/O interrupts are used for virtqueue
notifications.

\paragraph{Setting Up Classic Queue Indicators}\label{sec:Virtio Transport Options / Virtio over channel I/O / Device Initialization / Setting Up Indicators / Setting Up Classic Queue Indicators}

Indicators for notification via classic I/O interrupts are contained
in a 64 bit value per virtio-ccw proxy device.

To communicate the location of the indicator bits for host->guest
notification, the driver uses the CCW_CMD_SET_IND command,
pointing to a location containing the guest address of the
indicators in a 64 bit value.

If the driver has already set up two-staged queue indicators via the
CCW_CMD_SET_IND_ADAPTER command, the device MUST post a unit check
with command reject to any subsequent CCW_CMD_SET_IND command.

\paragraph{Setting Up Configuration Change Indicators}\label{sec:Virtio Transport Options / Virtio over channel I/O / Device Initialization / Setting Up Indicators / Setting Up Configuration Change Indicators}

Indicators for configuration change host->guest notification are
contained in a 64 bit value per virtio-ccw proxy device.

To communicate the location of the indicator bits used in the
configuration change host->guest notification, the driver issues the
CCW_CMD_SET_CONF_IND command, pointing to a location containing the
guest address of the indicators in a 64 bit value.

\paragraph{Setting Up Two-Stage Queue Indicators}\label{sec:Virtio Transport Options / Virtio over channel I/O / Device Initialization / Setting Up Indicators / Setting Up Two-Stage Queue Indicators}

Indicators for notification via adapter I/O interrupts consist of
two stages:
\begin{itemize}
\item a summary indicator byte covering the virtqueues for one or more
  virtio-ccw proxy devices
\item a set of contigous indicator bits for the virtqueues for a
  virtio-ccw proxy device
\end{itemize}

To communicate the location of the summary and queue indicator bits,
the driver uses the CCW_CMD_SET_IND_ADAPTER command with the following
payload:

\begin{lstlisting}
struct virtio_thinint_area {
        be64 summary_indicator;
        be64 indicator;
        be64 bit_nr;
        u8 isc;
} __attribute__ ((packed));
\end{lstlisting}

\field{summary_indicator} contains the guest address of the 8 bit summary
indicator.
\field{indicator} contains the guest address of an area wherein the indicators
for the devices are contained, starting at \field{bit_nr}, one bit per
virtqueue of the device. Bit numbers start at the left, i.e. the most
significant bit in the first byte is assigned the bit number 0.
\field{isc} contains the I/O interruption subclass to be used for the adapter
I/O interrupt. It MAY be different from the isc used by the proxy
virtio-ccw device's subchannel.
No padding is added at the end of the structure, it is exactly 25 bytes
in length.


\devicenormative{\subparagraph}{Setting Up Two-Stage Queue Indicators}{Virtio Transport Options / Virtio over channel I/O / Device Initialization / Setting Up Indicators / Setting Up Two-Stage Queue Indicators}
If the driver has already set up classic queue indicators via the
CCW_CMD_SET_IND command, the device MUST post a unit check with
command reject to any subsequent CCW_CMD_SET_IND_ADAPTER command.

\paragraph{Legacy Interfaces: A Note on Setting Up Indicators}\label{sec:Virtio Transport Options / Virtio over channel I/O / Device Initialization / Setting Up Indicators / Legacy Interfaces: A Note on Setting Up Indicators}

In some cases, legacy devices will only support classic queue indicators;
in that case, they will reject CCW_CMD_SET_IND_ADAPTER as they don't know that
command. Some legacy devices will support two-stage queue indicators, though,
and a driver will be able to successfully use CCW_CMD_SET_IND_ADAPTER to set
them up.

\subsection{Device Operation}\label{sec:Virtio Transport Options / Virtio over channel I/O / Device Operation}

\subsubsection{Host->Guest Notification}\label{sec:Virtio Transport Options / Virtio over channel I/O / Device Operation / Host->Guest Notification}

There are two modes of operation regarding host->guest notification,
classic I/O interrupts and adapter I/O interrupts. The mode to be
used is determined by the driver by using CCW_CMD_SET_IND respectively
CCW_CMD_SET_IND_ADAPTER to set up queue indicators.

For configuration changes, the driver always uses classic I/O
interrupts.

\paragraph{Notification via Classic I/O Interrupts}\label{sec:Virtio Transport Options / Virtio over channel I/O / Device Operation / Host->Guest Notification / Notification via Classic I/O Interrupts}

If the driver used the CCW_CMD_SET_IND command to set up queue
indicators, the device will use classic I/O interrupts for
host->guest notification about virtqueue activity.

For notifying the driver of virtqueue buffers, the device sets the
corresponding bit in the guest-provided indicators. If an
interrupt is not already pending for the subchannel, the device
generates an unsolicited I/O interrupt.

If the device wants to notify the driver about configuration
changes, it sets bit 0 in the configuration indicators and
generates an unsolicited I/O interrupt, if needed. This also
applies if adapter I/O interrupts are used for queue notifications.

\paragraph{Notification via Adapter I/O Interrupts}\label{sec:Virtio Transport Options / Virtio over channel I/O / Device Operation / Host->Guest Notification / Notification via Adapter I/O Interrupts}

If the driver used the CCW_CMD_SET_IND_ADAPTER command to set up
queue indicators, the device will use adapter I/O interrupts for
host->guest notification about virtqueue activity.

For notifying the driver of virtqueue buffers, the device sets the
bit in the guest-provided indicator area at the corresponding offset.
The guest-provided summary indicator is set to 0x01. An adapter I/O
interrupt for the corresponding interruption subclass is generated.

The recommended way to process an adapter I/O interrupt by the driver
is as follows:

\begin{itemize}
\item Process all queue indicator bits associated with the summary indicator.
\item Clear the summary indicator, performing a synchronization (memory
barrier) afterwards.
\item Process all queue indicator bits associated with the summary indicator
again.
\end{itemize}

\devicenormative{\subparagraph}{Notification via Adapter I/O Interrupts}{Virtio Transport Options / Virtio over channel I/O / Device Operation / Host->Guest Notification / Notification via Adapter I/O Interrupts}

The device SHOULD only generate an adapter I/O interrupt if the
summary indicator had not been set prior to notification.

\drivernormative{\subparagraph}{Notification via Adapter I/O Interrupts}{Virtio Transport Options / Virtio over channel I/O / Device Operation / Host->Guest Notification / Notification via Adapter I/O Interrupts}
The driver
MUST clear the summary indicator after receiving an adapter I/O
interrupt before it processes the queue indicators.

\paragraph{Legacy Interfaces: A Note on Host->Guest Notification}\label{sec:Virtio Transport Options / Virtio over channel I/O / Device Operation / Host->Guest Notification / Legacy Interfaces: A Note on Host->Guest Notification}

As legacy devices and drivers support only classic queue indicators,
host->guest notification will always be done via classic I/O interrupts.

\subsubsection{Guest->Host Notification}\label{sec:Virtio Transport Options / Virtio over channel I/O / Device Operation / Guest->Host Notification}

For notifying the device of virtqueue buffers, the driver
unfortunately can't use a channel command (the asynchronous
characteristics of channel I/O interact badly with the host block
I/O backend). Instead, it uses a diagnose 0x500 call with subcode
3 specifying the queue, as follows:

\begin{tabular}{ |l|l|l| }
\hline
GPR  &   Input Value     & Output Value \\
\hline \hline
  1   &       0x3         &              \\
\hline
  2   &  Subchannel ID    & Host Cookie  \\
\hline
  3   & Virtqueue number  &              \\
\hline
  4   &   Host Cookie     &              \\
\hline
\end{tabular}

\devicenormative{\paragraph}{Guest->Host Notification}{Virtio Transport Options / Virtio over channel I/O / Device Operation / Guest->Host Notification}
The device MUST ignore bits 0-31 (counting from the left) of GPR2.
This aligns passing the subchannel ID with the way it is passed
for the existing I/O instructions.

The device MAY return a 64-bit host cookie in GPR2 to speed up the
notification execution.

\drivernormative{\paragraph}{Guest->Host Notification}{Virtio Transport Options / Virtio over channel I/O / Device Operation / Guest->Host Notification}

For each notification, the driver SHOULD use GPR4 to pass the host cookie received in GPR2 from the previous notication.

\begin{note}
For example:
\begin{lstlisting}
info->cookie = do_notify(schid,
                         virtqueue_get_queue_index(vq),
                         info->cookie);
\end{lstlisting}
\end{note}

\subsubsection{Resetting Devices}\label{sec:Virtio Transport Options / Virtio over channel I/O / Device Operation / Resetting Devices}

In order to reset a device, a driver sends the
CCW_CMD_VDEV_RESET command.


\chapter{Device Types}\label{sec:Device Types}

On top of the queues, config space and feature negotiation facilities
built into virtio, several devices are defined.

The following device IDs are used to identify different types of virtio
devices.  Some device IDs are reserved for devices which are not currently
defined in this standard.

Discovering what devices are available and their type is bus-dependent.

\begin{tabular} { |l|c| }
\hline
Device ID  &  Virtio Device    \\
\hline \hline
0          & reserved (invalid) \\
\hline
1          &   network card     \\
\hline
2          &   block device     \\
\hline
3          &      console       \\
\hline
4          &  entropy source    \\
\hline
5          & memory ballooning (traditional)  \\
\hline
6          &     ioMemory       \\
\hline
7          &       rpmsg        \\
\hline
8          &     SCSI host      \\
\hline
9          &   9P transport     \\
\hline
10         &   mac80211 wlan    \\
\hline
11         &   rproc serial     \\
\hline
12         &   virtio CAIF      \\
\hline
13         &  memory balloon    \\
\hline
16         &   GPU device       \\
\hline
17         &   Timer/Clock device \\
\hline
18         &   Input device \\
\hline
\end{tabular}

Some of the devices above are unspecified by this document,
because they are seen as immature or especially niche.  Be warned
that some are only specified by the sole existing implementation;
they could become part of a future specification, be abandoned
entirely, or live on outside this standard.  We shall speak of
them no further.

\section{Network Device}\label{sec:Device Types / Network Device}

The virtio network device is a virtual ethernet card, and is the
most complex of the devices supported so far by virtio. It has
enhanced rapidly and demonstrates clearly how support for new
features are added to an existing device. Empty buffers are
placed in one virtqueue for receiving packets, and outgoing
packets are enqueued into another for transmission in that order.
A third command queue is used to control advanced filtering
features.

\subsection{Device ID}\label{sec:Device Types / Network Device / Device ID}

 1

\subsection{Virtqueues}\label{sec:Device Types / Network Device / Virtqueues}

\begin{description}
\item[0] receiveq1
\item[1] transmitq1
\item[\ldots]
\item[2N] receiveqN
\item[2N+1] transmitqN
\item[2N+2] controlq
\end{description}

 N=1 if VIRTIO_NET_F_MQ is not negotiated, otherwise N is set by
 \field{max_virtqueue_pairs}.

 controlq only exists if VIRTIO_NET_F_CTRL_VQ set.

\subsection{Feature bits}\label{sec:Device Types / Network Device / Feature bits}

\begin{description}
\item[VIRTIO_NET_F_CSUM (0)] Device handles packets with partial checksum.   This 
  ``checksum offload'' is a common feature on modern network cards.

\item[VIRTIO_NET_F_GUEST_CSUM (1)] Driver handles packets with partial checksum.

\item[VIRTIO_NET_F_CTRL_GUEST_OFFLOADS (2)] Control channel offloads
        reconfiguration support.

\item[VIRTIO_NET_F_MAC (5)] Device has given MAC address.

\item[VIRTIO_NET_F_GUEST_TSO4 (7)] Driver can receive TSOv4.

\item[VIRTIO_NET_F_GUEST_TSO6 (8)] Driver can receive TSOv6.

\item[VIRTIO_NET_F_GUEST_ECN (9)] Driver can receive TSO with ECN.

\item[VIRTIO_NET_F_GUEST_UFO (10)] Driver can receive UFO.

\item[VIRTIO_NET_F_HOST_TSO4 (11)] Device can receive TSOv4.

\item[VIRTIO_NET_F_HOST_TSO6 (12)] Device can receive TSOv6.

\item[VIRTIO_NET_F_HOST_ECN (13)] Device can receive TSO with ECN.

\item[VIRTIO_NET_F_HOST_UFO (14)] Device can receive UFO.

\item[VIRTIO_NET_F_MRG_RXBUF (15)] Driver can merge receive buffers.

\item[VIRTIO_NET_F_STATUS (16)] Configuration status field is
    available.

\item[VIRTIO_NET_F_CTRL_VQ (17)] Control channel is available.

\item[VIRTIO_NET_F_CTRL_RX (18)] Control channel RX mode support.

\item[VIRTIO_NET_F_CTRL_VLAN (19)] Control channel VLAN filtering.

\item[VIRTIO_NET_F_GUEST_ANNOUNCE(21)] Driver can send gratuitous
    packets.

\item[VIRTIO_NET_F_MQ(22)] Device supports multiqueue with automatic
    receive steering.

\item[VIRTIO_NET_F_CTRL_MAC_ADDR(23)] Set MAC address through control
    channel.
\end{description}

\subsubsection{Feature bit requirements}\label{sec:Device Types / Network Device / Feature bits / Feature bit requirements}

Some networking feature bits require other networking feature bits
(see \ref{drivernormative:Basic Facilities of a Virtio Device / Feature Bits}):

\begin{description}
\item[VIRTIO_NET_F_GUEST_TSO4] Requires VIRTIO_NET_F_GUEST_CSUM.
\item[VIRTIO_NET_F_GUEST_TSO6] Requires VIRTIO_NET_F_GUEST_CSUM.
\item[VIRTIO_NET_F_GUEST_ECN] Requires VIRTIO_NET_F_GUEST_TSO4 or VIRTIO_NET_F_GUEST_TSO6.
\item[VIRTIO_NET_F_GUEST_UFO] Requires VIRTIO_NET_F_GUEST_CSUM.

\item[VIRTIO_NET_F_HOST_TSO4] Requires VIRTIO_NET_F_CSUM.
\item[VIRTIO_NET_F_HOST_TSO6] Requires VIRTIO_NET_F_CSUM.
\item[VIRTIO_NET_F_HOST_ECN] Requires VIRTIO_NET_F_HOST_TSO4 or VIRTIO_NET_F_HOST_TSO6.
\item[VIRTIO_NET_F_HOST_UFO] Requires VIRTIO_NET_F_CSUM.

\item[VIRTIO_NET_F_CTRL_RX] Requires VIRTIO_NET_F_CTRL_VQ.
\item[VIRTIO_NET_F_CTRL_VLAN] Requires VIRTIO_NET_F_CTRL_VQ.
\item[VIRTIO_NET_F_GUEST_ANNOUNCE] Requires VIRTIO_NET_F_CTRL_VQ.
\item[VIRTIO_NET_F_MQ] Requires VIRTIO_NET_F_CTRL_VQ.
\item[VIRTIO_NET_F_CTRL_MAC_ADDR] Requires VIRTIO_NET_F_CTRL_VQ.
\end{description}

\subsubsection{Legacy Interface: Feature bits}\label{sec:Device Types / Network Device / Feature bits / Legacy Interface: Feature bits}
\begin{description}
\item[VIRTIO_NET_F_GSO (6)] Device handles packets with any GSO type.
\end{description}

This was supposed to indicate segmentation offload support, but
upon further investigation it became clear that multiple bits
were needed.

\subsection{Device configuration layout}\label{sec:Device Types / Network Device / Device configuration layout}
\label{sec:Device Types / Block Device / Feature bits / Device configuration layout}

Three driver-read-only configuration fields are currently defined. The \field{mac} address field
always exists (though is only valid if VIRTIO_NET_F_MAC is set), and
\field{status} only exists if VIRTIO_NET_F_STATUS is set. Two
read-only bits (for the driver) are currently defined for the status field:
VIRTIO_NET_S_LINK_UP and VIRTIO_NET_S_ANNOUNCE.

\begin{lstlisting}
#define VIRTIO_NET_S_LINK_UP     1
#define VIRTIO_NET_S_ANNOUNCE    2
\end{lstlisting}

The following driver-read-only field, \field{max_virtqueue_pairs} only exists if
VIRTIO_NET_F_MQ is set. This field specifies the maximum number
of each of transmit and receive virtqueues (receiveq1\ldots receiveqN
and transmitq1\ldots transmitqN respectively) that can be configured once VIRTIO_NET_F_MQ
is negotiated.

\begin{lstlisting}
struct virtio_net_config {
        u8 mac[6];
        le16 status;
        le16 max_virtqueue_pairs;
};
\end{lstlisting}

\devicenormative{\subsubsection}{Device configuration layout}{Device Types / Network Device / Device configuration layout}

The device MUST set \field{max_virtqueue_pairs} to between 1 and 0x8000 inclusive,
if it offers VIRTIO_NET_F_MQ.

\drivernormative{\subsubsection}{Device configuration layout}{Device Types / Network Device / Device configuration layout}

A driver SHOULD negotiate VIRTIO_NET_F_MAC if the device offers it.
If the driver negotiates the VIRTIO_NET_F_MAC feature, the driver MUST set
the physical address of the NIC to \field{mac}.  Otherwise, it SHOULD
use a locally-administered MAC address (see \hyperref[intro:IEEE 802]{IEEE 802},
``9.2 48-bit universal LAN MAC addresses'').

If the driver does not negotiate the VIRTIO_NET_F_STATUS feature, it SHOULD
assume the link is active, otherwise it SHOULD read the link status from
the bottom bit of \field{status}.

\subsubsection{Legacy Interface: Device configuration layout}\label{sec:Device Types / Network Device / Device configuration layout / Legacy Interface: Device configuration layout}
\label{sec:Device Types / Block Device / Feature bits / Device configuration layout / Legacy Interface: Device configuration layout}
When using the legacy interface, transitional devices and drivers
MUST format \field{status} and
\field{max_virtqueue_pairs} in struct virtio_net_config
according to the native endian of the guest rather than
(necessarily when not using the legacy interface) little-endian.

When using the legacy interface, \field{mac} is driver-writable
which provided a way for drivers to update the MAC without
negotiating VIRTIO_NET_F_CTRL_MAC_ADDR.

\subsection{Device Initialization}\label{sec:Device Types / Network Device / Device Initialization}

A driver would perform a typical initialization routine like so:

\begin{enumerate}
\item Identify and initialize the receive and
  transmission virtqueues, up to N of each kind. If
  VIRTIO_NET_F_MQ feature bit is negotiated,
  N=\field{max_virtqueue_pairs}, otherwise identify N=1.

\item If the VIRTIO_NET_F_CTRL_VQ feature bit is negotiated,
  identify the control virtqueue.

\item Fill the receive queues with buffers: see \ref{sec:Device Types / Network Device / Device Operation / Setting Up Receive Buffers}.

\item Even with VIRTIO_NET_F_MQ, only receiveq1, transmitq1 and
  controlq are used by default.  The driver would send the
  VIRTIO_NET_CTRL_MQ_VQ_PAIRS_SET command specifying the
  number of the transmit and receive queues to use.

\item If the VIRTIO_NET_F_MAC feature bit is set, the configuration
  space \field{mac} entry indicates the ``physical'' address of the
  network card, otherwise the driver would typically generate a random
  local MAC address.

\item If the VIRTIO_NET_F_STATUS feature bit is negotiated, the link
  status comes from the bottom bit of \field{status}.
  Otherwise, the driver assumes it's active.

\item A performant driver would indicate that it will generate checksumless
  packets by negotating the VIRTIO_NET_F_CSUM feature.

\item If that feature is negotiated, a driver can use TCP or UDP
  segmentation offload by negotiating the VIRTIO_NET_F_HOST_TSO4 (IPv4
  TCP), VIRTIO_NET_F_HOST_TSO6 (IPv6 TCP) and VIRTIO_NET_F_HOST_UFO
  (UDP fragmentation) features.

\item The converse features are also available: a driver can save
  the virtual device some work by negotiating these features.\note{For example, a network packet transported between two guests on
the same system might not need checksumming at all, nor segmentation,
if both guests are amenable.}
   The VIRTIO_NET_F_GUEST_CSUM feature indicates that partially
  checksummed packets can be received, and if it can do that then
  the VIRTIO_NET_F_GUEST_TSO4, VIRTIO_NET_F_GUEST_TSO6,
  VIRTIO_NET_F_GUEST_UFO and VIRTIO_NET_F_GUEST_ECN are the input
  equivalents of the features described above.
  See \ref{sec:Device Types / Network Device / Device Operation /
Setting Up Receive Buffers}~\nameref{sec:Device Types / Network
Device / Device Operation / Setting Up Receive Buffers} and
\ref{sec:Device Types / Network Device / Device Operation /
Processing of Incoming Packets}~\nameref{sec:Device Types /
Network Device / Device Operation / Processing of Incoming Packets} below.
\end{enumerate}

A truly minimal driver would only accept VIRTIO_NET_F_MAC and ignore
everything else.

\subsection{Device Operation}\label{sec:Device Types / Network Device / Device Operation}

Packets are transmitted by placing them in the
transmitq1\ldots transmitqN, and buffers for incoming packets are
placed in the receiveq1\ldots receiveqN. In each case, the packet
itself is preceded by a header:

\begin{lstlisting}
struct virtio_net_hdr {
#define VIRTIO_NET_HDR_F_NEEDS_CSUM    1
        u8 flags;
#define VIRTIO_NET_HDR_GSO_NONE        0
#define VIRTIO_NET_HDR_GSO_TCPV4       1
#define VIRTIO_NET_HDR_GSO_UDP         3
#define VIRTIO_NET_HDR_GSO_TCPV6       4
#define VIRTIO_NET_HDR_GSO_ECN      0x80
        u8 gso_type;
        le16 hdr_len;
        le16 gso_size;
        le16 csum_start;
        le16 csum_offset;
        le16 num_buffers;
};
\end{lstlisting}

The controlq is used to control device features such as
filtering.

\subsubsection{Legacy Interface: Device Operation}\label{sec:Device Types / Network Device / Device Operation / Legacy Interface: Device Operation}
When using the legacy interface, transitional devices and drivers
MUST format the fields in struct virtio_net_hdr
according to the native endian of the guest rather than
(necessarily when not using the legacy interface) little-endian.

The legacy driver only presented \field{num_buffers} in the struct virtio_net_hdr
when VIRTIO_NET_F_MRG_RXBUF was not negotiated; without that feature the
structure was 2 bytes shorter.

When using the legacy interface, the driver SHOULD ignore the
\field{len} value in used ring entries for the transmit queues
and the controlq queue.
\begin{note}
Historically, some devices put
the total descriptor length there, even though no data was
actually written.
\end{note}

\subsubsection{Packet Transmission}\label{sec:Device Types / Network Device / Device Operation / Packet Transmission}

Transmitting a single packet is simple, but varies depending on
the different features the driver negotiated.

\begin{enumerate}
\item The driver can send a completely checksummed packet.  In this case,
  \field{flags} will be zero, and \field{gso_type} will be VIRTIO_NET_HDR_GSO_NONE.

\item If the driver negotiated VIRTIO_NET_F_CSUM, it can skip
  checksumming the packet:
  \begin{itemize}
  \item \field{flags} has the VIRTIO_NET_HDR_F_NEEDS_CSUM set,

  \item \field{csum_start} is set to the offset within the packet to begin checksumming,
    and

  \item \field{csum_offset} indicates how many bytes after the csum_start the
    new (16 bit ones' complement) checksum is placed by the device.

  \item The TCP checksum field in the packet is set to the sum
    of the TCP pseudo header, so that replacing it by the ones'
    complement checksum of the TCP header and body will give the
    correct result.
  \end{itemize}

\begin{note}
For example, consider a partially checksummed TCP (IPv4) packet.
It will have a 14 byte ethernet header and 20 byte IP header
followed by the TCP header (with the TCP checksum field 16 bytes
into that header). \field{csum_start} will be 14+20 = 34 (the TCP
checksum includes the header), and \field{csum_offset} will be 16.
\end{note}

\item If the driver negotiated
  VIRTIO_NET_F_HOST_TSO4, TSO6 or UFO, and the packet requires
  TCP segmentation or UDP fragmentation, then \field{gso_type}
  is set to VIRTIO_NET_HDR_GSO_TCPV4, TCPV6 or UDP.
  (Otherwise, it is set to VIRTIO_NET_HDR_GSO_NONE). In this
  case, packets larger than 1514 bytes can be transmitted: the
  metadata indicates how to replicate the packet header to cut it
  into smaller packets. The other gso fields are set:

  \begin{itemize}
  \item \field{hdr_len} is a hint to the device as to how much of the header
    needs to be kept to copy into each packet, usually set to the
    length of the headers, including the transport header\footnote{Due to various bugs in implementations, this field is not useful
as a guarantee of the transport header size.
}.

  \item \field{gso_size} is the maximum size of each packet beyond that
    header (ie. MSS).

  \item If the driver negotiated the VIRTIO_NET_F_HOST_ECN feature,
    the VIRTIO_NET_HDR_GSO_ECN bit in \field{gso_type}
    indicates that the TCP packet has the ECN bit set\footnote{This case is not handled by some older hardware, so is called out
specifically in the protocol.}.
   \end{itemize}

\item \field{num_buffers} is set to zero.  This field is unused on transmitted packets.

\item The header and packet are added as one output descriptor to the
  transmitq, and the device is notified of the new entry
  (see \ref{sec:Device Types / Network Device / Device Initialization}~\nameref{sec:Device Types / Network Device / Device Initialization}).
\end{enumerate}

\drivernormative{\paragraph}{Packet Transmission}{Device Types / Network Device / Device Operation / Packet Transmission}

The driver MUST set \field{num_buffers} to zero.

If VIRTIO_NET_F_CSUM is not negotiated, the driver MUST set
\field{flags} to zero and SHOULD supply a fully checksummed
packet to the device.

If VIRTIO_NET_F_HOST_TSO4 is negotiated, the driver MAY set
\field{gso_type} to VIRTIO_NET_HDR_GSO_TCPV4 to request TCPv4
segmentation, otherwise the driver MUST NOT set
\field{gso_type} to VIRTIO_NET_HDR_GSO_TCPV4.

If VIRTIO_NET_F_HOST_TSO6 is negotiated, the driver MAY set
\field{gso_type} to VIRTIO_NET_HDR_GSO_TCPV6 to request TCPv6
segmentation, otherwise the driver MUST NOT set
\field{gso_type} to VIRTIO_NET_HDR_GSO_TCPV6.

If VIRTIO_NET_F_HOST_UFO is negotiated, the driver MAY set
\field{gso_type} to VIRTIO_NET_HDR_GSO_UDP to request UDP
segmentation, otherwise the driver MUST NOT set
\field{gso_type} to VIRTIO_NET_HDR_GSO_UDP.

The driver SHOULD NOT send to the device TCP packets requiring segmentation offload
which have the Explicit Congestion Notification bit set, unless the
VIRTIO_NET_F_HOST_ECN feature is negotiated, in which case the
driver MUST set the VIRTIO_NET_HDR_GSO_ECN bit in
\field{gso_type}.

If the VIRTIO_NET_F_CSUM feature has been negotiated, the
driver MAY set the VIRTIO_NET_HDR_F_NEEDS_CSUM bit in
\field{flags}, if so:
\begin{enumerate}
\item the driver MUST validate the packet checksum at
	offset \field{csum_offset} from \field{csum_start} as well as all
	preceding offsets;
\item the driver MUST set the packet checksum stored in the
	buffer to the TCP/UDP pseudo header;
\item the driver MUST set \field{csum_start} and
	\field{csum_offset} such that calculating a ones'
	complement checksum from \field{csum_start} up until the end of
	the packet and storing the result at offset \field{csum_offset}
	from  \field{csum_start} will result in a fully checksummed
	packet;
\end{enumerate}

If none of the VIRTIO_NET_F_HOST_TSO4, TSO6 or UFO options have
been negotiated, the driver MUST set \field{gso_type} to
VIRTIO_NET_HDR_GSO_NONE.

If \field{gso_type} differs from VIRTIO_NET_HDR_GSO_NONE, then
the driver MUST also set the VIRTIO_NET_HDR_F_NEEDS_CSUM bit in
\field{flags} and MUST set \field{gso_size} to indicate the
desired MSS.

If one of the VIRTIO_NET_F_HOST_TSO4, TSO6 or UFO options have
been negotiated, the driver SHOULD set \field{hdr_len} to a value
not less than the length of the headers, including the transport
header.

The driver MUST NOT set the VIRTIO_NET_HDR_F_DATA_VALID bit in
\field{flags}.

\devicenormative{\paragraph}{Packet Transmission}{Device Types / Network Device / Device Operation / Packet Transmission}
The device MUST ignore \field{flag} bits that it does not recognize.

If VIRTIO_NET_HDR_F_NEEDS_CSUM bit in \field{flags} is not set, the
device MUST NOT use the \field{csum_start} and \field{csum_offset}.

If one of the VIRTIO_NET_F_HOST_TSO4, TSO6 or UFO options have
been negotiated, the device MAY use \field{hdr_len} only as a hint about the
transport header size.
The device MUST NOT rely on \field{hdr_len} to be correct.
\begin{note}
This is due to various bugs in implementations.
\end{note}

If VIRTIO_NET_HDR_F_NEEDS_CSUM is not set, the device MUST NOT
rely on the packet checksum being correct.
\paragraph{Packet Transmission Interrupt}\label{sec:Device Types / Network Device / Device Operation / Packet Transmission / Packet Transmission Interrupt}

Often a driver will suppress transmission interrupts using the
VIRTQ_AVAIL_F_NO_INTERRUPT flag
 (see \ref{sec:General Initialization And Device Operation / Device Operation / Receiving Used Buffers From The Device}~\nameref{sec:General Initialization And Device Operation / Device Operation / Receiving Used Buffers From The Device})
and check for used packets in the transmit path of following
packets.

The normal behavior in this interrupt handler is to retrieve and
new descriptors from the used ring and free the corresponding
headers and packets.

\subsubsection{Setting Up Receive Buffers}\label{sec:Device Types / Network Device / Device Operation / Setting Up Receive Buffers}

It is generally a good idea to keep the receive virtqueue as
fully populated as possible: if it runs out, network performance
will suffer.

If the VIRTIO_NET_F_GUEST_TSO4, VIRTIO_NET_F_GUEST_TSO6 or
VIRTIO_NET_F_GUEST_UFO features are used, the maximum incoming packet
will be to 65550 bytes long (the maximum size of a
TCP or UDP packet, plus the 14 byte ethernet header), otherwise
1514 bytes.  The 12-byte struct virtio_net_hdr is prepended to this,
making for 65562 or 1526 bytes.

\drivernormative{\paragraph}{Setting Up Receive Buffers}{Device Types / Network Device / Device Operation / Setting Up Receive Buffers}

\begin{itemize}
\item If VIRTIO_NET_F_MRG_RXBUF is not negotiated:
  \begin{itemize}
    \item If VIRTIO_NET_F_GUEST_TSO4, VIRTIO_NET_F_GUEST_TSO6 or
      VIRTIO_NET_F_GUEST_UFO are negotiated, the driver SHOULD populate
      the receive queue(s) with buffers of at least 65562 bytes.
    \item Otherwise, the driver SHOULD populate the receive queue(s)
      with buffers of at least 1526 bytes.
  \end{itemize}
\item If VIRTIO_NET_F_MRG_RXBUF is negotiated, each buffer MUST be at
  greater than the size of the struct virtio_net_hdr.
\end{itemize}

\begin{note}
Obviously each buffer can be split across multiple descriptor elements.
\end{note}

If VIRTIO_NET_F_MQ is negotiated, each of receiveq1\ldots receiveqN
that will be used SHOULD be populated with receive buffers.

\devicenormative{\paragraph}{Setting Up Receive Buffers}{Device Types / Network Device / Device Operation / Setting Up Receive Buffers}

The device MUST set \field{num_buffers} to the number of descriptors used to
hold the incoming packet.

The device MUST use only a single descriptor if VIRTIO_NET_F_MRG_RXBUF
was not negotiated. \note{This means that \field{num_buffers} will always be 1 
if VIRTIO_NET_F_MRG_RXBUF is not negotiated.}

\subsubsection{Processing of Incoming Packets}\label{sec:Device Types / Network Device / Device Operation / Processing of Incoming Packets}
\label{sec:Device Types / Network Device / Device Operation / Processing of Packets}%old label for latexdiff

When a packet is copied into a buffer in the receiveq, the
optimal path is to disable further interrupts for the receiveq
(see \ref{sec:General Initialization And Device Operation / Device Operation / Receiving Used Buffers From The Device}~\nameref{sec:General Initialization And Device Operation / Device Operation / Receiving Used Buffers From The Device}) and process
packets until no more are found, then re-enable them.

Processing incoming packets involves:

\begin{enumerate}
\item \field{num_buffers} indicates how many descriptors
  this packet is spread over (including this one): this will
  always be 1 if VIRTIO_NET_F_MRG_RXBUF was not negotiated.
  This allows receipt of large packets without having to allocate large
  buffers. In this case, there will be at least \field{num_buffers} in
  the used ring, and the device chains them together to form a
  single packet. The other buffers will not begin with a struct
  virtio_net_hdr.

\item If
  \field{num_buffers} is one, then the entire packet will be
  contained within this buffer, immediately following the struct
  virtio_net_hdr.
\item If the VIRTIO_NET_F_GUEST_CSUM feature was negotiated, the
  VIRTIO_NET_HDR_F_DATA_VALID bit in \field{flags} can be
  set: if so, device has validated the packet checksum.
  In case of multiple encapsulated protocols, one level of checksums
  has been validated.
\end{enumerate}

Additionally, VIRTIO_NET_F_GUEST_CSUM, TSO4, TSO6, UDP and ECN
features enable receive checksum, large receive offload and ECN
support which are the input equivalents of the transmit checksum,
transmit segmentation offloading and ECN features, as described
in \ref{sec:Device Types / Network Device / Device Operation /
Packet Transmission}:
\begin{enumerate}
\item If the VIRTIO_NET_F_GUEST_CSUM feature was negotiated, the
  VIRTIO_NET_HDR_F_NEEDS_CSUM bit in \field{flags} can be
  set: if so, the packet checksum at offset \field{csum_offset}
  from \field{csum_start} and any preceding checksums
  have been validated.  The checksum on the packet is incomplete and
  \field{csum_start} and \field{csum_offset} indicate how to calculate
  it (see Packet Transmission point 1).

\item If the VIRTIO_NET_F_GUEST_TSO4, TSO6 or UFO options were
  negotiated, then \field{gso_type} MAY be something other than
  VIRTIO_NET_HDR_GSO_NONE, and \field{gso_size} field indicates the
  desired MSS (see Packet Transmission point 2).
\end{enumerate}

\devicenormative{\paragraph}{Processing of Incoming Packets}{Device Types / Network Device / Device Operation / Processing of Incoming Packets}
\label{devicenormative:Device Types / Network Device / Device Operation / Processing of Packets}%old label for latexdiff

If VIRTIO_NET_F_MRG_RXBUF has not been negotiated, the device MUST set
\field{num_buffers} to 1.

If VIRTIO_NET_F_MRG_RXBUF has been negotiated, the device MUST set
\field{num_buffers} to indicate the number of descriptors
the packet (including the header) is spread over.

The device MUST use all descriptors used by a single receive
packet together, by atomically incrementing \field{idx} in the
used ring by the \field{num_buffers} value.

If VIRTIO_NET_F_GUEST_CSUM is not negotiated, the device MUST set
\field{flags} to zero and SHOULD supply a fully checksummed
packet to the driver.

If VIRTIO_NET_F_GUEST_TSO4 is not negotiated, the device MUST NOT set
\field{gso_type} to VIRTIO_NET_HDR_GSO_TCPV4.

If VIRTIO_NET_F_GUEST_UDP is not negotiated, the device MUST NOT set
\field{gso_type} to VIRTIO_NET_HDR_GSO_UDP.

If VIRTIO_NET_F_GUEST_TSO6 is not negotiated, the device MUST NOT set
\field{gso_type} to VIRTIO_NET_HDR_GSO_TCPV6.

The device SHOULD NOT send to the driver TCP packets requiring segmentation offload
which have the Explicit Congestion Notification bit set, unless the
VIRTIO_NET_F_GUEST_ECN feature is negotiated, in which case the
device MUST set the VIRTIO_NET_HDR_GSO_ECN bit in
\field{gso_type}.

If the VIRTIO_NET_F_GUEST_CSUM feature has been negotiated, the
device MAY set the VIRTIO_NET_HDR_F_NEEDS_CSUM bit in
\field{flags}, if so:
\begin{enumerate}
\item the device MUST validate the packet checksum at
	offset \field{csum_offset} from \field{csum_start} as well as all
	preceding offsets;
\item the device MUST set the packet checksum stored in the
	receive buffer to the TCP/UDP pseudo header;
\item the device MUST set \field{csum_start} and
	\field{csum_offset} such that calculating a ones'
	complement checksum from \field{csum_start} up until the
	end of the packet and storing the result at offset
	\field{csum_offset} from  \field{csum_start} will result in a
	fully checksummed packet;
\end{enumerate}

If none of the VIRTIO_NET_F_GUEST_TSO4, TSO6 or UFO options have
been negotiated, the device MUST set \field{gso_type} to
VIRTIO_NET_HDR_GSO_NONE.

If \field{gso_type} differs from VIRTIO_NET_HDR_GSO_NONE, then
the device MUST also set the VIRTIO_NET_HDR_F_NEEDS_CSUM bit in
\field{flags} MUST set \field{gso_size} to indicate the desired MSS.

If one of the VIRTIO_NET_F_GUEST_TSO4, TSO6 or UFO options have
been negotiated, the device SHOULD set \field{hdr_len} to a value
not less than the length of the headers, including the transport
header.

If the VIRTIO_NET_F_GUEST_CSUM feature has been negotiated, the
device MAY set the VIRTIO_NET_HDR_F_DATA_VALID bit in
\field{flags}, if so, the device MUST validate the packet
checksum (in case of multiple encapsulated protocols, one level
of checksums is validated).

\drivernormative{\paragraph}{Processing of Incoming
Packets}{Device Types / Network Device / Device Operation /
Processing of Incoming Packets}

The driver MUST ignore \field{flag} bits that it does not recognize.

If VIRTIO_NET_HDR_F_NEEDS_CSUM bit in \field{flags} is not set, the
driver MUST NOT use the \field{csum_start} and \field{csum_offset}.

If one of the VIRTIO_NET_F_GUEST_TSO4, TSO6 or UFO options have
been negotiated, the driver MAY use \field{hdr_len} only as a hint about the
transport header size.
The driver MUST NOT rely on \field{hdr_len} to be correct.
\begin{note}
This is due to various bugs in implementations.
\end{note}

If neither VIRTIO_NET_HDR_F_NEEDS_CSUM nor
VIRTIO_NET_HDR_F_DATA_VALID is set, the driver MUST NOT
rely on the packet checksum being correct.
\subsubsection{Control Virtqueue}\label{sec:Device Types / Network Device / Device Operation / Control Virtqueue}

The driver uses the control virtqueue (if VIRTIO_NET_F_CTRL_VQ is
negotiated) to send commands to manipulate various features of
the device which would not easily map into the configuration
space.

All commands are of the following form:

\begin{lstlisting}
struct virtio_net_ctrl {
        u8 class;
        u8 command;
        u8 command-specific-data[];
        u8 ack;
};

/* ack values */
#define VIRTIO_NET_OK     0
#define VIRTIO_NET_ERR    1
\end{lstlisting}

The \field{class}, \field{command} and command-specific-data are set by the
driver, and the device sets the \field{ack} byte. There is little it can
do except issue a diagnostic if \field{ack} is not
VIRTIO_NET_OK.

\paragraph{Packet Receive Filtering}\label{sec:Device Types / Network Device / Device Operation / Control Virtqueue / Packet Receive Filtering}

If the VIRTIO_NET_F_CTRL_RX feature is negotiated, the driver can
send control commands for promiscuous mode, multicast receiving,
and filtering of MAC addresses.

\begin{note}
In general, these commands are best-effort: unwanted
packets could still arrive.
\end{note}

\paragraph{Setting Promiscuous Mode}\label{sec:Device Types / Network Device / Device Operation / Control Virtqueue / Setting Promiscuous Mode}

\begin{lstlisting}
#define VIRTIO_NET_CTRL_RX    0
 #define VIRTIO_NET_CTRL_RX_PROMISC      0
 #define VIRTIO_NET_CTRL_RX_ALLMULTI     1
\end{lstlisting}

The class VIRTIO_NET_CTRL_RX has two commands:
VIRTIO_NET_CTRL_RX_PROMISC turns promiscuous mode on and off, and
VIRTIO_NET_CTRL_RX_ALLMULTI turns all-multicast receive on and
off. The command-specific-data is one byte containing 0 (off) or
1 (on).

\paragraph{Setting MAC Address Filtering}\label{sec:Device Types / Network Device / Device Operation / Control Virtqueue / Setting MAC Address Filtering}

\begin{lstlisting}
struct virtio_net_ctrl_mac {
        le32 entries;
        u8 macs[entries][6];
};

#define VIRTIO_NET_CTRL_MAC    1
 #define VIRTIO_NET_CTRL_MAC_TABLE_SET        0
 #define VIRTIO_NET_CTRL_MAC_ADDR_SET         1
\end{lstlisting}

The device can filter incoming packets by any number of destination
MAC addresses\footnote{Since there are no guarantees, it can use a hash filter or
silently switch to allmulti or promiscuous mode if it is given too
many addresses.
}. This table is set using the class
VIRTIO_NET_CTRL_MAC and the command VIRTIO_NET_CTRL_MAC_TABLE_SET. The
command-specific-data is two variable length tables of 6-byte MAC
addresses (as described in struct virtio_net_ctrl_mac). The first table contains unicast addresses, and the second
contains multicast addresses.

The VIRTIO_NET_CTRL_MAC_ADDR_SET command is used to set the
default MAC address which rx filtering
accepts (and if VIRTIO_NET_F_MAC_ADDR has been negotiated,
this will be reflected in \field{mac} in config space).

The command-specific-data for VIRTIO_NET_CTRL_MAC_ADDR_SET is
the 6-byte MAC address.

\devicenormative{\subparagraph}{Setting MAC Address Filtering}{Device Types / Network Device / Device Operation / Control Virtqueue / Setting MAC Address Filtering}

The device MUST have an empty MAC filtering table on reset.

The device MUST update the MAC filtering table before it consumes
the VIRTIO_NET_CTRL_MAC_TABLE_SET command.

The device MUST update \field{mac} in config space before it consumes
the VIRTIO_NET_CTRL_MAC_ADDR_SET command, if VIRTIO_NET_F_MAC_ADDR has
been negotiated.

The device SHOULD drop incoming packets which have a destination MAC which
matches neither the \field{mac} (or that set with VIRTIO_NET_CTRL_MAC_ADDR_SET)
nor the MAC filtering table.

\drivernormative{\subparagraph}{Setting MAC Address Filtering}{Device Types / Network Device / Device Operation / Control Virtqueue / Setting MAC Address Filtering}

The driver MUST follow the VIRTIO_NET_CTRL_MAC_TABLE_SET command
by a le32 number, followed by that number of non-multicast
MAC addresses, followed by another le32 number, followed by
that number of multicast addresses.  Either number MAY be 0.

\subparagraph{Legacy Interface: Setting MAC Address Filtering}\label{sec:Device Types / Network Device / Device Operation / Control Virtqueue / Setting MAC Address Filtering / Legacy Interface: Setting MAC Address Filtering}
When using the legacy interface, transitional devices and drivers
MUST format \field{entries} in struct virtio_net_ctrl_mac
according to the native endian of the guest rather than
(necessarily when not using the legacy interface) little-endian.

Legacy drivers that didn't negotiate VIRTIO_NET_F_CTRL_MAC_ADDR
changed \field{mac} in config space when NIC is accepting
incoming packets. These drivers always wrote the mac value from
first to last byte, therefore after detecting such drivers,
a transitional device MAY defer MAC update, or MAY defer
processing incoming packets until driver writes the last byte
of \field{mac} in the config space.

\paragraph{VLAN Filtering}\label{sec:Device Types / Network Device / Device Operation / Control Virtqueue / VLAN Filtering}

If the driver negotiates the VIRTION_NET_F_CTRL_VLAN feature, it
can control a VLAN filter table in the device.

\begin{lstlisting}
#define VIRTIO_NET_CTRL_VLAN       2
 #define VIRTIO_NET_CTRL_VLAN_ADD             0
 #define VIRTIO_NET_CTRL_VLAN_DEL             1
\end{lstlisting}

Both the VIRTIO_NET_CTRL_VLAN_ADD and VIRTIO_NET_CTRL_VLAN_DEL
command take a little-endian 16-bit VLAN id as the command-specific-data.

\subparagraph{Legacy Interface: VLAN Filtering}\label{sec:Device Types / Network Device / Device Operation / Control Virtqueue / VLAN Filtering / Legacy Interface: VLAN Filtering}
When using the legacy interface, transitional devices and drivers
MUST format the VLAN id
according to the native endian of the guest rather than
(necessarily when not using the legacy interface) little-endian.

\paragraph{Gratuitous Packet Sending}\label{sec:Device Types / Network Device / Device Operation / Control Virtqueue / Gratuitous Packet Sending}

If the driver negotiates the VIRTIO_NET_F_GUEST_ANNOUNCE (depends
on VIRTIO_NET_F_CTRL_VQ), the device can ask the driver to send gratuitous
packets; this is usually done after the guest has been physically
migrated, and needs to announce its presence on the new network
links. (As hypervisor does not have the knowledge of guest
network configuration (eg. tagged vlan) it is simplest to prod
the guest in this way).

\begin{lstlisting}
#define VIRTIO_NET_CTRL_ANNOUNCE       3
 #define VIRTIO_NET_CTRL_ANNOUNCE_ACK             0
\end{lstlisting}

The driver checks VIRTIO_NET_S_ANNOUNCE bit in the device configuration \field{status} field
when it notices the changes of device configuration. The
command VIRTIO_NET_CTRL_ANNOUNCE_ACK is used to indicate that
driver has received the notification and device clears the
VIRTIO_NET_S_ANNOUNCE bit in \field{status}.

Processing this notification involves:

\begin{enumerate}
\item Sending the gratuitous packets (eg. ARP) or marking there are pending
  gratuitous packets to be sent and letting deferred routine to
  send them.

\item Sending VIRTIO_NET_CTRL_ANNOUNCE_ACK command through control
  vq.
\end{enumerate}

\drivernormative{\subparagraph}{Gratuitous Packet Sending}{Device Types / Network Device / Device Operation / Control Virtqueue / Gratuitous Packet Sending}

If the driver negotiates VIRTIO_NET_F_GUEST_ANNOUNCE, it SHOULD notify
network peers of its new location after it sees the VIRTIO_NET_S_ANNOUNCE bit
in \field{status}.  The driver MUST send a command on the command queue
with class VIRTIO_NET_CTRL_ANNOUNCE and command VIRTIO_NET_CTRL_ANNOUNCE_ACK.

\devicenormative{\subparagraph}{Gratuitous Packet Sending}{Device Types / Network Device / Device Operation / Control Virtqueue / Gratuitous Packet Sending}

If VIRTIO_NET_F_GUEST_ANNOUNCE is negotiated, the device MUST clear the
VIRTIO_NET_S_ANNOUNCE bit in \field{status} upon receipt of a command buffer
with class VIRTIO_NET_CTRL_ANNOUNCE and command VIRTIO_NET_CTRL_ANNOUNCE_ACK
before marking the buffer as used.

\paragraph{Automatic receive steering in multiqueue mode}\label{sec:Device Types / Network Device / Device Operation / Control Virtqueue / Automatic receive steering in multiqueue mode}

If the driver negotiates the VIRTIO_NET_F_MQ feature bit (depends
on VIRTIO_NET_F_CTRL_VQ), it MAY transmit outgoing packets on one
of the multiple transmitq1\ldots transmitqN and ask the device to
queue incoming packets into one of the multiple receiveq1\ldots receiveqN
depending on the packet flow.

\begin{lstlisting}
struct virtio_net_ctrl_mq {
        le16 virtqueue_pairs;
};

#define VIRTIO_NET_CTRL_MQ    4
 #define VIRTIO_NET_CTRL_MQ_VQ_PAIRS_SET        0
 #define VIRTIO_NET_CTRL_MQ_VQ_PAIRS_MIN        1
 #define VIRTIO_NET_CTRL_MQ_VQ_PAIRS_MAX        0x8000
\end{lstlisting}

Multiqueue is disabled by default. The driver enables multiqueue by
executing the VIRTIO_NET_CTRL_MQ_VQ_PAIRS_SET command, specifying
the number of the transmit and receive queues to be used up to
\field{max_virtqueue_pairs}; subsequently,
transmitq1\ldots transmitqn and receiveq1\ldots receiveqn where
n=\field{virtqueue_pairs} MAY be used.

When multiqueue is enabled, the device MUST use automatic receive steering
based on packet flow. Programming of the receive steering
classificator is implicit. After the driver transmitted a packet of a
flow on transmitqX, the device SHOULD cause incoming packets for that flow to
be steered to receiveqX. For uni-directional protocols, or where
no packets have been transmitted yet, the device MAY steer a packet
to a random queue out of the specified receiveq1\ldots receiveqn.

Multiqueue is disabled by setting \field{virtqueue_pairs} to 1 (this is
the default) and waiting for the device to use the command buffer.

\drivernormative{\subparagraph}{Automatic receive steering in multiqueue mode}{Device Types / Network Device / Device Operation / Control Virtqueue / Automatic receive steering in multiqueue mode}

The driver MUST configure the virtqueues before enabling them with the 
VIRTIO_NET_CTRL_MQ_VQ_PAIRS_SET command.

The driver MUST NOT request a \field{virtqueue_pairs} of 0 or
greater than \field{max_virtqueue_pairs} in the device configuration space.

The driver MUST queue packets only on any transmitq1 before the 
VIRTIO_NET_CTRL_MQ_VQ_PAIRS_SET command.

The driver MUST NOT queue packets on transmit queues greater than
\field{virtqueue_pairs} once it has placed the VIRTIO_NET_CTRL_MQ_VQ_PAIRS_SET command in the available ring.

\devicenormative{\subparagraph}{Automatic receive steering in multiqueue mode}{Device Types / Network Device / Device Operation / Control Virtqueue / Automatic receive steering in multiqueue mode}

The device MUST queue packets only on any receiveq1 before the 
VIRTIO_NET_CTRL_MQ_VQ_PAIRS_SET command.

The device MUST NOT queue packets on receive queues greater than
\field{virtqueue_pairs} once it has placed the VIRTIO_NET_CTRL_MQ_VQ_PAIRS_SET command in the used ring.

\subparagraph{Legacy Interface: Automatic receive steering in multiqueue mode}\label{sec:Device Types / Network Device / Device Operation / Control Virtqueue / Automatic receive steering in multiqueue mode / Legacy Interface: Automatic receive steering in multiqueue mode}
When using the legacy interface, transitional devices and drivers
MUST format \field{virtqueue_pairs}
according to the native endian of the guest rather than
(necessarily when not using the legacy interface) little-endian.

\paragraph{Offloads State Configuration}\label{sec:Device Types / Network Device / Device Operation / Control Virtqueue / Offloads State Configuration}

If the VIRTIO_NET_F_CTRL_GUEST_OFFLOADS feature is negotiated, the driver can
send control commands for dynamic offloads state configuration.

\subparagraph{Setting Offloads State}\label{sec:Device Types / Network Device / Device Operation / Control Virtqueue / Offloads State Configuration / Setting Offloads State}

\begin{lstlisting}
le64 offloads;

#define VIRTIO_NET_F_GUEST_CSUM       1
#define VIRTIO_NET_F_GUEST_TSO4       7
#define VIRTIO_NET_F_GUEST_TSO6       8
#define VIRTIO_NET_F_GUEST_ECN        9
#define VIRTIO_NET_F_GUEST_UFO        10

#define VIRTIO_NET_CTRL_GUEST_OFFLOADS       5
 #define VIRTIO_NET_CTRL_GUEST_OFFLOADS_SET   0
\end{lstlisting}

The class VIRTIO_NET_CTRL_GUEST_OFFLOADS has one command:
VIRTIO_NET_CTRL_GUEST_OFFLOADS_SET applies the new offloads configuration.

le64 value passed as command data is a bitmask, bits set define
offloads to be enabled, bits cleared - offloads to be disabled.

There is a corresponding device feature for each offload. Upon feature
negotiation corresponding offload gets enabled to preserve backward
compartibility.

\drivernormative{\subparagraph}{Setting Offloads State}{Device Types / Network Device / Device Operation / Control Virtqueue / Offloads State Configuration / Setting Offloads State}

A driver MUST NOT enable an offload for which the appropriate feature
has not been negotiated.

\subparagraph{Legacy Interface: Setting Offloads State}\label{sec:Device Types / Network Device / Device Operation / Control Virtqueue / Offloads State Configuration / Setting Offloads State / Legacy Interface: Setting Offloads State}
When using the legacy interface, transitional devices and drivers
MUST format \field{offloads}
according to the native endian of the guest rather than
(necessarily when not using the legacy interface) little-endian.


\subsubsection{Legacy Interface: Framing Requirements}\label{sec:Device
Types / Network Device / Legacy Interface: Framing Requirements}

When using legacy interfaces, transitional drivers which have not
negotiated VIRTIO_F_ANY_LAYOUT MUST use a single descriptor for the
struct virtio_net_hdr on both transmit and receive, with the
network data in the following descriptors.

Additionally, when using the control virtqueue (see \ref{sec:Device
Types / Network Device / Device Operation / Control Virtqueue})
, transitional drivers which have not
negotiated VIRTIO_F_ANY_LAYOUT MUST:
\begin{itemize}
\item for all commands, use a single 2-byte descriptor including the first two
fields: \field{class} and \field{command}
\item for all commands except VIRTIO_NET_CTRL_MAC_TABLE_SET
use a single descriptor including command-specific-data
with no padding.
\item for the VIRTIO_NET_CTRL_MAC_TABLE_SET command use exactly
two descriptors including command-specific-data with no padding:
the first of these descriptors MUST include the
virtio_net_ctrl_mac table structure for the unicast addresses with no padding,
the second of these descriptors MUST include the
virtio_net_ctrl_mac table structure for the multicast addresses
with no padding.
\item for all commands, use a single 1-byte descriptor for the
\field{ack} field
\end{itemize}

See \ref{sec:Basic
Facilities of a Virtio Device / Virtqueues / Message Framing}.

\section{Block Device}\label{sec:Device Types / Block Device}

The virtio block device is a simple virtual block device (ie.
disk). Read and write requests (and other exotic requests) are
placed in the queue, and serviced (probably out of order) by the
device except where noted.

\subsection{Device ID}\label{sec:Device Types / Block Device / Device ID}
  2

\subsection{Virtqueues}\label{sec:Device Types / Block Device / Virtqueues}
\begin{description}
\item[0] requestq
\end{description}

\subsection{Feature bits}\label{sec:Device Types / Block Device / Feature bits}

\begin{description}
\item[VIRTIO_BLK_F_SIZE_MAX (1)] Maximum size of any single segment is
    in \field{size_max}.

\item[VIRTIO_BLK_F_SEG_MAX (2)] Maximum number of segments in a
    request is in \field{seg_max}.

\item[VIRTIO_BLK_F_GEOMETRY (4)] Disk-style geometry specified in
    \field{geometry}.

\item[VIRTIO_BLK_F_RO (5)] Device is read-only.

\item[VIRTIO_BLK_F_BLK_SIZE (6)] Block size of disk is in \field{blk_size}.

\item[VIRTIO_BLK_F_TOPOLOGY (10)] Device exports information on optimal I/O
    alignment.
\end{description}

\subsubsection{Legacy Interface: Feature bits}\label{sec:Device Types / Block Device / Feature bits / Legacy Interface: Feature bits}

\begin{description}
\item[VIRTIO_BLK_F_BARRIER (0)] Device supports request barriers.

\item[VIRTIO_BLK_F_SCSI (7)] Device supports scsi packet commands.

\item[VIRTIO_BLK_F_FLUSH (9)] Cache flush command support.

\item[VIRTIO_BLK_F_CONFIG_WCE (11)] Device can toggle its cache between writeback
    and writethrough modes.
\end{description}

VIRTIO_BLK_F_FLUSH was also called VIRTIO_BLK_F_WCE: Legacy drivers
MUST only negotiate this feature if they are capable of sending
VIRTIO_BLK_T_FLUSH commands.

\subsection{Device configuration layout}\label{sec:Device Types / Block Device / Device configuration layout}

The \field{capacity} of the device (expressed in 512-byte sectors) is always
present. The availability of the others all depend on various feature
bits as indicated above.

\begin{lstlisting}
struct virtio_blk_config {
        le64 capacity;
        le32 size_max;
        le32 seg_max;
        struct virtio_blk_geometry {
                le16 cylinders;
                u8 heads;
                u8 sectors;
        } geometry;
        le32 blk_size;
        struct virtio_blk_topology {
                // # of logical blocks per physical block (log2)
                u8 physical_block_exp;
                // offset of first aligned logical block
                u8 alignment_offset;
                // suggested minimum I/O size in blocks
                le16 min_io_size;
                // optimal (suggested maximum) I/O size in blocks
                le32 opt_io_size;
        } topology;
        u8 reserved;
};
\end{lstlisting}


\subsubsection{Legacy Interface: Device configuration layout}\label{sec:Device Types / Block Device / Device configuration layout / Legacy Interface: Device configuration layout}
When using the legacy interface, transitional devices and drivers
MUST format the fields in struct virtio_blk_config
according to the native endian of the guest rather than
(necessarily when not using the legacy interface) little-endian.


\subsection{Device Initialization}\label{sec:Device Types / Block Device / Device Initialization}

\begin{enumerate}
\item The device size can be read from \field{capacity}.

\item If the VIRTIO_BLK_F_BLK_SIZE feature is negotiated,
  \field{blk_size} can be read to determine the optimal sector size
  for the driver to use. This does not affect the units used in
  the protocol (always 512 bytes), but awareness of the correct
  value can affect performance.

\item If the VIRTIO_BLK_F_RO feature is set by the device, any write
  requests will fail.

\item If the VIRTIO_BLK_F_TOPOLOGY feature is negotiated, the fields in the
  \field{topology} struct can be read to determine the physical block size and optimal
  I/O lengths for the driver to use. This also does not affect the units
  in the protocol, only performance.
\end{enumerate}

\subsubsection{Legacy Interface: Device Initialization}\label{sec:Device Types / Block Device / Device Initialization / Legacy Interface: Device Initialization}

The \field{reserved} field used to be called \field{writeback}.  If the
VIRTIO_BLK_F_CONFIG_WCE feature is offered, the cache mode can be
read from \field{writeback}; the
driver can also write to the field in order to toggle the cache
between writethrough (0) and writeback (1) mode.  If the feature is
not available, the driver can instead look at the result of
negotiating VIRTIO_BLK_F_FLUSH: the cache will be in writeback mode
after reset if and only if VIRTIO_BLK_F_FLUSH is negotiated.

Some older legacy devices did not operate in writethrough mode even
after a driver announced lack of support for VIRTIO_BLK_F_FLUSH.

\subsection{Device Operation}\label{sec:Device Types / Block Device / Device Operation}

The driver queues requests to the virtqueue, and they are used by
the device (not necessarily in order). Each request is of form:

\begin{lstlisting}
struct virtio_blk_req {
        le32 type;
        le32 reserved;
        le64 sector;
        u8 data[][512];
        u8 status;
};
\end{lstlisting}

The type of the request is either a read (VIRTIO_BLK_T_IN), a write
(VIRTIO_BLK_T_OUT), or a flush (VIRTIO_BLK_T_FLUSH).

\begin{lstlisting}
#define VIRTIO_BLK_T_IN           0
#define VIRTIO_BLK_T_OUT          1
#define VIRTIO_BLK_T_FLUSH        4
\end{lstlisting}

The \field{sector} number indicates the offset (multiplied by 512) where
the read or write is to occur. This field is unused and set to 0
for scsi packet commands and for flush commands.

The final \field{status} byte is written by the device: either
VIRTIO_BLK_S_OK for success, VIRTIO_BLK_S_IOERR for device or driver
error or VIRTIO_BLK_S_UNSUPP for a request unsupported by device:

\begin{lstlisting}
#define VIRTIO_BLK_S_OK        0
#define VIRTIO_BLK_S_IOERR     1
#define VIRTIO_BLK_S_UNSUPP    2
\end{lstlisting}

\drivernormative{\subsubsection}{Device Operation}{Device Types / Block Device / Device Operation}

A driver MUST NOT submit a request which would cause a read or write
beyond \field{capacity}.

A driver SHOULD accept the VIRTIO_BLK_F_RO feature if offered.

A driver MUST set \field{sector} to 0 for a VIRTIO_BLK_T_FLUSH request.
A driver SHOULD NOT include any data in a VIRTIO_BLK_T_FLUSH request.

\devicenormative{\subsubsection}{Device Operation}{Device Types / Block Device / Device Operation}

A device MUST set the \field{status} byte to VIRTIO_BLK_S_IOERR
for a write request if the VIRTIO_BLK_F_RO feature if offered, and MUST NOT
write any data.

Upon receipt of a VIRTIO_BLK_T_FLUSH request, the driver SHOULD ensure
that any writes which were completed are committed to non-volatile storage.

\subsubsection{Legacy Interface: Device Operation}\label{sec:Device Types / Block Device / Device Operation / Legacy Interface: Device Operation}
When using the legacy interface, transitional devices and drivers
MUST format the fields in struct virtio_blk_req
according to the native endian of the guest rather than
(necessarily when not using the legacy interface) little-endian.

When using the legacy interface, transitional drivers
SHOULD ignore the \field{len} value in used ring entries.
\begin{note}
Historically, some devices put the total descriptor length,
or the total length of device-writable buffers there,
even when only the status byte was actually written.
\end{note}

The \field{reserved} field was previously called \field{ioprio}.  \field{ioprio}
is a hint about the relative priorities of requests to the device:
higher numbers indicate more important requests.

\begin{lstlisting}
#define VIRTIO_BLK_T_FLUSH_OUT    5
\end{lstlisting}

The command VIRTIO_BLK_T_FLUSH_OUT was a synonym for VIRTIO_BLK_T_FLUSH;
a driver MUST treat it as a VIRTIO_BLK_T_FLUSH command.

\begin{lstlisting}
#define VIRTIO_BLK_T_BARRIER     0x80000000
\end{lstlisting}

If the device has VIRTIO_BLK_F_BARRIER
feature the high bit (VIRTIO_BLK_T_BARRIER) indicates that this
request acts as a barrier and that all preceding requests SHOULD be
complete before this one, and all following requests SHOULD NOT be
started until this is complete.

\begin{note} A barrier does not flush
caches in the underlying backend device in host, and thus does not
serve as data consistency guarantee.  Only a VIRTIO_BLK_T_FLUSH request
does that.
\end{note}

If the device has VIRTIO_BLK_F_SCSI feature, it can also support
scsi packet command requests, each of these requests is of form:

\begin{lstlisting}
/* All fields are in guest's native endian. */
struct virtio_scsi_pc_req {
        u32 type;
        u32 ioprio;
        u64 sector;
        u8 cmd[];
        u8 data[][512];
#define SCSI_SENSE_BUFFERSIZE   96
        u8 sense[SCSI_SENSE_BUFFERSIZE];
        u32 errors;
        u32 data_len;
        u32 sense_len;
        u32 residual;
        u8 status;
};
\end{lstlisting}

A request type can also be a scsi packet command (VIRTIO_BLK_T_SCSI_CMD or
VIRTIO_BLK_T_SCSI_CMD_OUT).  The two types are equivalent, the device
does not distinguish between them:

\begin{lstlisting}
#define VIRTIO_BLK_T_SCSI_CMD     2
#define VIRTIO_BLK_T_SCSI_CMD_OUT 3
\end{lstlisting}

The \field{cmd} field is only present for scsi packet command requests,
and indicates the command to perform. This field MUST reside in a
single, separate device-readable buffer; command length can be derived
from the length of this buffer.

Note that these first three (four for scsi packet commands)
fields are always device-readable: \field{data} is either device-readable
or device-writable, depending on the request. The size of the read or
write can be derived from the total size of the request buffers.

\field{sense} is only present for scsi packet command requests,
and indicates the buffer for scsi sense data.

\field{data_len} is only present for scsi packet command
requests, this field is deprecated, and SHOULD be ignored by the
driver. Historically, devices copied data length there.

\field{sense_len} is only present for scsi packet command
requests and indicates the number of bytes actually written to
the \field{sense} buffer.

\field{residual} field is only present for scsi packet command
requests and indicates the residual size, calculated as data
length - number of bytes actually transferred.

\subsubsection{Legacy Interface: Framing Requirements}\label{sec:Device
Types / Block Device / Legacy Interface: Framing Requirements}

When using legacy interfaces, transitional drivers which have not
negotiated VIRTIO_F_ANY_LAYOUT:

\begin{itemize}
\item MUST use a single 8-byte descriptor containing \field{type},
  \field{reseved} and \field{sector}, followed by descriptors
  for \field{data}, then finally a separate 1-byte descriptor
  for \field{status}.

\item For SCSI commands there are additional constraints.
  \field{errors}, \field{data_len}, \field{sense_len} and
  \field{residual} MUST reside in a single, separate
  device-writable descriptor, \field{sense} MUST reside in a
  single separate device-writable descriptor of size 96 bytes,
  and \field{errors}, \field{data_len}, \field{sense_len} and
  \field{residual} MUST reside a single separate
  device-writable descriptor.
\end{itemize}

See \ref{sec:Basic Facilities of a Virtio Device / Virtqueues / Message Framing}.

\section{Console Device}\label{sec:Device Types / Console Device}

The virtio console device is a simple device for data input and
output. A device MAY have one or more ports. Each port has a pair
of input and output virtqueues. Moreover, a device has a pair of
control IO virtqueues. The control virtqueues are used to
communicate information between the device and the driver about
ports being opened and closed on either side of the connection,
indication from the device about whether a particular port is a
console port, adding new ports, port hot-plug/unplug, etc., and
indication from the driver about whether a port or a device was
successfully added, port open/close, etc. For data IO, one or
more empty buffers are placed in the receive queue for incoming
data and outgoing characters are placed in the transmit queue.

\subsection{Device ID}\label{sec:Device Types / Console Device / Device ID}

  3

\subsection{Virtqueues}\label{sec:Device Types / Console Device / Virtqueues}

\begin{description}
\item[0] receiveq(port0)
\item[1] transmitq(port0)
\item[2] control receiveq
\item[3] control transmitq
\item[4] receiveq(port1)
\item[5] transmitq(port1)
\item[\ldots]
\end{description}

The port 0 receive and transmit queues always exist: other queues
only exist if VIRTIO_CONSOLE_F_MULTIPORT is set.

\subsection{Feature bits}\label{sec:Device Types / Console Device / Feature bits}

\begin{description}
\item[VIRTIO_CONSOLE_F_SIZE (0)] Configuration \field{cols} and \field{rows}
    are valid.

\item[VIRTIO_CONSOLE_F_MULTIPORT (1)] Device has support for multiple
    ports; \field{max_nr_ports} is valid and control virtqueues will be used.

\item[VIRTIO_CONSOLE_F_EMERG_WRITE (2)] Device has support for emergency write.
    Configuration field emerg_wr is valid.
\end{description}

\subsection{Device configuration layout}\label{sec:Device Types / Console Device / Device configuration layout}

  The size of the console is supplied
  in the configuration space if the VIRTIO_CONSOLE_F_SIZE feature
  is set. Furthermore, if the VIRTIO_CONSOLE_F_MULTIPORT feature
  is set, the maximum number of ports supported by the device can
  be fetched.

  If VIRTIO_CONSOLE_F_EMERG_WRITE is set then the driver can use emergency write
  to output a single character without initializing virtio queues, or even
  acknowledging the feature.

\begin{lstlisting}
struct virtio_console_config {
        le16 cols;
        le16 rows;
        le32 max_nr_ports;
        le32 emerg_wr;
};
\end{lstlisting}

\subsubsection{Legacy Interface: Device configuration layout}\label{sec:Device Types / Console Device / Device configuration layout / Legacy Interface: Device configuration layout}
When using the legacy interface, transitional devices and drivers
MUST format the fields in struct virtio_console_config
according to the native endian of the guest rather than
(necessarily when not using the legacy interface) little-endian.

\subsection{Device Initialization}\label{sec:Device Types / Console Device / Device Initialization}

\begin{enumerate}
\item If the VIRTIO_CONSOLE_F_EMERG_WRITE feature is offered,
  \field{emerg_wr} field of the configuration can be written at any time.
  Thus it works for very early boot debugging output as well as
  catastophic OS failures (eg. virtio ring corruption).

\item If the VIRTIO_CONSOLE_F_SIZE feature is negotiated, the driver
  can read the console dimensions from \field{cols} and \field{rows}.

\item If the VIRTIO_CONSOLE_F_MULTIPORT feature is negotiated, the
  driver can spawn multiple ports, not all of which are necessarily
  attached to a console. Some could be generic ports. In this
  case, the control virtqueues are enabled and according to
  \field{max_nr_ports}, the appropriate number
  of virtqueues are created. A control message indicating the
  driver is ready is sent to the device. The device can then send
  control messages for adding new ports to the device. After
  creating and initializing each port, a
  VIRTIO_CONSOLE_PORT_READY control message is sent to the device
  for that port so the device can let the driver know of any additional
  configuration options set for that port.

\item The receiveq for each port is populated with one or more
  receive buffers.
\end{enumerate}

\devicenormative{\subsubsection}{Device Initialization}{Device Types / Console Device / Device Initialization}

The device MUST allow a write to \field{emerg_wr}, even on an
unconfigured device.

The device SHOULD transmit the lower byte written to \field{emerg_wr} to
an appropriate log or output method.

\subsection{Device Operation}\label{sec:Device Types / Console Device / Device Operation}

\begin{enumerate}
\item For output, a buffer containing the characters is placed in
  the port's transmitq\footnote{Because this is high importance and low bandwidth, the current
Linux implementation polls for the buffer to be used, rather than
waiting for an interrupt, simplifying the implementation
significantly. However, for generic serial ports with the
O_NONBLOCK flag set, the polling limitation is relaxed and the
consumed buffers are freed upon the next write or poll call or
when a port is closed or hot-unplugged.
}.

\item When a buffer is used in the receiveq (signalled by an
  interrupt), the contents is the input to the port associated
  with the virtqueue for which the notification was received.

\item If the driver negotiated the VIRTIO_CONSOLE_F_SIZE feature, a
  configuration change interrupt indicates that the updated size can
  be read from the configuration fields.  This size applies to port 0 only.

\item If the driver negotiated the VIRTIO_CONSOLE_F_MULTIPORT
  feature, active ports are announced by the device using the
  VIRTIO_CONSOLE_PORT_ADD control message. The same message is
  used for port hot-plug as well.
\end{enumerate}

\drivernormative{\subsubsection}{Device Operation}{Device Types / Console Device / Device Operation}

The driver MUST NOT put a device-readable in a receiveq. The driver
MUST NOT put a device-writable buffer in a transmitq.

\subsubsection{Multiport Device Operation}\label{sec:Device Types / Console Device / Device Operation / Multiport Device Operation}

If the driver negotiated the VIRTIO_CONSOLE_F_MULTIPORT, the two
control queues are used to manipulate the different console ports: the
control receiveq for messages from the device to the driver, and the
control sendq for driver-to-device messages.  The layout of the
control messages is:

\begin{lstlisting}
struct virtio_console_control {
        le32 id;    /* Port number */
        le16 event; /* The kind of control event */
        le16 value; /* Extra information for the event */
};
\end{lstlisting}

The values for \field{event} are:
\begin{description}
\item [VIRTIO_CONSOLE_DEVICE_READY (0)] Sent by the driver at initialization
  to indicate that it is ready to receive control messages.  A value of
  1 indicates success, and 0 indicates failure.  The port number \field{id} is unused.
\item [VIRTIO_CONSOLE_DEVICE_ADD (1)] Sent by the device, to create a new
  port.  \field{value} is unused.
\item [VIRTIO_CONSOLE_DEVICE_REMOVE (2)] Sent by the device, to remove an
  existing port. \field{value} is unused.
\item [VIRTIO_CONSOLE_PORT_READY (3)] Sent by the driver in response
  to the device's VIRTIO_CONSOLE_PORT_ADD message, to indicate that
  the port is ready to be used. A \field{value} of 1 indicates success, and 0
  indicates failure.
\item [VIRTIO_CONSOLE_CONSOLE_PORT (4)] Sent by the device to nominate
  a port as a console port.  There MAY be more than one console port.
\item [VIRTIO_CONSOLE_RESIZE (5)] Sent by the device to indicate
  a console size change.  \field{value} is unused.  The buffer is followed by the number of columns and rows:
\begin{lstlisting}
struct virtio_console_resize {
        le16 cols;
        le16 rows;
};
\end{lstlisting}
\item [VIRTIO_CONSOLE_PORT_OPEN (6)] This message is sent by both the
  device and the driver.  \field{value} indicates the state: 0 (port
  closed) or 1 (port open).  This allows for ports to be used directly
  by guest and host processes to communicate in an application-defined
  manner.
\item [VIRTIO_CONSOLE_PORT_NAME (7)] Sent by the device to give a tag
  to the port.  This control command is immediately
  followed by the UTF-8 name of the port for identification
  within the guest (without a NUL terminator).
\end{description}

\devicenormative{\paragraph}{Multiport Device Operation}{Device Types / Console Device / Device Operation / Multiport Device Operation}

The device MUST NOT specify a port which exists in a
VIRTIO_CONSOLE_DEVICE_ADD message, nor a port which is equal or
greater than \field{max_nr_ports}.

The device MUST NOT specify a port in VIRTIO_CONSOLE_DEVICE_REMOVE
which has not been created with a previous VIRTIO_CONSOLE_DEVICE_ADD.

\drivernormative{\paragraph}{Multiport Device Operation}{Device Types / Console Device / Device Operation / Multiport Device Operation}

The driver MUST send a VIRTIO_CONSOLE_DEVICE_READY message if
VIRTIO_CONSOLE_F_MULTIPORT is negotiated.

Upon receipt of a VIRTIO_CONSOLE_CONSOLE_PORT message, the driver
SHOULD treat the port in a manner suitable for text console access
and MUST respond with a VIRTIO_CONSOLE_PORT_OPEN message, which MUST
have \field{value} set to 1.

\subsubsection{Legacy Interface: Device Operation}\label{sec:Device Types / Console Device / Device Operation / Legacy Interface: Device Operation}
When using the legacy interface, transitional devices and drivers
MUST format the fields in struct virtio_console_control
according to the native endian of the guest rather than
(necessarily when not using the legacy interface) little-endian.

When using the legacy interface, the driver SHOULD ignore the
\field{len} value in used ring entries for the transmit queues
and the control transmitq.
\begin{note}
Historically, some devices put the total descriptor length there,
even though no data was actually written.
\end{note}

\subsubsection{Legacy Interface: Framing Requirements}\label{sec:Device
Types / Console Device / Legacy Interface: Framing Requirements}

When using legacy interfaces, transitional drivers which have not
negotiated VIRTIO_F_ANY_LAYOUT MUST use only a single
descriptor for all buffers in the control receiveq and control transmitq.

\section{Entropy Device}\label{sec:Device Types / Entropy Device}

The virtio entropy device supplies high-quality randomness for
guest use.

\subsection{Device ID}\label{sec:Device Types / Entropy Device / Device ID}
  4

\subsection{Virtqueues}\label{sec:Device Types / Entropy Device / Virtqueues}
\begin{description}
\item[0] requestq
\end{description}

\subsection{Feature bits}\label{sec:Device Types / Entropy Device / Feature bits}
  None currently defined

\subsection{Device configuration layout}\label{sec:Device Types / Entropy Device / Device configuration layout}
  None currently defined.

\subsection{Device Initialization}\label{sec:Device Types / Entropy Device / Device Initialization}

\begin{enumerate}
\item The virtqueue is initialized
\end{enumerate}

\subsection{Device Operation}\label{sec:Device Types / Entropy Device / Device Operation}

When the driver requires random bytes, it places the descriptor
of one or more buffers in the queue. It will be completely filled
by random data by the device.

\drivernormative{\subsubsection}{Device Operation}{Device Types / Entropy Device / Device Operation}

The driver MUST NOT place driver-readable buffers into the queue.

The driver MUST examine the length written by the device to determine
how many random bytes were received.

\devicenormative{\subsubsection}{Device Operation}{Device Types / Entropy Device / Device Operation}

The device MUST place one or more random bytes into the buffer, but it
MAY use less than the entire buffer length.

\section{Traditional Memory Balloon Device}\label{sec:Device Types / Memory Balloon Device}

This is the traditional balloon device.  The device number 13 is
reserved for a new memory balloon interface, with different
semantics, which is expected in a future version of the standard.

The traditional virtio memory balloon device is a primitive device for
managing guest memory: the device asks for a certain amount of
memory, and the driver supplies it (or withdraws it, if the device
has more than it asks for). This allows the guest to adapt to
changes in allowance of underlying physical memory. If the
feature is negotiated, the device can also be used to communicate
guest memory statistics to the host.

\subsection{Device ID}\label{sec:Device Types / Memory Balloon Device / Device ID}
  5

\subsection{Virtqueues}\label{sec:Device Types / Memory Balloon Device / Virtqueues}
\begin{description}
\item[0] inflateq
\item[1] deflateq
\item[2] statsq.
\end{description}

  Virtqueue 2 only exists if VIRTIO_BALLON_F_STATS_VQ set.

\subsection{Feature bits}\label{sec:Device Types / Memory Balloon Device / Feature bits}
\begin{description}
\item[VIRTIO_BALLOON_F_MUST_TELL_HOST (0)] Host has to be told before
    pages from the balloon are used.

\item[VIRTIO_BALLOON_F_STATS_VQ (1)] A virtqueue for reporting guest
    memory statistics is present.
\item[VIRTIO_BALLOON_F_DEFLATE_ON_OOM (2) ] Deflate balloon on
    guest out of memory condition.

\end{description}

\drivernormative{\subsubsection}{Feature bits}{Device Types / Memory Balloon Device / Feature bits}
The driver SHOULD accept the VIRTIO_BALLOON_F_MUST_TELL_HOST
feature if offered by the device.

\devicenormative{\subsubsection}{Feature bits}{Device Types / Memory Balloon Device / Feature bits}
If the device offers the VIRTIO_BALLOON_F_MUST_TELL_HOST feature
bit, and if the driver did not accept this feature bit, the
device MAY signal failure by failing to set FEATURES_OK
\field{device status} bit when the driver writes it.
\subparagraph{Legacy Interface: Feature bits}\label{sec:Device
Types / Memory Balloon Device / Feature bits / Legacy Interface:
Feature bits}
As the legacy interface does not have a way to gracefully report feature
negotiation failure, when using the legacy interface,
transitional devices MUST support guests which do not negotiate
VIRTIO_BALLOON_F_MUST_TELL_HOST feature, and SHOULD
allow guest to use memory before notifying host if
VIRTIO_BALLOON_F_MUST_TELL_HOST is not negotiated.

\subsection{Device configuration layout}\label{sec:Device Types / Memory Balloon Device / Device configuration layout}
  Both fields of this configuration
  are always available.

\begin{lstlisting}
struct virtio_balloon_config {
        le32 num_pages;
        le32 actual;
};
\end{lstlisting}

\subparagraph{Legacy Interface: Device configuration layout}\label{sec:Device Types / Memory Balloon Device / Device
configuration layout / Legacy Interface: Device configuration layout}
When using the legacy interface, transitional devices and drivers
MUST format the fields in struct virtio_balloon_config
according to the little-endian format.
\begin{note}
This is unlike the usual convention that legacy device fields are guest endian.
\end{note}

\subsection{Device Initialization}\label{sec:Device Types / Memory Balloon Device / Device Initialization}

The device initialization process is outlined below:

\begin{enumerate}
\item The inflate and deflate virtqueues are identified.

\item If the VIRTIO_BALLOON_F_STATS_VQ feature bit is negotiated:
  \begin{enumerate}
  \item Identify the stats virtqueue.
  \item Add one empty buffer to the stats virtqueue.
  \item DRIVER_OK is set: device operation begins.
  \item Notify the device about the stats virtqueue buffer.
  \end{enumerate}
\end{enumerate}

\subsection{Device Operation}\label{sec:Device Types / Memory Balloon Device / Device Operation}

The device is driven either by the receipt of a configuration
change interrupt, or by changing guest memory needs, such as
performing memory compaction or responding to out of memory
conditions.

\begin{enumerate}
\item \field{num_pages} configuration field is examined. If this is
  greater than the \field{actual} number of pages, the balloon wants
  more memory from the guest.  If it is less than \field{actual},
  the balloon doesn't need it all.

\item To supply memory to the balloon (aka. inflate):
  \begin{enumerate}
  \item The driver constructs an array of addresses of unused memory
    pages. These addresses are divided by 4096\footnote{This is historical, and independent of the guest page size.
} and the descriptor
    describing the resulting 32-bit array is added to the inflateq.
  \end{enumerate}

\item To remove memory from the balloon (aka. deflate):
  \begin{enumerate}
  \item The driver constructs an array of addresses of memory pages
    it has previously given to the balloon, as described above.
    This descriptor is added to the deflateq.

  \item If the VIRTIO_BALLOON_F_MUST_TELL_HOST feature is negotiated, the
    guest informs the device of pages before it uses them.

  \item Otherwise, the guest is allowed to re-use pages previously
    given to the balloon before the device has acknowledged their
    withdrawal\footnote{In this case, deflation advice is merely a courtesy.
}.
  \end{enumerate}

\item In either case, the device acknowledges inflate and deflate
requests by using the descriptor.
\item Once the device has acknowledged the inflation or
  deflation, the driver updates \field{actual} to reflect the new number of pages in the balloon.
\end{enumerate}

\drivernormative{\subsubsection}{Device Operation}{Device Types / Memory Balloon Device / Device Operation}
The driver SHOULD supply pages to the balloon when \field{num_pages} is
greater than the actual number of pages in the balloon.

The driver MAY use pages from the balloon when \field{num_pages} is
less than the actual number of pages in the balloon.

The driver MAY supply pages to the balloon when \field{num_pages} is
greater than or equal to the actual number of pages in the balloon.

If VIRTIO_BALLOON_F_DEFLATE_ON_OOM has not been negotiated, the
driver MUST NOT use pages from the balloon when \field{num_pages}
is less than or equal to the actual number of pages in the
balloon.

If VIRTIO_BALLOON_F_DEFLATE_ON_OOM has been negotiated, the
driver MAY use pages from the balloon when \field{num_pages}
is less than or equal to the actual number of pages in the
balloon if this is required for system stability
(e.g. if memory is required by applications running within
 the guest).

The driver MUST use the deflateq to inform the device of pages that it
wants to use from the balloon.

If the VIRTIO_BALLOON_F_MUST_TELL_HOST feature is negotiated, the
driver MUST NOT use pages from the balloon until
the device has acknowledged the deflate request.

Otherwise, if the VIRTIO_BALLOON_F_MUST_TELL_HOST feature is not
negotiated, the driver MAY begin to re-use pages previously
given to the balloon before the device has acknowledged the
deflate request.

In any case, the driver MUST NOT use pages from the balloon
after adding the pages to the balloon, but before the device has
acknowledged the inflate request.

The driver MUST NOT request deflation of pages in
the balloon before the device has acknowledged the inflate
request.

The driver MUST update \field{actual} after changing the number
of pages in the balloon.

The driver MAY update \field{actual} once after multiple
inflate and deflate operations.

\devicenormative{\subsubsection}{Device Operation}{Device Types / Memory Balloon Device / Device Operation}

The device MAY modify the contents of a page in the balloon
after detecting its physical number in an inflate request
and before acknowledging the inflate request by using the inflateq
descriptor.

If the VIRTIO_BALLOON_F_MUST_TELL_HOST feature is negotiated, the
device MAY modify the contents of a page in the balloon
after detecting its physical number in an inflate request
and before detecting its physical number in a deflate request
and acknowledging the deflate request.

\paragraph{Legacy Interface: Device Operation}\label{sec:Device
Types / Memory Balloon Device / Device Operation / Legacy
Interface: Device Operation}
When using the legacy interface, the driver SHOULD ignore the \field{len} value in used ring entries.
\begin{note}
Historically, some devices put the total descriptor length there,
even though no data was actually written.
\end{note}
When using the legacy interface, the driver MUST write out all
4 bytes each time it updates the \field{actual} value in the
configuration space, using a single atomic operation.

When using the legacy interface, the device SHOULD NOT use the
\field{actual} value written by the driver in the configuration
space, until the last, most-significant byte of the value has been
written.
\begin{note}
Historically, devices used the \field{actual} value, even though
when using Virtio Over PCI Bus the device-specific configuration
space was not guaranteed to be atomic. Using intermediate
values during update by driver is best avoided, except for
debugging.

Historically, drivers using Virtio Over PCI Bus wrote the
\field{actual} value by using multiple single-byte writes in
order, from the least-significant to the most-significant value.
\end{note}
\subsubsection{Memory Statistics}\label{sec:Device Types / Memory Balloon Device / Device Operation / Memory Statistics}

The stats virtqueue is atypical because communication is driven
by the device (not the driver). The channel becomes active at
driver initialization time when the driver adds an empty buffer
and notifies the device. A request for memory statistics proceeds
as follows:

\begin{enumerate}
\item The device pushes the buffer onto the used ring and sends an
  interrupt.

\item The driver pops the used buffer and discards it.

\item The driver collects memory statistics and writes them into a
  new buffer.

\item The driver adds the buffer to the virtqueue and notifies the
  device.

\item The device pops the buffer (retaining it to initiate a
  subsequent request) and consumes the statistics.
\end{enumerate}

  Within the buffer, statistics are an array of 6-byte entries.
  Each statistic consists of a 16 bit
  tag and a 64 bit value. All statistics are optional and the
  driver chooses which ones to supply. To guarantee backwards
  compatibility, devices omit unsupported statistics.

\begin{lstlisting}
struct virtio_balloon_stat {
#define VIRTIO_BALLOON_S_SWAP_IN  0
#define VIRTIO_BALLOON_S_SWAP_OUT 1
#define VIRTIO_BALLOON_S_MAJFLT   2
#define VIRTIO_BALLOON_S_MINFLT   3
#define VIRTIO_BALLOON_S_MEMFREE  4
#define VIRTIO_BALLOON_S_MEMTOT   5
        le16 tag;
        le64 val;
} __attribute__((packed));
\end{lstlisting}

\drivernormative{\paragraph}{Memory Statistics}{Device Types / Memory Balloon Device / Device Operation / Memory Statistics}
Normative statements in this section apply if and only if the
VIRTIO_BALLOON_F_STATS_VQ feature has been negotiated.

The driver MUST make at most one buffer available to the device
in the statsq, at all times.

After initializing the device, the driver MUST make an output
buffer available in the statsq.

Upon detecting that device has used a buffer in the statsq, the
driver MUST make an output buffer available in the statsq.

Before making an output buffer available in the statsq, the
driver MUST initialize it, including one struct
virtio_balloon_stat entry for each statistic that it supports.

Driver MUST use an output buffer size which is a multiple of 6
bytes for all buffers submitted to the statsq.

Driver MAY supply struct virtio_balloon_stat entries in the
output buffer submitted to the statsq in any order, without
regard to \field{tag} values.

Driver MAY supply a subset of all statistics in the output buffer
submitted to the statsq.

Driver MUST supply the same subset of statistics in all buffers
submitted to the statsq.

\devicenormative{\paragraph}{Memory Statistics}{Device Types / Memory Balloon Device / Device Operation / Memory Statistics}
Normative statements in this section apply if and only if  the
VIRTIO_BALLOON_F_STATS_VQ feature has been negotiated.

Within an output buffer submitted to the statsq,
the device MUST ignore entries with \field{tag} values that it does not recognize.

Within an output buffer submitted to the statsq,
the device MUST accept struct virtio_balloon_stat entries in any
order without regard to \field{tag} values.

\paragraph{Legacy Interface: Memory Statistics}\label{sec:Device Types / Memory Balloon Device / Device Operation / Memory Statistics / Legacy Interface: Memory Statistics}

When using the legacy interface, transitional devices and drivers
MUST format the fields in struct virtio_balloon_stat
according to the native endian of the guest rather than
(necessarily when not using the legacy interface) little-endian.

When using the legacy interface,
the device SHOULD ignore all values in the first buffer in the
statsq supplied by the driver after device initialization.
\begin{note}
Historically, drivers supplied an uninitialized buffer in the
first buffer.
\end{note}

\subsubsection{Memory Statistics Tags}\label{sec:Device Types / Memory Balloon Device / Device Operation / Memory Statistics Tags}

\begin{description}
\item[VIRTIO_BALLOON_S_SWAP_IN (0)] The amount of memory that has been
  swapped in (in bytes).

\item[VIRTIO_BALLOON_S_SWAP_OUT (1)] The amount of memory that has been
  swapped out to disk (in bytes).

\item[VIRTIO_BALLOON_S_MAJFLT (2)] The number of major page faults that
  have occurred.

\item[VIRTIO_BALLOON_S_MINFLT (3)] The number of minor page faults that
  have occurred.

\item[VIRTIO_BALLOON_S_MEMFREE (4)] The amount of memory not being used
  for any purpose (in bytes).

\item[VIRTIO_BALLOON_S_MEMTOT (5)] The total amount of memory available
  (in bytes).
\end{description}

\section{SCSI Host Device}\label{sec:Device Types / SCSI Host Device}

The virtio SCSI host device groups together one or more virtual
logical units (such as disks), and allows communicating to them
using the SCSI protocol. An instance of the device represents a
SCSI host to which many targets and LUNs are attached.

The virtio SCSI device services two kinds of requests:
\begin{itemize}
\item command requests for a logical unit;

\item task management functions related to a logical unit, target or
  command.
\end{itemize}

The device is also able to send out notifications about added and
removed logical units. Together, these capabilities provide a
SCSI transport protocol that uses virtqueues as the transfer
medium. In the transport protocol, the virtio driver acts as the
initiator, while the virtio SCSI host provides one or more
targets that receive and process the requests.

This section relies on definitions from \hyperref[intro:SAM]{SAM}.

\subsection{Device ID}\label{sec:Device Types / SCSI Host Device / Device ID}
  8

\subsection{Virtqueues}\label{sec:Device Types / SCSI Host Device / Virtqueues}

\begin{description}
\item[0] controlq
\item[1] eventq
\item[2\ldots n] request queues
\end{description}

\subsection{Feature bits}\label{sec:Device Types / SCSI Host Device / Feature bits}

\begin{description}
\item[VIRTIO_SCSI_F_INOUT (0)] A single request can include both
    device-readable and device-writable data buffers.

\item[VIRTIO_SCSI_F_HOTPLUG (1)] The host SHOULD enable reporting of
    hot-plug and hot-unplug events for LUNs and targets on the SCSI bus.
    The guest SHOULD handle hot-plug and hot-unplug events.

\item[VIRTIO_SCSI_F_CHANGE (2)] The host will report changes to LUN
    parameters via a VIRTIO_SCSI_T_PARAM_CHANGE event; the guest
    SHOULD handle them.

\item[VIRTIO_SCSI_F_T10_PI (3)] The extended fields for T10 protection
    information (DIF/DIX) are included in the SCSI request header.
\end{description}

\subsection{Device configuration layout}\label{sec:Device Types / SCSI Host Device / Device configuration layout}

  All fields of this configuration are always available.

\begin{lstlisting}
struct virtio_scsi_config {
        le32 num_queues;
        le32 seg_max;
        le32 max_sectors;
        le32 cmd_per_lun;
        le32 event_info_size;
        le32 sense_size;
        le32 cdb_size;
        le16 max_channel;
        le16 max_target;
        le32 max_lun;
};
\end{lstlisting}

\begin{description}
\item[\field{num_queues}] is the total number of request virtqueues exposed by
    the device. The driver MAY use only one request queue,
    or it can use more to achieve better performance.

\item[\field{seg_max}] is the maximum number of segments that can be in a
    command. A bidirectional command can include \field{seg_max} input
    segments and \field{seg_max} output segments.

\item[\field{max_sectors}] is a hint to the driver about the maximum transfer
    size to use.

\item[\field{cmd_per_lun}] is tells the driver the maximum number of
    linked commands it can send to one LUN.

\item[\field{event_info_size}] is the maximum size that the device will fill
    for buffers that the driver places in the eventq. It is
    written by the device depending on the set of negotiated
    features.

\item[\field{sense_size}] is the maximum size of the sense data that the
    device will write. The default value is written by the device
    and MUST be 96, but the driver can modify it. It is
    restored to the default when the device is reset.

\item[\field{cdb_size}] is the maximum size of the CDB that the driver will
    write. The default value is written by the device and MUST
    be 32, but the driver can likewise modify it. It is
    restored to the default when the device is reset.

\item[\field{max_channel}, \field{max_target} and \field{max_lun}] can be
    used by the driver as hints to constrain scanning the logical units
    on the host to channel/target/logical unit numbers that are less than
    or equal to the value of the fields.  \field{max_channel} SHOULD
    be zero.  \field{max_target} SHOULD be less than or equal to 255.
    \field{max_lun} SHOULD be less than or equal to 16383.
\end{description}

\drivernormative{\subsubsection}{Device configuration layout}{Device Types / SCSI Host Device / Device configuration layout}

The driver MUST NOT write to device configuration fields other than
\field{sense_size} and \field{cdb_size}.

The driver MUST NOT send more than \field{cmd_per_lun} linked commands
to one LUN, and MUST NOT send more than the virtqueue size number of
linked commands to one LUN.

\devicenormative{\subsubsection}{Device configuration layout}{Device Types / SCSI Host Device / Device configuration layout}

On reset, the device MUST set \field{sense_size} to 96 and
\field{cdb_size} to 32.

\subsubsection{Legacy Interface: Device configuration layout}\label{sec:Device Types / SCSI Host Device / Device configuration layout / Legacy Interface: Device configuration layout}
When using the legacy interface, transitional devices and drivers
MUST format the fields in struct virtio_scsi_config
according to the native endian of the guest rather than
(necessarily when not using the legacy interface) little-endian.

\devicenormative{\subsection}{Device Initialization}{Device Types / SCSI Host Device / Device Initialization}

On initialization the driver SHOULD first discover the
device's virtqueues.

If the driver uses the eventq, the driver SHOULD place at least one
buffer in the eventq.

The driver MAY immediately issue requests\footnote{For example, INQUIRY
or REPORT LUNS.} or task management functions\footnote{For example, I_T
RESET.}.

\subsection{Device Operation}\label{sec:Device Types / SCSI Host Device / Device Operation}

Device operation consists of operating request queues, the control
queue and the event queue.

\paragraph{Legacy Interface: Device Operation}\label{sec:Device
Types / SCSI Host Device / Device Operation / Legacy
Interface: Device Operation}
When using the legacy interface, the driver SHOULD ignore the \field{len} value in used ring entries.
\begin{note}
Historically, devices put the total descriptor length,
or the total length of device-writable buffers there,
even when only part of the buffers were actually written.
\end{note}

\subsubsection{Device Operation: Request Queues}\label{sec:Device Types / SCSI Host Device / Device Operation / Device Operation: Request Queues}

The driver queues requests to an arbitrary request queue, and
they are used by the device on that same queue. It is the
responsibility of the driver to ensure strict request ordering
for commands placed on different queues, because they will be
consumed with no order constraints.

Requests have the following format:

\begin{lstlisting}
struct virtio_scsi_req_cmd {
        // Device-readable part
        u8 lun[8];
        le64 id;
        u8 task_attr;
        u8 prio;
        u8 crn;
        u8 cdb[cdb_size];
        // The next two fields are only present if VIRTIO_SCSI_F_T10_PI
        // is negotiated.
        le32 pi_bytesout;
        le32 pi_bytesin;
        u8 pi_out[pi_bytesout];
        u8 dataout[];

        // Device-writable part
        le32 sense_len;
        le32 residual;
        le16 status_qualifier;
        u8 status;
        u8 response;
        u8 sense[sense_size];
        // The next two fields are only present if VIRTIO_SCSI_F_T10_PI
        // is negotiated
        u8 pi_in[pi_bytesin];
        u8 datain[];
};


/* command-specific response values */
#define VIRTIO_SCSI_S_OK                0
#define VIRTIO_SCSI_S_OVERRUN           1
#define VIRTIO_SCSI_S_ABORTED           2
#define VIRTIO_SCSI_S_BAD_TARGET        3
#define VIRTIO_SCSI_S_RESET             4
#define VIRTIO_SCSI_S_BUSY              5
#define VIRTIO_SCSI_S_TRANSPORT_FAILURE 6
#define VIRTIO_SCSI_S_TARGET_FAILURE    7
#define VIRTIO_SCSI_S_NEXUS_FAILURE     8
#define VIRTIO_SCSI_S_FAILURE           9

/* task_attr */
#define VIRTIO_SCSI_S_SIMPLE            0
#define VIRTIO_SCSI_S_ORDERED           1
#define VIRTIO_SCSI_S_HEAD              2
#define VIRTIO_SCSI_S_ACA               3
\end{lstlisting}

\field{lun} addresses the REPORT LUNS well-known logical unit, or
a target and logical unit in the virtio-scsi device's SCSI domain.
When used to address the REPORT LUNS logical unit, \field{lun} is 0xC1,
0x01 and six zero bytes.  The virtio-scsi device SHOULD implement the
REPORT LUNS well-known logical unit.

When used to address a target and logical unit, the only supported format
for \field{lun} is: first byte set to 1, second byte set to target,
third and fourth byte representing a single level LUN structure, followed
by four zero bytes. With this representation, a virtio-scsi device can
serve up to 256 targets and 16384 LUNs per target.  The device MAY also
support having a well-known logical units in the third and fourth byte.

\field{id} is the command identifier (``tag'').

\field{task_attr} defines the task attribute as in the table above, but
all task attributes MAY be mapped to SIMPLE by the device.  Some commands
are defined by SCSI standards as "implicit head of queue"; for such
commands, all task attributes MAY also be mapped to HEAD OF QUEUE.
Drivers and applications SHOULD NOT send a command with the ORDERED
task attribute if the command has an implicit HEAD OF QUEUE attribute,
because whether the ORDERED task attribute is honored is vendor-specific.

\field{crn} may also be provided by clients, but is generally expected
to be 0. The maximum CRN value defined by the protocol is 255, since
CRN is stored in an 8-bit integer.

The CDB is included in \field{cdb} and its size, \field{cdb_size},
is taken from the configuration space.

All of these fields are defined in \hyperref[intro:SAM]{SAM} and are
always device-readable.

\field{pi_bytesout} determines the size of the \field{pi_out} field
in bytes.  If it is nonzero, the \field{pi_out} field contains outgoing
protection information for write operations.  \field{pi_bytesin} determines
the size of the \field{pi_in} field in the device-writable section, in bytes.
All three fields are only present if VIRTIO_SCSI_F_T10_PI has been negotiated.

The remainder of the device-readable part is the data output buffer,
\field{dataout}.

\field{sense} and subsequent fields are always device-writable. \field{sense_len}
indicates the number of bytes actually written to the sense
buffer.

\field{residual} indicates the residual size,
calculated as ``data_length - number_of_transferred_bytes'', for
read or write operations. For bidirectional commands, the
number_of_transferred_bytes includes both read and written bytes.
A \field{residual} that is less than the size of \field{datain} means that
\field{dataout} was processed entirely. A \field{residual} that
exceeds the size of \field{datain} means that \field{dataout} was
processed partially and \field{datain} was not processed at
all.

If the \field{pi_bytesin} is nonzero, the \field{pi_in} field contains
incoming protection information for read operations.  \field{pi_in} is
only present if VIRTIO_SCSI_F_T10_PI has been negotiated\footnote{There
  is no separate residual size for \field{pi_bytesout} and
  \field{pi_bytesin}.  It can be computed from the \field{residual} field,
  the size of the data integrity information per sector, and the sizes
  of \field{pi_out}, \field{pi_in}, \field{dataout} and \field{datain}.}.

The remainder of the device-writable part is the data input buffer,
\field{datain}.


\devicenormative{\paragraph}{Device Operation: Request Queues}{Device Types / SCSI Host Device / Device Operation / Device Operation: Request Queues}

The device MUST write the \field{status} byte as the status code as
defined in \hyperref[intro:SAM]{SAM}.

The device MUST write the \field{response} byte as one of the following:

\begin{description}

\item[VIRTIO_SCSI_S_OK] when the request was completed and the \field{status}
  byte is filled with a SCSI status code (not necessarily
  ``GOOD'').

\item[VIRTIO_SCSI_S_OVERRUN] if the content of the CDB (such as the
  allocation length, parameter length or transfer size) requires
  more data than is available in the datain and dataout buffers.

\item[VIRTIO_SCSI_S_ABORTED] if the request was cancelled due to an
  ABORT TASK or ABORT TASK SET task management function.

\item[VIRTIO_SCSI_S_BAD_TARGET] if the request was never processed
  because the target indicated by \field{lun} does not exist.

\item[VIRTIO_SCSI_S_RESET] if the request was cancelled due to a bus
  or device reset (including a task management function).

\item[VIRTIO_SCSI_S_TRANSPORT_FAILURE] if the request failed due to a
  problem in the connection between the host and the target
  (severed link).

\item[VIRTIO_SCSI_S_TARGET_FAILURE] if the target is suffering a
  failure and to tell the driver not to retry on other paths.

\item[VIRTIO_SCSI_S_NEXUS_FAILURE] if the nexus is suffering a failure
  but retrying on other paths might yield a different result.

\item[VIRTIO_SCSI_S_BUSY] if the request failed but retrying on the
  same path is likely to work.

\item[VIRTIO_SCSI_S_FAILURE] for other host or driver error. In
  particular, if neither \field{dataout} nor \field{datain} is empty, and the
  VIRTIO_SCSI_F_INOUT feature has not been negotiated, the
  request will be immediately returned with a response equal to
  VIRTIO_SCSI_S_FAILURE.
\end{description}

All commands must be completed before the virtio-scsi device is
reset or unplugged.  The device MAY choose to abort them, or if
it does not do so MUST pick the VIRTIO_SCSI_S_FAILURE response.

\drivernormative{\paragraph}{Device Operation: Request Queues}{Device Types / SCSI Host Device / Device Operation / Device Operation: Request Queues}

\field{task_attr}, \field{prio} and \field{crn} SHOULD be zero.

Upon receiving a VIRTIO_SCSI_S_TARGET_FAILURE response, the driver
SHOULD NOT retry the request on other paths.

\paragraph{Legacy Interface: Device Operation: Request Queues}\label{sec:Device Types / SCSI Host Device / Device Operation / Device Operation: Request Queues / Legacy Interface: Device Operation: Request Queues}
When using the legacy interface, transitional devices and drivers
MUST format the fields in struct virtio_scsi_req_cmd
according to the native endian of the guest rather than
(necessarily when not using the legacy interface) little-endian.

\subsubsection{Device Operation: controlq}\label{sec:Device Types / SCSI Host Device / Device Operation / Device Operation: controlq}

The controlq is used for other SCSI transport operations.
Requests have the following format:

{
\lstset{escapechar=\$}
\begin{lstlisting}
struct virtio_scsi_ctrl {
        le32 type;
$\ldots$
        u8 response;
};

/* response values valid for all commands */
#define VIRTIO_SCSI_S_OK                       0
#define VIRTIO_SCSI_S_BAD_TARGET               3
#define VIRTIO_SCSI_S_BUSY                     5
#define VIRTIO_SCSI_S_TRANSPORT_FAILURE        6
#define VIRTIO_SCSI_S_TARGET_FAILURE           7
#define VIRTIO_SCSI_S_NEXUS_FAILURE            8
#define VIRTIO_SCSI_S_FAILURE                  9
#define VIRTIO_SCSI_S_INCORRECT_LUN            12
\end{lstlisting}
}

The \field{type} identifies the remaining fields.

The following commands are defined:

\begin{itemize}
\item Task management function.
\begin{lstlisting}
#define VIRTIO_SCSI_T_TMF                      0

#define VIRTIO_SCSI_T_TMF_ABORT_TASK           0
#define VIRTIO_SCSI_T_TMF_ABORT_TASK_SET       1
#define VIRTIO_SCSI_T_TMF_CLEAR_ACA            2
#define VIRTIO_SCSI_T_TMF_CLEAR_TASK_SET       3
#define VIRTIO_SCSI_T_TMF_I_T_NEXUS_RESET      4
#define VIRTIO_SCSI_T_TMF_LOGICAL_UNIT_RESET   5
#define VIRTIO_SCSI_T_TMF_QUERY_TASK           6
#define VIRTIO_SCSI_T_TMF_QUERY_TASK_SET       7

struct virtio_scsi_ctrl_tmf
{
        // Device-readable part
        le32 type;
        le32 subtype;
        u8   lun[8];
        le64 id;
        // Device-writable part
        u8   response;
}

/* command-specific response values */
#define VIRTIO_SCSI_S_FUNCTION_COMPLETE        0
#define VIRTIO_SCSI_S_FUNCTION_SUCCEEDED       10
#define VIRTIO_SCSI_S_FUNCTION_REJECTED        11
\end{lstlisting}

  The \field{type} is VIRTIO_SCSI_T_TMF; \field{subtype} defines which
  task management function. All
  fields except \field{response} are filled by the driver.

  Other fields which are irrelevant for the requested TMF
  are ignored but they are still present. \field{lun}
  is in the same format specified for request queues; the
  single level LUN is ignored when the task management function
  addresses a whole I_T nexus. When relevant, the value of \field{id}
  is matched against the id values passed on the requestq.

  The outcome of the task management function is written by the
  device in \field{response}. The command-specific response
  values map 1-to-1 with those defined in \hyperref[intro:SAM]{SAM}.

  Task management function can affect the response value for commands that
  are in the request queue and have not been completed yet.  For example,
  the device MUST complete all active commands on a logical unit
  or target (possibly with a VIRTIO_SCSI_S_RESET response code)
  upon receiving a "logical unit reset" or "I_T nexus reset" TMF.
  Similarly, the device MUST complete the selected commands (possibly
  with a VIRTIO_SCSI_S_ABORTED response code) upon receiving an "abort
  task" or "abort task set" TMF.  Such effects MUST take place before
  the TMF itself is successfully completed, and the device MUST use
  memory barriers appropriately in order to ensure that the driver sees
  these writes in the correct order.

\item Asynchronous notification query.
\begin{lstlisting}
#define VIRTIO_SCSI_T_AN_QUERY                    1

struct virtio_scsi_ctrl_an {
    // Device-readable part
    le32 type;
    u8   lun[8];
    le32 event_requested;
    // Device-writable part
    le32 event_actual;
    u8   response;
}

#define VIRTIO_SCSI_EVT_ASYNC_OPERATIONAL_CHANGE  2
#define VIRTIO_SCSI_EVT_ASYNC_POWER_MGMT          4
#define VIRTIO_SCSI_EVT_ASYNC_EXTERNAL_REQUEST    8
#define VIRTIO_SCSI_EVT_ASYNC_MEDIA_CHANGE        16
#define VIRTIO_SCSI_EVT_ASYNC_MULTI_HOST          32
#define VIRTIO_SCSI_EVT_ASYNC_DEVICE_BUSY         64
\end{lstlisting}

  By sending this command, the driver asks the device which
  events the given LUN can report, as described in paragraphs 6.6
  and A.6 of \hyperref[intro:SCSI MMC]{SCSI MMC}. The driver writes the
  events it is interested in into \field{event_requested}; the device
  responds by writing the events that it supports into
  \field{event_actual}.

  The \field{type} is VIRTIO_SCSI_T_AN_QUERY. \field{lun} and \field{event_requested}
  are written by the driver. \field{event_actual} and \field{response}
  fields are written by the device.

  No command-specific values are defined for the \field{response} byte.

\item Asynchronous notification subscription.
\begin{lstlisting}
#define VIRTIO_SCSI_T_AN_SUBSCRIBE                2

struct virtio_scsi_ctrl_an {
        // Device-readable part
        le32 type;
        u8   lun[8];
        le32 event_requested;
        // Device-writable part
        le32 event_actual;
        u8   response;
}
\end{lstlisting}

  By sending this command, the driver asks the specified LUN to
  report events for its physical interface, again as described in
   \hyperref[intro:SCSI MMC]{SCSI MMC}. The driver writes the events it is
  interested in into \field{event_requested}; the device responds by
  writing the events that it supports into \field{event_actual}.

  Event types are the same as for the asynchronous notification
  query message.

  The \field{type} is VIRTIO_SCSI_T_AN_SUBSCRIBE. \field{lun} and
  \field{event_requested} are written by the driver.
  \field{event_actual} and \field{response} are written by the device.

  No command-specific values are defined for the response byte.
\end{itemize}

\paragraph{Legacy Interface: Device Operation: controlq}\label{sec:Device Types / SCSI Host Device / Device Operation / Device Operation: controlq / Legacy Interface: Device Operation: controlq}

When using the legacy interface, transitional devices and drivers
MUST format the fields in struct virtio_scsi_ctrl, struct
virtio_scsi_ctrl_tmf, struct virtio_scsi_ctrl_an and struct
virtio_scsi_ctrl_an
according to the native endian of the guest rather than
(necessarily when not using the legacy interface) little-endian.


\subsubsection{Device Operation: eventq}\label{sec:Device Types / SCSI Host Device / Device Operation / Device Operation: eventq}

The eventq is populated by the driver for the device to report information on logical
units that are attached to it. In general, the device will not
queue events to cope with an empty eventq, and will end up
dropping events if it finds no buffer ready. However, when
reporting events for many LUNs (e.g. when a whole target
disappears), the device can throttle events to avoid dropping
them. For this reason, placing 10-15 buffers on the event queue
is sufficient.

Buffers returned by the device on the eventq will be referred to
as ``events'' in the rest of this section. Events have the
following format:

\begin{lstlisting}
#define VIRTIO_SCSI_T_EVENTS_MISSED   0x80000000

struct virtio_scsi_event {
        // Device-writable part
        le32 event;
        u8  lun[8];
        le32 reason;
}
\end{lstlisting}

The devices sets bit 31 in \field{event} to report lost events
due to missing buffers.

The meaning of \field{reason} depends on the
contents of \field{event}. The following events are defined:

\begin{itemize}
\item No event.
\begin{lstlisting}
#define VIRTIO_SCSI_T_NO_EVENT         0
\end{lstlisting}

  This event is fired in the following cases:

\begin{itemize}
\item When the device detects in the eventq a buffer that is
    shorter than what is indicated in the configuration field, it
    MAY use it immediately and put this dummy value in \field{event}.
    A well-written driver will never observe this
    situation.

\item When events are dropped, the device MAY signal this event as
    soon as the drivers makes a buffer available, in order to
    request action from the driver. In this case, of course, this
    event will be reported with the VIRTIO_SCSI_T_EVENTS_MISSED
    flag.
\end{itemize}

\item Transport reset
\begin{lstlisting}
#define VIRTIO_SCSI_T_TRANSPORT_RESET  1

#define VIRTIO_SCSI_EVT_RESET_HARD         0
#define VIRTIO_SCSI_EVT_RESET_RESCAN       1
#define VIRTIO_SCSI_EVT_RESET_REMOVED      2
\end{lstlisting}

  By sending this event, the device signals that a logical unit
  on a target has been reset, including the case of a new device
  appearing or disappearing on the bus. The device fills in all
  fields. \field{event} is set to
  VIRTIO_SCSI_T_TRANSPORT_RESET. \field{lun} addresses a
  logical unit in the SCSI host.

  The \field{reason} value is one of the three \#define values appearing
  above:

  \begin{description}
  \item[VIRTIO_SCSI_EVT_RESET_REMOVED] (``LUN/target removed'') is used
    if the target or logical unit is no longer able to receive
    commands.

  \item[VIRTIO_SCSI_EVT_RESET_HARD] (``LUN hard reset'') is used if the
    logical unit has been reset, but is still present.

  \item[VIRTIO_SCSI_EVT_RESET_RESCAN] (``rescan LUN/target'') is used if
    a target or logical unit has just appeared on the device.
  \end{description}

  The ``removed'' and ``rescan'' events can happen when
  VIRTIO_SCSI_F_HOTPLUG feature was negotiated; when sent for LUN 0,
  they MAY apply to the entire target so the driver can ask the
  initiator to rescan the target to detect this.

  Events will also be reported via sense codes (this obviously
  does not apply to newly appeared buses or targets, since the
  application has never discovered them):

  \begin{itemize}
  \item ``LUN/target removed'' maps to sense key ILLEGAL REQUEST, asc
    0x25, ascq 0x00 (LOGICAL UNIT NOT SUPPORTED)

  \item ``LUN hard reset'' maps to sense key UNIT ATTENTION, asc 0x29
    (POWER ON, RESET OR BUS DEVICE RESET OCCURRED)

  \item ``rescan LUN/target'' maps to sense key UNIT ATTENTION, asc
    0x3f, ascq 0x0e (REPORTED LUNS DATA HAS CHANGED)
  \end{itemize}

  The preferred way to detect transport reset is always to use
  events, because sense codes are only seen by the driver when it
  sends a SCSI command to the logical unit or target. However, in
  case events are dropped, the initiator will still be able to
  synchronize with the actual state of the controller if the
  driver asks the initiator to rescan of the SCSI bus. During the
  rescan, the initiator will be able to observe the above sense
  codes, and it will process them as if it the driver had
  received the equivalent event.

  \item Asynchronous notification
\begin{lstlisting}
#define VIRTIO_SCSI_T_ASYNC_NOTIFY     2
\end{lstlisting}

  By sending this event, the device signals that an asynchronous
  event was fired from a physical interface.

  All fields are written by the device. \field{event} is set to
  VIRTIO_SCSI_T_ASYNC_NOTIFY. \field{lun} addresses a logical
  unit in the SCSI host. \field{reason} is a subset of the
  events that the driver has subscribed to via the ``Asynchronous
  notification subscription'' command.

  \item LUN parameter change
\begin{lstlisting}
#define VIRTIO_SCSI_T_PARAM_CHANGE  3
\end{lstlisting}

  By sending this event, the device signals a change in the configuration parameters
  of a logical unit, for example the capacity or caching mode.
  \field{event} is set to VIRTIO_SCSI_T_PARAM_CHANGE.
  \field{lun} addresses a logical unit in the SCSI host.

  The same event SHOULD also be reported as a unit attention condition.
  \field{reason} contains the additional sense code and additional sense code qualifier,
  respectively in bits 0\ldots 7 and 8\ldots 15.
  \begin{note}
  For example, a change in capacity will be reported as asc 0x2a, ascq 0x09
  (CAPACITY DATA HAS CHANGED).
  \end{note}

  For MMC devices (inquiry type 5) there would be some overlap between this
  event and the asynchronous notification event, so for simplicity the host never
  reports this event for MMC devices.
\end{itemize}

\drivernormative{\paragraph}{Device Operation: eventq}{Device Types / SCSI Host Device / Device Operation / Device Operation: eventq}

The driver SHOULD keep the eventq populated with buffers.  These
buffers MUST be device-writable, and SHOULD be at least
\field{event_info_size} bytes long, and MUST be at least the size of
struct virtio_scsi_event.

If \field{event} has bit 31 set, the driver SHOULD
poll the logical units for unit attention conditions, and/or do
whatever form of bus scan is appropriate for the guest operating
system and SHOULD poll for asynchronous events manually using SCSI commands.

When receiving a VIRTIO_SCSI_T_TRANSPORT_RESET message with
\field{reason} set to VIRTIO_SCSI_EVT_RESET_REMOVED or
VIRTIO_SCSI_EVT_RESET_RESCAN for LUN 0, the driver SHOULD ask the
initiator to rescan the target, in order to detect the case when an
entire target has appeared or disappeared.

\devicenormative{\paragraph}{Device Operation: eventq}{Device Types / SCSI Host Device / Device Operation / Device Operation: eventq}

The device MUST set bit 31 in \field{event} if events were lost due to
missing buffers, and it MAY use a VIRTIO_SCSI_T_NO_EVENT event to report
this.

The device MUST NOT send VIRTIO_SCSI_T_TRANSPORT_RESET messages
with \field{reason} set to VIRTIO_SCSI_EVT_RESET_REMOVED or
VIRTIO_SCSI_EVT_RESET_RESCAN unless VIRTIO_SCSI_F_HOTPLUG was negotiated.

The device MUST NOT report VIRTIO_SCSI_T_PARAM_CHANGE for MMC devices.

\paragraph{Legacy Interface: Device Operation: eventq}\label{sec:Device Types / SCSI Host Device / Device Operation / Device Operation: eventq / Legacy Interface: Device Operation: eventq}
When using the legacy interface, transitional devices and drivers
MUST format the fields in struct virtio_scsi_event
according to the native endian of the guest rather than
(necessarily when not using the legacy interface) little-endian.

\subsubsection{Legacy Interface: Framing Requirements}\label{sec:Device
Types / SCSI Host Device / Legacy Interface: Framing Requirements}

When using legacy interfaces, transitional drivers which have not
negotiated VIRTIO_F_ANY_LAYOUT MUST use a single descriptor for the
\field{lun}, \field{id}, \field{task_attr}, \field{prio},
\field{crn} and \field{cdb} fields, and MUST only use a single
descriptor for the \field{sense_len}, \field{residual},
\field{status_qualifier}, \field{status}, \field{response} and
\field{sense} fields.

\chapter{Reserved Feature Bits}\label{sec:Reserved Feature Bits}

Currently there are three device-independent feature bits defined:

\begin{description}
  \item[VIRTIO_F_RING_INDIRECT_DESC (28)] Negotiating this feature indicates
  that the driver can use descriptors with the VIRTQ_DESC_F_INDIRECT
  flag set, as described in \ref{sec:Basic Facilities of a Virtio Device / Virtqueues / The Virtqueue Descriptor Table / Indirect Descriptors}~\nameref{sec:Basic Facilities of a Virtio Device / Virtqueues / The Virtqueue Descriptor Table / Indirect Descriptors}.

  \item[VIRTIO_F_RING_EVENT_IDX(29)] This feature enables the \field{used_event}
  and the \field{avail_event} fields as described in \ref{sec:Basic Facilities of a Virtio Device / Virtqueues / Virtqueue Interrupt Suppression} and \ref{sec:Basic Facilities of a Virtio Device / Virtqueues / The Virtqueue Used Ring}.

  \item[VIRTIO_F_VERSION_1(32)] This indicates compliance with this
    specification, giving a simple way to detect legacy devices or drivers.
\end{description}

\drivernormative{\section}{Reserved Feature Bits}{Reserved Feature Bits}

A driver MUST accept VIRTIO_F_VERSION_1 if it is offered.  A driver
MAY fail to operate further if VIRTIO_F_VERSION_1 is not offered.

\devicenormative{\section}{Reserved Feature Bits}{Reserved Feature Bits}

A device MUST offer VIRTIO_F_VERSION_1.  A device MAY fail to operate further
if VIRTIO_F_VERSION_1 is not accepted.

\section{Legacy Interface: Reserved Feature Bits}\label{sec:Reserved Feature Bits / Legacy Interface: Reserved Feature Bits}

Transitional devices MAY offer the following:
\begin{description}
\item[VIRTIO_F_NOTIFY_ON_EMPTY (24)] If this feature
  has been negotiated by driver, the device MUST issue
  an interrupt if the device runs
  out of available descriptors on a virtqueue, even though
  interrupts are suppressed using the VIRTQ_AVAIL_F_NO_INTERRUPT
  flag or the \field{used_event} field.
\begin{note}
  An example of a driver using this feature is the legacy
  networking driver: it doesn't need to know every time a packet
  is transmitted, but it does need to free the transmitted
  packets a finite time after they are transmitted. It can avoid
  using a timer if the device interrupts it when all the packets
  are transmitted.
\end{note}
\end{description}

Transitional devices MUST offer, and if offered by the device
transitional drivers MUST accept the following:
\begin{description}
\item[VIRTIO_F_ANY_LAYOUT (27)] This feature indicates that the device
  accepts arbitrary descriptor layouts, as described in Section
  \ref{sec:Basic Facilities of a Virtio Device / Virtqueues / Message Framing / Legacy Interface: Message Framing}~\nameref{sec:Basic Facilities of a Virtio Device / Virtqueues / Message Framing / Legacy Interface: Message Framing}.

\item[UNUSED (30)] Bit 30 is used by qemu's implementation to check
  for experimental early versions of virtio which did not perform
  correct feature negotiation, and SHOULD NOT be negotiated.
\end{description}
